\documentclass[12pt,french]{beamer}

\usepackage[utf8]{inputenc}
\usepackage[T1]{fontenc}
\usepackage{lmodern}
\usepackage{amsmath,amsfonts,amssymb}
\usepackage{enumitem}

\usepackage{babel}
\usepackage[lastexercise]{exercise}

\renewcommand{\ExerciseName}{Exercice}
\renewcommand{\AnswerName}{Réponse de l'exercice}

\newcommand{\N}{\mathbf{N}}

\everymath{\displaystyle\everymath{}}

\title{Correction exercices}
\date{7 octobre 2015}

\beamerdefaultoverlayspecification{<+->}

\begin{document}

\begin{frame}
  \maketitle
\end{frame}

\begin{frame}
  \begin{block}{$0\leq u_n \leqslant \frac1n$}
    \begin{itemize}
      \item $n! = 1\times2\times3\times4\times5\times\cdots\times n$
      \item $n^n = n\times n\times n\times n\times n\times\cdots\times n$
      \item $\forall 1 < k\leqslant n,\ \frac{k}{n} \leqslant 1$
      \item Premier terme : $\frac1n$
      \item d'où la majoration !
      \item $0<u_n$ !
    \end{itemize}
  \end{block}
  \begin{block}{Comportement en $+\infty$}
    \begin{itemize}
      \item suite encadrée par deux suites !
      \item $v_n = (0)_{n\in\N}$
      \item $w_n = \frac1n,\ \forall n \in\N^*$
      \item $\lim_{n\to+\infty}u_n = 0$
    \end{itemize}
  \end{block}
\end{frame}

\end{document}
