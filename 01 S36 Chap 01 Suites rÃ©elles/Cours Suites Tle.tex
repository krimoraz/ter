\documentclass[12pt,a4paper]{article}
\usepackage[utf8]{inputenc}
\usepackage[T1]{fontenc}
\usepackage[frenchb]{babel}
\usepackage{ntheorem}
\usepackage{amsmath}
\usepackage{amsfonts}

\usepackage{kpfonts}

\usepackage[bookmarks=false,colorlinks,linkcolor=blue]{hyperref}

\pdfminorversion 7
\pdfobjcompresslevel 3

\usepackage{tabularx}

\usepackage{pgf}
\usepackage{tikz}
\usepackage{tkz-euclide}
\usetkzobj{all}
\usepackage[top=1.7cm,bottom=2cm,left=2cm,right=2cm]{geometry}

\usepackage{lastpage}

\usepackage{marginnote}

\usepackage{wrapfig}

\usepackage[autolanguage]{numprint}
\newcommand{\np}{\numprint}

\makeatletter
\renewcommand{\@evenfoot}%
        {\hfil \upshape \small page {\thepage} de \pageref{LastPage}}
\renewcommand{\@oddfoot}{\@evenfoot}

\renewcommand{\maketitle}%
{\framebox{%
    \begin{minipage}{1.0\linewidth}%
      \begin{center}%
        \Large \@title ~-- \@author \\%
        \@date%
      \end{center}%
    \end{minipage}}%
  \normalsize%
  %\vspace{1cm}%
}

\newcommand{\R}{\mathbf{R}}
\newcommand{\Vecteur}{\overrightarrow}
\newcommand{\norme}[1]{\lVert #1 \rVert}
\newcommand{\vabs}[1]{\lvert #1 \rvert}

\makeatother


\theoremstyle{break}
\newtheorem{definition}{Définition}
\newtheorem{propriete}{Propriété}
\newtheorem{propdef}{Propriété - Définition}
\theoremstyle{plain}
\theorembodyfont{\normalfont}
\newtheorem{exercice}{Exercice}
\theoremstyle{nonumberplain}
\newtheorem{remarque}{Remarque}
\newtheorem{probleme}{Problème}

\newtheorem{preuve}{Preuve}

% Mise en forme des labels dans les énumérations
\renewcommand{\labelenumi}{\textbf{\theenumi.}}
\renewcommand{\labelenumii}{\textbf{\theenumii)}}
\renewcommand{\theenumi}{\arabic{enumi}}
\renewcommand{\theenumii}{\alph{enumii}}

\renewcommand{\labelitemi}{$\bullet$}

\setlength{\parsep}{0pt}
\setlength{\parskip}{5pt}
\setlength{\parindent}{0pt}
\setlength{\itemsep}{7pt}

\usepackage{multicol}
\setlength{\columnseprule}{0pt}

\everymath{\displaystyle\everymath{}}

\title{La notion de suite}
\author{J.B.S. Terminale S}
\date{}

\begin{document}

\maketitle

\section{Ensembles, familles, applications et suites}

Le but ici n'est pas de rentrer dans les détails des quatre notions
mentionnées dans le titre, mais de se donner quelques élements et pistes
de réflexions sur ceux-ci.

\subsection{L'ensemble, une collection qui ne peut se contenir}

Lorsque les mathématiciens, au tournant du 20\ieme{} siècle élaborèrent
une première théorie des ensembles celles-ci souffraient de failles,
dont la principale fut mise en évidence par le philosophe et logicien
Bertrand Russel, sur une variation du paradoxe crétois ou paradoxe du
barbier. Ainsi, il démontra qu'une théorie trop naïve des ensembles,
pouvait, simplement en s'interrogeant sur l'appartenance d'un ensemble à
un autre conduire à des incohérences.

Lorsque la théorie des ensembles fut axiomatisée, on prit garde à y
ajouter l'axiome $x\not\in x$, évitant ainsi les paradoxes précédents.

En revanche, on définit la notion de parties d'un ensemble, comme l'ensemble
de tous les sous-ensembles que l'on peut former dans celui-ci. Deux ensembles
font toujours partie de $\mathscr{P}(E)$, c'est $E$ et $\emptyset$.

\subsection{La famille, un cas particulier de l'ensemble}

\begin{definition}
	Soient deux ensembles $E$ et $I$, une famille d'éléments de $E$ est un sous
	ensemble $F$ de $E$ tel que pour tout élément de $I$ il existe un élément de
	$F$.
	
	On note alors la famille $\mathscr{F}_I$  ou $(F_i)_{i\in I}$ et ses éléments
	$F_i$
\end{definition}

Attention, aucune hypothèse n'est ici faite sur $I$, ni fini, ni même dénombrable.

\subsection{La suite, un cas particulier des familles}

\end{document}
