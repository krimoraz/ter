\documentclass[frenchb]{beamer}

\usepackage[utf8]{inputenc}
\usepackage[T1]{fontenc}
\usepackage{lmodern}
\usepackage{kpfonts}

\usepackage{multicol}

\usepackage{babel}

\usetheme{Madrid}

\title{Programme de mathématiques de Terminale Scientifique}
\author{Vincent-Xavier \bsc{Jumel}}
\institute{LaSalle Saint-Denis}
\date{5 septembre 2017}

\begin{document}

\begin{frame}
  \maketitle
\end{frame}

\begin{frame}
  \frametitle{Généralités}
  \begin{itemize}
    \item mettre en œuvre une recherche de façon autonome ;
    \item mener des raisonnements ;
    \item avoir une attitude critique vis-à-vis des résultats obtenus ;
    \item communiquer à l'écrit et à l'oral.
  \end{itemize}
\end{frame}

\begin{frame}
  \frametitle{Les 6 compétences du lycée}
  \begin{itemize}[<+->]
    \item Chercher ;
    \item Modéliser ;
    \item Représenter ;
    \item Calculer ;
    \item Raisonner ;
    \item Communiquer.
  \end{itemize}
\end{frame}

\begin{frame}
  \frametitle{Organisation du programme}
  \begin{itemize}[<+->]
    \item 3 grands pôles : analyse, géométrie et probabilité ;
    \item analyse :
      \begin{itemize}
        \item notion de limite d'une suite, puis d'une fonction ;
        \item continuité des fonctions ;
        \item fonctions logarithmes et exponentielles ;
        \item intégration des fonctions ;
      \end{itemize}
    \item géométrie :
      \begin{itemize}
        \item dans l'espace ;
        \item en généralisant le produit scalaire ;
        \item nombres complexes ;
      \end{itemize}
    \item probabilités :
      \begin{itemize}
        \item loi normale ;
        \item échantillonnage et tests d'hypothèses.
      \end{itemize}
  \end{itemize}
\end{frame}

\begin{frame}
  \frametitle{Quelques compléments}
  \begin{itemize}[<+->]
    \item ne font pas l'objet d'un chapitre mais serons suivis
    \item utilisation des notations mathématiques
    \item utilisation de l'algorithmique
    \item un peu de \alert{logique}
  \end{itemize}
\end{frame}

\begin{frame}
  \frametitle{Organisation de l'année}
  \begin{itemize}[<+->]
    \item Progression parallèle : on aborde deux chapitres assez
      rapidement
    \item Documents distribués : classeur souple
    \item Recommandation : bloc feuilles simples petits carreaux
    \item Archivage des chapitres précédents à la maison
    \item Évaluation en classe et devoir à la maison
    \item Prestation orale (en français et en anglais)
  \end{itemize}
\end{frame}

\begin{frame}
  \frametitle{Progression indicative}
  \begin{multicols}{2}
  \begin{itemize}
    \item<1-> Raisonnement par récurrence et suites
    \item<2-> Limites de fonctions et continuité
    \item<3-> Dérivation
    \item<4-> Exponentielle
    \item<5-> Logarithme
    \item<6-> Intégration
    \item<7-> Fonctions trigonométriques
    \item[]<8-> ~
      \columnbreak
    \item<1-> Probabilités discrètes
    \item<2-> Géométrie dans l'espace (classique et vectorielle)
    \item[]<3-> ~
    \item<4-> Produit scalaire
    \item[]<5-> ~
    \item<6-> Nombres complexes
    \item<7-> Lois de probabilités
    \item<8-> Échantillonnage
  \end{itemize}
\end{multicols}
\end{frame}

\begin{frame}
  \begin{center}
    \Large Au travail \normalsize
  \end{center}
\end{frame}

\end{document}
