\documentclass[12pt,a4paper,french]{article}
\usepackage[utf8]{inputenc}
\usepackage[T1]{fontenc}
\usepackage{babel}
\usepackage[thmmarks]{ntheorem}
\usepackage{amsmath}
\usepackage{amsfonts}
\usepackage{amssymb}

\usepackage{array}

\usepackage{lmodern}
\usepackage{kpfonts}
\usepackage{cmbright}

\usepackage[bookmarks=false,colorlinks,linkcolor=blue,pdfusetitle]{hyperref}

\pdfminorversion 7
\pdfobjcompresslevel 3

\usepackage{tabularx}
\usepackage[autolanguage,np]{numprint}
\usepackage{enumitem}

\usepackage{tipfr}
\usepackage{pgf}
\usepackage{tikz}
\usepackage{tkz-euclide}
\usetkzobj{all}
%\usetikzlibrary{hobby}
\usepackage{tkz-tab}

\usepackage[top=1.7cm,bottom=2cm,left=2cm,right=2cm]{geometry}

\usepackage{lastpage}

\usepackage{esvect}
\usepackage{marginnote}

\usepackage{wrapfig}

\usepackage[defaultlines=5,all]{nowidow}


\makeatletter
\renewcommand{\@evenfoot}%
        {\hfil \upshape \small page {\thepage} de \pageref{LastPage}}
\renewcommand{\@oddfoot}{\@evenfoot}

\renewcommand{\maketitle}%
{\framebox{%
    \begin{minipage}{1.0\linewidth}%
      \begin{center}%
        \Large \@title ~-- \@author \\%
        \@date%
      \end{center}%
    \end{minipage}}%
  \normalsize%
  %\vspace{1cm}%
}

\pgfdeclarepatternformonly{mes_hachures}
{\pgfpoint{-0.1cm}{-0.1cm}}
{\pgfpoint{0.9cm}{0.5cm}}
{\pgfpoint{0.8cm}{0.4cm}}
{\pgfpathmoveto{\pgfpointorigin}
  \pgfpathlineto{\pgfpoint{0.8cm}{0.4cm}}
\pgfusepath{stroke}}

%Des macros pour les noms d'ensmbles
\newcommand{\R}{\mathbf{R}}
\newcommand{\Q}{\mathbf{Q}}
\newcommand{\Z}{\mathbf{Z}}
\newcommand{\C}{\mathbf{C}}
\newcommand{\N}{\mathbf{N}}

\newcommand{\norme}[1]{\left\lVert #1 \right\rVert}
\newcommand{\abs}[1]{\left\lvert #1 \right\rvert}

%Une macro récursive pour l'intérieru des vecteurs
%http://tex.stackexchange.com/questions/19693/arguments-of-custom-commands-as-comma-separated-list

\newcommand\vecteur[2][\\]{%
    \global\def\my@delim{#1}%
    \left(\negthinspace\begin{matrix}
        \my@vector #2,\relax\noexpand\@eolst%
    \end{matrix}\right)}

%Une macro pour les vecteurs
\def\my@vector #1,#2\@eolst{%
   \ifx\relax#2\relax
      #1
   \else
      #1\my@delim
      \my@vector #2\@eolst%
   \fi}

%Une macro récursive pour mettre formater l'intérieur des intervalles
\def\my@intervalle #1;#2\@eolst{%
  \ifx\relax#2\relax
    #1
  \else
    \my@intervalle #2\@eolst%
  \fi}

%Quatre macros pour les quatres types d'intervalles
\newcommand{\interff}[1]{%
  \left[\my@intervalle #1;\relax\noexpand\@eolst%
  \right]
}
\newcommand{\interfo}[1]{%
  \left[\my@intervalle #1;\relax\noexpand\@eolst%
  \right[}
\newcommand{\interof}[1]{%
  \left]\my@intervalle #1;\relax\noexpand\@eolst%
  \right]}
\newcommand{\interoo}[1]{%
  \left]\my@intervalle #1;\relax\noexpand\@eolst%
  \right[}

\makeatother


\usepackage{mdframed}

\theoremstyle{break}
\newtheorem{definition}{Définition}
\newtheorem{propriete}{Propriété}
\newtheorem{corollaire}{Corollaire}
\newtheorem{propdef}{Propriété - Définition}
\newtheorem{theoreme}{Théorème}
\theoremstyle{plain}
\theorembodyfont{\normalfont}
\newtheorem{exerciceT}{Exercice}
\theoremstyle{nonumberplain}
\newtheorem{remarque}{Remarque}
\newtheorem{notation}{Notation}
\newtheorem{probleme}{Problème}
\theoremsymbol{\ensuremath{\blacksquare}}
\newtheorem{preuve}{Preuve}
\theoremsymbol{}
\theoremstyle{nonumberbreak}
\newtheorem{exemple}{Exemple}

\newenvironment{exercice}{\begin{framed}\begin{exerciceT}}{\end{exerciceT}\end{framed}}

\setlength{\parsep}{0pt}
\setlength{\parskip}{5pt}
\setlength{\parindent}{0pt}
\setlength{\itemsep}{7pt}

\setlist{noitemsep}
%\setlist[1]{\labelindent=\parindent} % < Usually a good idea
\setlist[itemize]{leftmargin=*}
\setlist[itemize,1]{label=$\triangleright$}
\setlist[enumerate]{labelsep=*, leftmargin=1.5pc}
\setlist[enumerate,1]{label=\arabic*., ref=\arabic*}
\setlist[enumerate,2]{label=\emph{\alph*}),
ref=\theenumi.\emph{\alph*}}
\setlist[enumerate,3]{label=\roman*), ref=\theenumii.\roman*}
\setlist[description]{font=\sffamily\bfseries}

\usepackage{multicol}
\setlength{\columnseprule}{0pt}

\usepackage[]{exsheets}
\SetupExSheets{
  headings = block,
  question/pre-hook = \mdframed,
  question/post-hook = \endmdframed,
}

\everymath{\displaystyle\everymath{}}

\title{Devoir à la maison \no{2} : Dénombrements et autres formules}
\author{\bsc{Ts 2}}
\date{pour le 13 octobre 2017}

\begin{document}

\maketitle

\section*{Introduction}

Le but de ce devoir est de présenter l'art de compter (dénombrer) dans
le but de résoudre des problèmes de probabilités dans un univers fini. À
cet effet, il est bon de rappeller un peu de vocabulaire.

\begin{definition}
  On appelle \emph{couleur d'une carte} une des quatre «valeurs» :
  \begin{itemize}[label=--]
    \item cœur
    \item carreau
    \item pique
    \item trèfle
  \end{itemize}

  On appelle \emph{figure} une des trois «valeurs» :
  \begin{itemize}[label=--]
    \item roi
    \item reine
    \item valet
  \end{itemize}

  Un \emph{jeu de 32 cartes} est constitué, pour chacune des quatre
  couleurs, des 3 figures, des cartes de 10 à 7 et de l'as.
\end{definition}

\section{Quelques techniques de dénombrement}

\begin{definition}
  On appelle \emph{factorielle} d'un nombre $n$ l'entier, noté $n!$
  défini par \[ \left\lbrace\begin{matrix}1 & \text{si } n = 1 \\
  n\times (n - 1)! & \text{si } n > 1\end{matrix}\right. \]

\end{definition}

\begin{question}[ID=factorielle]
  Combien dispose-t-on de façon d'ordonner 6 objets ?
\end{question}
\begin{solution}
  On dispose d'exactement 6! = 720 façon de trier ces objets. En effet,
  pour le premier, il y'a 6 choix possibles, puis 5 choix pour le second
  etc jusqu'à 2 puis 1.
\end{solution}

\begin{definition}
  On appelle \emph{arrangement} de $p$ parmi $n$ le nombre de choix
  ordonné de $p$ éléments dans un ensemble à $n$ éléments.
\end{definition}

\begin{question}[ID=arrangement]
  Combien de façons existe-il d'arranger 3 éléments parmi 4. Attention
  l'ordre compte.

  \emph{Faire le décompte avec des feuilles de papier numérotées.}

  Proposer une expression du nombre d'arrangement en fonction de $n$ et
  $p$.
\end{question}
\begin{solution}
  Il existe 12 façons de constituer des listes (ordonnées) de 3 éléments
  parmi 4.

  On peut définir le nombre d'arrangement \[ A_n^p = \frac{n!}{(n-p)!}
  \]
\end{solution}

\begin{definition}
  On rappelle que le nombre $\binom{n}{p} = \frac{n!}{(n-p)! p!}$
  correspond au \emph{nombre de combinaisons} de $p$ parmi $n$.
\end{definition}

\begin{question}[ID=combinaison_arrangement]
  Expliquer le lien entre le nombre d'arrangements et le nombre de
  combinaisons
\end{question}
\begin{solution}
  Dans les combinaisons, l'ordre des $p$ éléments ne «compte pas». Il
  faut donc diviser le nombre d'arrangements par le nombre de façon
  d'ordonner les éléments de la $p$-liste.
\end{solution}

\section{Autour du nombre de combinaison}

On peut démontrer une première propriété de symétrie du nombre de
combinaisons.

\begin{question}[ID=symétrie]
  Démontrer que pour tout $n$ entier naturel et pour tout $p$ entier
  naturel inférieur ou égal à $n$, on a \[ \binom{n}{p} = \binom{n}{n -
  p} \]
\end{question}
\begin{solution}
  Il suffit de remplacer $p$ par $n -p$.
\end{solution}

On peut également démontrer une autre propriété qui permet de calculer
assez aisément les nombres de combinaisons.

\begin{question}[ID=pascal]
  Démontrer que pour tout $n$ entier naturel et pour tout $p$ entier
  naturel inférieur ou égal à $n$, on a \[ \binom{n}{p} + \binom{n}{p+1}
  = \binom{n+1}{p+1}. \] Quel est le nom de cette construction ?

  Proposer une mise en œuvre visuelle de cette propriété en construisant
  les coefficients binomiaux pour $n$ variant de 1 à 5 et $p$ variant de
  1 à $n$.
\end{question}
\begin{solution}
  On écrit la somme des deux coefficients binomiaux en utilisant la
  définition par les factorielles, puis on mets astucieusement au même
  dénominateur.

  La construction s'appelle le triangle de Pascal.

  On a :\\
  1 \\
  1 \hspace{1em} 1 \\
  1 \hspace{1em} 2 \hspace{1em} 1 \\
  1 \hspace{1em} 3 \hspace{1em} 3 \hspace{1em} 1 \\
  1 \hspace{1em} 4 \hspace{1em} 6 \hspace{1em} 4 \hspace{1em} 1 \\
  1 \hspace{1em} 5 \hspace{1em} 10 \hspace{0.5em} 10 \hspace{0.5em} 5
  \hspace{1em} 1
\end{solution}

On peut aussi vérifier que $\binom{n}{n} = \binom{n}{0} = 1$.

\begin{question}[ID=1choix]
  Donner une interprétation en terme de choix des deux résultats
  précédents.
\end{question}
\begin{solution}
  Il n'y a qu'une seule façon de choisir tous les éléments ou aucun.
\end{solution}

\section{Ils sont partout}

Pour finir ce devoir, un dernier exercice qui va faire apparaître les
coefficients binomiaux là où on ne les attends pas.

\begin{question}[ID=Newton]
  \begin{enumerate}
    \item Démontrer par récurrence que pour tout $n$ entier \[ (a + b)^n
      = \sum_{k = 0}^n \binom{n}{k} a^k b^{n - k} \]
    \item Que vaut cette expression pour $a = b = 1$.
    \item En déduire la somme des coefficients binomiaux pour $n$ donné.
  \end{enumerate}
\end{question}
\begin{solution}
  \begin{itemize}
    \item Pour $n =0$, on a l'égalité $0=0$ pour tout $a \neq -b$. Avec
      la convention $0^0$ on conclut dans le cas $a = -b$. La
      proposition $P_n$ : «$(a + b)^n = \sum_{k = 0}^n \binom{n}{k} a^k
      b^{n - k}$» est donc vraie pour $n =0$.
    \item Soit $n$ un entier naturel. Montrons que $P_n \implies
      P_{n+1}$

      $(a + b)^{n+1} = (a+b)(a + b)^n = (a+b)\sum_{k = 0}^n \binom{n}{k}
      a^k b^{n - k} = \sum_{k = 0}^n \binom{n}{k}a^{k+1}b^{n - k} +
      \sum_{k = 0}^n \binom{n}{k}a^k b^{n+1 - k}$

      En extrayant les premiers et derniers termes de la somme et
      en utilisant l'exercice 5 on trouve le résultat attendu.
    \item La proposition $P_n$ est vraie pour tout entier $n$.
  \end{itemize}
  On parle de formule du binôme de Newton.
\end{solution}
\end{document}

\pagebreak

\maketitle
\vspace{-0.5cm}

\section*{Solutions des exercices}

\vspace{-7mm}

\printsolutions

\end{document}
