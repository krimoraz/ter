\documentclass[12pt,a4paper,french]{article}
\usepackage{etex}
\usepackage[utf8]{inputenc}
\usepackage[T1]{fontenc}
\usepackage{babel}
\usepackage[thmmarks]{ntheorem}
\usepackage{amsmath}
\usepackage{amsfonts}
\usepackage{amssymb}
\usepackage{array}
\usepackage{lmodern}
\usepackage{kpfonts}
\usepackage[bookmarks=false,colorlinks,linkcolor=blue,pdfusetitle]{hyperref}
\usepackage{tabularx}
\usepackage[autolanguage,np]{numprint}
\usepackage{enumitem}
\usepackage{lastpage}
\usepackage{pgf}
\usepackage{tikz}
\usepackage{tkz-euclide}
\usetkzobj{all}
\usetikzlibrary{arrows}
\usetikzlibrary{shadows}
\usetikzlibrary{shapes}
\usetikzlibrary{intersections}
\usetikzlibrary{plotmarks}
\usetikzlibrary{positioning}
\usetikzlibrary{datavisualization}
\usetikzlibrary{decorations.pathmorphing}
\usetikzlibrary{patterns}
\usetikzlibrary{calc}
\usepackage{tkz-tab}
\usepackage{tkz-fct}
\usepackage{tkz-base}
\usepackage{bclogo}
\usepackage[top=0.5cm,bottom=2cm,left=1.5cm,right=1.3cm]{geometry}

\usepackage{lastpage}
\usepackage{marginnote}
%\usepackage{pas-tableur}
%\usepackage{picins}
\usepackage{xparse}
\usepackage[extdef]{delimset}

\usepackage[defaultlines=5,all]{nowidow}

\newcommand{\R}{\mathbb{R}} 
\newcommand{\N}{\mathbb{N}} 
\renewcommand{\D}{\mathbb{D}} 
\newcommand{\Z}{\mathbb{Z}} 
\newcommand{\Q}{\mathbb{Q}} 
\newcommand{\C}{\mathbb{C}} 
\newcommand{\vect}[1]{\mathchoice% 
{\overrightarrow{\displaystyle\mathstrut#1\,\,}}% 
{\overrightarrow{\textstyle\mathstrut#1\,\,}}% 
{\overrightarrow{\scriptstyle\mathstrut#1\,\,}}% 
{\overrightarrow{\scriptscriptstyle\mathstrut#1\,\,}}} 
\def\Oij{$\left(\text{O},~\vect{\imath},~\vect{\jmath}\right)$} 
\def\Oijk{$\left(\text{O},~\vect{\imath},~ \vect{\jmath},~ \vect{k}\right)$} 
\def\Ouv{$\left(\text{O},~\vect{u},~\vect{v}\right)$}

\usepackage{mdframed}
\setlength{\parsep}{0pt}
\setlength{\parskip}{5pt}
\setlength{\parindent}{0pt}
\setlength{\itemsep}{7pt}
\setlist{noitemsep}
\usepackage{multicol}
\setlength{\columnseprule}{0pt}
\usepackage[]{exsheets}
\SetupExSheets{
  counter-format = {qu[1] :} ,
  headings = block-subtitle ,
  points/name = {pt/s} ,
  solution/print = false ,%true,%true
  %solution/pre-hook = {\mdframed} ,
  %solution/post-hook = {\endmdframed} ,
  blank/style = dotted ,
}

\DeclareExSheetsHeadingContainer{autopoints}{%
  \textbf{(\GetQuestionProperty{points}{\CurrentQuestionID} points)}%
}

 \DeclareInstance{exsheets-heading}{block-subtitle}{default}{
 join = {
 title[r,B]number[l,B](.333em,0pt) ;
 title[r,B]subtitle[l,B](1em,0pt)
 } ,
 attach = {
 main[l,vc]title[l,vc](0pt,0pt) ;
 main[r,vc]autopoints[l,vc](\marginparsep-55pt,0pt)
 },
}


\everymath{\displaystyle\everymath{}}

\newcommand{\brm}[1]{\marginpar{\addpoints*{#1}}}% étoile n'affiche pas les points sur la copie
\newcommand{\cel}{\celtxt[c]}
\title{BBlanc \no{1}}
\renewcommand{\theenumi}{\textbf{\arabic{enumi}}} 
\renewcommand{\labelenumi}{\textbf{\theenumi.}} 
\renewcommand{\theenumii}{\textbf{\alph{enumii}}} 
\renewcommand{\labelenumii}{\textbf{\theenumii.}}
\usetikzlibrary{backgrounds}
\frenchbsetup{StandardLists=true}
\def\coeff{2}

\newcommand{\itemTriangle}{%
   \scalebox{0.8}{$\blacktriangleright$}%
}
\usepackage{ifthen}
\usepackage{fancyhdr}
\pagestyle{fancy}


\count1=\year \count2=\year
\ifnum\month<8\advance\count1by-1\else\advance\count2by1\fi

\setlength{\headheight}{14.5pt}
\cfoot{\textsl{\footnotesize{Année \number\count1/\number\count2}}}
\lfoot{\textsl{\footnotesize{Lycée La salle \textsc{Saint-Denis}}}}
%\rfoot{%
 % \ifthenelse{\value{page}=1}
 % {%
 %  \footnotesize{Page \thepage/ \pageref{LastPage}}
% }
%}
%\rhead{\textbf{16LYJBSBB1}}
\renewcommand{\headrulewidth}{1pt}
\renewcommand{\footrulewidth}{0pt}
\fancyfoot[r]{\textsl{\footnotesize{Page \thepage{} sur \pageref{LastPage}}}}

\begin{document}



\renewcommand\labelitemi{\textbullet}
\thispagestyle{empty} 
\begin{flushright}
\textbf{16LYJBSBB1}
\end{flushright}
\begin{center}
 \begin{tikzpicture}
 \draw (0,8) node[color=black,draw,fill=black!20]%
 {\fontsize{30}{30}\selectfont{\upshape BACCALAUR\'EAT G\'EN\'ERAL }};
  \node[name    = s,%
        shape   = rectangle,%
        rotate  = 35,%
        color   = black!50]%
  {\fontsize{100}{150}\selectfont{\upshape  \textbf{\textit{obligatoire}}}};
\draw (0,7)    node[color=black] {\LARGE \textbf{Bac blanc n°1  Lycée La Salle -Saint Denis} };%   
\draw (0,6)    node[color=black] {\Large \textbf{Session 2016} };%
\draw (0,4)    node[color=black] {\Huge \textbf{MATH\'EMATIQUES} };%
\draw (0,2)    node[color=black] {\huge \textbf{- Série S -}};
\draw (0,0)    node[color=black] {\LARGE \textbf{\'EPREUVE DU  MERCREDI 14 D\'ECEMBRE 2016 }};
\draw (0,-2)    node[color=black] {\Large \textbf{ENSEIGNEMENT SP\'ECIFIQUE}};
\draw (0,-3)    node[color=black] {\Large \textit{ \textbf{Durée de l'épreuve : 4 heures}}};
\draw (6.3,-2.1)    node[color=black] {\large \textit{ \textbf{Coefficient : 7}}};
\draw (0,-7)  node[text centered]%
                  {\textbf%
                   {Les calculatrices éléctroniques de poche sont autorisées,}
                  };
\draw (0,-7.4)  node[text centered]%
                  {\textbf%
                   {conformément à la réglementation en vigueur.}
                  };

 \draw (0,-10) node[inner sep=5pt,draw,text width = 15cm, text justified] {
\textbf{Le sujet est composé de 4 exercices indépendants. Le candidat doit traiter tous les exercices.}

\textbf{Dans chaque exercice, le candidat peut admettre un résultat précédemment donné dans le texte pour aborder les questions suivantes, à condition de l'indiquer clairement sur la copie. La qualité et la précision de la rédaction seront prises en compte dans l'appréciation des copies.}
};
                

\draw (0,-13.6)  node[text centered]%
                 {\textbf%
                   {Avant de composer, le candidat s'assurera que le sujet comporte bien \pageref{LastPage} pages numérotées de 1 à \pageref{LastPage}}
                 };
\end{tikzpicture} 
\end{center} 
\begin{flushright}
\begin{tabular}{@{}p{10cm}@{}}
 {\raggedleft \itshape 
\og Les gens de mérite sont des gens de mérite tant qu’ils ne voient pas leurs mérites. \fg \par}\\
 
\end{tabular}
\end{flushright}

\vfill
\newpage
\pagenumbering{arabic}
\begin{question}
\vspace{-5.8mm}
\begin{center}\textbf{Commun à tous les candidats}\end{center}
\textbf{Dans cet exercice,  on cherche à représenter dans un repère orthonormé l’ensemble des points  $M$ du plan , de coordonnées $( x;y)$ , vérifiant l’équation :}
\[(E):\dfrac{x^2}{4}+y^2=1 \]
\begin{enumerate}[itemsep=8pt]
\item Justifier que , pour tout $x\in[~-2~;~2~]$ , l’équation $(E)$ est équivalente  à : \[ y=\sqrt[]{1-\dfrac{x^2}{4}}\quad  \mbox{ou}\quad y =-\hspace{2mm}\sqrt[]{1-\dfrac{x^2}{4}}\]  \brm{0.5}
\item Soit $f$  la fonction définie sur l’intervalle $[~-2~;~2~]$ par : 
$f(x)=\sqrt[]{1-\dfrac{x^2}{4}}$ 
\begin{enumerate}[itemsep=5pt]
\item Montrer que  $ f$ en n'est pas dérivable en  -2 .  Quelle est l’interprétation  graphique de ce résultat ? \brm{1}

On admettra pour la suite que $f$ n'est pas dérivable en 2.
\item Pour tout $x\in]~-2~;~2~[$, calculer $f'(x)$ et étudier son signe. \brm{1}

\item Dresser le tableau de variations de $f$. \brm{0.5}
\end{enumerate}
\item Sur la feuille \textbf{annexe n°1} tracer $\mathscr{C}_{f}$ la courbe représentative de la fonction $f$. \brm{1}
\item Comment obtient-on , à partir de $\mathscr{C}_{f} $ la courbe représentative $ \mathscr{C}_{g}$  de la fonction \brm{1}

$g:x \mapsto -\hspace{2mm} \sqrt[]{1-\dfrac{x^2}{4}}$ ? Tracer sur le même graphique cette courbe.

\textbf{La réunion des deux courbes est une ellipse}
\item \textbf{Question bonus:}

Soit $F(~\sqrt[]{3}~  ;~0 )\quad \mbox{et}\quad F' (-~\sqrt{3}~;~0)$ ; Montrer que , pour tout point de cette ellipse :
$MF+MF'=4$

\end{enumerate}

\end{question}
\begin{solution}
	\begin{enumerate}
    	\item L'équation $(E)$ est équivalente à $y^2 = 1 - \frac{x^2}4
        \iff y = \pm\sqrt{1 - \frac{x^2}4}$
        \item 
        \begin{enumerate}
        \item $f = \sqrt{u}$ avec $u \colon x \mapsto 1 - \frac{x^2}4$.
        $f$ est dérivable si $u$ est dérivable et $u(x) > 0$. Or $u(-2) = 
        0$ donc $f$ n'est pas dérivable en 0. On peut cependant calculer le taux
        d'accroissement $\tau_x$ et sa  limite. Par définition, $\tau_x(h) =
        \frac{f(x+h) - f(x)}{h}$. Dans notre cas, on obtient : \[\tau_{-2}(h) = 
        \frac{\sqrt{\brk*{1-\frac{h}{\sqrt{2}}}\brk*{1+\frac{h}{\sqrt{2}}}}}{h} \]
        Sous cette forme là, on peut déterminer la limite et ici, on obtient $\lim_{h\to 
        0}\tau_{-2}(h) = +\infty$. Cela signifie que la tangente en $-2$ est verticale.
        
        \item Pour tout $x$ de ${]{-2},2[}$, $f'(x) = \frac{-x}{4\sqrt{1 -
        \frac{x^2}4}}$.
        \item On a le tableau suivant :
        \begin{center}
        	\begin{tikzpicture}
            	\tkzTabInit{$x$/1,$f'(x)$/1,$f$/2}{${-2}$,0,2}
                \tkzTabLine{d,+,0,-,d}
                \tkzTabVar{-/0,+/$1$,-/0}       	
        	\end{tikzpicture}	
        \end{center}
        \end{enumerate}
        \item Voir graphique ci-dessous
        
        \begin{center}
\begin{tikzpicture}[xscale=2,yscale=2]
\tkzInit[xmin=-2,xmax=2,ymin=-1.,ymax=1.]
\tkzGrid 
\tkzAxeXY 
\draw [very thick,red,<->] (-0.5,1) -- (0.5,1) ;
\draw [very thick,red,->] (2,0) -- (2,0.5) ;
\draw [very thick,red,->] (-2,0) -- (-2,0.5) ;
\draw [very thick,blue] plot [domain=-2:2,smooth,samples=250] (\x,{sqrt(1 - (\x/2)^2)} ) ;
\draw [very thick,green] plot [domain=-2:2,smooth,samples=250] (\x,{-sqrt(1 - (\x/2)^2)} ) ;
\end{tikzpicture}
\end{center}
\item La courbe de $\mathscr{C}_g$ se déduit par symétrie axiale d'axe l'axe des abscisses.
\item Soit $\mathcal{E}$ l'ensemble des points $M$ tel que $MF+MF'=4$.

Montrons $\mathcal{E}$ est bien notre ellipse.

\[ \begin{array}{ccl} MF + MF' = 4 & \iff & MF^2 + 2MF \times MF' + MF'^2 = 16 \\
 & \iff & \brk*{x - \sqrt{3}}^2 + y^2 + \brk*{x + \sqrt{3}}^2 + y^2 + 2\sqrt{\brk*{ 
 \brk*{x - \sqrt{3}}^2 + y^2} \brk*{\brk*{x + \sqrt{3}}^2 + y^2}} = 16 \\
 & \iff & \sqrt{\brk*{x^2 - 2\sqrt{3}x +3 + y^2} \brk*{x^2 + 2\sqrt{3}x +3 + y^2}} =
 5 - x^2 - y^2 \\
 & \iff & \sqrt{\brk*{x^2 + y^2 + 3}^2 - 12x^2 } = 5 - x^2 - y^2
\end{array} \]
Une condition nécessaire pour trouver une solution est que $\brk*{5 - \brk*{x^2 + y^2}
}^2 = \brk*{x^2 + y^2 + 3} - 12x^2$. En développant astucieusement, on obtient $25 - 10
\brk*{x^2 + y^2} = 6\brk*{x^2 + y^2} + 9 -12x^2$ qui peut se mettre sous la forme $4x^2 
+ 16y^2 = 16$, où encore $\frac{x^2}{4} + y^2 = 1$. La condition est suffisante si on  
ajoute que $5 - x^2 - y^2 \geqslant 0 \iff x^2 + y^2 \leqslant 5$ 
    \end{enumerate}
\end{solution}

\vspace{4cm}
\begin{question}
\vspace{-5.8mm}
\begin{center}\textbf{Commun à tous les candidats}\end{center}

On présente ici un modèle probabiliste d'évolution de population proche du
modèle de \bsc{Galton}-\bsc{Watson}. On suppose ici qu'un individu donné a,
au cours de sa vie :
\begin{itemize}
\item aucun enfant avec une probabilité $\frac18$ ;
\item un enfant avec une probabilité $\frac38$ ;
\item deux enfants avec une probabilité $\frac38$ ;
\item trois enfants avec une probabilité $\frac18$.
\end{itemize}
On suppose que les nombres d'enfants d'individus différents est indépendant d'un individu à l'autre.

{\bfseries On admet qu'aucun individu n'a plus de trois enfants.}


\textbf{Première partie : calculs de probabilités}

Partons d'un seul individu (génération 0). Nous étudions la population de ses
descendants.

{\bfseries Soit $q_{n}$ la probabilité que la population s'éteigne au plus à la n-ième génération}


\`A partir des informations données plus haut répondre aux questions suivantes:
\begin{enumerate}[itemsep=5pt]
\item Quelle est la probabilité $q_{1}$ que la population s'éteigne dès la première
génération ?% est $q_1 = \frac18$ 
\brm{0.25}

\item On souhaite calculer $q_{2}$ la probabilité que la population s'éteigne à la deuxième génération.
    \begin{enumerate}[itemsep=5pt]
      \item Déterminer la probabilité que la population s'éteigne si cet individu n'a qu'un seul enfant. \brm{0.25}
      \item Déterminer la probabilité que la population s'éteigne si cet individu a deux enfants. \brm{0.25}
      \item Déterminer la probabilité que la population s'éteigne si cet individu a trois enfants. \brm{0.25}
      \item Montrer que $q_{2}= \dfrac{729}{4096}$ \brm{0.5}
    \end{enumerate}
  \item En déduire que $q_{2}= \dfrac{1}{8}+\dfrac{3}{8}q_{1}+\dfrac{3}{8}q_{1}^{2}+\dfrac{1}{8} q_{1}^{3}$ \brm{0.5}
  
  
  
  \item Exprimer alors $q_{3}$, la probabilité que la population s'éteigne au plus à la 3-ième génération, en fonction de $q_{2}$. \brm{0.25}
  \end{enumerate}
\textbf{Deuxième partie : étude de la suite $(q_n)$}

\begin{enumerate}
  \item Soit $f$ la fonction définie sur l'intervalle $\intv{0}{1}$
    \[ f(x) = \frac18x^3 + \frac38x^2 + \frac38x + \frac18 \]
    \begin{enumerate}
      \item Étudier les variations de $f$ sur l'intervalle $\intv{0}{1}$ \brm{0.5}
      \item Soit $g$ la fonction définie pour tout $x$ de $\intv{0}{1}$
        par $g(x) = f(x) - x$ \brm{0.5}

        En remarquant que $g(1) = 0$, trouver trois nombres réels $a,b$
        et $c$ tels que pour tout $x$ de $\intv{0}{1}$, \[ g(x) =
        (x-1)(ax^2 + bx + c) .\] En déduire le signe de $g(x)$ sur
        l'intervalle $\intv{0}{1}$.
    \end{enumerate}
  \item On admet que la suite $(q_n)$ est définie par la relation de
    récurrence \[ \left\lbrace\begin{matrix}q_{n+1} = f(q_n) \\ q_1 =
    \frac18\end{matrix}\right. \]

    \begin{enumerate}
      \item À l'aide de la représentation graphique de la fonction $f$ en \textbf{annexe n°2} ainsi qu'à l'aide de la droite d'équation $y=x$, représenter sur l'axe des abscisses les quatre premiers
        termes de la suite $(q_n)$. \brm{0.5}
      \item Quelles conjectures peut-on alors faire sur le comportement de la suite $(q_{n})$ ? \brm{0.25}

      \item Montrer par récurrence que pour tout entier $n$ non nul , \[ 0 \leqslant q_n \leqslant \sqrt{5} - 2 \] \brm{0.5}

      \item   Montrer que la suite $(q_n)$ est croissante. \brm{0.5}

      \item Justifier que la suite $(q_n)$ est convergente et déterminer \brm{0.5}
        sa limite.
    \end{enumerate}
\end{enumerate}
\end{question}

\begin{solution}
	\textbf{Partie A}
    
    \begin{enumerate}
    \item La probabilité $q_1$ est $q_1 = \frac18$.
    \item \begin{enumerate}
    \item Si le parent n'a eu qu'un enfant, il a eu avec une
    probabilité de $\frac38$. La probabilité que cet enfant n'ait
    pas de descendance est $\frac18$. Les événements étant
    indépendant, la probabilité est donc de $\frac38 \times \frac18
    = \frac3{64}$.
    \item Si le parent a eu deux enfants, c'est cette fois avec une
    probabilité de $\frac38$. Pour chacun des deux descendants, la
    probabilité que la population s'éteigne ensuite est de
    $\frac18$ par enfant, et en utilisant à nouveau l'indépendance,
    on a $\frac38 \times \brk*{\frac18}^2 = \frac3{512}$.
    \item Par un raisonnement similaire, on trouve $\frac18 \times 
    \brk*{\frac18}^3 = \frac1{4096}$.
    \item Les 4 événements précédents étant incompatibles, on peut
    donc les additionner pour obtenir la probabilité que la
    population ne dépasse pas la deuxième génération.
    \end{enumerate}
    \item D'après ce qui précède, on a $q_2 = \frac18 + \frac38
    \times \frac18 + \frac38 \brk*{\frac18}^2 + \frac18 
    \brk*{\frac18}^3$. \`A l'exception du terme constant 
    correspondant à l'extinction à la première génération, on 
    reconnaît $q_1$ et on en déduit $q_2 = \frac18 + \frac38
    \times q_1 + \frac38 q_1^2 + \frac18 q_1^3$.
    \item La probabilité de s'éteindre à la 3\up{ième} génération
    est fonction la probabilité de s'éteindre à la 2\up{ième}
    génération. Il est raisonnable de penser que $q_3 = \frac18 +
    \frac38\times q_2 + \frac38 q_2^2 + \frac18 q_3^3$.
    \end{enumerate}
    \textbf{Partie B}
    
    \begin{enumerate}
    \item \begin{enumerate}
    \item $f$ est une fonction polynôme de degré 3, somme monômes 
    croissants sur $\intv{0}{1}$ donc elle est croissante sur 
    $\intv{0}{1}$.
    \item $f(1) = 1$, donc $g(1) = 0$ et $g(x)$ est donc 
    factorisable par $(x-1)$.
    
    $(x-1)(ax^2 +bx + c) = ax^3 +bx^2 + cx - ax^2 -bx - c = ax^3 +
    (b-a)x^2 + (c-b)x - c$. Au facteur $\frac18$ près, identifions 
    les coefficients : $a = 1$, $-c = 1$ et $(b-a) = 3$. On trouve 
    donc $a = \frac18$, $b=\frac12$ et $c=-\frac18$.
    
    $g(x) = \frac18(x-1)(x^2 +4x -1)$. $x^2 +4x -1$ possède un 
    discriminant positif égal à 20, il se factorise en $(x + 2 + 
    \sqrt{5})(x + 2 - \sqrt{5})$.
    
    \begin{center}
    \begin{tikzpicture}
    	\tkzTabInit{$x$/1,$x-1$/1,$x+2 - \sqrt{5}$/1,$g(x)$/1}
        {0,$\sqrt{5}-2$,1}
        \tkzTabLine{,-,,-,0}
        \tkzTabLine{,-,0,+,}
        \tkzTabLine{,+,0,-,0}
    \end{tikzpicture}
    \end{center}
    On en déduit les solutions de l'équation $f(x) = $ sur 	
    $\intv{0}{1}$ et que la courbe de $f$ est en dessous de la
    droite $y =x$ sur $\intv*{\sqrt{5}-2}{1}$.
    
    \end{enumerate}
    \item \begin{enumerate}
    \item On a le tracé suivant :
    \begin{center}
\begin{tikzpicture}[scale=50,>=latex]
	\draw [dotted,thin] (0,0) grid [step=0.05] (0.25,0.25) ;
	\draw [very thick] plot [smooth,domain=0:0.25] (\x,{1/8*\x^3 + 3/8*\x*\x + 3/8 *\x + 1/8}) ;
    \draw [very thick] (0,0) -- (0.25,0.25) ;
    \draw [thick,->] (-0.01,0) -- (0.25,0) ;
    \draw [thick,->] (0,-0.01) -- (0,0.25) ;
    \draw (0,0) node [anchor=north west] {O} ;
    
    \draw [dashed,thick] (0.125,0) -- (0.125,0.177978515625) -- (0.177978515625,0.177978515625)
    -- (0.177978515625,0.20432528913261194) -- (0.20432528913261194,0.20432528913261194)
    -- (0.20432528913261194,0.21834408490310356) -- (0.21834408490310356,0.21834408490310356) ;
    \draw (0.125,0) node [below] {$u_1$} ;
    \draw [dashed,thick] (0.177978515625,0.177978515625) -- (0.177978515625,0) node [below] {$u_2$} ;
    \draw [dashed,thick] (0.20432528913261194,0.20432528913261194) -- (0.20432528913261194,0) node [below] {$u_3$} ;
    \draw [dashed,thick] (0.21834408490310356,0.21834408490310356) -- (0.21834408490310356,0) node [below] {$u_4$} ;
\end{tikzpicture}
% Calculs effectués avec 
%def u(n):
%    if n == 1:
%        return 1/8
%    else:
%        return 1/8 + 3/8*u(n-1) + 3/8*u(n-1)**2 + 1/8*u(n-1)**3
\end{center}
\item La suite semble converger vers la première racine de $g$, c'est à dire $\sqrt{5} - 
2$.
\item Montrons par récurrence que «les termes de la suite $(q_n)$ appartiennent à 
l'intervalle $\intv*{0}{\sqrt{5}-2}$», ce qui constitue notre proposition.
\begin{itemize}
\item Initialisation : $q_1 = \frac{1}{8}$ appartient bien à $\intv*{0}{\sqrt{5}-2}$
\item Hérédité : Supposons la proposition vraie au rang $n$. Alors $q_n \in 
\intv*{0}{\sqrt{5}-2}$. Comme $f$ est strictement croissante, alors $q_{n+1} = f(q_n)$ appartient à $\intv*{f(0)}{f(\sqrt{5} -2)} \subset \intv*{0}{\sqrt{5}-2}$. En effet, $f(0) = \frac18$ et $f(\sqrt{5} -2) = \sqrt{5} -2$.
\item Conclusion : Pour tout $n$ entier naturel différent de 0, $ 0 \leqslant q_n 
\leqslant \sqrt{5} - 2$
\end{itemize}
\item Soit $n$ un entier naturel, $q_{n+1} - q_n = f(q_n) - q_n = g(q_n)$. Or $g(x) > 0$ 
pour $x \in \intv*{0}{\sqrt{5}-2}$, donc $q_{n+1} - q_n > 0$ et donc la suite est 
croissante.
\item $(q_n)$ est une suite croissante majorée, donc elle converge. De plus, si on note 
$\ell$ sa limite, on a, par définition $\ell = f(\ell)$ donc $\ell = \sqrt{5} - 2$.
    \end{enumerate}
    \end{enumerate}
\end{solution}

\newpage
\begin{question}
\vspace{-5.8mm}
\begin{center}\textbf{Commun à tous les candidats}\end{center}
\textbf{Partie A}

Soit $g$ la fonction définie sur $\mathbf{R}$ par $g(x)=-3x^{4}+3x^{3}+1$.
\begin{enumerate}[itemsep=5pt]
\item Déterminer les limites de $g$ en $-\infty\;\mbox{et en}\,+\infty$. \brm{0.5}
\item \'Etudier les variations de la fonction $g$. \brm{0.5}
\item \begin{enumerate}[itemsep=5pt]
\item Déterminer que l'équation  $g(x)=0$ a exactement deux solutions dans $\mathbf{R}$. \brm{0.75}
\item Donner un encadrement d'amplitude $\np{0,1}$ de chaque solution. \brm{0.5}
\end{enumerate}
\item Déterminer le signe de $g(x)$ selon les valeurs de $x$. \brm{0.5}
\end{enumerate}
\medskip
\textbf{Partie B}

Soit $f$ la fonction définie sur $\mathbf{R}$ par  $f(x)=2+\dfrac{4x-3}{x^{4}+1}$ et $\mathscr{C}$ sa courbe représentative.
\begin{center}
\definecolor{qqwuqq}{rgb}{0.,0.39215686274509803,0.}
\begin{tikzpicture}[line cap=round,line join=round,>=triangle 45,x=2cm,y=0.8cm]
\draw[->,color=black] (-2.96,0.) -- (5.,0.);
\foreach \x in {-2.,-1.,1.,2.,3.,4.}
\draw[shift={(\x,0)},color=black] (0pt,2pt) -- (0pt,-2pt) node[below] {\footnotesize $\x$};
\draw[->,color=black] (0.,-2.7) -- (0.,3.06);
\foreach \y in {-2.,-1.,1.,2.,3.}
\draw[shift={(0,\y)},color=black] (2pt,0pt) -- (-2pt,0pt) node[left] {\footnotesize $\y$};
\draw[color=black] (0pt,-10pt) node[right] {\footnotesize $0$};
\clip(-3,-3) rectangle (5.,3.06);
\draw[line width=1.2pt,color=qqwuqq,smooth,samples=100,domain=-2.96:5.0] plot(\x,{2.0+(4.0*(\x)-3.0)/((\x)^(4.0)+1.0)});
\draw [color=qqwuqq](1.56,3.12) node[anchor=north west] {$\mathscr{C}$};
\end{tikzpicture}
\end{center}
\begin{enumerate}[itemsep=5pt]
\item \begin{enumerate}[itemsep=5pt]
\item Démontrer que la courbe $\mathscr{C} $ admet une asymptote $\mathscr{D}$ parallèle à l'axe des abscisses. \brm{0.25}
\item \'Etudier la position de $\mathscr{C}$ par rapport à $\mathscr{D}$. \brm{0.25}
\end{enumerate}
\item \begin{enumerate}[itemsep=5pt]
\item Démontrer que $f'(x)=\dfrac{4g(x)}{(x^4+1)^2}$, $g$ étant la fonction définie dans la \textbf{partie A.} \brm{0.5}
\item \'Etudier les variations de la fonction $f$. \brm{0.25}
\end{enumerate}
\item Soit $h$ une fonction telle que pour tout réel $x$ de l'intervalle $\left[\dfrac{3}{4};+\infty \right[ $ ,
\[ 2 \leq h(x) \leq f(x) \]
Déterminer la limite de la fonction $h$ en $+\infty$. \brm{0.5}
\end{enumerate}
\end{question}
\begin{solution}
\textbf{Partie A}

\begin{enumerate}
\item Étudions la limite de $g$ en $+\infty$. Pour $x > 0$, on $g(x) = x^4\brk*{-3 + \frac{3}{x} + \frac{1}{x^4}}$ et donc $\lim_{x\to +\infty}g(x) = -\infty$.

Par symétrie, on trouve $\lim_{x\to -\infty}g(x) = -\infty$.
\item $g$ est une somme de fonction dérivables, donc elle est dérivable et sa dérivée est $-12x^3 + 9x^2 = 3x^2(-4x + 3)$. On a donc le tableau suivant :
\begin{center}
\begin{tikzpicture}
\tkzTabInit[espcl=5]
{$x$/1,$g'(x)$/1,$g$/3}
{$-\infty$,$\frac34$,$+\infty$}
\tkzTabLine{,+,z,-,}
\tkzTabVar{-/$-\infty$,+/$\frac{337}{256}$,-/$-\infty$}
\tkzTabVal{1}{2}{0.6}{$\alpha$}{0}
\tkzTabVal{2}{3}{0.4}{$\beta$}{0}
\end{tikzpicture}
\end{center}
\item \begin{enumerate}
\item Sur l'intervalle $\intv[l]*{-\infty}{\frac34}$, $g$ est strictement croissante, 
continue donc d'après le théorème des valeurs :
\begin{itemize}
\item son image est un intervalle $J$ ;
\item pour tout $k\in J$, l'équation $g(x) = k$ admet une unique solution.
\end{itemize}
Ici, $J = \intv[l]*{-\infty}{\frac{337}{256}} \ni 0$ donc il existe un unique réel 
$\alpha$ tel que $g(\alpha) = 0$.

De la même façon, on en déduit l'existence et l'unicité de $\beta$ tel que $g(\beta) = 
0$.
\item On trouve $\alpha \approx -0,6$ et $\beta \approx 1,2$.
\end{enumerate}
\item On a pour le signe de $g$ le tableau suivant :
\begin{center}
\begin{tikzpicture}
\tkzTabInit[espcl=3]{$x$/1,$g(x)$/1}{$-\infty$,$\alpha$,$\beta$,$+\infty$}
\tkzTabLine{,-,z,+,z,-,}
\end{tikzpicture}
\end{center}
\end{enumerate}
\textbf{Partie B}
\begin{enumerate}
\item \begin{enumerate}
\item On obtient assez facilement que $\lim_{x\to\pm\infty}f(x) = 2$, ce qui se traduit
graphiquement par la présence d'une asymptote horizontale d'équation $y = 2$.
\item Étudions le signe de $f(x) - y$. $x^4 + 1$ étant positif, il suffit donc d'étudier 
le signe de $4x -3$. On en déduit aisément que la courbe est sous son asymptote pour $x$ 
négatif et au dessus pour $x$ positif.
\end{enumerate}
\item \begin{enumerate}
\item $f$ est de la forme $\frac{u}{v}$ où $u$ et $v$ sont les fonctions définies par 
$u(x) = 4x -3$, de dérivée $u'(x) = 4$ et $v(x)= x^4 + 1$ de dérivée $4x^3$. Utilisons 
la relation $\frac{u'v - uv'}{v^2}$. On a donc $f'(x) = \frac{4(x^4 + 1) - 4x^3(4x 
-3)}{(x^4+1)^2} = \frac{4x^4 + 4 - 16x^4 +12x^3}{(x^4+1)^2} = \frac{4 - 4\times 3x^4 
+4\times 3x^3}{(x^4+1)^2}$
\item On peut reprendre le tableau précédent et le compléter :
\begin{center}
\begin{tikzpicture}
\tkzTabInit[espcl=3]{$x$/1,$g(x)$/3}{$-\infty$,$\alpha$,$\beta$,$+\infty$}
\tkzTabVar{+/2,-/$f(\alpha)$,+/$f(\beta)$,-/2}
\end{tikzpicture}
\end{center}
\end{enumerate}
\item Par encadrement, on a que $\lim_{x\to+\infty}h(x) = 2$.
\end{enumerate}
\end{solution}
\newpage
\begin{question}
\vspace{-5.8mm}
\begin{flushright}\hfill\textbf{Candidats n'ayant pas suivi l'enseignement de spécialité}\hfill ~\end{flushright}



On considère le cube $ABCDEFGH$ de côté $a$, avec $I$, $J$ les
    milieux respectifs des segments $[CD]$ et $[GH]$ et $L$ est le
    milieu du segment $[JG]$ \brm{5}
    \vspace{1cm}
    \begin{center}
    \definecolor{uuuuuu}{rgb}{0.26666666666666666,0.26666666666666666,0.26666666666666666}
\definecolor{ffqqtt}{rgb}{1.,0.,0.2}
\begin{tikzpicture}[line cap=round,line join=round,>=triangle 45,x=1.0cm,y=1.0cm]
\clip(0.42,1.5) rectangle (5.64,6.7);
\draw (1.,2.)-- (4.,2.);
\draw [dash pattern=on 3pt off 3pt,color=ffqqtt] (5.,3.)-- (2.,3.);
\draw [color=uuuuuu] (1.,5.)-- (4.,5.);
\draw (5.,6.)-- (2.,6.);
\draw (1.,2.)-- (1.,5.);
\draw (4.,2.)-- (4.,5.);
\draw (4.,5.)-- (5.,6.);
\draw (4.,2.)-- (5.,3.);
\draw (1.,5.)-- (2.,6.);
\draw [dash pattern=on 3pt off 3pt,color=ffqqtt] (2.,6.)-- (2.,3.);
\draw [dash pattern=on 3pt off 3pt,color=ffqqtt] (2.,3.)-- (1.,2.);
\draw (5.,6.)-- (5.,3.);
\draw (3.48,6.74) node[anchor=north west] {$\mathbf{J}$};
\draw (4.1,6.78) node[anchor=north west] {$\mathbf{L}$};
\draw (0.54,2.24) node[anchor=north west] {$\mathbf{A}$};
\draw (4.06,2.18) node[anchor=north west] {$\mathbf{B}$};
\draw (5.06,3.18) node[anchor=north west] {$\mathbf{C}$};
\draw (1.44,3.62) node[anchor=north west] {$\mathbf{D}$};
\draw (0.48,5.44) node[anchor=north west] {$\mathbf{E}$};
\draw (4.06,5.18) node[anchor=north west] {$\mathbf{F}$};
\draw (5.1,6.74) node[anchor=north west] {$\mathbf{G}$};
\draw (1.52,6.66) node[anchor=north west] {$\mathbf{H}$};
\draw (3.5,3.74) node[anchor=north west] {$\mathbf{I}$};
\begin{scriptsize}
\draw [fill=uuuuuu] (3.5,6.) circle (1.5pt);
\draw [fill=uuuuuu] (3.5,3.) circle (1.5pt);
\draw [fill=uuuuuu] (4.25,6.) circle (1.5pt);
\end{scriptsize}
\end{tikzpicture}
    \end{center}
\begin{enumerate}[itemsep=8pt]
\item 

 Démontrer que la droite $(BI)$ est orthogonale à $(IJ)$


\item Quelle est la nature de la section du plan $(BIL)$ avec le plan $(EFG)$ ?(Donner deux caractéristiques de cette section)

\item Démontrer que la section du cube $ABCDEFGH$ par le plan $(BIL)$ est un trapèze

\item On se place dans  le repère
    $(A;\overrightarrow{AB},\overrightarrow{AD},\overrightarrow{AE})$

\begin{enumerate}[itemsep=5pt]
\item Les points $L$, $I$, $B$ et $F$ sont-ils coplanaires ? 
\item On considère le point  $K (~a~;~a~;~\frac{a}{2}~)$

Les droites $(IL)$ et $(BK)$ sont-elles coplanaires ?
\item Répondre à la question {\bf1.} en utilisant les coordonnées des points $B$,\,$J$ et $I$.
\end{enumerate}
\end{enumerate}

\end{question}
\begin{solution}
\begin{enumerate}
\item \textbf{Montrons que $(BI)$ est orthogonale à $(IJ)$}:

On a $(IJ) \perp (DC)$ d'une part.

D'autre part Comme $(IJ)\:\parallel \: (CG)$ et $(CG) \perp (BC)$ alors $(IJ)$ et orthogonale à $(BC)$. 

$(IJ)$ est orthogonale à deux droites sécantes du plan $(BCD)$ donc $(IJ) \perp (BCD)$.
Or $(BI)$ est une droite du plan $(BCD)$ Donc $(IJ) \perp (BI)$.

\item \textbf{Nature de la section:}
\begin{itemize}
\item $(BIL)$ coupe $(EFG)$ en une droite $d$. Comme $L \in (BIL)$ et $ L \in (EFG)$ donc $ L \in d$.
\item $(BIL)$ coupe $(ABC)$ suivant la droite $(IB)$, et $(ABC) \parallel (EFG)$ donc $(IB) \parallel d$.
\item Finalement $L \in d$ et $d \parallel (IB)$
\end{itemize}
\item \textbf{Section du cube:}
\begin{itemize}
\item $(BIL)$ coupe la face $ABCD$ suivant le segment $[IB]$.
\item On a vu question précédente que $(BIL)$ coupe $(EFG)$ suivant une droite $d$ passant par $L$ et parallèle à $(IB)$. On appelera $M$ le point d'intersection de $d$ avec $[FG]$

Donc $(BIL)$ coupe la face $EFGH$ suivant le segment 	$LM]$.
\item $(BIL)$ coupe la face $DCGH$ suivant le segment	$[IL]$.
\item $(BIL)$ coupe la face $BCGF$ suivant le segment $[BM]$.
\end{itemize}
D'après ce qui précède la section du cube $ABCDEFGH$ est du plan $(BIL)$ est le trapèze $BILM$.
\begin{center}
    \definecolor{uuuuuu}{rgb}{0.26666666666666666,0.26666666666666666,0.26666666666666666}
\definecolor{ffqqtt}{rgb}{1.,0.,0.2}
\begin{tikzpicture}[line cap=round,line join=round,>=triangle 45,x=1.0cm,y=1.0cm]
\draw (0,0) node [below] {B} -- 
	({sqrt(3)},0) node [below] {I} -- 
    ({sqrt(3)},{sqrt(2)}) node [above] {L} -- 
    ({sqrt(2)-1/2},{sqrt(2)}) node [above] {M} -- 
    cycle;
\end{tikzpicture}
    \end{center}.
\item \begin{enumerate}
\item \textbf{Nature des points L,I,B et F:} 

On à $L (~\frac{3}{4} a; a ; a ~)$,\: $B(~ a; 0; 0~)$,\;$I(~\frac{1}{2} a; a ; 	0~)$ et $F(~	 a; 0	; a ~)$

$L,I,B \mbox{et}, F$ Sont coplanaires ssi $\vect{IL}$, $\vect{IB}$\;et $\vect{IF}$ sont coplanaires.

Or il	n'exite pas de rélles	$\alpha\,\mbox{et} \beta$ vérifiant $\vect{IL}=\alpha	\vect{IB}+\beta \vect{IF}$ Donc les vecteures ne sont pas coplanaires.
\item \textbf{Coplanarité des droites $(IL)\,et\,(BK)$:}

Les droites  $(IL)\,et\,(BK)$ sont coplanaires ssi les points $I, L, B	\;\mbox{et} \,K$ sont coplanaires. 

Même raisonnement il	n'exite pas de rélles	$\alpha\,\mbox{et} \beta$ vérifiant $\vect{IL}=\alpha	\vect{IB}+\beta \vect{IK}$ Donc les vecteures ne sont pas coplanaires.
\item Il suffit d'utilier le théorème de Pythagore dans le triangle $IBJ$.


\end{enumerate}
\end{enumerate}
\end{solution}
\newpage 
\textbf{Annexe \no{1} :}\hfill\textbf{\`A rendre avec la copie}\hfill \textbf{Numéro:}$\cdots$

\vspace{2cm}
\begin{tikzpicture}[xscale=4,yscale=5.5]
\tkzInit[xmin=-2,xmax=2,ymin=-1.,ymax=1.]
\tkzGrid 
\tkzAxeXY 
\end{tikzpicture}
\newpage 
\textbf{Annexe \no{2} :}\hfill\textbf{\`A rendre avec la copie}\hfill \textbf{Numéro:}$\cdots$
\vfill

\begin{center}
\begin{tikzpicture}[scale=13,>=latex]
	\draw [dotted,thin] (0,0) grid [step=0.1] (1,1) ;
	\draw [very thick] plot [smooth,domain=0:1.1] (\x,{1/8*\x^3 + 3/8*\x*\x + 3/8 *\x + 1/8}) ;
    \draw [very thick] (0,0) -- (1.1,1.1) ;
    \draw [thick,->] (-0.1,0) -- (1.1,0) ;
    \draw [thick,->] (0,-0.1) -- (0,1.1) ;
    \draw (0,0) node [anchor=north west] {O} ;
\end{tikzpicture}
\end{center}
\vfill
\newpage
\printsolutions
\end{document}
Pour chaque question plusieurs réponses sont proposées. Déterminer celles qui sont corrects.

Une bonne réponse rapporte \np[pt]{1}.

Une mauvaise réponse enlève \np[pt]{0,5}

Une absence de réponse ne donne ni n'enlève de points.

\textbf{Sur votre copie écrire la question ainsi que la réponse}

\bcbombe {\bfseries Toute réponse non justifiée ne sera pas prise en compte.}
\vspace{1cm}
