\documentclass[12pt,a4paper,french]{exam}

\usepackage[utf8]{inputenc}
\usepackage[T1]{fontenc}
\usepackage{lmodern}
\usepackage{cmbright}

\usepackage{babel}

\frenchbsetup{}

\newcounter{exo}
\newcommand{\Exo}{\refstepcounter{exo}\fullwidth{\noindent%
\Large \textbf{Exercice \theexo}}}

\newcounter{partie}[exo]
\newcommand{\Partie}{%
  \refstepcounter{partie}%
  \fullwidth{%
    \noindent \textbf{Partie \Alph{partie}}
  }
}
\renewcommand{\questionlabel}{\textbf{\thequestion.}}
\renewcommand{\partlabel}{\textbf{\thepartno.}}


% Commande pour se faciliter la vie:
\newcommand{\N}{\mathbf{N}}
\newcommand{\R}{\mathbf{R}}

\title{Bac blanc décembre 2015}


\begin{document}

\pagestyle{headandfoot}

\firstpageheader{}
  {}
  {}
\runningheader{}{Jean-Baptiste de la Salle\\ Bac blanc, décembre 2015\\
Mathématiques}{}

\footer{}{Page \thepage{} sur \numpages}{\iflastpage{Fin du
sujet.}{T.S.V.P.}}

\begin{coverpages}

  \maketitle

\end{coverpages}

\Exo
\begin{questions}
  \question Donner la définition 

  \Partie
  On considère la fonction définie sur $\R$ par $f(x) = x + \frac1x$.

  \question

  \begin{parts}
    \part On considère la suite $(u_n)$ définie sur $\N$ par $u_{n+1} =
    f(u_n)$.
  \end{parts}
\end{questions}

\Exo
\begin{questions}
  \question 2 + 2 = 
\end{questions}

\end{document}
