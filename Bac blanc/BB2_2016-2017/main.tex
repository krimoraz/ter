\documentclass[12pt,a4paper,french]{article}
\usepackage{etex}
\usepackage[utf8]{inputenc}
\usepackage[T1]{fontenc}
\usepackage{babel}
\usepackage[thmmarks]{ntheorem}
\usepackage{amsmath}
\usepackage{amsfonts}
\usepackage{amssymb}
\usepackage{array}
\usepackage{lmodern}
\usepackage{kpfonts}
\usepackage[bookmarks=false,colorlinks,linkcolor=blue]{hyperref}
\usepackage{tabularx}
\usepackage[autolanguage,np]{numprint}
\usepackage{enumitem}
\usepackage{lastpage}
\usepackage{pgf}
\usepackage{tikz}
\usepackage{tkz-euclide}
\usepackage{esvect}
\usetkzobj{all}
\usetikzlibrary{arrows}
\usetikzlibrary{shadows}
\usetikzlibrary{shapes}
\usetikzlibrary{intersections}
\usetikzlibrary{plotmarks}
\usetikzlibrary{positioning}
\usetikzlibrary{datavisualization}
\usetikzlibrary{decorations.pathmorphing}
\usetikzlibrary{patterns}
\usetikzlibrary{calc}
\usepackage{tkz-tab}
\usepackage{tkz-fct}
\usepackage{tkz-base}
\usepackage{bclogo}
\DeclareMathOperator{\e}{e}
\usepackage[top=0.5cm,bottom=2cm,left=1.5cm,right=1.3cm]{geometry}
\usepackage{pstricks,pst-plot,pst-text,pst-tree,pstricks-add}

\usepackage{lastpage}
\usepackage{marginnote}
%\usepackage{pas-tableur}
\usepackage{picins}
\usepackage{xparse}
\usepackage[extdef]{delimset}

\usepackage[defaultlines=5,all]{nowidow}

\newcommand{\R}{\mathbb{R}} 
\newcommand{\N}{\mathbb{N}} 
\renewcommand{\D}{\mathbb{D}} 
\newcommand{\Z}{\mathbb{Z}} 
\newcommand{\Q}{\mathbb{Q}} 
%\renewcommand{\C}{\mathbb{C}} 
\newcommand{\vect}[1]{\mathchoice% 
{\overrightarrow{\displaystyle\mathstrut#1\,\,}}% 
{\overrightarrow{\textstyle\mathstrut#1\,\,}}% 
{\overrightarrow{\scriptstyle\mathstrut#1\,\,}}% 
{\overrightarrow{\scriptscriptstyle\mathstrut#1\,\,}}} 
\def\Oij{$\left(\text{O},~\vect{\imath},~\vect{\jmath}\right)$} 
\def\Oijk{$\left(\text{O},~\vect{\imath},~ \vect{\jmath},~ \vect{k}\right)$} 
\def\Ouv{$\left(\text{O},~\vect{u},~\vect{v}\right)$}

\usepackage{mdframed}
\setlength{\parsep}{0pt}
\setlength{\parskip}{5pt}
\setlength{\parindent}{0pt}
\setlength{\itemsep}{7pt}
\setlist{noitemsep}
\usepackage{multicol}
\setlength{\columnseprule}{0pt}
\usepackage[]{exsheets}
\SetupExSheets{
  counter-format = {qu[1] :} ,
  headings = block-subtitle ,
  points/name = {pt/s} ,
  solution/print = false ,%true,%true
  %solution/pre-hook = {\mdframed} ,
  %solution/post-hook = {\endmdframed} ,
  blank/style = dotted ,
}

\DeclareExSheetsHeadingContainer{autopoints}{%
  \textbf{(\GetQuestionProperty{points}{\CurrentQuestionID} points)}%
}

 \DeclareInstance{exsheets-heading}{block-subtitle}{default}{
 join = {
 title[r,B]number[l,B](.333em,0pt) ;
 title[r,B]subtitle[l,B](1em,0pt)
 } ,
 attach = {
 main[l,vc]title[l,vc](0pt,0pt) ;
 main[r,vc]autopoints[l,vc](\marginparsep-55pt,0pt)
 },
}


\everymath{\displaystyle\everymath{}}

\newcommand{\brm}[1]{\marginpar{\addpoints{#1}}}% étoile n'affiche pas les points sur la copie
\newcommand{\cel}{\celtxt[c]}
\title{BBlanc \no{2}}
\renewcommand{\theenumi}{\textbf{\arabic{enumi}}} 
\renewcommand{\labelenumi}{\textbf{\theenumi.}} 
\renewcommand{\theenumii}{\textbf{\alph{enumii}}} 
\renewcommand{\labelenumii}{\textbf{\theenumii.}}
\usetikzlibrary{backgrounds}
\frenchbsetup{StandardLists=true}
\def\coeff{2}

\newcommand{\itemTriangle}{%
   \scalebox{0.8}{$\blacktriangleright$}%
}
\usepackage{ifthen}
\usepackage{fancyhdr}
\pagestyle{fancy}


\count1=\year \count2=\year
\ifnum\month<8\advance\count1by-1\else\advance\count2by1\fi

\setlength{\headheight}{14.5pt}
\cfoot{\textsl{\footnotesize{Année \number\count1/\number\count2}}}
\lfoot{\textsl{\footnotesize{Lycée La salle \textsc{Saint-Denis}}}}
%\rfoot{%
 % \ifthenelse{\value{page}=1}
 % {%
 %  \footnotesize{Page \thepage/ \pageref{LastPage}}
% }
%}
%\rhead{\textbf{16LYJBSBB1}}
\renewcommand{\headrulewidth}{1pt}
\renewcommand{\footrulewidth}{0pt}
\fancyfoot[r]{\textsl{\footnotesize{Page \thepage{} sur \pageref{LastPage}}}}

%\usepackage{fontspec}

\begin{document}



\renewcommand\labelitemi{\textbullet}
\thispagestyle{empty} 
\begin{flushright}
\textbf{17LYJBSBB2}
\end{flushright}
\begin{center}
 \begin{tikzpicture}
 \draw (0,8) node[color=black,draw,fill=black!5]%
 {\fontsize{30}{30}\selectfont{\upshape BACCALAUR\'EAT G\'EN\'ERAL }};
  \node[name    = s,%
        shape   = rectangle,%
        rotate  = 35,%
        color   = black!50]%
  {\fontsize{100}{150}\selectfont{\upshape  \textbf{\textit{obligatoire}}}};
\draw (0,7)    node[color=black] {\LARGE \textbf{Bac blanc n°2  Lycée La Salle -Saint Denis} };%   
\draw (0,6)    node[color=black] {\Large \textbf{Session 2017} };%
\draw (0,4)    node[color=black] {\Huge \textbf{MATH\'EMATIQUES} };%
\draw (0,2)    node[color=black] {\huge \textbf{- Série S -}};
\draw (0,0)    node[color=black] {\LARGE \textbf{\'EPREUVE DU 29 MARS 2017 }};
\draw (0,-2)    node[color=black] {\Large \textbf{ENSEIGNEMENT SP\'ECIFIQUE}};
\draw (0,-3)    node[color=black] {\Large \textit{ \textbf{Durée de l'épreuve : 4 heures}}};
\draw (6.3,-2.1)    node[color=black] {\large \textit{ \textbf{Coefficient : 7}}};
\draw (0,-7)  node[text centered]%
                  {\textbf%
                   {Les calculatrices éléctroniques de poche sont autorisées,}
                  };
\draw (0,-7.4)  node[text centered]%
                  {\textbf%
                   {conformément à la réglementation en vigueur.}
                  };

 \draw (0,-10) node[inner sep=5pt,draw,text width = 15cm, text justified] {
\textbf{Le sujet est composé de 4 exercices indépendants. Le candidat doit traiter tous les exercices.}

\textbf{Dans chaque exercice, le candidat peut admettre un résultat précédemment donné dans le texte pour aborder les questions suivantes, à condition de l'indiquer clairement sur la copie. La qualité et la précision de la rédaction seront prises en compte dans l'appréciation des copies.}
};
                

\draw (0,-13.6)  node[text centered]%
                 {\textbf%
                   {Avant de composer, le candidat s'assurera que le sujet comporte bien \pageref{LastPage} pages numérotées de 1 à \pageref{LastPage}}
                 };
\end{tikzpicture} 
\end{center} 
\begin{flushright}
\begin{tabular}{@{}p{13cm}@{}}
 {\raggedleft \itshape 
\og  Suivre ses passions et espérer vivre longtemps est le moteur de tout mal. En effet, suivre ses passions détourne de la connaissance de la vérité et de la volonté de la mettre en pratique. Quant à l’espérance d’une longue vie, elle fait oublier l’au-delà et empêche la personne de s’y préparer. \fg \par}\\
 
\end{tabular}
\end{flushright}

\vfill
\newpage
\pagenumbering{arabic}
\begin{question}
\vspace{-5.8mm}
\begin{center}\textbf{Commun à tous les candidats}\end{center}
Soit la fonction $f$ définie sur $\left] 0 \, ; +\infty\right[ $ par
  $f(x)=\ln x$. On note $\mathscr{C}$ la courbe représentative de $f$
  dans un repère orthonormé $(O;I,J)$. Soit $M$ un point quelconque
  sur~$\mathscr{C}$.
\begin{center}
\definecolor{qqwuqq}{rgb}{0.,0.39215686274509803,0.}
\definecolor{cqcqcq}{rgb}{0.7529411764705882,0.7529411764705882,0.7529411764705882}
\begin{tikzpicture}[line cap=round,line join=round,>=triangle 45,x=1.0cm,y=1.0cm]
\draw [color=cqcqcq,, xstep=1.0cm,ystep=1.0cm] (-1.,-2.) grid (5.,3.);
\draw[->,color=black] (-1.,0.) -- (5.,0.);
\foreach \x in {-1.,1.,2.,3.,4.}
\draw[shift={(\x,0)},color=black] (0pt,2pt) -- (0pt,-2pt);
\draw[->,color=black] (0.,-2.) -- (0.,3.);
\foreach \y in {-2.,-1.,1.,2.}
\draw[shift={(0,\y)},color=black] (2pt,0pt) -- (-2pt,0pt);
\clip(-1.,-2.) rectangle (5.,3.);
\draw[line width=1.2pt,color=qqwuqq,smooth,samples=100,domain=0.001:5.0] plot(\x,{ln((\x))});
\draw (0.,1.)-- (1.54,0.43178241642553783);
\draw (1.22,1.22) node[anchor=north west] {$\mathbf{M}$};
\draw (-0.5,1.44) node[anchor=north west] {$\mathbf{J}$};
\draw(4,1.5) node[anchor=north west]{\color{qqwuqq}{$\mathscr{C}$}};
\begin{scriptsize}
\draw[color=qqwuqq] (0.26,-6.61) node {$f$};
\end{scriptsize}
\end{tikzpicture}
\end{center} 
\begin{enumerate}

\item Soit $x$ l'abscisse de $M$ avec $x>0$. Exprimer $JM$ en fonction de $x$. \brm{0.5}
\item On pose $g(x)=x^2-1+\ln x$.
\begin{enumerate}
\item Déterminer les limites de $g$ en 0 et en $+\infty$. \brm{0.25+0.25}
\item Dresser le tableau de variation de $g$ sur $\left]  0 \, ; +\infty\right[ $.  \brm{0.5}
\item Montrer que l'équation $g(x)=0$ admet une unique solution \brm{1}
  $\alpha$ sur $\left]  0 \, ; +\infty\right[ $.

 Donner la valeur exacte de $\alpha$.
\item En déduire le signe de $g(x)$. \brm{0.5}
\end{enumerate} 
\item On pose $h(x)=x^2+(1-\ln x)^2$.
\begin{enumerate}
\item Montrer que pour tout réel $x>0$, $h^{\prime}(x)=\dfrac{2g(x)}{x}$. \brm{0.5}
\item Étudier les variations de la fonction $h$ sur $\left]  0 \, ; +\infty\right[ $. \brm{0.5}
\item En déduire la position de $M$ qui rend la distance $JM$ minimale, et calculer cette distance. \brm{0.5}
\item Dans la configuration de la question \textbf{3.c}, que peut-on dire sur la droite $(JM)$ et  la tangente à la courbe $\mathscr{C}$ au point $M$ ? prouvez-le. \brm{0.5}
\end{enumerate}
\end{enumerate}
\end{question}
\begin{solution}
  \begin{enumerate}
    \item $JM = \sqrt{x^2 + (1 - \ln x)^2}$
    \item $\lim_{x\to 0}g(x) = - \infty$ ; $\lim_{x\to +\infty}g(x) = + \infty$
      \begin{enumerate}
        \item $x^2 - 1$ et $\ln $ sont des fonctions croissantes sur
          l'intervalle considéré, donc $g$ est croissante.
        \item $g$ est strictement croissante et continue. D'après le
          théorème des valeurs intermédiaires, pour tout $y$ dans
          l'intervalle image (ici, $\R$), l'équation $f(x) = y$ possède
          exactement une solution.

          On trouve que $g(1) = 0$ donc $\alpha = 1$.
        \item ~\\
          \begin{center}
            \begin{tikzpicture}
              \tkzTabInit{$x$/1,$g(x)$/1}{$-\infty$,1,$+\infty$}
              \tkzTabLine{,-,0,+,}
            \end{tikzpicture}	
          \end{center}
      \end{enumerate}
      \begin{enumerate}
        \item $h'(x) = 2x + 2(1 - \ln x)\left(\frac{-1}x\right) =
          \frac{2x^2 - 2 + 2 \ln x}x$
        \item ~\\
          \begin{center}
            \begin{tikzpicture}
              \tkzTabInit{$x$/1,$g(x)$/1,$h(x)$/2}{0,1,$+\infty$}
              \tkzTabLine{d,-,0,+,}
              \tkzTabVar{+/$-\infty$,-,+/$+\infty$}
            \end{tikzpicture}	
          \end{center}
        \item $JM$ est minimale pour $h$ minimale et ce minimum est
          atteint en 1. On a donc $JM = \sqrt{2}$.
        \item $(JM) \perp (T)$. En effet, $(JM) : y = -x + 1$ et $(T) :
          y = x - 1$
      \end{enumerate}
  \end{enumerate}
\end{solution}

\begin{question}
\vspace{-5.8mm}
\begin{center}\textbf{Commun à tous les candidats}\end{center}
En l'absence de prédateurs et dans des conditions de vie favorables ( abondance de nourriture...), on peut modéliser l'évolution d'une population par une fonction $f$ définie sur $\left[  0 \, ; +\infty\right[ $ de la forme $f(t)=k\e^{at}$, où $a$ et $k$ sont des constantes réelles. Il s'agit du modèle de l'économiste anglais \textit{Malthus ( 1766 - 1834)}. 

\begin{enumerate}
\item Montrer que $f^\prime$ est proportionnelle à $f$. \brm{0.5}
\end{enumerate}

On étudie l'évolution d'une population  de lapin dans une île où les lapins n'ont pas de prédateurs.

On note $g(t)$ le nombre de milliers de lapins à l'instant $t$ (exprimé en années). 

On définie ainsi une fonction $g$ sur l'intervalle $[0;+\infty[$.

On prend comme hypothèse que la \og vitesse de croissance \fg, $g^\prime(t)$ est proportionnelle à $g(t)$ à tout instant $t$.

Les observations faites conduisent à prendre pour coefficient de proportionnalité \np{0,34}.

On va déterminer une expressions de $g$.

\subsection*{\'Etude théorique:}
On cherche à déterminer les solutions de l'équation $(E)$: $g^\prime =\np{0,34} g$ ,\; où l'inconnue est $g$.
\begin{enumerate}[resume]
\item Montrer que les fonctions $g$, définies sur $[0,+\infty [$ par $g(t)=C\,\e^{0,34t}$, où $C$ est un réel, sont solutions de l'équation $(E)$. \brm{0.5}
\item Soit $h$ la fonction définie sur  $[0,+\infty [$ par $h(t)=\e^{-0,34 t}\times g(t)$.
\begin{enumerate}
\item Montrer que $h^\prime(t)=0$. \brm{0.5}
\item En déduire que si la fonction $g$ est solution de $(E)$ alors, pour tout réel $t$ de  $[0,+\infty [$ , $\e^{-0,34t}\times g(t)=C$ avec $C$ un réel. \brm{1.0}
\end{enumerate}
\item \`A l'aide de la question \textbf{2.} et \textbf{3.} , montrer que les fonctions $g$, définies  $[0,+\infty [$  par $g(t)=C\,\e^{0,34t}$, où $C$ est un réel, sont les seules solutions de l'équation $(E)$. \brm{0.5}
\end{enumerate}

\subsection*{Utilisation d'un modèle :}
En \np{1859}, un britannique, Thomas Austin, a importé en Australie 12 couples de lapins de garenne.

On suppose que la population de lapins s'est développée en suivant le modèle ci-dessous.
\begin{enumerate}
\item On choisit de modéliser l'évolution du nombre de lapins au terme de l'année $\np{1859}+n$ où $n$ est un entier naturel par la suite suivante :

\[\left\lbrace\begin{matrix} P_{n+1} - P_n = 0,4P_n\\ P_0 = 24 \end{matrix} \right. \]

\textit{La question suivante est une question à choix multiples. Le candidat recopiera sur sa copie la réponse qu'il pense être correcte.} \brm{0.5}
\begin{center}
\begin{tabular}{|p{5cm}|p{5cm}|p{5cm}|}\hline
La suite $(P_n)$ est géométrique de raison $1,4$ & La suite $(P_n)$ est arithmétique de raison $0,4$ & La suite $(P_n)$ est géométrique de raison $0,4$ \\
\hline
\end{tabular}
\end{center}
\item On considère l'algorithme suivant :
	\begin{center}
        \begin{minipage}{0.7\linewidth}
        \begin{mdframed}
        \begin{tabular}{p{2.5cm}p{8cm}}
		\textbf{Variables :} & $P$, $N$ \\    
        \textbf{Traitement :} & \\
        & Affecter 0 à $N$ \\
		& Affecter $\cdots$ à $P$ \\
        & Tant que $\dotfill$ \\
        & \hspace{5mm} Affecter $\cdots$ à $N$ \\
        & \hspace{5mm} Affecter $P + 0,4P$ à $P$ \\
        & Fin Tant que \\
        \textbf{Sortie :} & Afficher $P$ et $N$ \\
		\end{tabular}
    	\end{mdframed}
    	\end{minipage}
    \end{center}
      
\begin{enumerate}
\item Recopier et compléter cet algorithme pour déterminer au bout de combien d'année le nombre de lapins a pour la première fois dépassé 600 millions. \brm{0.75}
\item Quelle est cette année ? Préciser les valeurs renvoyées par l'algorithme. \brm{0.25}
\end{enumerate}
\item Sans les mesures prises pour freiner cette évolution, combien y aurait-il de lapins en Australie aujourd'hui? \brm{0.5}
\end{enumerate}

\end{question}

\begin{solution}
  \section*{Étude théorique}
  \begin{enumerate}
    \item $f(t) = ke^{at} \implies f'(t) = kae^{at} = af(t)$
    \item $g'(t) = 0,34Ce^{0,34t} = 0,34g(t)$ donc $g$ est une solution
      de $(E)$.
    \item $h'(t) = e^{-0.34t} g'(t) - 0,34e^{-0.34t}g(t) = 0$
    \item La question précédente entraîne que $h$ est constante et comme
      $e^{-0.34\times 0}g(0) = C$, on en déduit $e^{-0.34t}g(t) = C$.
    \item La question 2. a permis de montrer que les fonctions de la
      forme $Ce^{0,34t}$ étaient solution de $(E)$. Réciproquement, si
      on a une fonction $g$ solution de $(E)$, alors $e^{-0.34t}g(t) =
      C$ et donc $g(t) = Ce^{0,34t}$ est une solution. Ce sont donc les
      seules solutions.
  \end{enumerate}
  \section*{Utilisation d'un modèle}
  \begin{enumerate}
    \item $(P_n)$ est géométrique de raison 1,4.
    \item On complète par :
      \begin{itemize}
        \item Affecter 24 à $P$
        \item Tant que $P < \np{600000000}$
        \item Affecter $N + 1$ à $N$.
      \end{itemize}
    \item $P$ représente la population de lapins et $N$ le nombre d'années.

      On trouve $N = 51$ pour $P = 6,8\cdot 10^{8}$ lapins.
    \item avec un tel modèle, on aurait $2,94\cdot 10^{24}$ lapins de
      nos jours en Australie.
  \end{enumerate}
\end{solution}

\pagebreak
\begin{question}
\vspace{-5.8mm}
\begin{center}\textbf{Commun à tous les candidats}\end{center}
Soient $f$ et $g$ les fonctions définies sur l'ensemble $\mathbf{R}$ des nombres réels par :
$ f(x) = x\e^{1 - x}$ et $g(x) = x^2\e^{1 - x}$.

Les courbes représentatives des fonctions $f$ et $g$ dans un repère orthogonal \Oij\, sont respectivement notées $\mathcal{C}$ et $\mathcal{C}'$.

\begin{center}
  \psset{unit=1cm}
  \begin{pspicture}(-3,-3)(4,4)
    \psgrid[gridlabels=0pt,subgriddiv=1,gridcolor=cyan](-3,-3)(4,4)
    \psaxes[linewidth=1.5pt,arrowsize=2pt 3]{->}(0,0)(-3,-3)(4,4)
    \psaxes[linewidth=1.5pt](0,0)(-3,-3)(4,4)
    \psplot[plotpoints=8000,linewidth=1.25pt,linecolor=red]{-0.606}{4}{x
      2.71828 x 1 sub exp div}
    \psplot[plotpoints=8000,linewidth=1.25pt,linecolor=green]{-0.81}{4}{x
      dup mul 2.71828 x 1 sub exp div} \uput[dl](0,0){O}
    \uput[l](-0.5,-2){$\mathcal{C}$} \uput[l](-0.53,2){$\mathcal{C}'$}
  \end{pspicture}
\end{center}

\begin{enumerate}

\item \textbf{Calcul d'intégrales}

Pour tout entier naturel $n$, on définit l'intégrale $I_{n}$ par :

$I_{n} = \int_{0}^1 x^{n} \e^{1 - x} dx$ (en particulier $ I_{0} = \int_{0}^1 \e^{1 - x}dx$).


	\begin{enumerate}
		\item Calculer la valeur exacte de $I_{0}$. \brm{0.75}
		\item Pour tout réel $x$ et tout entier naturel $n$, on pose $g_{n}(x)=x^n\e^{1-x}$ (en particulier $g_{0}=\e^{1-x}$).\\
Établir que pour tout réel $x$ et tout entier naturel $n$, $g'_{n+1}(x)=(n+1)g_{n}(x)-g_{n+1}(x)$.\\
En déduire que pour tout entier naturel $n$:
		
$I_{n+1} = - 1 + (n + 1)I_{n}.$		\brm{1.}
		\item En déduire la valeur exacte de $I_{1}$, puis celle de $I_{2}$. \brm{0.5}
	\end{enumerate}


\item \textbf{Calcul d'une aire plane}


	\begin{enumerate}
		\item Étudier la position relative des courbes $\mathcal{C}$ et $\mathcal{C}'$. \brm{0.5}
		\item On désigne par $\mathcal{A}$ l'aire, exprimée en unité d'aire, de la partie du plan comprise d'une part entre les courbes $\mathcal{C}$ et $\mathcal{C}'$, d'autre part entre les droites d'équations respectives $x = 0$ et $x = 1$. \brm{1}
		
En exprimant $\mathcal{A}$ comme différence de deux aires que l'on précisera, démontrer l'égalité :

$\mathcal{A} = 3 - \e.$
	\end{enumerate}

\pagebreak

\item \textbf{Étude de l'égalité de deux aires}

Soit $a$ un réel strictement supérieur à  1.

On désigne par $S(a)$ l'aire, exprimée en unité d'aire, de la partie du plan comprise d'une part entre les courbes $\mathcal{C}$ et $\mathcal{C}'$, d'autre part entre les droites d'équations respectives $x = 1$ et $x = a$.

On admet que $S(a)$ s'exprime par :

$S(a) = 3 - \text{e}^{1 - a}\left(a^2 + a + 1\right)$.\\

L'objectif de cette question est de prouver qu'il existe une et une seule valeur de $a$ pour laquelle les aires $\mathcal{A}$ et $S(a)$ sont égales.


	\begin{enumerate}
		\item Démontrer que l'équation $S(a) = \mathcal{A}$ est équivalente \`a  l'équation :
		
$\e^a = a^2 + a + 1$. \brm{0.5}
		\item \emph{Dans cette question, toute trace d'argumentation, même incomplète, ou d'initiative, même non fructueuse, sera prise en compte dans l'évaluation.}
		
Conclure, quant à l'existence et l'unicité du réel $a$, solution du problème posé. \brm{0.75}
	\end{enumerate}
\end{enumerate}
\end{question}

\begin{solution}
  \begin{enumerate}
    \item Calcul d'intégrales
    \begin{enumerate}
      \item $I_0 = \int_0^1 e^{1 - X}\mathrm{d}x = \left[-e^{1 -
        x}\right]_0^1 = e - 1$
      \item $g'_{n+1}(x) = (n+1)x^ne^{1-x} - x^{n+1}e^{1-x} =
        (n+1)g_n(x) - g_{n+1}(x)$. On en déduit que $g_{n+1}(x) =
        (n+1)g_n(x) - g'_{n+1}(x)$, ce qui entraîne en intégrant que
        $\int_0^1 g_{n+1}(x)\mathrm{d}x = (n+1)\int_0^1
        g_n(x)\mathrm{d}x - \int_0^1 g'_{n+1}(x)\mathrm{d}x$ et donc
        $I_{n+1} = (n+1)I_n + \left[g_{n+1}(x)\right]_0^1$. Or on a le
        résultat suivante : $\left[g_{n+1}(x)\right]_0^1$ et donc
        $I_{n+1} = (n+1)I_n -1$. $I_1 = -1 + I_0 = -2 + e$ et $I_2 = -1
        + 2I_1 = -5 +2e$.
    \end{enumerate}
  \item Calcul d'une aire plane
    \begin{enumerate}
      \item $f(x) - g(x) = x(x-1)e^{1-x}$
        \begin{center}
          \begin{tikzpicture}
            \tkzTabInit{$x$/1,$f(x) - g(x)$/1}{0,1,$+\infty$}
            \tkzTabLine{,-,0,+,}
          \end{tikzpicture}
        \end{center}
        On en déduit que $\mathcal{C}$ est au dessus de $\mathcal{C}'$
        sur $[0;1]$ et que $\mathcal{C}'$ est au dessus de $\mathcal{C}$
        sur $[1;+\infty[$
      \item $\mathcal{A} = I_1 - I_2$. $I_0 = e - 1$, $I_1 = -1 +
        I_0 = -2 + e$ et donc $I_2 = -1 + 2I_1 = -5 +2e$.
        On obtient $\mathcal{A} =  -2 + e -(-5 +2e) = +3 -e$
    \end{enumerate}
  \item Étude de l'égalité de deux aires
    \begin{enumerate}
      \item $S(a) = \mathcal{A} \iff 3 - e^{1-a}(a^2 + a + 1) = 3 -
        e \iff e^1e^{-a}(a^2 + a + 1) = e^1 \iff a^2 + a + 1 = e^a$.
      \item Une piste de résolution : on considère une fonction $h$
        définie par $h(x) = e^x - x^2 - x - 1$. On montre que cette
        fonction est strictement monotone sur un intervalle
        contenant 0 et sur un seul. On conclue avec le TVI.
    \end{enumerate}
\end{enumerate}
\end{solution}

\begin{question}
\vspace{-5.8mm}
\begin{flushright}\hfill\textbf{Candidats n'ayant pas suivi l'enseignement de spécialité}\hfill ~\end{flushright}

On considère un cube ABCDEFGH de côté 1.

\parpic[r]{
\psset{unit=0.7cm}
\begin{pspicture}(11,11)
\psframe(3,0.5)(10,7.5)%DAEH
\pspolygon(10,7.5)(8.2,9.3)(1.2,9.3)%EFG
\pspolygon(3,7.5)(1.2,9.3)(3,0.5)%HGD
\psline(1.2,9.3)(1.2,2.3)(3,0.5)(10,7.5)%GCDE
\psline[linestyle=dashed](8.2,9.3)(8.2,2.3)(1.2,2.3)%FBC
\psline[linestyle=dashed](8.2,2.3)(10,0.5)%BA
\uput[dr](10,0.5){A}\uput[ur](8.2,2.3){B}
\uput[l](1.2,2.3){C}\uput[dl](3,0.5){D}
\uput[r](10,7.5){E}\uput[ur](8.2,9.3){F}
\uput[ul](1.2,9.3){G}\uput[dr](3,7.5){H}
\end{pspicture}}

On se place dans le repère orthonormé $\left(\text{B}~;~\vect{\text{BA}},\: \vect{\text{BC}},\: \vect{\text{BF}}\right)$.

\medskip

\begin{enumerate}
\item Déterminer une représentation paramétrique de

la droite (BH). \brm{1}
\item Démontrer que la droite (BH) est perpendiculaire

au plan (DEG). \brm{1.0}
\item Déterminer une équation cartésienne du plan (DEG). \brm{1}
\item On note P le point d'intersection du plan (DEG) et de 

la droite (BH).

Déduire des questions précédentes les coordonnées

du point P. \brm{1.0}

\item Que représente le point P pour le triangle DEG ?

Justifier la réponse. \brm{1.0}
\end{enumerate}

\end{question}
\begin{solution}
  \begin{enumerate}
  \item $\vv{BH} : \left( \begin{matrix} 1\\1\\1 \end{matrix} \right)$.
    On en déduit que \[ (BH) : \left\{ \begin{matrix} x = t \\ y = t\\ z
      = t \end{matrix}, \ \ t \in \R \right. .\]
    \item $(DEG)$ est dirigé par les vecteurs $\vv{DG} :  \left(
    \begin{matrix} 0\\-1\\1 \end{matrix} \right)$ et $\vv{DE} :  \left(
    \begin{matrix} -1\\0\\1 \end{matrix} \right)$. On vérifie aisément
      que $\vv{DE} \cdot \vv{BH} = 0$ et $\vv{DG} \cdot \vv{BH} = 0$.
    \item \[ (DEG) : \left\{ \begin{matrix} x = 1-r \\ y = 1 - s\\ z = r
        + s \end{matrix},\ \  (r,s) \in \R^2 \right. .\]
    \item $P: \left(\frac12, \frac12, \frac12\right)$
    \item $P$ est le centre de gravité du triangle. On peut le démontrer
      en vérifiant la somme vectorielle $\vv{PE} + \vv{PD} + \vv{PG} =
      \vv{0}$.
  \end{enumerate}
\end{solution}

\pagebreak

\printsolutions

\end{document}
Le but de cet exercice est de montrer  que l'équation :\\
\begin{center}(E)\quad $\e^x = \dfrac{1}{x}$\end{center}
admet une unique solution dans  $\mathbf{R}$ et de construire une suite qui converge vers elle.\\

\subsection*{Partie A: Existence et unicité de la solution}


Soit $f$ la fonction définie sur $\mathbf{R}$ par :
\begin{center}$f(x)=x - \e^{-x}$.\end{center}
\begin{enumerate}
\item  Démonter que $x$ est solution de  l'équation (E) si, et seulement si, $f(x)=0$.
\item
	\begin{enumerate}
		\item  Étudier le sens de variations de $f$ sur $\mathbf{R}$.
		\item  En déduire que l'équation (E) possède une unique solution sur $\mathbf{R}$. On la note $\alpha$.\vspace{1mm}
		\item  Démontrer que $\dfrac{1}{2}\leqslant \alpha \leqslant 1$.
		\item  Étudier le signe de $f$  sur l'intervalle $[0~;~\alpha]$.
	\end{enumerate}
\end{enumerate}

\subsection*{Partie B: Deuxième approche}


On note $g$ la fonction définie sur l'intervalle $[0~;~1]$ par :
\begin{center}$g(x) = \dfrac{1 + x}{1 + \e^x}$.\end{center}
\begin{enumerate}
\item  Démontrer l'équivalence :
\begin{center}$f(x) = 0 \Leftrightarrow g(x) = x$.\end{center}
\item  En déduire que $\alpha$ est l'unique solution de $g(x)=x$.
\item  Calculer $g'(x)$ et en déduire que la fonction $g$ est croissante sur l'intervalle $[0~;~\alpha]$.
\end{enumerate}

\subsection*{Partie C: Approximation de $\boldsymbol{\alpha}$}

Soit la suite $\left(u_{n}\right)$ définie par :
\begin{center}$\left\{\begin{array}{@{}l@{}c@{}l@{}} u_{0} & \,\,=\,\, & 0 \\ u_{n+1} & = &   g\left(u_{n}\right)\end{array}\right.$.\end{center}
\begin{enumerate}
\item  Démontrer par récurrence que, pour tout $n\in\mathbf{N}$ :
\begin{center}$0 \leqslant  u_{n} \leqslant u_{n+1} \leqslant  \alpha$.\end{center}
\item  En déduire que la suite $\left(u_{n}\right)$ est convergente.
\item  On note $\ell$ la limite de la suite $(u_n)$.\par
Justifier que $g(\ell) = \ell$. En déduire la valeur  de $\ell$.
\item  À l'aide de la calculatrice, déterminer une valeur \mbox{approchée} de $\alpha$ arrondie à la  sixième décimale.
\end{enumerate}
