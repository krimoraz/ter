\documentclass[12pt,a4paper,french]{article}
\usepackage[utf8]{inputenc}
\usepackage[T1]{fontenc}
\usepackage{babel}
\usepackage[thmmarks]{ntheorem}
\usepackage{amsmath}
\usepackage{amsfonts}
\usepackage{amssymb}

\usepackage{array}

\usepackage{lmodern}
\usepackage{kpfonts}

\usepackage[bookmarks=false,colorlinks,linkcolor=blue,pdfusetitle]{hyperref}

\pdfminorversion 7
\pdfobjcompresslevel 3

\usepackage{tabularx}
\usepackage[autolanguage,np]{numprint}
\usepackage{enumitem}

\usepackage{tipfr}
\usepackage{pgf}
\usepackage{tikz}
\usepackage{tkz-euclide}
\usetkzobj{all}
\usetikzlibrary{hobby}
\usepackage{tkz-tab}

\usepackage[top=1.9cm,bottom=2cm,left=2cm,right=2cm]{geometry}

\usepackage{lastpage}

%\usepackage[gobble=auto]{pythontex}

%%%%%%%%%%%%%%%%%%%%%%%%%%%%%%%%%%%%%%%%%%%%%%%%%%%%%%%
\usepackage{fancybox,graphicx}
\usepackage{tabularx}
\usepackage{dcolumn}
%\usepackage{textcomp}
%\usepackage{diagbox}
\usepackage{tabularx}
\usepackage{lscape}
%\newcommand{\euro}{\eurologo{}}
%Tapuscrit : Denis Vergès
\usepackage{pstricks,pst-plot,pst-text,pst-tree,pstricks-add}
\newcommand{\Oij}{$(0,\vv{\imath},\vv{\jmath})$}
\newcommand{\Vect}{\vv}
\newcommand{\vect}{\vv}
%%%%%%%%%%%%%%%%%%%%%%%%%%%%%%%%%%%%%%%%%%%%%%%%%%%%%%%%%


\usepackage{esvect}
\usepackage{marginnote}

\usepackage{wrapfig}

\usepackage[defaultlines=5,all]{nowidow}


\usepackage[]{algorithm2e}

\usepackage{ifthen}
\usepackage{fancyhdr}
\pagestyle{fancy}


\makeatletter

\count1=\year \count2=\year
\ifnum\month<8\advance\count1by-1\else\advance\count2by1\fi

\setlength{\headheight}{14.5pt}
\renewcommand{\headrulewidth}{0pt}
\renewcommand{\footrulewidth}{0pt}
\cfoot{\textsl{\footnotesize{Année \number\count1/\number\count2}}}

\rfoot{%
  \ifthenelse{\value{page}=1}{%
  }
  {%
    \footnotesize{Page \thepage/ \pageref{LastPage}}
  }
}

\rhead{}

\lhead{%
  \ifthenelse{\value{page}=1}{%
    Nom:\dotfill\hfill Prénom: \dotfill \hfill
    Classe: \@author \dots%
  }
  { }
}

\renewcommand{\maketitle}%
{\framebox{%
    \begin{minipage}{1.0\linewidth}%
      \begin{center}%
        \Large \@title ~-- \@author \\%
        \@date%
      \end{center}%
    \end{minipage}}%
  \normalsize%
  %\vspace{1cm}%
}

%Des macros pour les noms d'ensmbles
\newcommand{\R}{\mathbf{R}}
\newcommand{\Q}{\mathbf{Q}}
\newcommand{\Z}{\mathbf{Z}}
\newcommand{\C}{\mathbf{C}}
\newcommand{\N}{\mathbf{N}}

\newcommand{\norme}[1]{\left\lVert #1 \right\rVert}
\newcommand{\abs}[1]{\left\lvert #1 \right\rvert}

%Une macro récursive pour l'intérieru des vecteurs
%http://tex.stackexchange.com/questions/19693/arguments-of-custom-commands-as-comma-separated-list

\newcommand\vecteur[2][\\]{%
    \global\def\my@delim{#1}%
    \left(\negthinspace\begin{matrix}
        \my@vector #2,\relax\noexpand\@eolst%
    \end{matrix}\right)}

%Une macro pour les vecteurs
\def\my@vector #1,#2\@eolst{%
   \ifx\relax#2\relax
      #1
   \else
      #1\my@delim
      \my@vector #2\@eolst%
   \fi}

%Une macro récursive pour mettre formater l'intérieur des intervalles
\def\my@intervalle #1;#2\@eolst{%
  \ifx\relax#2\relax
    #1
  \else
    \my@intervalle #2\@eolst%
  \fi}

%Quatre macros pour les quatres types d'intervalles
\newcommand{\interff}[1]{%
  \left[\my@intervalle #1;\relax\noexpand\@eolst%
  \right]
}
\newcommand{\interfo}[1]{%
  \left[\my@intervalle #1;\relax\noexpand\@eolst%
  \right[}
\newcommand{\interof}[1]{%
  \left]\my@intervalle #1;\relax\noexpand\@eolst%
  \right]}
\newcommand{\interoo}[1]{%
  \left]\my@intervalle #1;\relax\noexpand\@eolst%
  \right[}

\makeatother


\usepackage{framed}

\theoremstyle{break}
\newtheorem{definition}{Définition}
\newtheorem{propriete}{Propriété}
\newtheorem{corollaire}{Corollaire}
\newtheorem{propdef}{Propriété - Définition}
\newtheorem{theoreme}{Théorème}
\theoremstyle{plain}
\theorembodyfont{\normalfont}
\newtheorem{exerciceT}{Exercice}
\theoremstyle{nonumberplain}
\newtheorem{remarque}{Remarque}
\newtheorem{notation}{Notation}
\newtheorem{probleme}{Problème}
\theoremsymbol{\ensuremath{\blacksquare}}
\newtheorem{preuve}{Preuve}
\theoremsymbol{}
\theoremstyle{nonumberbreak}
\newtheorem{exemple}{Exemple}

\newenvironment{exercice}{\begin{framed}\begin{exerciceT}}{\end{exerciceT}\end{framed}}

\setlength{\parsep}{0pt}
\setlength{\parskip}{5pt}
\setlength{\parindent}{0pt}
\setlength{\itemsep}{7pt}

\setlist{noitemsep}
%\setlist[1]{\labelindent=\parindent} % < Usually a good idea
\setlist[itemize]{leftmargin=*}
\setlist[itemize,1]{label=$\triangleright$}
\setlist[enumerate]{labelsep=*, leftmargin=1.5pc}
\setlist[enumerate,1]{label=\arabic*., ref=\arabic*}
\setlist[enumerate,2]{label=\emph{\alph*}),
ref=\theenumi.\emph{\alph*}}
\setlist[enumerate,3]{label=\roman*), ref=\theenumii.\roman*}
\setlist[description]{font=\sffamily\bfseries}

\usepackage{multicol}
\setlength{\columnseprule}{0pt}

\usepackage[]{exsheets}

\SetupExSheets{
  counter-format = {qu[1] :} ,
  headings = block-subtitle ,
  points/name = {pt/s} ,
  solution/pre-hook = \mdframed ,
  solution/post-hook = \endmdframed ,
  blank/style = dotted ,
}

 \DeclareInstance{exsheets-heading}{block-subtitle}{default}{
 indent-first = false,
 join = {
 title[r,B]number[l,B](.333em,0pt) ;
 title[r,B]subtitle[l,B](1em,0pt)
 } ,
 attach = {
 main[l,vc]title[l,vc](0pt,0pt) 
 } ,
}


\everymath{\displaystyle\everymath{}}

\title{Fiche exercice intégration}
\author{\bsc{Ts}}
\date{janvier 2018}

\begin{document}

\maketitle

\begin{question}
  Justifier que les fonctions dont les expressions suivent sont
  intégrables sur $I$.
  \begin{enumerate}[label=\textbf{\alph*)}]
    \item $\cos x$, $I = \interff{0;\pi}$
    \item $x^2 - 3x + 1$, $I = \interff{0;1}$
    \item $\frac3{x^2}$, $I = \interff{1,\sqrt{2}}$
  \end{enumerate}
\end{question}

\begin{question}
  Proposer une primitive des fonctions dont les expressions suivent.
  \begin{enumerate}[label=\textbf{\alph*)}]
    \item $e^x$
    \item $x^2 - 3x + 1$
    \item $\frac3{x^2}$
  \end{enumerate}
\end{question}

\begin{question}
  Pour chacune des fonctions précédentes, donner la forme générale des
  primitives puis préciser la primitive qui s'annule en 1.
\end{question}

\begin{question}
La durée de vie, en années, d'un composant électronique fabriqué dans cette usine est une variable aléatoire $T$ qui suit la loi exponentielle de paramètre $\lambda$ (où $\lambda$ est un nombre réel strictement positif).

On note $f$ la fonction densité associée à la variable aléatoire $T$. On rappelle que :

\setlength\parindent{8mm}
\begin{itemize}
\item pour tout nombre réel $x \geqslant 0,\: f(x) = \lambda\text{e}^{-\lambda x}$.
\item pour tout nombre réel $a \geqslant 0,\: p(T \leqslant a) = \displaystyle\int_0^a f(x)\:\text{d}x$.
\end{itemize}
\setlength\parindent{0mm}

\medskip

\begin{enumerate}
\item  La courbe représentative $\mathcal{C}$ de la fonction $f$ est donnée ci-dessous.

\begin{center}
\psset{unit=1cm}
\begin{pspicture*}(-0.5,-0.5)(7.5,6.5)
\psgrid[gridlabels=0pt,subgriddiv=1,gridwidth=0.3pt,griddots=10](0,0)(8,7)
\psaxes[linewidth=1.25pt,Dx=20,Dy=20]{->}(0,0)(0,0)(7.5,6.5)[$x$,-110][$y$,200]
\psline(3.4,-0.1)(3.4,0.1)
\uput[d](3.4,0){$a$}
%\uput[d](7.35,0){$x$}\uput[l](0,6.35){$y$}
\uput[dl](0.6,3.5){\blue ${C}$}
\psplot[plotpoints=3000,linewidth=1.25pt,linecolor=blue]{0}{7.5}{5 2.71828 0.510826 x mul exp div}
\end{pspicture*}
\end{center}

	\begin{enumerate}
		\item Interpréter graphiquement $P(T \leqslant a)$ où $a > 0$.
		\item Montrer que pour tout nombre réel $t \geqslant 0 \::\: P(T \leqslant t) = 1 - \text{e}^{- \lambda t}$.
		\item En déduire que $\displaystyle\lim_{t \to + \infty} P(T \leqslant t) = 1$.
	\end{enumerate}
\item On suppose que $P(T \leqslant 7) = 0,5$. Déterminer $\lambda$ à $10^{-3}$ près.
\item Dans cette question on prend $\lambda = 0,099$ et on arrondit les résultats des probabilités au centième.
	\begin{enumerate}
		\item On choisit au hasard un composant fabriqué dans cette usine.
		
Déterminer la probabilité que ce composant fonctionne au moins 5 ans.
		\item On choisit au hasard un composant parmi ceux qui fonctionnent encore au bout de 2 ans.
		
Déterminer la probabilité que ce composant ait une durée de vie supérieure à 7 ans.
		\item Donner l'espérance mathématique E($T$) de la variable aléatoire $T$ à l'unité près.
		
Interpréter ce résultat.
	\end{enumerate}
\end{enumerate}
\end{question}

\begin{question}
\medskip

\parbox{0.52\linewidth}{\psset{unit=0.2cm}
\begin{pspicture}(-1.5,-1.5)(29,19)
\psaxes[linewidth=1.pt,labels=none,tickstyle=bottom]{->}(0,0)(29,19)
\psplot[plotpoints=5000,linewidth=1.25pt]{0}{20}{x 1 add ln x 1 add mul 3 x mul sub 7 add}
\rput(7.07,7.07){\psplot[plotpoints=5000,linewidth=1.25pt]{0}{20}{x 1 add ln x 1 add mul 3 x mul sub 7 add}}
\pspolygon[showpoints](20,0)(27.07,7.07)(27.07,18.005)(20,10.935)%DD'C'C
\psline[showpoints](0,7.07)(7.07,14.14)%BB'
\psline[showpoints,linestyle=dashed](0,0)(7.07,7.07)(27.07,7.07)
\psline[linestyle=dashed,showpoints](7.07,7.07)(7.07,14.14)
\uput[dl](0,0){O} \uput[ul](7.07,7.07){A} \uput[l](0,7.07){B} 
\uput[ul](7.07,14.14){B$'$} \uput[dr](20,10.935){C} \uput[dr](27.07,18.005){C$'$} 
\uput[d](20,0){D} \uput[dr](27.07,7.07){D$'$} \uput[d](1,0){I} 
\uput[l](0,1){J}
\end{pspicture} }\hfill 
\parbox{0.45\linewidth}{Une municipalité a décidé d'installer un module de skateboard dans un parc de la commune.

Le dessin ci-contre en fournit une perspective
cavalière. Les quadrilatères OAD$'$D, DD$'$C$'$C, et OAB$'$B sont des rectangles.

Le plan de face (OBD) est muni d'un repère orthonormé (O, I, J).

L'unité est le mètre. La largeur du module est de 10 mètres, autrement dit, DD$'$ = 10, sa
longueur OD est de 20~mètres.}
\bigskip

\textbf{Le but dit problème est de déterminer l'aire des différentes surfaces à peindre.}

\medskip

Le profil du module de skateboard a été modélisé à partir d'une photo par la fonction $f$ définie sur l'intervalle [0~;~20] par

\[f(x) = (x + 1)\ln (x + 1) - 3x + 7.\]

On note $f'$ la fonction dérivée de la fonction $f$ et $\mathcal{C}$ la courbe représentative de la fonction $f$ dans le repère (O, I, J).
\medskip

\parbox{0.52\linewidth}{\textbf{Partie 1} 

\begin{enumerate}
\item Montrer que pour tout réel $x$ appartenant à l'intervalle
[0~;~20], on a $f'(x) = \ln (x + 1) -2$.
\item En déduire les variations de $f$ sur l'intervalle [0 ; 20]
et dresser son tableau de variation.
\item  Calculer le coefficient directeur de la tangente à la courbe $\mathcal{C}$ au point d'abscisse $0$.

La valeur absolue de ce coefficient est appelée l'inclinaison du module de skateboard au point B.
 \end{enumerate}}
\hfill
\parbox{0.45\linewidth}{\psset{unit=0.222cm}
\begin{pspicture}(-1.5,-1.5)(23,13.5)
\psaxes[linewidth=1.25pt,labels=none,tickstyle=bottom]{->}(0,0)(23,13.5)
\psplot[plotpoints=5000,linewidth=1.25pt]{0}{20}{x 1 add ln x 1 add mul 3 x mul sub 7 add}
\uput[u](15,7){$\mathcal{C}$}\uput[d](20,0){D}\uput[l](0,7.07){B}\uput[dr](20,10.935){C}\uput[dl](0,0){O}\uput[d](1,0){I} 
\uput[l](0,1){J}  
\end{pspicture}}

\medskip
 
 
\textbf{4.} On admet que la fonction $g$ définie sur l'intervalle [0~;~20]  par

\[g(x) = \dfrac{1}{2}(x + 1)^2 \ln (x + 1) - \dfrac{1}{4}x^2 - \dfrac{1}{2}x\]

a pour dérivée la fonction   $g'$ définie sur l'intervalle
[0~;~20] par $g'(x) = (x + 1)\ln (x + 1)$.

Déterminer une primitive de la fonction $f$ sur l'intervalle [0~;~20].

\bigskip

\textbf{Partie 2}

\medskip

\emph{Les trois questions de cette partie sont indépendantes}

\medskip

\begin{enumerate}
\item Les propositions suivantes sont-elles exactes ? Justifier les réponses.

\setlength\parindent{9mm}
\begin{description}
\item[ ] P$_1$ : La différence de hauteur entre le point le plus haut et le point le plus bas de la piste est au moins égale à 8 mètres.
\item[ ] P$_2$ : L'inclinaison de la piste est presque deux fois plus grande en B qu'en C.
\end{description}
\setlength\parindent{0mm}

\item On souhaite recouvrir les quatre faces latérales de ce module d'une couche de peinture rouge. La peinture utilisée permet de couvrir une surface de 5 m$^2$ par litre.

Déterminer, à 1 litre près, le nombre minimum de litres de peinture nécessaires.

\medskip

\item~

\parbox{0.48\linewidth}{On souhaite peindre en noir la piste roulante, autrement dit la surface supérieure
du module.

Afin de déterminer une valeur approchée de l'aire de la partie à peindre, on considère
dans le repère (O, I, J) du plan de face, les points $B_k(k~;~f(k))$ pour $k$ variant de 0 à 20.

Ainsi, $B_0 =$ B.



}\hfill
\parbox{0.48\linewidth}{\psset{unit=0.18cm}
\begin{pspicture}(-1.5,-1.5)(29,19)
\psaxes[linewidth=1.25pt,labels=none,tickstyle=bottom]{->}(0,0)(29,19)
\psplot[plotpoints=5000,linewidth=1.25pt]{0}{20}{x 1 add ln x 1 add mul 3 x mul sub 7 add}
\rput(7.07,7.07){\psplot[plotpoints=5000,linewidth=1.25pt]{0}{20}{x 1 add ln x 1 add mul 3 x mul sub 7 add}}
\pspolygon(20,0)(27.07,7.07)(27.07,18.005)(20,10.935)%DD'C'C
\psline(0,7.07)(7.07,14.14)%BB'
\psline[linestyle=dashed](0,0)(7.07,7.07)(27.07,7.07)
\psline[linestyle=dashed](7.07,7.07)(7.07,14.14)
\uput[dl](0,0){\scriptsize O} \uput[ur](7.07,7.07){\scriptsize A} \uput[l](0,7.07){\scriptsize B} 
\uput[ul](7.07,14.14){\scriptsize B$'$} \uput[dr](20,10.935){\scriptsize C} \uput[dr](27.07,18.005){\scriptsize C$'$} 
\uput[d](20,0){\scriptsize D} \uput[dr](27.07,7.07){\scriptsize D$'$} \uput[d](1,0){\scriptsize I}
\psline[linestyle=dashed,linewidth=0.6pt](1,5.39)(8.07,12.46)\uput[d](1,5.68){\scriptsize $B_1$}\uput[ur](7.8,12.06){\scriptsize $B'_1$}
\psline[linestyle=dashed,linewidth=0.6pt](2,4.3)(9.07,11.37) \uput[d](2,4.54){\scriptsize $B_2$}\uput[ur](8.9,10.6){\scriptsize $B'_2$}
\psline[linestyle=dashed,linewidth=0.6pt](7,2.64)(14.07,9.71)\uput[dl](7.8,2.94){\scriptsize $B_k$}\uput[ul](14.37,9.71){\scriptsize $B'_k$} 
\psline[linestyle=dashed,linewidth=0.6pt](8,2.78)(15.07,9.85)\uput[d](9,3.33){\scriptsize $B_{k+1}$}\uput[u](15.9,9.85){\scriptsize $B'_{k+1}$}  
\uput[l](0,1){\scriptsize J}
\end{pspicture}}

\medskip

On décide d'approcher l'arc de la courbe $\mathcal{C}$
allant de $B_k$ à $B_{k+1}$ par le segment $\biggl[B_kB_{k+1}\biggr]$.

Ainsi l'aire de la surface à peindre sera  approchée par la somme des aires des
rectangles du type $B_k B_{k+1} B'_{k+1}B'_k$ (voir figure).
	\begin{enumerate}
		\item Montrer que pour tout entier $k$ variant de 0 à 19, 
		
		$B_kB_{k+1} = \displaystyle\sqrt{1 + \left (f(k + 1) - f(k)\right )^2}$.
		\item Compléter l'algorithme suivant pour qu'il affiche une estimation de l'aire de la partie roulante.
		
\begin{center}
\begin{tabularx}{0.9\linewidth}{|l|X|}\hline		
Variables 	&$S$ : réel\\
			&$K$ : entier\\
Fonction 	&$f$ : définie par $f(x) = (x + 1)\ln(x + 1)- 3x + 7$\\ \hline
Traitement	&$S$ prend pour valeur $0$\\
			&Pour $K$ variant de \ldots \:à\: \ldots\\
			&\hspace{1cm}$S$ prend pour valeur \:\ldots \ldots\\
			&Fin Pour\\ \hline
Sortie 		&Afficher \ldots\\ \hline
\end{tabularx}
\end{center}
	\end{enumerate}
\end{enumerate}

\end{question}

\begin{question}
\textbf{Partie A} 

\medskip

Dans le plan muni d'un repère orthonormé, on désigne par  $\mathcal{C}_1$ la courbe représentative de la fonction $f_1$  définie sur $\R$ par : 

\[f_1(x) = x +  \text{e}^{-x}.\]


\begin{enumerate}
\item  Justifier que $\mathcal{C}_1$ passe par le point A de coordonnées (0~;~1). 
\item  Déterminer le tableau de variation de la fonction $f_1$. On précisera les limites de $f_1$ en $+ \infty$ et en $- \infty$. 
\end{enumerate}

\bigskip

\textbf{Partie B}

\medskip

L’objet de cette partie est d'étudier la suite $\left(I_n\right)$ définie sur $\N$ par : 

\[I_n = \int_0^1 \left(x + \text{e}^{- nx}\right)\:\text{d}x.\] 


\begin{enumerate}
\item  Dans le plan muni d'un repère orthonormé \Oij , pour tout entier naturel $n$, on note 
$\mathcal{C}_n$ la courbe représentative de la fonction $f_n $ définie sur $\R$ par 

\[f_n(x) = x + \text{e}^{- nx}. \]

Sur le graphique ci-dessous on a tracé la courbe  $\mathcal{C}_n$ pour plusieurs valeurs de l'entier $n$ et la droite $\mathcal{D}$ d'équation $x = 1$. 

\begin{center}
\psset{unit=5cm}
\begin{pspicture*}(-0.3,-0.4)(1.3,1.4)
\psaxes[linewidth=1.5pt](0,0)(-0.3,-0.1)(1.4,1.4)
\psaxes[linewidth=1.5pt]{->}(0,0)(1,1)
\psline(1,0)(1,1.4)\uput[r](1,0.5){$\mathcal{D}$}
\uput[dl](0,0){O}\uput[dl](0,1){A}
%\multido{\n=1+1}{4}{\psplot[plotpoints=4000,linewidth=1.25pt]{-0.2}{1.3}{2.71828 x \n mul  neg exp x add}}
\psplot[plotpoints=4000,linewidth=1.25pt]{-0.4}{1.4}{2.71828 x   neg exp x add}
\psplot[plotpoints=4000,linewidth=1.25pt]{-0.4}{1.4}{2.71828 x 2 mul  neg exp x add}
\psplot[plotpoints=4000,linewidth=1.25pt]{-0.4}{1.4}{2.71828 x 3 mul  neg exp x add}
\psplot[plotpoints=4000,linewidth=1.25pt]{-0.4}{1.4}{2.71828 x 4 mul  neg exp x add}
\psplot[plotpoints=4000,linewidth=1.25pt]{-0.4}{1.4}{2.71828 x 6 mul  neg exp x add}
\psplot[plotpoints=4000,linewidth=1.25pt]{-0.4}{1.4}{2.71828 x 15 mul  neg exp x add}
\psplot[plotpoints=4000,linewidth=1.25pt]{-0.4}{1.4}{2.71828 x 60 mul  neg exp x add}
\uput[u](0.6,1.2){$\mathcal{C}_{1}$}\uput[u](0.6,0.9){$\mathcal{C}_{2}$}
\uput[u](0.5,0.7){$\mathcal{C}_{3}$}\uput[u](0.4,0.6){$\mathcal{C}_{4}$}
\uput[u](0.3,0.45){$\mathcal{C}_{6}$}\uput[u](0.2,0.25){$\mathcal{C}_{15}$}
\uput[u](0.1,0.15){$\mathcal{C}_{60}$}
\uput[d](0.5,0){$\vect{\imath}$}\uput[l](0,0.5){$\vect{\jmath}$}
\end{pspicture*}
\end{center}
	\begin{enumerate}
		\item Interpréter géométriquement l'intégrale $I_{n}$. 
		\item En utilisant cette interprétation, formuler une conjecture sur le sens de   variation de la suite $\left(I_n\right)$ et sa limite éventuelle. On précisera les éléments sur lesquels on s’appuie   pour conjecturer. 
	\end{enumerate}
\item Démontrer que pour tout entier naturel $n$ supérieur ou égal à 1, 

 
\[I_{n+1} - I_{n} = \int_{0}^1 \text{e}^{-(n + 1)x} \left(1 - \text{e}^{x}\right)\:\text{d}x.\] 
 

En déduire le signe de $I_{n+1} - I_{n}$ puis démontrer que la suite $\left(I_n\right)$ est convergente. 
\item Déterminer l'expression de $I_{n}$ en fonction de $n$ et déterminer la limite de la suite $\left(I_n\right)$. 
\end{enumerate}

\end{question}



\end{document}
