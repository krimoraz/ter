\documentclass[12pt,a4paper,french]{article}
\usepackage[utf8]{inputenc}
\usepackage[T1]{fontenc}
\usepackage[french]{babel}
\usepackage{ntheorem}
\usepackage{amsmath}
\usepackage{amsfonts}
\usepackage{amssymb}

\usepackage[top=1.2cm,bottom=1.2cm,left=2cm,right=2cm,includehead,
includefoot]{geometry}

\usepackage{lmodern}
\usepackage[upright]{kpfonts}
\usepackage{textcomp}
\newcommand{\euro}{\eurologo{}}
\frenchbsetup{og=«,fg=»}

\usepackage{ifthen}
\usepackage{fancyhdr}
\pagestyle{fancy}


\count1=\year \count2=\year
\ifnum\month<8\advance\count1by-1\else\advance\count2by1\fi
\pagestyle{fancy}
\cfoot{\textsl{\footnotesize{Année \number\count1/\number\count2}}}
\lfoot{\textsl{\footnotesize{Lycée \textsc{LaSalle Saint-Denis}}}}
\rfoot{\footnotesize{Page \thepage/ \pageref{LastPage}}}
\rhead{}
%\lhead{\ifthenelse{\value{page}=1}{Nom:\dotfill\hfill Prénom: \dotfill
%\hfill \no{\dots}}{}}
\renewcommand{\headrulewidth}{0pt}
\renewcommand{\footrulewidth}{0pt}


\usepackage[bookmarks=false,colorlinks,linkcolor=blue,pdfusetitle]{hyperref}

\pdfminorversion 7
\pdfobjcompresslevel 3

%\frenchbsetup{ItemLabels=\textbullet,}

\usepackage{tabularx}
\usepackage[inline]{enumitem}

\usepackage[autolanguage,np]{numprint}

\usepackage{tipfr}
\usepackage{pgf}
\usepackage{tikz}
\usepackage{tkz-euclide}
\usetkzobj{all}
\usetikzlibrary{hobby}
\usepackage{tkz-tab}
\usepackage{lastpage}
\usepackage{esvect}
\usepackage{delimset}

\usepackage{fancybox}


\usepackage{pdfmarginpar}

\makeatletter
\renewcommand{\maketitle}%
{\framebox{%
    \begin{minipage}{0.98\linewidth}%
      \begin{center}%
        \Large \@title ~-- \@author \\%
        \@date%
      \end{center}%
  \end{minipage}}%
  \normalsize%
  %\vspace{1cm}%
}
\hypersetup{
  pdftitle={\scantokens\expandafter{\jobname\noexpand}},
  %pdfsubject={Modèle de document LaTeX},
  %pdfkeywords={LaTeX, modèle},
  pdfauthor={Vincent-Xavier Jumel}
}


\makeatother

\theoremstyle{break}
\newtheorem{definition}{Définition}
\theorembodyfont{\normalfont}
\theoremstyle{break}
\newtheorem{exercice}{Exercice}

%\theoremstyle{plain}
\theoremstyle{nonumberplain}
\newtheorem{probleme}{Problème}

\setlength{\parsep}{0pt}
\setlength{\parskip}{5pt}
\setlength{\parindent}{0pt}
\setlength{\itemsep}{7pt}

\setlist{noitemsep}
%\setlist[1]{\labelindent=\parindent} % < Usually a good idea
\setlist[itemize]{leftmargin=*}
\setlist[itemize,1]{label=$\triangleright$}
\setlist[enumerate]{labelsep=*, leftmargin=1.5pc}
\setlist[enumerate,1]{label=\arabic*., ref=\arabic*}
\setlist[enumerate,2]{label=\emph{\alph*}),
ref=\theenumi.\emph{\alph*}}
\setlist[enumerate,3]{label=\roman*), ref=\theenumii.\roman*}
\setlist[description]{font=\sffamily\bfseries}

\usepackage{multicol}
\setlength{\columnseprule}{0pt}

\everymath{\displaystyle\everymath{}}

\newcommand{\N}{\mathbf{N}}
\newcommand{\Z}{\mathbf{Z}}
\newcommand{\Q}{\mathbf{Q}}
\newcommand{\R}{\mathbf{R}}
\newcommand{\C}{\mathbf{C}}
\newcommand{\e}{\varepsilon}

\newcommand{\rep}[1]{\textcolor{red}{#1}}
\newcommand{\com}[1]{\textcolor{green!60!black}{#1}}

\newcommand{\ligne}[1]{%
  \begin{tikzpicture}[]
    \draw[white] (0,#1+0.8) -- (15,#1+0.8) ;
    \foreach \i in {1,...,#1}
    { \draw[dotted] (0,\i) -- (\linewidth,\i) ; }
    \draw[white] (0,0.6) -- (15,0.6) ;
  \end{tikzpicture}%
}

\usepackage{framed}
\newcommand{\maximaout}[3]{%
  \begin{framed}%
    \begin{minipage}{0.5\linewidth}
    \ttfamily \textcolor{red}{(\%i#1)} \textcolor{blue}{#2}\\
      \textcolor{brown}{(\%o#1)} #3 %
    \end{minipage}
  \end{framed}%
}

\usepackage{array,multirow,makecell}
\setcellgapes{1pt}
\makegapedcells
\newcolumntype{R}[1]{>{\raggedleft\arraybackslash }b{#1}}
\newcolumntype{L}[1]{>{\raggedright\arraybackslash }b{#1}}
\newcolumntype{C}[1]{>{\centering\arraybackslash }b{#1}}

\usepackage{alterqcm}
\usepackage{needspace}


\title{Activité : primitive de la fonction inverse}
\author{Terminale}
\date{janvier 2018}

\begin{document}
\maketitle

\bigskip

Le but de cette activité est de montrer, d'une autre façon que celle du
cours sur les logarithmes que \[ \int_1^t \frac1x \mathrm{d}x = \ln t
.\]

Pour cela, on étudie les propriétés de la primitive de la fonction $x
\mapsto \frac1x$ sur $\brk[s]{0;+\infty}$ après avoir justifié que la
fonction y était intégrable.

On démontre alors que la fonction ainsi définie possède les propriétés
des fonctions logarithmes, puis en étudiant le valeur en un point
particulier, on conclue.

Cette approche est d'ailleurs plus proche de l'histoire des
démonstrations puisque le logartihme a été étudié, historiquement
parlant, avant l'exponentielle et avant toute formalisation des
équations différentielles comme $f' = f$ qui définit actuellement
l'exponentielle.

\begin{center}
  \begin{tikzpicture}[>=latex,scale=1.96]
    \draw[->] (-1,0) -- (5,0) ;
    \draw[->] (0,-1) -- (0,5) ;
    \draw (0,0) node [below left]{$O$} ;
    \draw (-0.05,1) node [left] {1} -- (0.05,1) ;

    \draw [very thick] plot [smooth,domain = 0.2:5] (\x,1/\x) ;
    \draw (1,-0.05) node [below] {1} -- (1,0.05) ;
    \draw (1.3,-0.05) node [below] {a} -- (1.3,0.05) ;
    \draw (2.7,-0.05) node [below] {b} -- (2.7,0.05) ;
    \draw (3.51,-0.05) node [below] {ab} -- (3.51,0.05) ;
  \end{tikzpicture}
\end{center}

On veut donc montrer que, pour $a$ et $b$ deux réels supérieurs à 1, \[
\int_1^{ab} \frac1x \mathrm{d}x = \int_1^a \frac1x \mathrm{d}x +
\int_1^b \frac 1x \mathrm{d}x .\]

\begin{enumerate}
  \item Proposer une représentation graphique qui illustre cette
    égalité.
  \item Vérifier cette égalité avec la calculatrice sur quelques
    exemples numériques.
\end{enumerate}

\pagebreak

\begin{center}
  \begin{tikzpicture}[>=latex,scale=1.96]
    \draw[->] (-1,0) -- (5,0) ;
    \draw[->] (0,-1) -- (0,5) ;
    \draw (0,0) node [below left]{$O$} ;
    \draw (-0.05,1) node [left] {1} -- (0.05,1) ;

    \draw [very thick] plot [smooth,domain = 0.2:5] (\x,1/\x) ;
    \draw (1,-0.05) node [below] {1} -- (1,0.05) ;
    \draw (1.3,-0.05) node [below] {$a$} -- (1.3,0.05) ;
    \draw (2.7,-0.05) node [below] {$b$} -- (2.7,0.05) ;
    \draw (3.51,-0.05) node [below] {$ab$} -- (3.51,0.05) ;

    \draw (0.05,2.7) -- (-0.05,2.7) node [left] {$b$} ;
    \draw (0.37,0.05) -- (0.37,-0.05) node [below] {$\frac1b$} ;
    \draw [very thick] plot [domain=0:4] (\x,\x) ;
    \draw [dashed] (1,0) -- (1,1) -- (0,1) ;
    \draw [dashed] (0.37,0) -- (0.37,2.7) ;
    \draw [dashed] (0,2.7) -- (0.37,2.7) ;
  \end{tikzpicture}
\end{center}

On convient de noter $\frac1x \mathrm{d}x $ sous la forme
$\frac{\mathrm{d}x}x$.

On appelle «$\mathrm{d}$» l'opérateur différentiel et $\mathrm{d}\lambda x
= \lambda \mathrm{d}x$ pour tout $\lambda$ réel.

\begin{enumerate}[resume]
  \item Si on pose $x = bt$, que vaut $\mathrm{d}x$ en fonction de
    $\mathrm{d}t$.

    De même, pour $x = 1$ et $x = ab$, que vaut $t$ ?
  \item Exprimer alors $\int_1^{ab} \frac{\mathrm{d}x}x$ comme une
    intégrale en $t$.
  \item Justifier que $\int_{\frac1b}^a \frac{\mathrm{d}t}t =
    \int_{\frac1b}^1 \frac{\mathrm{d}t}t + \int_1^a
    \frac{\mathrm{d}t}t$.
  \item \begin{enumerate}
      \item Proposer une justification géométrique permettant de
        transformer $\int_{\frac1b}^1 \frac{\mathrm{d}t}t$.
      \item On pose $t = \frac1y$. Que vaut $\mathrm{d}y$.

        Que deviennes les bornes ?
      \item Écrire les calculs qui permettent de justifier que
        $\int_{\frac1b}^1 \frac{\mathrm{d}t}t = \int_1^b
        \frac{\mathrm{d}y}y $
    \end{enumerate}
  \item Que faut $\int_1^1 \frac{\mathrm{d}x}x$ ?
  \item \begin{enumerate}
      \item Conclure pour les réels strictement supérieurs à 1.
      \item Proposer une transformation pour démontrer cette égalité si
        $a$ ou $b$ est compris strictement entre 0 et 1.
      \item Conclure.
    \end{enumerate}
  \item En déduire la dérivée de la fonction $\ln$ (citer un théorème.)
\end{enumerate}

\end{document}
