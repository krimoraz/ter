\documentclass[12pt,french]{beamer}

\usepackage[utf8]{inputenc}
\usepackage[T1]{fontenc}
\usepackage{lmodern}
\usepackage{amsmath,amsfonts,amssymb}
\usepackage{enumitem}

\usepackage{babel}
\usepackage[lastexercise]{exercise}

\renewcommand{\ExerciseName}{Exercice}
\renewcommand{\AnswerName}{Réponse de l'exercice}

\newcommand{\N}{\mathbf{N}}

\everymath{\displaystyle\everymath{}}

\title{Correction exercices}
\date{13 octobre 2015}

\mode<presentation> { \usetheme{boxes} }
\beamerdefaultoverlayspecification{<+->}

\begin{document}

\begin{frame}
  \maketitle
\end{frame}

\begin{frame}
  \begin{block}{Question 1}
    \begin{enumerate}[label=\alph*)]
      \item \begin{itemize}
          \item «Toutes les boules tirées sont de la même couleur» :
            $\frac2{2^n} = \frac{1}{2^{n-1}}$
          \item «On obtient exactement une boule blanche» :
            $n\times\frac1{2^n}$
        \end{itemize}
      \item \begin{itemize}
          \item $A$ contraire du premier $\implies P(A) = 1 -
            \frac1{2^{n-1}}$
          \item $A\cap B$ = «On obtient exactement une boule blanche»
          \item $P(B) = P(A\cap B) + \frac1{2^n}$
        \end{itemize}
    \end{enumerate}
  \end{block}
\end{frame}

\begin{frame}
  \begin{block}{Question 2}
    \begin{itemize}[label=$\iff$]
      \item $A$ et $B$ indépendants
      \item $P(A)\times P(B) = P(A\cap B)$
      \item $\left(1 - \frac1{2^{n-1}}\right)\times \frac{n+1}{2^n} =
        \frac{n}{2^n}$
      \item $\left(1 - \frac1{2^{n-1}}\right)(n+1) = n$
      \item $\left(2^{n-1} - 1\right)(n+1) = n\times 2^{n-1}$
      \item $n\times 2^{n-1} + 2^{n-1} - (n+1) = n\times 2^{n-1}$
      \item $2^{n-1} = n+1$
    \end{itemize}
  \end{block}
\end{frame}

\begin{frame}
  \begin{block}{Question 3}
    \begin{enumerate}[label=\alph*)]
      \item $u_2 = -1 ; u_3 = 0 ; u_4 = 3$
      \item $2(2^{n-1} -1) \geqslant 2^{n-1} -1 \implies 2^n -2
        \geqslant 2^{n-1} -1 \implies 2^n - (n+2) \geqslant 2^{n-1} -
        (n+1) \implies u_{n+1} \geqslant u_n$
    \end{enumerate}
  \end{block}
  \begin{block}{Question 4}
    $u_3 = 0 \implies 2^{n-1} = n+1$ pour $n=3$, ce qui donne
    l'indépendance pour $n = 3$.
  \end{block}
\end{frame}

\end{document}
