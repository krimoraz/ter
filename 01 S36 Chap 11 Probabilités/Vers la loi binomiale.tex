\documentclass[a4ppaer,12pt,french]{article}

\usepackage[utf8]{inputenc}
\usepackage[T1]{fontenc}
\usepackage{lmodern}
\usepackage{kpfonts}

\usepackage[a4paper,margin=2cm]{geometry}
\usepackage{amsmath,amssymb,amsfonts,amsthm}
\usepackage{mdframed}

\usepackage{babel}

\title{Vers la loi binomiale}
\author{\bsc{Jumel}}
\date{19 septembre}

\makeatletter

\renewcommand{\maketitle}{%
  \begin{mdframed}\begin{center}\Large \@title ~-- \@author \\ \@date
  \normalsize \end{center} \end{mdframed}
}
\newcommand{\R}{\mathbf{R}}

\makeatother

\newtheorem{definition}{Définition}
\newtheorem{exemple}{Exemple}
\newtheorem{exercice}{Exercice}

\everymath{\displaystyle}

\parindent0pt

\begin{document}

\maketitle

\section*{Introduction}

On considère dans ce cours des expériences aléatoires dont le nombre de
résultat possible est «relativement grand» tout en restant fini. Comme
il est illusoire de préciser \emph{in extenso} la totalité des
événements, on va chercher à caractériser les différentes expériences
aléatoires et leurs probabilités associées.

\section{Variable aléatoire discrète}

\subsection{Généralités}

\begin{definition}[Variable aléatoire discrète]
  Une variable aléatoire discrète est une \emph{fonction} d'un univers de
  probabilité $\Omega$ dans un ensemble fini, inclus dans $\R$
\end{definition}

Dans le cas d'un univers «simple», on peut donner en extension cette
fonction.
\begin{exemple}
  Pour l'expérience aléatoire «jeter un dé à 6 faces, on considère
  généralement la variable aléatoire qui à chaque réalisation de
  l'expérience aléatoire associe la valeur du dé.
\end{exemple}

On note «la variable aléatoire $X$ prend la valeur $k$» par $X = k$.

Il n'est cependant pas rare de composer la variable aléatoire avec une
autre fonction.

\begin{mdframed}
  \begin{exemple}
    On considère l'expérience aléatoire suivante : «on lance un dé à 6
    faces. Si le résultat est premier, le lanceur gagne un euro. Si le
    résultat est un quatre, le lanceur gagne deux euros. Dans les autres
    cas, il perd quatre euros.

    \emph{Quelles sont les valeurs prises par la variable aléatoire ?}
    \\[1.6cm]
  \end{exemple}
\end{mdframed}

\begin{definition}[fonction de probabilité]
  On appelle \emph{fonction de probabilité} la fonction qui à chaque
  valeur prise par la variable aléatoire associe sa probabilité.
\end{definition}

\begin{mdframed}
  \begin{exemple}
    \emph{Préciser sous forme de tableau la fonction de probabilité de
    l'exemple précédent.}\\[1.6cm]
  \end{exemple}
\end{mdframed}

\subsection{Quantités caractéristiques}

\begin{definition}[espérance]
  On appelle \emph{espérance} la moyenne pondérée par les probabilités
  des réalisation d'une variable aléatoire. On la note $\mathbf{E}[X]$
  et elle vaut $\sum_{k\in X(\Omega)} p(X=k)\times k$
\end{definition}

\begin{mdframed}
  \begin{exemple}
    \emph{Calculer l'espérance dans le cas précédent}\\[1.6cm]
    \emph{Interpréter le résultat en terme de gain moyen}\\[1.6cm]
  \end{exemple}
\end{mdframed}

\section{Lois discrètes de probabilité}

Jusqu'à présent nous avons étudié des univers de probabilité de «taille»
relativement petite, pour lesquelles une étude exhaustive était possible.
On se pose la question de modéliser le comportement d'une probabilité
sur un ensemble plus grand où l'exploration exhaustive de tous les cas
de figure n'est pas envisageable. Pour cela, on va chercher à dégager un
modèle de comportement que nous appellerons loi de probabilité.

Étudions ainsi une première loi de probabilité que nous qualifierons de
discrète, car le nombre de réalisation de l'expérience aléatoire est
fini. On verra plus tard dans l'année des lois de probabilité qualifiées
de continues, notion qui sera explicité plus tard dans l'année.

\subsection{Nombre de combinaisons}

\begin{mdframed}
  \begin{exercice}
    \begin{enumerate}
      \item \emph{On lance $n$ fois une pièce équilibrée. Dessiner
        l'arbre de probabilité dans le cas $n = 4$.}\\[4.8cm]
      \item \emph{Combien de séries de lancer on permis d'obtenir
        exactement une fois «pile» ?}\\[1.6cm]
      \item \emph{Même question avec 0, 2, 3 et 4 fois «pile».}\\[1.6cm]
    \end{enumerate}
  \end{exercice}
\end{mdframed}

\begin{mdframed}
  \begin{exercice}
    On considère un jeu de 32 cartes dans lequel on distingue les 4
    «couleurs» usuelles --- pique et trèfle pour les noirs et carreau et
    cœur pour les rouges.
    \begin{enumerate}
      \item Combien y a-t-il de façon de choisir deux couleurs parmi les
        4 disponibles ?
      \item On se limite aux figures --- roi, dame, valet. On tire 3
        cartes au hasard. Quelle est la probabilité de choisir 2 rois ?
        3 rois ?
        \vspace{3.2cm}
    \end{enumerate}
  \end{exercice}
\end{mdframed}

Les deux exercices précédent ont permis de dégager la notion de
nombre de combinaisons.

\begin{definition}[nombre de combinaison]
  On appelle \emph{nombre de combinaison} de $k$ parmi $n$ le nombre de
  chemins permettant d'obtenir $k$ réalisations d'une des deux issues
  possibles d'un arbre de probabilité binaire à $n$ répétition. On note
  ce nombre $\binom{n}{k}$.
\end{definition}

En pratique, ce nombre se calcule avec une machine. On peut cependant
noter que pour tout $n$ et pour tout $k$ entiers naturels,
$\binom{n}{k} = \frac{n!}{k!(n-k)!}$.

Un devoir à la maison permettra d'explorer différentes propriétés autour
du nombre de combinaisons.

\subsection{Loi binomiale}

\begin{mdframed}
  \begin{exercice}
    On considère une pièce déséquilibrée --- elle tombe 9 fois sur 10
    sur «face».
    \begin{enumerate}
      \item On effectue deux lancers successifs. Quelle est la
        probabilité d'obtenir une fois «pile» ? deux fois «pile»
        ?\\[2.4cm]
      \item On effectue 5 lancers successifs. Quelle est la probabilité
        d'obtenir une fois «pile» ? deux fois «pile» ?\\[3.2cm]
      \item \begin{enumerate}
          \item Proposer une expression permettant de calculer la
            probabilité d'obtenir deux fois pile dans un arbre à $n$
            répétitions.\\[1.6cm]
          \item Proposer une expression permettant de calculer la
            probabilité d'obtenir $k$ fois pile dans un arbre à 5
            répétitions.\\[1.6cm]
          \item Proposer une expression permettant de calculer la
            probabilité d'obtenir $k$ fois pile dans un arbre à $n$
            répétitions.\\[1.6cm]
      \end{enumerate}
    \end{enumerate}
  \end{exercice}
\end{mdframed}

On peut réaliser une simulation numérique avec Python.

\begin{exercice}
  On considère le code Python suivant :
  \begin{verbatim}
  from random import randint
  def pf():
      if randint(1,10) > 9:
          return 0
      else:
          return 1

  tirages = [pf() for i in range(100)]
  print(tirages.count(1))
  \end{verbatim}

  Que permet de simuler le code ci-dessus ?

  \vspace{1.6cm}
  Écrire une boucle qui permet de répéter un grand nombre de fois cette
  expérience aléatoire. Vérifier la formule proposée à l'exercice
  précédent.
\end{exercice}

\end{document}
