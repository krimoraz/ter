\documentclass[12pt,a4paper,french]{article}
\usepackage[utf8]{inputenc}
\usepackage[T1]{fontenc}
\usepackage[french]{babel}
\usepackage{ntheorem}
\usepackage{amsmath}
\usepackage{amsfonts}
\usepackage{amssymb}

\usepackage[top=1.4cm,bottom=1.4cm,left=2cm,right=2cm,includehead,
includefoot]{geometry}

\usepackage{lmodern}
\usepackage[upright]{kpfonts}
\usepackage{textcomp}
\newcommand{\euro}{\eurologo{}}
\frenchbsetup{og=«,fg=»}

\usepackage{ifthen}
\usepackage{fancyhdr}
\pagestyle{fancy}


\count1=\year \count2=\year
\ifnum\month<8\advance\count1by-1\else\advance\count2by1\fi
\pagestyle{fancy}
\cfoot{\textsl{\footnotesize{Année \number\count1/\number\count2}}}
\lfoot{\textsl{\footnotesize{Lycée \textsc{LaSalle Saint-Denis}}}}
\rfoot{\footnotesize{Page \thepage/ \pageref{LastPage}}}
\rhead{}
\lhead{\ifthenelse{\value{page}=1}{Nom:\dotfill\hfill Prénom: \dotfill
\hfill \no{\dots}}{}}
\renewcommand{\headrulewidth}{0pt}
\renewcommand{\footrulewidth}{0pt}


\usepackage[bookmarks=false,colorlinks,linkcolor=blue,pdfusetitle]{hyperref}

\pdfminorversion 7
\pdfobjcompresslevel 3

%\frenchbsetup{ItemLabels=\textbullet,}

\usepackage{tabularx}
\usepackage{enumitem}

\usepackage[autolanguage,np]{numprint}

\usepackage{tipfr}
\usepackage{pgf}
\usepackage{tikz}
\usepackage{tkz-euclide}
\usetkzobj{all}
\usetikzlibrary{hobby}
\usepackage{tkz-tab}
\usepackage{lastpage}

\usepackage{fancybox}


\usepackage{pdfmarginpar}

\makeatletter
\renewcommand{\maketitle}%
{\framebox{%
    \begin{minipage}{0.98\linewidth}%
      \begin{center}%
        \Large \@title ~-- \@author \\%
        \@date%
      \end{center}%
  \end{minipage}}%
  \normalsize%
  %\vspace{1cm}%
}
\hypersetup{
  pdftitle={\scantokens\expandafter{\jobname\noexpand}},
  %pdfsubject={Modèle de document LaTeX},
  %pdfkeywords={LaTeX, modèle},
  pdfauthor={Vincent-Xavier Jumel}
}


\makeatother

\theoremstyle{break}
\newtheorem{definition}{Définition}
\theorembodyfont{\normalfont}
\theoremstyle{break}
\newtheorem{exercice}{Exercice}

%\theoremstyle{plain}
\theoremstyle{nonumberplain}
\newtheorem{probleme}{Problème}

\setlength{\parsep}{0pt}
\setlength{\parskip}{5pt}
\setlength{\parindent}{0pt}
\setlength{\itemsep}{7pt}

\setlist{noitemsep}
%\setlist[1]{\labelindent=\parindent} % < Usually a good idea
\setlist[itemize]{leftmargin=*}
\setlist[itemize,1]{label=$\triangleright$}
\setlist[enumerate]{labelsep=*, leftmargin=1.5pc}
\setlist[enumerate,1]{label=\arabic*., ref=\arabic*}
\setlist[enumerate,2]{label=\emph{\alph*}),
ref=\theenumi.\emph{\alph*}}
\setlist[enumerate,3]{label=\roman*), ref=\theenumii.\roman*}
\setlist[description]{font=\sffamily\bfseries}

\usepackage{multicol}
\setlength{\columnseprule}{0pt}

\everymath{\displaystyle\everymath{}}

\newcommand{\N}{\mathbf{N}}
\newcommand{\Z}{\mathbf{Z}}
\newcommand{\Q}{\mathbf{Q}}
\newcommand{\R}{\mathbf{R}}
\newcommand{\C}{\mathbf{C}}
\newcommand{\e}{\varepsilon}

\newcommand{\rep}[1]{\textcolor{red}{#1}}
\newcommand{\com}[1]{\textcolor{green!60!black}{#1}}

\newcommand{\ligne}[1]{%
  \begin{tikzpicture}[]
    \draw[white] (0,#1+0.8) -- (15,#1+0.8) ;
    \foreach \i in {1,...,#1}
    { \draw[dotted] (0,\i) -- (\linewidth,\i) ; }
    \draw[white] (0,0.6) -- (15,0.6) ;
  \end{tikzpicture}%
}

\usepackage{framed}
\newcommand{\maximaout}[3]{%
  \begin{framed}%
    \begin{minipage}{0.5\linewidth}
    \ttfamily \textcolor{red}{(\%i#1)} \textcolor{blue}{#2}\\
      \textcolor{brown}{(\%o#1)} #3 %
    \end{minipage}
  \end{framed}%
}

\usepackage{array,multirow,makecell}
\setcellgapes{1pt}
\makegapedcells
\newcolumntype{R}[1]{>{\raggedleft\arraybackslash }b{#1}}
\newcolumntype{L}[1]{>{\raggedright\arraybackslash }b{#1}}
\newcolumntype{C}[1]{>{\centering\arraybackslash }b{#1}}

\usepackage{alterqcm}
\usepackage{needspace}


\title{Évaluation \no{3}}
\author{Terminale S 3}
\date{lundi 10 octobre 2016}

\begin{document}
\maketitle

\medskip

\begin{exercice}~\\[-5mm]
  Une chocolaterie fabrique deux sortes de chocolats: des chocolats
  noirs et des chocolats au lait.

  60\,\% des chocolats fabriqués sont noirs. Parmi ceux-ci, 70\,\% sont
  fourrés, tandis que 30\,\% seulement
  des chocolats au lait sont fourrés.

  \bigskip

  \textbf{Partie A}

  \medskip

  \textbf{Dans cette partie, pour chaque probabilité demandée, on
  donnera sa valeur exacte}

  \medskip

  Un client fait une dégustation de chocolats et il en choisit un au
  hasard.

  On considère les évènements suivants :
  \setlength\parindent{8mm}
  \begin{description}
    \item[ ]$N$ : \og le chocolat choisi est noir\fg
    \item[ ]$F$ : \og le chocolat choisi est fourré \fg. 
  \end{description}
  \setlength\parindent{0mm}
  \medskip

  \begin{enumerate}
    \item Donner la probabilité $P_1$ que le chocolat choisi soit noir.
    \item Déterminer la probabilité $P_2$ que le chocolat choisi soit
      noir et fourré.

      Justifier la réponse.
    \item On note $P_3$ la probabilité que le chocolat choisi soit
      fourré.

      Justifier que $P_3 = 0,54$.
  \end{enumerate}

  \bigskip

  \textbf{Partie B}

  \medskip

  Un client achète une boîte de $n$ chocolats, où $n$ est un entier
  naturel non nul.

  Chaque chocolat mis dans la boîte est choisi au hasard et on suppose
  le nombre de chocolats suffisamment grand pour que l'on puisse
  considérer que les choix successifs sont faits de façon identique et
  indépendante.

  On note $X_n$ la variable aléatoire représentant le nombre de
  chocolats fourrés contenus dans la boîte.

  \medskip

  \begin{enumerate}
    \item $X_n$ suit une loi binomiale dont on précisera les paramètres.
    \item \textbf{Dans cette question n = 12 et, pour chaque probabilité
        demandée, on
      donnera une valeur approchée à \boldmath $10^{-4}$\unboldmath\: près.}
      \begin{enumerate}
        \item Donner la probabilité $P_4$ que la moitié des chocolats de
          la boîte soient fourrés.
        \item Donner la probabilité $P_5$ que la boîte contienne au
          moins un chocolat fourré.
        \item Donner la probabilité $P_6$ que la boîte contienne au plus
          trois chocolats fourrés.
      \end{enumerate}
    \item \textbf{Dans cette question, $n$ est quelconque.}
      \begin{enumerate}
        \item Donner, en fonction de $n$, la probabilité $q_n$ que la
          boîte contienne au moins un chocolat fourré.
        \item Déterminer le nombre minimum $n_0$ de chocolats que doit
          acheter le client afin que la probabilité que la boîte
          contienne au moins un chocolat fourré soit strictement
          supérieure à $0,98$. Détailler les calculs.
      \end{enumerate}
  \end{enumerate}
  \ligne{24}

\end{exercice}


\end{document}
