\documentclass[12pt,a4paper,french]{article}

\input{../../commons.tex.inc}

\newcommand{\Z}{\mathbf{Z}}
\newcommand{\Q}{\mathbf{Q}}
\newcommand{\e}{\varepsilon}

\newcommand{\rep}[1]{\textcolor{red}{#1}}
\newcommand{\com}[1]{\textcolor{green!60!black}{#1}}


\title{Évaluation \no{2}}
\author{Terminale S 2}
\date{lundi 25 septembre 2016}

\begin{document}
\maketitle
\thispagestyle{fancy}

\medskip

\begin{question}~\\[-5mm]
  Une chocolaterie fabrique deux sortes de chocolats: des chocolats
  noirs et des chocolats au lait.

  60\,\% des chocolats fabriqués sont noirs. Parmi ceux-ci, 70\,\% sont
  fourrés, tandis que 30\,\% seulement
  des chocolats au lait sont fourrés.

  \bigskip

  \textbf{Partie A}

  \medskip

  \textbf{Dans cette partie, pour chaque probabilité demandée, on
  donnera sa valeur exacte}

  \medskip

  Un client fait une dégustation de chocolats et il en choisit un au
  hasard.

  On considère les évènements suivants :
  \setlength\parindent{8mm}
  \begin{description}
    \item[ ]$N$ : \og le chocolat choisi est noir\fg
    \item[ ]$F$ : \og le chocolat choisi est fourré \fg. 
  \end{description}
  \setlength\parindent{0mm}
  \medskip

  \begin{enumerate}
    \item Donner la probabilité $P_1$ que le chocolat choisi soit noir.
    \item Déterminer la probabilité $P_2$ que le chocolat choisi soit
      noir et fourré.

      Justifier la réponse.
    \item On note $P_3$ la probabilité que le chocolat choisi soit
      fourré.

      Justifier que $P_3 = 0,54$.
  \end{enumerate}
\end{question}

%  \bigskip
%
%  \textbf{Partie B}
%
%  \medskip
%
%  Un client achète une boîte de $n$ chocolats, où $n$ est un entier
%  naturel non nul.
%
%  Chaque chocolat mis dans la boîte est choisi au hasard et on suppose
%  le nombre de chocolats suffisamment grand pour que l'on puisse
%  considérer que les choix successifs sont faits de façon identique et
%  indépendante.
%
%  On note $X_n$ la variable aléatoire représentant le nombre de
%  chocolats fourrés contenus dans la boîte.
%
%  \medskip
%
%  \begin{enumerate}
%    \item $X_n$ suit une loi binomiale dont on précisera les paramètres.
%    \item \textbf{Dans cette question n = 12 et, pour chaque probabilité
%        demandée, on
%      donnera une valeur approchée à \boldmath $10^{-4}$\unboldmath\: près.}
%      \begin{enumerate}
%        \item Donner la probabilité $P_4$ que la moitié des chocolats de
%          la boîte soient fourrés.
%        \item Donner la probabilité $P_5$ que la boîte contienne au
%          moins un chocolat fourré.
%        \item Donner la probabilité $P_6$ que la boîte contienne au plus
%          trois chocolats fourrés.
%      \end{enumerate}
%    \item \textbf{Dans cette question, $n$ est quelconque.}
%      \begin{enumerate}
%        \item Donner, en fonction de $n$, la probabilité $q_n$ que la
%          boîte contienne au moins un chocolat fourré.
%        \item Déterminer le nombre minimum $n_0$ de chocolats que doit
%          acheter le client afin que la probabilité que la boîte
%          contienne au moins un chocolat fourré soit strictement
%          supérieure à $0,98$. Détailler les calculs.
%      \end{enumerate}
%  \end{enumerate}
  \ligne{24}



\end{document}
