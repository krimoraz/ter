% vim: set ft=tex:
\documentclass[12pt,french,a4paper]{article}

\input{../commons.tex.inc}

\title{Suites et probabilités : évaluation \no{2} }
\date{10 octobre 2017}
\author{}

\begin{document}

\maketitle
\thispagestyle{fancy}

\begin{center}
  \tikz{
    \draw (0,0) rectangle (18,3) ;
    \draw (3,0) rectangle (15,3) ;
    \draw (3,0) node [anchor=south east] {\Huge /5} ;
    \draw (3,3) node [anchor=north west] { \large Appréciation } ;
    \draw (15,3) node [anchor=north west] { \large Signature } ;
  }

%  Pour certaines questions, plusieurs réponses peuvent être possible.
\end{center}

\begin{question}
  Un joueur débute un jeu vidéo et effectue plusieurs parties successives.
  On admet que :
  \begin{itemize}
    \item la probabilité qu'il gagne la première partie est de \np{0.1}. ;
    \item s'il gagne une partie, la probabilité de gagner la suivante est
      égale à \np{0,8} ;
    \item s'il perd une partie, la probabilité de gagner la suivant est
      égale à \np{0,6}
  \end{itemize}
  On note, pour tout entier naturel $n$ non nul :
  \begin{itemize}
    \item $G_n$ l'événement « le joueur gagne la $n$-ième partie » ;
    \item $p_n$ la probabilité de l'événement $G_n$.
  \end{itemize}
  On a donc $p_1 = \np{0,1}$.
  \begin{enumerate}
    \item Montrer que $p_2 = \np{0,62}$.
    \item Le joueur a gagné la deuxième partie. Calculer la probabilité
      qu'il ait perdu la première.
    \item Montrer que pour tout entier naturel $n$, non nul : \[ p_{n+1} =
      \frac15 p_n + \frac35 . \]
    \item Montrer par récurrence que, pour tout entier naturel $n$ non nul :
      $ p_n = \frac34 - \frac{13}4 \brk*{\frac15}^n$.
    \item Déterminer la limite de la suite $(p_n)$ quand $n$ tend vers
      $+\infty$.
  \end{enumerate}
\end{question}


\end{document}
