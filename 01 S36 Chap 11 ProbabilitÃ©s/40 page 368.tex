\documentclass[12pt,french]{beamer}

\usepackage[utf8]{inputenc}
\usepackage[T1]{fontenc}
\usepackage{lmodern}
\usepackage{amsmath,amsfonts,amssymb}
\usepackage{enumitem}

\usepackage{babel}
\usepackage[lastexercise]{exercise}

\renewcommand{\ExerciseName}{Exercice}
\renewcommand{\AnswerName}{Réponse de l'exercice}

\newcommand{\N}{\mathbf{N}}

\everymath{\displaystyle\everymath{}}

\title{Correction exercices}
\date{7 octobre 2015}

\mode<presentation> { \usetheme{boxes} }
\beamerdefaultoverlayspecification{<+->}

\begin{document}

\begin{frame}
  \maketitle
\end{frame}

\begin{frame}
  \begin{block}{Initialisation}
    $n = 2$ : triplet pythagoricien, donc égalité, donc vraie
  \end{block}
  \begin{block}{Hérédité}
    \begin{itemize}
      \item $5^n \geqslant 4^n + 3^n$
      \item $5^{n+1} = 5\times 5^n \geqslant 5\times (4^n + 3^n) =
        5\times 4^n + 5\times 3^n$
      \item $5\times 4^n \geqslant 4\times 4^n = 4^{n+1}$ et idem pour 3
      \item $5\times 4^n + 5\times 3^n \geqslant 4^{n+1} + 3^{n+1}$
      \item $P_n \implies P_{n+1}$
    \end{itemize}
  \end{block}
  \begin{block}{Conclusion}
    Par récurrence, on a montré que $\forall n\in\N,\ n\geqslant 2,\ 5^n
    \geqslant 4^n + 3^n$
  \end{block}
\end{frame}

\end{document}
