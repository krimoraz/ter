\documentclass[12pt,a4paper,french]{article}

\usepackage[utf8]{inputenc}
\usepackage[T1]{fontenc}
\usepackage{lmodern}
%\usepackage{fourier}

\usepackage[margin=1.4cm]{geometry}
\usepackage{titling}
\usepackage{amsmath,amsfonts,amssymb,amsthm}
\usepackage{tikz}
\usepackage{tkz-base}
\usetikzlibrary{intersections}
\usepackage{tabularx}
\usepackage{esvect}
\usepackage{delimset}

\usepackage{sesamanuel} % pour les commandes spéciales du manuel 2de
\usepackage{sesamanuelTIKZ} %  pour les commandes spéciales de la figure

%\usepackage{enumitem}
%\setlist{noitemsep}
%%\setlist[1]{\labelindent=\parindent} % < Usually a good idea
%\setlist[itemize]{leftmargin=*}
%\setlist[itemize,1]{label=$\triangleright$}
%\setlist[enumerate]{labelsep=*, leftmargin=1.5pc}
%\setlist[enumerate,1]{label=\textbf{\arabic*.}, ref=\arabic*}
%\setlist[enumerate,2]{label=\emph{\alph*}),
%ref=\theenumi.\emph{\alph*}}
%\setlist[enumerate,3]{label=\roman*), ref=\theenumii.\roman*}
%\setlist[description]{font=\sffamily\bfseries}

\usepackage{exsheets}
\usepackage{numprint}
\newcommand{\np}{\numprint}
\usepackage{tabularx}

\usepackage[pdfusetitle]{hyperref}

\usepackage{babel}

\newcommand{\N}{\mathbf{N}}
\newcommand{\norme}[1]{\left\lVert #1 \right\rVert}
%\newcommand{\abs}[1]{\left\lvert #1 \right\rvert}
\renewcommand{\u}{(u_n)_{n\in\N}}

\author{Vincent-Xavier \bsc{Jumel}}
\title{Terminale S, semaine 36 : Probabilités}
\date{septembre 2017}

\setlength{\parindent}{0pt}
%\everymath{\displaystyle}

\SetupExSheets{solution/print=false}

\newcommand{\arbre}[8]{%
  \begin{tikzpicture}[level distance=25mm,sibling distance=10mm]
    \node {} [grow=right]
      child[sibling distance=25mm] {
        node {$\overline{#1}$}
        child[sibling distance=15mm] {
          node {$\overline{#2}$}
          edge from parent node[below] { #8 }
        }
        child[sibling distance=15mm] {
          node {$#2$}
          edge from parent node[above] { #7 }
        }
        edge from parent node[below] { #4 }
      }
      child[sibling distance=25mm] { node {$#1$}
        child[sibling distance=15mm] {
          node {$\overline{#2}$}
          edge from parent node[below] { #6 }
        }
        child[sibling distance=15mm] {
             node {$#2$}
          edge from parent node[above] { #5 }
        }
        edge from parent node[above] { #3 }
      } ;
  \end{tikzpicture}
}


\begin{document}

\maketitle

\newpage

\section{Exercices d'entraînement}

\begin{question}On considère deux évènements $A$ et $B$ tels que $P(A)=\nombre{0,1}$ et $P(A\cap B)=\nombre{0,06}$.
  Calculer $P_A(B)$.
\end{question}


\begin{question}On considère deux évènements $C$ et $D$ tels que $P(D)=\nombre{0,6}$ et $P(C\cap \overline{D})=\nombre{0,35}$.
  Calculer $P_{\overline{D}}(C)$.
\end{question}

\begin{question}On considère deux évènements disjoints $E$ et $F$ de probabilités non nulles.
  Calculer $P_E(F)$.

\end{question}


\begin{question}On considère deux évènements $A$ et $B$ tels que $P(A)=\nombre{0,37}$, $P(B)=\nombre{0,68}$  et $P(A\cup B)=\nombre{0,84}$.
  Calculer :
  \begin{enumerate}
  \item $P_A(B)$
  \item $P_B(A)$
  \end{enumerate}
\end{question}


\begin{question}On considère deux évènements $A$ et $B$ tels que $P(A)=\nombre{0,63}$ et $P_A(B)=\nombre{0,06}$. Calculer :
  \begin{enumerate}
  \item $P(A\cap B)$
  \item $P\left(A\cap \overline{B}\right)$
  \end{enumerate}
\end{question}

\begin{question}
  \begin{enumerate}
    \item On considère deux évènements $R$ et $S$ tels que
      $P(R)=\dfrac{1}{4}$, $P_R(S)=\dfrac{5}{6}$ et
      $P_{\overline{R}}\left(\overline{S}\right)=\dfrac{11}{12}$.

      Construire un arbre pondéré avec ces évènements $R$ et $S$.
    \item Tao ne sait pas s'il lui reste de quoi préparer à manger dans son
      réfrigérateur.

      Il estime la probabilité que ce soit le cas à \nombre{0,8}.
      \begin{itemize}
        \item Dans ce cas (s'il a de quoi préparer à manger), il estime que
          la probabilité que le repas qu'il se préparera soit bon est de
          \nombre{0,65}.
        \item Sinon, il ira dans son restaurant favori dans lequel il estime
          que la probabilité que le repas servi soit bon est de
          \nombre{0,99}.
      \end{itemize}

      Construire un arbre pondéré représentant la situation après avoir
      explicité les notations des évènements apparaissant dans cet arbre.
  \end{enumerate}
\end{question}
\begin{solution}
  \begin{enumerate}
    \item ~~
      \begin{center}
        \arbre{R}{S}{$\frac14$}{$\frac34$}
        {$\frac56$}{$\frac16$}{$\frac1{12}$}{$\frac{11}{12}$}
      \end{center}
    \item On considère les évènements :
      \begin{itemize}
        \item $R$ : \og Tao a de quoi se préparer à manger dans son
          réfrigérateur \fg
        \item $B$ : \og Le repas est bon \fg\medskip
      \end{itemize}
      \begin{center}
        \arbre{R}{B}{\nombre{0.8}}{\nombre{0.2}}
        {\nombre{0.65}}{\nombre{0.35}}{\nombre{0.99}}{\nombre{0.01}}
      \end{center}
  \end{enumerate}
\end{solution}

\begin{question}
    Sur une chaîne de production d'un composant électronique, on effectue des tests qualité :
  \begin{itemize}
    \item Un premier examen visuel est effectué éliminant \upc{5} des composants, qui sont détruits.
    \item Les composants restants passent un test de fiabilité qui est réussi par \upc{90} des composants qui sont alors mis en vente.
    \item Parmi les composants n'ayant pas réussi le test de fiabilité, \upc{30} peuvent être réparés facilement et mis en vente, le reste est détruit.
  \end{itemize}
  On prélève un composant au hasard sur cette chaîne.
  \begin{enumerate}
    \item Représenter la situation par un arbre de probabilité.

      On notera $E$ l'événement \og le composant réussit l'examen
      visuel \fg{}, $F$ \og le composant réussit le test de fiabilité
      \fg{} et $V$ \og le composant est mis en vente \fg{}.
    \item Calculer $P\left(\overline{F}\cap V\right)$, $P(V)$ et
      $P_V\left(\overline{F}\right)$.
    \item Un composant :
        \begin{itemize}
        \item coûte \ueuro{0,05} s'il est détruit ;
        \item rapporte \ueuro{0.5} s'il est mis en vente sans
          réparation et \ueuro{0,25} s'il est mis en vente après
          réparation.
        \end{itemize}
        \vspace{-\baselineskip}\begin{enumerate}
        \item Donner la loi de probabilité de $X$, la variable
          aléatoire donnant la somme algébrique rapportée par un
          composant produit et éventuellement vendu.
        \item Combien d'argent peut-on \og espérer \fg{} gagner par
          composant ?
        \end{enumerate}
  \end{enumerate}
\end{question}

\subsection{Indépendance}

\begin{question}
  On considère deux évènements indépendants $A$ et $B$ tels que
  $P(A)=\nombre{0,15}$ et $P(A\cap B)=\nombre{0,085}$.

  Calculer $P(B)$.
\end{question}

\begin{question}
  On considère deux évènements indépendants $E$ et $F$ tels que
  $P\left(\overline{F}\right)=\nombre{0,53}$ et
  $P(E\cap F)=\nombre{0,25}$.

  Calculer $P(E)$.
\end{question}

\begin{question}
  On considère deux évènements indépendants $C$ et $D$ tels que
  $P(C\cup D)=\nombre{0,23}$ et $P(C)=\nombre{0,11}$.

  Calculer $P(D)$.
\end{question}

\begin{question}[subtitle={Indépendants et incompatibles ?}]
  Deux évènements disjoints de probabilités non nulles peuvent-ils
  être indépendants ?
\end{question}

\begin{question}
  Dans un magasin de meubles, il y a \upc{55} de canapés dont \upc{14} en
  cuir, \upc{30} de fauteuils dont \upc{20} en cuir et le reste est
  constitué de poufs dont \upc{42} en cuir.

  Un client se présente et choisit un meuble.

  On considère les évènements :
  \begin{itemize}
    \item $F$ : \og le meuble choisi est un fauteuil \fg{} ;
    \item $C$ : \og le meuble choisi est en cuir \fg.
  \end{itemize}

  Montrer que ces deux évènements sont indépendants.
\end{question}

\begin{question}
  Lily a dans sa poche deux pièces de 20 centimes, trois de 50 centimes et
  une de 1 euro.

  Elle tire successivement (sans remise) deux pièces de sa poche. Les
  évènements \og les deux pièces sont du même montant \fg{} et \og les deux
  pièces lui permettent d'acheter un croissant à 1 euro \fg{} sont-ils
  indépendants ?
\end{question}

\section{Problèmes}

\begin{question}[subtitle=Obstétrique--Question ouverte]
  D'après l'\og Enquête nationale prénatale \fg{} de 2010 réalisée par
  l'Inserm, la probabilité d'une grossesse donnant lieu à une naissance
  prématurée en France est de \upc{6,6} mais est accentuée par le fait que
  la grossesse soit multiple (jumeaux, triplés, etc) ou non.

  Plus précisément, cette probabilité est de \upc{41,7} en cas de grossesse
  multiple contre \upc{5,5} sinon.

  Déterminer la probabilité d'une grossesse multiple.
\end{question}

\begin{question}[subtitle={D'après Bac (Pondichéry - 2013)}]
  Dans une entreprise, on s'intéresse à la probabilité qu'un salarié soit
  absent durant une période d'épidémie de grippe.
  \begin{itemize}
    \item Un salarié malade est absent.
    \item La première semaine de travail, le salarié n'est pas malade.
    \item Si la semaine $n$ le salarié n'est pas malade, il tombe malade la
      semaine $n+1$ avec une probabilité égale à $\nombre{0,04}$.
    \item Si la semaine $n$ le salarié est malade, il reste malade la
      semaine $n+1$ avec une probabilité égale à $\nombre{0,24}$.
  \end{itemize}
  On désigne, pour tout entier naturel $n$ supérieur ou égal à 1, par $E_n$
  l'évènement \og le salarié est absent pour cause de maladie la $n$\ieme{}
  semaine \fg{}. On note $p_n$ la probabilité de l'évènement $E_n$.

  On a ainsi : $p_1=0$ et, pour tout entier naturel $n$ supérieur ou égal à
  1 : $0\leqslant p_n<1$.
  \begin{enumerate}
    \item
      \begin{enumerate}
        \item Déterminer la valeur de $p_3$ à l'aide d'un arbre de
          probabilité.
        \item Sachant que le salarié a été absent pour cause de maladie la
          troisième semaine, déterminer la probabilité qu'il ait été aussi
          absent pour cause de maladie la deuxième semaine.
      \end{enumerate}
    \item
      \begin{enumerate}
        \item Recopier sur la copie et compléter l'arbre de probabilité
          donné ci-dessous

          \begin{center}
            \arbre{E_n}{E_{n+1}}{$p_n$}{$\ldots$}
            {$\ldots$}{$\ldots$}{$\ldots$}{$\ldots$}
          \end{center}
        \item Montrer que, pour tout entier naturel $n$ supérieur ou égal à
          1, $p_{n+ 1}=\nombre{0,2}p_n+\nombre{0,04}$.
        \item Montrer que la suite $(u_n)$ définie pour tout entier naturel
          $n$ supérieur ou égal à 1 par $u_n=p_n-\nombre{0,05}$ est une
          suite géométrique dont on donnera le premier terme et la raison
          $r$.

          En déduire l'expression de $u_n$ puis de $p_n$ en fonction de $n$
          et $r$.
        \item En déduire la limite de la suite $(p_n)$.
        \item On admet dans cette question que la suite $(p_n)$ est
          croissante. On considère l'algorithme  suivant :

          \begin{center}
            \begin{oldalgorithme}
              \textbf{Variables}
              K et J sont des entiers naturels
              P est un nombre réel
              \textbf{Initialisation}
              P prend la valeur 0
              J prend la valeur 1
              \textbf{Entrée}
              Saisir la valeur de K
              \textbf{Traitement}
              Tant que P<\nombre{0,05}-10$^\textrm{-K}$
              P prend la valeur 0,2*P+0,04
              J prend la valeur J+1
              Fin Tant que
              \textbf{Sortie}
              Afficher J
            \end{oldalgorithme}
          \end{center}

          À quoi correspond l'affichage final J ?

          Pourquoi est-on sûr que cet algorithme s'arrête ?
    \end{enumerate}
  \end{enumerate}

\end{question}

\newpage

\centering{\bfseries \Large \thetitle{} -- solutions }

\bigskip

\printsolutions

\end{document}
