\documentclass[12pt,a4paper,french]{article}

\usepackage[utf8]{inputenc}
\usepackage[T1]{fontenc}
\usepackage{lmodern}
%\usepackage{fourier}

\usepackage[margin=1.4cm]{geometry}
\usepackage{titling}
\usepackage{amsmath,amsfonts,amssymb,amsthm}
\usepackage{tikz}
\usepackage{tkz-base}
\usetikzlibrary{intersections}
\usepackage{tabularx}
\usepackage{esvect}
\usepackage{delimset}

\usepackage{sesamanuel} % pour les commandes spéciales du manuel 2de
\usepackage{sesamanuelTIKZ} %  pour les commandes spéciales de la figure

%\usepackage{enumitem}
%\setlist{noitemsep}
%%\setlist[1]{\labelindent=\parindent} % < Usually a good idea
%\setlist[itemize]{leftmargin=*}
%\setlist[itemize,1]{label=$\triangleright$}
%\setlist[enumerate]{labelsep=*, leftmargin=1.5pc}
%\setlist[enumerate,1]{label=\textbf{\arabic*.}, ref=\arabic*}
%\setlist[enumerate,2]{label=\emph{\alph*}),
%ref=\theenumi.\emph{\alph*}}
%\setlist[enumerate,3]{label=\roman*), ref=\theenumii.\roman*}
%\setlist[description]{font=\sffamily\bfseries}

\usepackage{exsheets}
\usepackage{numprint}
\newcommand{\np}{\numprint}
\usepackage{tabularx}

\usepackage[pdfusetitle]{hyperref}

\usepackage{babel}

\newcommand{\N}{\mathbf{N}}
\newcommand{\norme}[1]{\left\lVert #1 \right\rVert}
%\newcommand{\abs}[1]{\left\lvert #1 \right\rvert}
\renewcommand{\u}{(u_n)_{n\in\N}}

\author{Vincent-Xavier \bsc{Jumel}}
\title{Terminale S, semaine 36 : Probabilités}
\date{septembre 2017}

\setlength{\parindent}{0pt}
\everymath{\displaystyle}

\SetupExSheets{solution/print=false}


\begin{document}

\maketitle

\newpage

\section{Exercices d'entraînement}

\begin{question}
  \begin{enumerate}
    \item On considère deux évènements $R$ et $S$ tels que
      $P(R)=\dfrac{1}{4}$, $P_R(S)=\dfrac{5}{6}$ et
      $P_{\overline{R}}\left(\overline{S}\right)=\dfrac{11}{12}$.

      Construire un arbre pondéré avec ces évènements $R$ et $S$.
    \item Tao ne sait pas s'il lui reste de quoi préparer à manger dans son
      réfrigérateur.

      Il estime la probabilité que ce soit le cas à \nombre{0,8}.
      \begin{itemize}
        \item Dans ce cas (s'il a de quoi préparer à manger), il estime que
          la probabilité que le repas qu'il se préparera soit bon est de
          \nombre{0,65}.
        \item Sinon, il ira dans son restaurant favori dans lequel il estime
          que la probabilité que le repas servi soit bon est de
          \nombre{0,99}.
      \end{itemize}

      Construire un arbre pondéré représentant la situation après avoir
      explicité les notations des évènements apparaissant dans cet arbre.
  \end{enumerate}
\end{question}
\begin{solution}
  \begin{enumerate}
    \item ~~
      \begin{center}
        \begin{tikzpicture}[level distance=25mm,sibling distance=10mm]
          \node {} [grow=right]
            child[sibling distance=25mm] {
              node {$\overline{R}$}
              child[sibling distance=15mm] {
                node {$\overline{S}$}
                edge from parent node[below] {$p_{\overline{R}}(\overline{S})$}
              }
              child[sibling distance=15mm] {
                node {$S$}
                edge from parent node[above] {$p_{\overline{R}}(S)$}
              }
              edge from parent node[below] { $p(\overline{R})$ }
            }
            child[sibling distance=25mm] { node {$R$}
              child[sibling distance=15mm] {
              node {$\overline{S}$}
                edge from parent node[below] { $p_R(\overline{S})$ }
              }
              child[sibling distance=15mm] {
                node {$S$}
                edge from parent node[above] { $p_R(S)$ }
              }
              edge from parent node[above] { $p(R) = \nombre{0.25}$ }
            } ;
          \end{tikzpicture}
        \end{center}
    \end{enumerate}
  \end{solution}


\section{Problèmes}

\centering{\bfseries \Large \thetitle{} -- solutions }

\bigskip

\printsolutions

\end{document}
