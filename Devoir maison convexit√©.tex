\documentclass[12pt,a4paper,french]{article}
\usepackage[utf8]{inputenc}
\usepackage[T1]{fontenc}
\usepackage{babel}
\usepackage[thmmarks]{ntheorem}
\usepackage{amsmath}
\usepackage{amsfonts}
\usepackage{amssymb}

\usepackage{array}

\usepackage{lmodern}
\usepackage{kpfonts}

\usepackage[bookmarks=false,colorlinks,linkcolor=blue]{hyperref}

\pdfminorversion 7
\pdfobjcompresslevel 3

\usepackage{tabularx}
\usepackage[autolanguage,np]{numprint}
\usepackage{enumitem}

\usepackage{tipfr}
\usepackage{pgf}
\usepackage{tikz}
\usepackage{tkz-euclide}
\usetkzobj{all}
%\usetikzlibrary{hobby}
\usepackage{tkz-tab}

\usepackage[top=1.7cm,bottom=2cm,left=2cm,right=2cm]{geometry}

\usepackage{lastpage}

\usepackage{esvect}
\usepackage{marginnote}

\usepackage{wrapfig}

\usepackage[defaultlines=5,all]{nowidow}


\makeatletter
\renewcommand{\@evenfoot}%
        {\hfil \upshape \small page {\thepage} de \pageref{LastPage}}
\renewcommand{\@oddfoot}{\@evenfoot}

\renewcommand{\maketitle}%
{\framebox{%
    \begin{minipage}{1.0\linewidth}%
      \begin{center}%
        \Large \@title ~-- \@author \\%
        \@date%
      \end{center}%
    \end{minipage}}%
  \normalsize%
  %\vspace{1cm}%
}

\pgfdeclarepatternformonly{mes_hachures}
{\pgfpoint{-0.1cm}{-0.1cm}}
{\pgfpoint{0.9cm}{0.5cm}}
{\pgfpoint{0.8cm}{0.4cm}}
{\pgfpathmoveto{\pgfpointorigin}
  \pgfpathlineto{\pgfpoint{0.8cm}{0.4cm}}
\pgfusepath{stroke}}

%Des macros pour les noms d'ensmbles
\newcommand{\R}{\mathbf{R}}
\newcommand{\Q}{\mathbf{Q}}
\newcommand{\Z}{\mathbf{Z}}
\newcommand{\C}{\mathbf{C}}
\newcommand{\N}{\mathbf{N}}

\newcommand{\norme}[1]{\left\lVert #1 \right\rVert}
\newcommand{\abs}[1]{\left\lvert #1 \right\rvert}

%Une macro récursive pour l'intérieru des vecteurs
%http://tex.stackexchange.com/questions/19693/arguments-of-custom-commands-as-comma-separated-list

\newcommand\vecteur[2][\\]{%
    \global\def\my@delim{#1}%
    \left(\negthinspace\begin{matrix}
        \my@vector #2,\relax\noexpand\@eolst%
    \end{matrix}\right)}

%Une macro pour les vecteurs
\def\my@vector #1,#2\@eolst{%
   \ifx\relax#2\relax
      #1
   \else
      #1\my@delim
      \my@vector #2\@eolst%
   \fi}

%Une macro récursive pour mettre formater l'intérieur des intervalles
\def\my@intervalle #1;#2\@eolst{%
  \ifx\relax#2\relax
    #1
  \else
    \my@intervalle #2\@eolst%
  \fi}

%Quatre macros pour les quatres types d'intervalles
\newcommand{\interff}[1]{%
  \left[\my@intervalle #1;\relax\noexpand\@eolst%
  \right]
}
\newcommand{\interfo}[1]{%
  \left[\my@intervalle #1;\relax\noexpand\@eolst%
  \right[}
\newcommand{\interof}[1]{%
  \left]\my@intervalle #1;\relax\noexpand\@eolst%
  \right]}
\newcommand{\interoo}[1]{%
  \left]\my@intervalle #1;\relax\noexpand\@eolst%
  \right[}

\makeatother


\usepackage{mdframed}

\theoremstyle{break}
\newtheorem{definition}{Définition}
\newtheorem{propriete}{Propriété}
\newtheorem{corollaire}{Corollaire}
\newtheorem{propdef}{Propriété - Définition}
\newtheorem{theoreme}{Théorème}
\theoremstyle{plain}
\theorembodyfont{\normalfont}
\newtheorem{exerciceT}{Exercice}
\theoremstyle{nonumberplain}
\newtheorem{remarque}{Remarque}
\newtheorem{notation}{Notation}
\newtheorem{probleme}{Problème}
\theoremsymbol{\ensuremath{\blacksquare}}
\newtheorem{preuve}{Preuve}
\theoremsymbol{}
\theoremstyle{nonumberbreak}
\newtheorem{exemple}{Exemple}

\newenvironment{exercice}{\begin{framed}\begin{exerciceT}}{\end{exerciceT}\end{framed}}

\setlength{\parsep}{0pt}
\setlength{\parskip}{5pt}
\setlength{\parindent}{0pt}
\setlength{\itemsep}{7pt}

\setlist{noitemsep}
%\setlist[1]{\labelindent=\parindent} % < Usually a good idea
\setlist[itemize]{leftmargin=*}
\setlist[itemize,1]{label=$\triangleright$}
\setlist[enumerate]{labelsep=*, leftmargin=1.5pc}
\setlist[enumerate,1]{label=\arabic*., ref=\arabic*}
\setlist[enumerate,2]{label=\emph{\alph*}),
ref=\theenumi.\emph{\alph*}}
\setlist[enumerate,3]{label=\roman*), ref=\theenumii.\roman*}
\setlist[description]{font=\sffamily\bfseries}

\usepackage{multicol}
\setlength{\columnseprule}{0pt}

\usepackage[]{exsheets}
\SetupExSheets{
  headings = block,
  question/pre-hook = \mdframed,
  question/post-hook = \endmdframed,
}

\everymath{\displaystyle\everymath{}}

\title{Devoir à la maison \no{3} : convexité, théorie et applications}
\author{\bsc{Ts 3}}
\date{pour le 30 janvier 2017}

\begin{document}

\maketitle

\section{Petit préambule géométrique}

\subsection{Approche qualitative}

\begin{definition}
  On dit qu'un domaine du plan est convexe lorsque le segment qui joint
  deux points arbitraires de ce domaine est contenu dedans.
\end{definition}

\begin{exemple}
  Parmi les parties convexes du plan, on peut citer :
  \begin{itemize}
    \item le segment ;
    \item l'intérieur d'un triangle ;
    \item le disque ;
    \item l'intérieur d'un quadrilatère non-croisé.
  \end{itemize}
\end{exemple}

\begin{question}[ID=reformulation]
  Reformuler cette définition avec les quantificateurs $\forall$ et
  $\exists$
\end{question}
\begin{solution}
  On dit qu'un domaine $A$ du plan est convexe lorsque $\forall (M,M')
  \in A^2,\ [MM']\subset A$.
\end{solution}

Le vocabulaire «ancien» des fonctions définissait la notion de graphe et
d'épigraphe d'une fonction. Cependant, nous pouvons donner la définition
suivante.
\begin{definition}
  \begin{itemize}
    \item On appelle graphe $\Gamma$ d'une fonction l'ensemble des
      points du plan $\Gamma = \lbrace (x,y) \mid y = f(x) \rbrace$.
    \item On appelle épigraphe d'une fonction l'ensemble des points du
      plans $\lbrace (x,y) \mid y \geqslant f(x)\rbrace$.
  \end{itemize}
\end{definition}

On dira donc qu'une fonction est convexe si son épigraphe est convexe.

\begin{question}[ID=convexe.parties]
  \begin{enumerate}
    \item Parmi les ensembles suivants, préciser s'ils sont convexes ou
      non.
      \begin{itemize}
        \item un carré ;
        \item un disque ;
        \item un segment ;
      \end{itemize}
    \item Proposer une figure géométrique dont l'intérieur n'est pas
      convexe. Illustrer par un dessin.
  \end{enumerate}
\end{question}
\begin{solution}
  \begin{enumerate}
    \item Les 3 propositions sont convexes.
    \item

      \tikz{
        \draw (0,0) -- (2,0) -- (2,2) -- (1,1) -- (0,2) -- cycle ;
      }
  \end{enumerate}
\end{solution}


\subsection{Décrire un segment : le barycentre}

On ne traitera qu'une vision simplifiée de la notion de barycentre ici,
en se limitant au plan et au barycentre de deux points. D'autres
démonstrations peuvent être menés avec cette notion.

Cependant, il est pertinent de donner ici la définition du barycentre
d'un système de points pondérés.

\begin{definition}
  On appelle \emph{barycentre} du système de points $(A_i, \alpha_i)$,
  avec $1 \leqslant i \leqslant n$ et $\sum \alpha_i \neq 0$ le point
  $G$ tel que \[ \sum_{i=1}^n \alpha_i \vv{GA_i} = \vv{0}.\]
\end{definition}

On peut remarquer que ce point est unique et que le système de
pondération des points permet, $n$ points étant donnés, de déterminer
tous les points du plan.

En particulier, on peut même réduire ce nombre de points à $n + 1$ pour
un espace de dimension $n$, ce qui nous donne deux points pour définir
un point d'une droite.

Plus précisément même encore, dans le cas d'un segment, on peut chercher
à raffiner les poids à considérer.

\begin{question}[ID=barycentre.segment]
  \begin{enumerate}
    \item Démontrer que si $x\in\interff{a;b}$, alors il existe
      $t\in\interff{0;1}$ tel que $x=f(t)$. \emph{On explicitera $f$}.
    \item Démontrer que $x$ s'écrit $x = (1-t)a + tb$.
    \item Donner une autre expression de $x$. \emph{Indication : on peut
      parcourir le segment «dans l'autre sens».}
  \end{enumerate}
\end{question}
\begin{solution}
  \begin{enumerate}
    \item Soit $x\in\interff{a;b}$.
      \begin{align*}
        a\leqslant x \leqslant b \iff \\
        0\leqslant x - a \leqslant b - a \iff \\
        0\leqslant \frac{x-a}{b-a} \leqslant 1.
      \end{align*}
    \item D'après ce qui précède, $t = \frac{x - a}{b - a}$
      \begin{align*}
        \iff t(b - a) = x - a \\
        \iff tb - ta + a = x\\
        \iff x = (1-t)a + tb.
      \end{align*}
  \end{enumerate}
\end{solution}

On dit que cette dernière expression est l'expression barycentrique de
$x$.

\begin{definition}
  On appelle barycentre de deux points $A$ et $B$ pondérés
  respectivement par $\alpha$ et $\beta$ le point $G$ tel que pour tout
  point $M$ du plan, \[\alpha \vv{MA} + \beta \vv{MB} = (\alpha + \beta)
  \vv{MG}.\]
\end{definition}

\begin{remarque}
  Cette définition ne possède un sens que si $\alpha + \beta \neq 0$.
\end{remarque}

\section{Fonctions convexes}

On va maintenant chercher à exploiter davantage la définition de la
convexité qu'on s'est donné, ainsi que l'outillage qui en découle.

\subsection{Caractérisation en utilisant les cordes}

Comencons par une petite définition de la notion de corde d'une courbe.

\begin{definition}
  On appelle corde d'une courbe le segment de coordonnées
  $(x_A,f(x_A))$, $(x_B,f(x_B))$.
\end{definition}

\begin{center}
  \begin{tikzpicture}[scale=0.7]
    \tkzInit
    \tkzGrid
    \tkzDrawXY

    \draw [thick] plot [smooth,domain=0:10] (\x,{(\x/2 - 2)^2 + 1/2}) ;
    \draw [thick] (1,{(1/2 - 2)^2 + 1/2}) node [circle,fill=black,inner
    sep=1pt] {} -- (8,{(8/2 - 2)^2 + 1/2}) node [circle,fill=black,inner
    sep=1pt] {} ;
  \end{tikzpicture}
\end{center}

\begin{question}[ID=inégalité.des.pentes]
  \begin{enumerate}
    \item Soit $M$ un point de la corde. Écrire ses coordonnées
      barycentriques.
    \item Donner l'image par la fonction $f$ d'un élément de même
      abscise que ce point de la corde.
    \item Quelle relation de comparaison peut-on écrire entre la
      valeur de la fonction et l'ordonnée d'un point de la corde, si
      celle-ci est convexe.
    \item On considère une fonction convexe sur le segment
      $\interff{a;c}$ et $b$ est un élément de $\interoo{a,c}$.
      \begin{enumerate}
        \item Exprimer les coordonnées barycentriques de $b$.
        \item Exprimer, en fonction de $b$, la valeur du paramètre.
        \item En utilisant l'inégalité de convexité \[ f(ta + (1-t)c)
            \leqslant tf(a) + (1-t)f(c) \] démontrer que \[\frac{f(c) -
          f(a)}{c-a} \leqslant \frac{f(c) - f(b)}{c-b}.\]
          \emph{On fera attention au signe des différences.}
      \end{enumerate}
    \item Démontrer, de la même façon que \[\frac{f(b) - f(a)}{b-a}
      \leqslant \frac{f(c) - f(a)}{c-a}.\]
      \emph{On pourra changer le paramétrage du segment $\interff{a;c}$.}
  \end{enumerate}
\end{question}
\begin{solution}
  \begin{enumerate}
    \item $M$ a pour coordonnées $(x = tx_A + (1-t)x_B ; y = tf(x_A) +
      (1-t)f(x_B))$.
    \item L'image par $f$ d'un point du segment est $f(tx_A +
      (1-t)x_B)$.
    \item Puisque la fonction est convexe, on a $f(tx_A + (1-t)x_B)
      \leqslant tf(x_A) + (1-t)f(x_B))$.
    \item Puisque la fonction est convexe sur le segment
      $\interff{a;b}$, on a $f(ta + (1-t)c) \leqslant f(a) + (1-t)f(c)$.
      \begin{enumerate}
        \item $b$ s'écrit $b = ta + (1-t)c$.
        \item $t = \frac{b - c}{a - c}$.
        \item $f(ta + (1-t)c) \leqslant tf(a) + (1-t)f(c)$ devient
          $f(b) \leqslant \frac{b - c}{a - c}\left(f(a) - f(c)\right) +
          f(c)$

          D'où on tire que $f(b) - f(c) \leqslant (b - c)\frac{f(a) -
          f(c)}{a-c}$.

          Comme $b-c<0$, on obtient $\frac{f(b) - f(c)}{b-c} \geqslant
          \frac{f(a) - f(c)}{a-c}$.

          \begin{center}
            \begin{tikzpicture}[scale=0.7]
              \tkzInit
              \tkzGrid
              \tkzDrawXY

              \draw [thick] plot [smooth,domain=0:10] (\x,{(\x/2 - 2)^2 + 1/2}) ;

              \draw [thick] (1,{(1/2 - 2)^2 + 1/2}) node [circle,fill=black,inner
              sep=1pt] {} -- (3,{(3/2 - 2)^2 + 1/2}) node [circle,fill=black,inner
              sep=1pt] {} ;
               \draw [thick] (3,{(3/2 - 2)^2 + 1/2}) node [circle,fill=black,inner
              sep=1pt] {} -- (8,{(8/2 - 2)^2 + 1/2}) node [circle,fill=black,inner
              sep=1pt] {} ;
               \draw [thick] (1,{(1/2 - 2)^2 + 1/2}) node [circle,fill=black,inner
              sep=1pt] {} -- (8,{(8/2 - 2)^2 + 1/2}) node [circle,fill=black,inner
              sep=1pt] {} ;

              \draw (1,0) node [below] {$a$} ;
              \draw (3,0) node [below] {$b$} ;
              \draw (8,0) node [below] {$c$} ;
            \end{tikzpicture}
          \end{center}

          L'inégalité précédente peut s'écrire $\frac{f(a) - f(c)}{a -
          c} \leqslant \frac{f(b) - f(c)}{b - c}$.
      \end{enumerate}
  \end{enumerate}
\end{solution}

\begin{propriete}[inégalité des trois pentes]
  Si $f$ est une fonction convexe sur un segment $\interff{a;c}$ et $b$
  un élément de l'ouvert $\interoo{a;b}$, alors elle vérifie l'inégalité
  suivante : \[ \frac{f(b) - f(a)}{b-a} \leqslant \frac{f(c) -
  f(a)}{c-a} \leqslant \frac{f(c) - f(b)}{c-b}.\]
\end{propriete}

\subsection{Caractérisation par les variations de la dérivée.}

On peut démontrer, mais c'est difficile, que si $f$ est convexe, alors,
$f$ est dérivable, donc continue. Admettons ce résultat et intéressons
nous à la dérivée.

On va démontrer dans cette partie qu'on peut être encore plus précis et
que la dérivée seconde, si elle existe est positive.

\begin{question}[ID=df.croissante]
  Démontrer, en utilisant l'égalité des pentes, que la fonction dérivée
  est strictement croissante.

  En déduire, que si $f$ est deux fois dérivable, alors $f''(x) > 0$.
\end{question}
\begin{solution}
  \begin{center}
    \begin{tikzpicture}[scale=1.4]
      \tkzInit[xmin=1,xmax=6,ymax=3]
      \tkzGrid
      \tkzDrawXY

      \draw [thick] plot [smooth,domain=1:5] (\x,{(\x/2 - 2)^2 + 1/2}) ;

      \draw [red,thick] (2,{(2/2 - 2)^2 + 1/2}) node
      [circle,fill=black,inner sep=1pt] {} -- (3,{(3/2 - 2)^2 + 1/2})
      node [circle,fill=black,inner sep=1pt] {} ;
      \draw [red,thick] (3,{(3/2 - 2)^2 + 1/2}) node
      [circle,fill=black,inner sep=1pt] {} -- (4,{(4/2 - 2)^2
      + 1/2}) node [circle,fill=black,inner sep=1pt] {} ;

      \draw (2,0) node [below] {$a$} ;
      \draw (3,0) node [below] {$b$} ;
      \draw (4,0) node [below] {$c$} ;
    \end{tikzpicture}
  \end{center}
\end{solution}


\subsection{Caractérisation par les tangentes}

Le but de cette partie est de démontrer que la courbe représentative
d'une fonction convexe est au dessus de ses tangentes.

\begin{question}[ID=tangentes]
  \begin{enumerate}
    \item Rappeller l'expression de l'équation de la tangente en un
      point.
    \item On pose $h(x) = \frac{f(x) - f(b)}{x -b}$
      \begin{enumerate}
        \item Justifier que $f$ convexe entraîne $h$ croissante.
        \item Dériver $h$. Que peut-on dire du signe du quotient obtenu
          ?
        \item En déduire que $f(b) \geqslant f(a) + (b -a)f'(a)$.
      \end{enumerate}
    \item Peut-on utiliser des équivalences dans le raisonnement
      précédent ? Conclure.
  \end{enumerate}
\end{question}
\begin{solution}
  \begin{enumerate}
    \item L'équation de la tangente $T$ est $y = f'(x_0) (x - x_0) +
      f(x_0)$.
      \begin{enumerate}
        \item Puisque $f$ est convexe, d'après l'inégalité des trois
          pentes, on a, pour tout $a < x \leqslant y < b$, \[ \frac{f(x)
          - f(y)}{x -y} \leqslant \frac{f(b) - f(x)}{b - x} \leqslant
          \frac{f(y) - f(b)}{y -b}. \] En particulier, on retrouve $h(x)
          \leqslant h(y)$.
        \item $h'(x) = \frac{f'(x)(x -b) - f(x) + f(b)}{(x - b)^2}$.

          Comme $h$ est croisstante, on en déduit que $f'(x)(x -b) -
          f(x) + f(b) \geqslant 0$.
        \item L'inégalité précédente entraîne que $f'(x)(x -b) - f(x)
          \geqslant -f(b) \implies f(b) \geqslant f'(x)(b - x) + f(x)$.
      \end{enumerate}
    \item Oui, on peut mener le raisonnement précédent par équivalence.
      En fait, il est valable pour tout $a < x < y < b$ et on trouve que
      $f(y) \geqslant f'(x)(t - x) + f(x)$, ce qui prouve bien que la
      tangente est sous la courbe.
  \end{enumerate}
\end{solution}

\section{Développements autour de cette notion}

\subsection{Concavité}

On dit qu'une fonction est concave si $-f$ est convexe.

\begin{question}[ID=concave]
  \begin{enumerate}
    \item En supposant que $f$ est deux fois dérivable, donner une
      caractérisation de $f$ est concave pour de telles fonctions.
    \item Donner un interprétation de la concavité en termes de
      \begin{enumerate}
        \item cordes ;
        \item tangentes.
      \end{enumerate}
  \end{enumerate}
\end{question}
\begin{solution}
  \begin{enumerate}
    \item On a de façon assez immédiate que $f'' < 0$
    \item On peut dire que
      \begin{enumerate}
        \item les cordes sont sous la fonction ;
        \item les tangentes sont au dessus de la fonction.
      \end{enumerate}
  \end{enumerate}
\end{solution}

\subsection{Tracés de courbes, étude des points d'inflexion d'une
courbe}

Une application intéressante de la convexité est le tracé des courbes.
En particulier, par comparaison locale avec un cercle, on peut se poser
la question de la convexité des fonctions de références, tout simplement
car il est plus facile de tracer une fonction lorsqu'on connaît sa
convexité.

\begin{question}[ID=convexite.reference]
  Pour les fonctions de références (affine, carrée, racine, inverse,
  cube, exponentielle, logarithme), préciser sur quels intervalles elles
  sont convexe ou concave. Vous justifierez votre réponse.
\end{question}
\begin{solution}
  \begin{itemize}
    \item Les fonctions affines sont convexes (utiliser la définition.)
    \item La fonction carrée est convexe, il suffit de prendre la
      dérivée seconde.
    \item La fonction racine est concave, car la dérivée seconde est
      négative.
    \item La fonction inverse est convexe sur $\interoo{0;+\infty}$,
      concave sur $\interoo{-\infty;0}$.
    \item La fonction exponentielle est convexe.
    \item La fonction logarithme est concave.
  \end{itemize}
\end{solution}

Sans rentrer dans les détails, nous avons vu que les inégalités qui
portaient sur la dérivée seconde était stricte. Si on se pose la
question de savoir ce qui se passe lorsque la dérivée seconde s'annule,
on peut observer par exemple le comportement de la fonction cube.

On parle alors de point d'inflexion : c'est un point où la tangente
«tarverse» la courbe.

\subsection{Équilibre en physique : stabilité, instabilité}

L'étude d'une situation physique, par des raisonnements de minimisation
d'énergie amène souvent à plusieurs solutions. Prenons un exemple
concret :

On cherche à placer une bille immobile dans une parabole. Pour cela, on
peut placer la parabole dans deux sens différents. Intuitivement, on se
doute la meilleure situation d'équilibre correspond à la situation où la
bille est placée dans la partie convexe de la parabole.

\begin{question}
  Pourquoi choisir une solution convexe dans un problème de physique ?
\end{question}
\begin{solution}
  Il s'agit d'une question relativement difficile telle que posée ainsi.
  En effet, l'idée générale, mais qui n'est pas forcément facile a
  démontrer est que si $f$ est convexe sur un convexe de $\R$, alors un
  minimum local de $f$ est en fait un minimum global.
  %https://who.rocq.inria.fr/Jean-Charles.Gilbert/ensta/04-co.pdf
  %https://who.rocq.inria.fr/Jean-Charles.Gilbert/ensta/optim.html
  %http://dumas.perso.math.cnrs.fr/LSMA651.html
\end{solution}

%\end{document}

\pagebreak
\setcounter{page}{1}
\printsolutions

\end{document}
