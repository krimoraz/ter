\documentclass[12pt,a4paper,french]{article}
\usepackage[utf8]{inputenc}
\usepackage[T1]{fontenc}
\usepackage{babel}
\usepackage[thmmarks]{ntheorem}
\usepackage{amsmath}
\usepackage{amsfonts}
\usepackage{amssymb}

\usepackage{array}

\usepackage{lmodern}
\usepackage{kpfonts}

\usepackage[bookmarks=false,colorlinks,linkcolor=blue]{hyperref}

\pdfminorversion 7
\pdfobjcompresslevel 3

\usepackage{tabularx}
\usepackage[autolanguage,np]{numprint}
\usepackage{enumitem}

\usepackage{tipfr}
\usepackage{pgf}
\usepackage{tikz}
\usepackage{tkz-euclide}
\usetkzobj{all}
\usetikzlibrary{hobby}
\usepackage{tkz-tab}

\usepackage[top=1.7cm,bottom=2cm,left=2cm,right=2cm]{geometry}

\usepackage{lastpage}

\usepackage{esvect}
\usepackage{marginnote}

\usepackage{wrapfig}

\usepackage[defaultlines=5,all]{nowidow}


\makeatletter
\renewcommand{\@evenfoot}%
        {\hfil \upshape \small page {\thepage} de \pageref{LastPage}}
\renewcommand{\@oddfoot}{\@evenfoot}

\renewcommand{\maketitle}%
{\framebox{%
    \begin{minipage}{1.0\linewidth}%
      \begin{center}%
        \Large \@title ~-- \@author \\%
        \@date%
      \end{center}%
    \end{minipage}}%
  \normalsize%
  %\vspace{1cm}%
}

\pgfdeclarepatternformonly{mes_hachures}
{\pgfpoint{-0.1cm}{-0.1cm}}
{\pgfpoint{0.9cm}{0.5cm}}
{\pgfpoint{0.8cm}{0.4cm}}
{\pgfpathmoveto{\pgfpointorigin}
  \pgfpathlineto{\pgfpoint{0.8cm}{0.4cm}}
\pgfusepath{stroke}}

%Des macros pour les noms d'ensmbles
\newcommand{\R}{\mathbf{R}}
\newcommand{\Q}{\mathbf{Q}}
\newcommand{\Z}{\mathbf{Z}}
\newcommand{\C}{\mathbf{C}}
\newcommand{\N}{\mathbf{N}}

\newcommand{\norme}[1]{\left\lVert #1 \right\rVert}
\newcommand{\abs}[1]{\left\lvert #1 \right\rvert}

%Une macro récursive pour l'intérieru des vecteurs
%http://tex.stackexchange.com/questions/19693/arguments-of-custom-commands-as-comma-separated-list

\newcommand\vecteur[2][\\]{%
    \global\def\my@delim{#1}%
    \left(\negthinspace\begin{matrix}
        \my@vector #2,\relax\noexpand\@eolst%
    \end{matrix}\right)}

%Une macro pour les vecteurs
\def\my@vector #1,#2\@eolst{%
   \ifx\relax#2\relax
      #1
   \else
      #1\my@delim
      \my@vector #2\@eolst%
   \fi}

%Une macro récursive pour mettre formater l'intérieur des intervalles
\def\my@intervalle #1;#2\@eolst{%
  \ifx\relax#2\relax
    #1
  \else
    \my@intervalle #2\@eolst%
  \fi}

%Quatre macros pour les quatres types d'intervalles
\newcommand{\interff}[1]{%
  \left[\my@intervalle #1;\relax\noexpand\@eolst%
  \right]
}
\newcommand{\interfo}[1]{%
  \left[\my@intervalle #1;\relax\noexpand\@eolst%
  \right[}
\newcommand{\interof}[1]{%
  \left]\my@intervalle #1;\relax\noexpand\@eolst%
  \right]}
\newcommand{\interoo}[1]{%
  \left]\my@intervalle #1;\relax\noexpand\@eolst%
  \right[}

\makeatother


\usepackage{framed}

\theoremstyle{break}
\newtheorem{definition}{Définition}
\newtheorem{propriete}{Propriété}
\newtheorem{corollaire}{Corollaire}
\newtheorem{propdef}{Propriété - Définition}
\newtheorem{theoreme}{Théorème}
\theoremstyle{plain}
\theorembodyfont{\normalfont}
\newtheorem{exerciceT}{Exercice}
\theoremstyle{nonumberplain}
\newtheorem{remarque}{Remarque}
\newtheorem{notation}{Notation}
\newtheorem{probleme}{Problème}
\theoremsymbol{\ensuremath{\blacksquare}}
\newtheorem{preuve}{Preuve}
\theoremsymbol{}
\theoremstyle{nonumberbreak}
\newtheorem{exemple}{Exemple}

\newenvironment{exercice}{\begin{framed}\begin{exerciceT}}{\end{exerciceT}\end{framed}}

\setlength{\parsep}{0pt}
\setlength{\parskip}{5pt}
\setlength{\parindent}{0pt}
\setlength{\itemsep}{7pt}

\setlist{noitemsep}
%\setlist[1]{\labelindent=\parindent} % < Usually a good idea
\setlist[itemize]{leftmargin=*}
\setlist[itemize,1]{label=$\triangleright$}
\setlist[enumerate]{labelsep=*, leftmargin=1.5pc}
\setlist[enumerate,1]{label=\arabic*., ref=\arabic*}
\setlist[enumerate,2]{label=\emph{\alph*}),
ref=\theenumi.\emph{\alph*}}
\setlist[enumerate,3]{label=\roman*), ref=\theenumii.\roman*}
\setlist[description]{font=\sffamily\bfseries}

\usepackage{multicol}
\setlength{\columnseprule}{0pt}

\usepackage[]{exsheets}
\SetupExSheets{headings=block}

\everymath{\displaystyle\everymath{}}

\title{Évaluation \no{6} : exponentielle}
\author{\bsc{Ts 3}}
\date{23 janvier 2016}

\begin{document}

\maketitle

\begin{tabular}{|p{6em}|p{26em}|p{6em}|}\hline
   & & \\
   & & \\
   \hfill\Huge /\totalpoints* & & \\
   & & \\
   & & \\ \hline
\end{tabular}


\begin{question}[ID=4p90,use=false]
  ~\\[-6ex]
  \phantom{a}\hfill\textbf{(\GetQuestionProperty{points}{4p90} points)}\\
  $f$ est la fonction définie sur $\R$ par : \[f(x) = \frac{e^{2x} -
  1}{2e^x}.\]
  \begin{enumerate}
    \item \begin{enumerate}
        \item Démontrez que pour tout nombre $x$ : \[f(x) = \frac{e^x -
          e^{-x}}2.\] \addpoints*{1}
        \item Calculez la limite de $f$ en $-\infty$ et $+\infty$.
          \addpoints*{1}
      \end{enumerate}
    \item \begin{enumerate}
        \item Étudiez le sens de variation de $f$. \addpoints*{1}
        \item Donner la valeur de $f$ et son nombre dérivé en 0. \addpoints*{1}
      \end{enumerate}
%    \item Pour tout nombre $x$, on pose \[d(x) = f(x) - x.\]
%      \begin{enumerate}
%        \item Vérifiez que pour tout nombre $x$, \[ d'(x) = \frac{(e^x
%          -1)^2}{2e^x}.\] \addpoints*{1}
%        \item Dressez le tableau de variation de $d$. \addpoints*{1}
%        \item Déduisez-en la position de $\mathscr{C}$ par rapport à la
%          tangente $T_0$. \addpoints*{1}
%      \end{enumerate}
  \end{enumerate}
\end{question}
\begin{solution}
  \begin{enumerate}
    \item \begin{enumerate}
        \item Soit $x\in\R$ : \[f(x) = \frac{e^{2x} - 1}{2e^x} =
          \frac{e^{2x} - 1}{2}e^{-x} = \frac{e^x - e^{-x}}2.\]
        \item Avec la deuxième forme, on trouve :
          \begin{itemize}
            \item $\lim_{x\to+\infty}f(x) = +\infty$
            \item $\lim_{x\to-\infty}f(x) = -\infty$
          \end{itemize}
      \end{enumerate}
    \item \begin{enumerate}
        \item Soit $x \in \R$, $f'(x) = \frac{e^x -(-e^{-x})}2 =
          \frac{e^x + e^{-x}}2.$

          $\forall x\in\R,\ e^x > 0$ et $e^{-x} > 0 \implies \forall
          x\in \R,\ f'(x) > 0$ donc $f$ est strictement croissante.
        \item
          \begin{itemize}
            \item $f(0) = \frac{e^0 - e^{-0}}2 = 0$
            \item $f'(1) = \frac{e^0 + e^{-0}}2 = \frac22 = 1$
          \end{itemize}
      \end{enumerate}
  \end{enumerate}
\end{solution}

\begin{question}[ID=limites]
  ~\\[-6ex]
  \phantom{a}\hfill\textbf{(\GetQuestionProperty{points}{limites} points)}\\
  Donnez la limite en $+\infty$ et $-\infty$ des fonctions suivantes :
  \begin{enumerate}
    \item $f(x) = e^x - e^{-x}$ \addpoints*{1}
    \item $f(x) = e^{3x^2 -2x +1}$ \addpoints*{1}
    \item $f(x) = xe^{-x}$ \addpoints*{1}
  \end{enumerate}
\end{question}
\begin{solution}
  \begin{enumerate}
    \item \begin{itemize}
        \item Par somme, on a$\lim_{x\to+\infty}f(x) = +\infty$
        \item Par somme, on a $\lim_{x\to-\infty}f(x) = -\infty$
      \end{itemize}
    \item \begin{itemize}
        \item D'une part, $\lim_{x\to+\infty} 3x^2 -2x +1 =
          \lim_{x\to+\infty} x\left(3x -2 +\frac1x\right) = +\infty$ ;
        \item d'autre part $\lim_{x\to-\infty} 3x^2 -2x +1 = +\infty$ ;
        \item d'ù $\lim_{x\to\pm\infty} f(x) = +\infty$.
      \end{itemize}
    \item \begin{itemize}
        \item $\lim_{x\to+\infty}xe^x =
          -\lim_{x\to+\infty}\frac{-x}{e^{-x}} = 0$

          Le produit ici aurait donné une forme indéterminée et donc, il
          faut se ramener à l'inverse du quotient $\frac{e^x}x$, au
          signe près pour conclure.
        \item Par produit des limites, on a $\lim_{x\to-\infty}xe^{x} =
          -\infty$.
      \end{itemize}
  \end{enumerate}
\end{solution}

\begin{question}[ID=ROC]
  ~\\[-6ex]
  \phantom{a}\hfill\textbf{(\GetQuestionProperty{points}{ROC} points)}\\
  Démontrez que s'il existe une fonction $f$ telle que $f = f'$ et $f(0)
  = 1$, alors $f$ est unique et est strictement positive.

  \emph{Indication :} Considérez $\varphi(x) = f(x)\times f(-x)$ puis si
  $f$ et $g$ sont deux fonctions différentes, étudier $\psi(x) =
  f(x)\times g(-x)$.
  \addpoints*{3}
\end{question}
\begin{solution}
  \begin{itemize}
    \item Premier point : si $f$ est dérivable sur $I\subset \R$, alors
      $x\mapsto f(-x)$ aussi et $\forall x \in I,\ (x\mapsto f(-x))'  =
      -f'(-x)$
    \item Deuxième point : on considère la fonction $\varphi$ définie
      par $\varphi(x) = f(x)\times f(-x)$.

      Cette fonction est dérivable comme produit de deux fonctions
      dérivables et $\varphi'(x) = f'(x)\times f(-x) - f(x)\times
      f'(-x)$. Mais $f'=f$ par hypthèse, d'où $\varphi'(x) = f(x)\times
      f(-x) - f(x)\times f(-x) = 0$.

      $\varphi$ est donc constante, on peut la déterminer avec la valeur
      en 0 : $\varphi(0) = f(0) \times f(-0) = 1$.

      On en déduit que $\forall x\in \R,\ f(x) \neq 0$ et $f(-x) =
      \frac1{f(x)}$.
    \item Troisième point (l'unicité à proprement parler) : on suppose
      désormais que $f$ et $g$ sont deux fonctions qui satisfont aux
      hypothèses $f = f'$ et $f(0) = 1$. Considérons la fonction $\psi =
      \frac{f}{g}$. Cette fonction est bien définie car $g$ ne s'annule
      pas, cf. le  point précédent.

      Montrons que $\psi$ est constante égale à 1. Pour cela, calculons
      la dérivée qui existe car on a supposé $f$ et $g$ dérivables.
      $\psi' = \frac{f'g - g'f}{f^2} = \frac{fg -gf}{f^2} \equiv 0$
      (notations fonctionnelles.) $\psi$ est donc constante. De plus,
      $\psi(0) = 1$ donc $\frac{f}g = 1$ et donc $f=g$.
  \end{itemize}
\end{solution}


\begin{question}[ID=BAC,use=true]
  ~\\[-6ex]
  \phantom{a}\hfill\textbf{(\GetQuestionProperty{points}{BAC} points)}\\

  $f$ est la fonction définie sur $\interfo{0;+\infty}$ par : \[ f(x) =
  \frac{x-1}{x+1} - e^{-x}. \]
  On note $\mathscr{C}$ sa courbe représentative dans un repère.

  \begin{enumerate}
    \item Démontrez que $\mathscr{C}$ admet une asymptote horinzontale
      dont on donnera une équation. \addpoints*{1}
    \item Étudiez les variations de $f$ sur $\interfo{0;+\infty}$.
      \addpoints*{1}
    \item Déterminez une équation de la tangente $T$ à $\mathscr{C}$ au
      point d'abscisse 0. \addpoints*{1}
    \item Démontrez que l'équation $f(x) = 0$ admet une solution unique
      dans l'intervalle $\interff{1;2}$. \addpoints*{1} On note $u$
      cette solution.

      Déterminez un encadrement d'amplitude $10^{-1}$ de $u$.
      \addpoints*{1}
  \end{enumerate}
\end{question}

\begin{question}[ID=Vecteurs,use=true]
  ~\\[-6ex]
  \phantom{a}\hfill\textbf{(\GetQuestionProperty{points}{Vecteurs} points)}\\

  Dans un repère orthonormé, on donne les points $A (1;0;1)$, $B
  (0;-1;2)$ et $C (2;5;2)$.

  \begin{enumerate}
    \item Vérifier que ces points ne sont pas alignés. \addpoints*{1}
    \item Déterminer la valeur de $a$ pour que le vecteur $\vec{n} : (a
      ; -1 ; 2)$ soit normal au plan $(ABC)$. \addpoints*{3}
  \end{enumerate}
\end{question}


%\pagebreak
%\printsolutions

\end{document}
