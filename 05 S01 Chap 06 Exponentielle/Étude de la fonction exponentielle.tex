\documentclass[12pt,a4paper,french]{article}
\usepackage[utf8]{inputenc}
\usepackage[T1]{fontenc}
\usepackage{babel}
\usepackage[thmmarks]{ntheorem}
\usepackage{amsmath}
\usepackage{amsfonts}
\usepackage{amssymb}

\usepackage{array}

\usepackage{kpfonts}

\usepackage[bookmarks=false,colorlinks,linkcolor=blue,pdfusetitle]{hyperref}

\pdfminorversion 7
\pdfobjcompresslevel 3

\usepackage{tabularx}
\usepackage[autolanguage,np]{numprint}
\usepackage{enumitem}

\usepackage{tipfr}
\usepackage{pgf}
\usepackage{tikz}
\usepackage{tkz-euclide}
\usetkzobj{all}
\usetikzlibrary{hobby}
\usepackage{tkz-tab}

\usepackage[top=1.7cm,bottom=2cm,left=2cm,right=2cm]{geometry}

\usepackage{lastpage}

\usepackage{esvect}
\usepackage{marginnote}

\usepackage{wrapfig}

\usepackage[defaultlines=5,all]{nowidow}


\makeatletter
\renewcommand{\@evenfoot}%
        {\hfil \upshape \small page {\thepage} de \pageref{LastPage}}
\renewcommand{\@oddfoot}{\@evenfoot}

\renewcommand{\maketitle}%
{\framebox{%
    \begin{minipage}{1.0\linewidth}%
      \begin{center}%
        \Large \@title ~-- \@author \\%
        \@date%
      \end{center}%
    \end{minipage}}%
  \normalsize%
  %\vspace{1cm}%
}

\pgfdeclarepatternformonly{mes_hachures}
{\pgfpoint{-0.1cm}{-0.1cm}}
{\pgfpoint{0.9cm}{0.5cm}}
{\pgfpoint{0.8cm}{0.4cm}}
{\pgfpathmoveto{\pgfpointorigin}
  \pgfpathlineto{\pgfpoint{0.8cm}{0.4cm}}
\pgfusepath{stroke}}

%Des macros pour les noms d'ensmbles
\newcommand{\R}{\mathbf{R}}
\newcommand{\Q}{\mathbf{Q}}
\newcommand{\Z}{\mathbf{Z}}
\newcommand{\C}{\mathbf{C}}
\newcommand{\N}{\mathbf{N}}

\newcommand{\norme}[1]{\left\lVert #1 \right\rVert}
\newcommand{\abs}[1]{\left\lvert #1 \right\rvert}

%Une macro récursive pour l'intérieru des vecteurs
%http://tex.stackexchange.com/questions/19693/arguments-of-custom-commands-as-comma-separated-list

\newcommand\vecteur[2][\\]{%
    \global\def\my@delim{#1}%
    \left(\negthinspace\begin{matrix}
        \my@vector #2,\relax\noexpand\@eolst%
    \end{matrix}\right)}

%Une macro pour les vecteurs
\def\my@vector #1,#2\@eolst{%
   \ifx\relax#2\relax
      #1
   \else
      #1\my@delim
      \my@vector #2\@eolst%
   \fi}

%Une macro récursive pour mettre formater l'intérieur des intervalles
\def\my@intervalle #1;#2\@eolst{%
  \ifx\relax#2\relax
    #1
  \else
    \my@intervalle #2\@eolst%
  \fi}

%Quatre macros pour les quatres types d'intervalles
\newcommand{\interff}[1]{%
  \left[\my@intervalle #1;\relax\noexpand\@eolst%
  \right]
}
\newcommand{\interfo}[1]{%
  \left[\my@intervalle #1;\relax\noexpand\@eolst%
  \right[}
\newcommand{\interof}[1]{%
  \left]\my@intervalle #1;\relax\noexpand\@eolst%
  \right]}
\newcommand{\interoo}[1]{%
  \left]\my@intervalle #1;\relax\noexpand\@eolst%
  \right[}

\makeatother


\usepackage{framed}

\theoremstyle{break}
\newtheorem{definition}{Définition}
\newtheorem{propriete}{Propriété}
\newtheorem{corollaire}{Corollaire}
\newtheorem{propdef}{Propriété - Définition}
\newtheorem{theoreme}{Théorème}
\theoremstyle{plain}
\theorembodyfont{\normalfont}
\newtheorem{exerciceT}{Exercice}
\theoremstyle{nonumberplain}
\newtheorem{remarque}{Remarque}
\newtheorem{notation}{Notation}
\newtheorem{probleme}{Problème}
\theoremsymbol{\ensuremath{\blacksquare}}
\newtheorem{preuve}{Preuve}
\theoremsymbol{}
\theoremstyle{nonumberbreak}
\newtheorem{exemple}{Exemple}

\newenvironment{exercice}{\begin{framed}\begin{exerciceT}}{\end{exerciceT}\end{framed}}

\setlength{\parsep}{0pt}
\setlength{\parskip}{5pt}
\setlength{\parindent}{0pt}
\setlength{\itemsep}{7pt}

\setlist{noitemsep}
%\setlist[1]{\labelindent=\parindent} % < Usually a good idea
\setlist[itemize]{leftmargin=*}
\setlist[itemize,1]{label=$\triangleright$}
\setlist[enumerate]{labelsep=*, leftmargin=1.5pc}
\setlist[enumerate,1]{label=\arabic*., ref=\arabic*}
\setlist[enumerate,2]{label=\emph{\alph*}),
ref=\theenumi.\emph{\alph*}}
\setlist[enumerate,3]{label=\roman*), ref=\theenumii.\roman*}
\setlist[description]{font=\sffamily\bfseries}

\usepackage{multicol}
\setlength{\columnseprule}{0pt}

\everymath{\displaystyle\everymath{}}

\title{Étude de la fonction exponentielle}
\author{\bsc{Jumel}}
\date{novembre 2017}

\begin{document}

\maketitle

On a montré, en admettant son existence, qu'il existait une et une seule
fonction définie, continue et dérivable sur $\R$ telle que : \[ f' = f \
\text{et}\ f(0) = 1.\] Cette fonction s'appelle la fonction
exponentielle et est notée $\exp$.

\section{Propriétés «fonctionnelles»}

Les premières propriétés relatives à la fonction exponentielle sont
celles qui explicitent la relation fonctionnelle de l'exponentielle,
c'est à dire la transformation des sommes en produits. Explicitons-les.


\begin{propriete}
  Pour tout $a$ et $b$ réels, on a \[\exp(a+ b) = \exp(a)\times\exp(b)
  .\]
  \begin{preuve}
    Soit $b$ un nombre réel.

    On considère la fonction $g_b$ définie sur $\R$ par $g_b(x) =
    \frac{\exp(x+b)}{\exp(b)}$. En dérivant $g_b$ et en calculant
    $g_b(0)$, on démontre que $g_b = \exp$.
  \end{preuve}
\end{propriete}

\begin{remarque}
  On appelle cette propriété un morphisme de groupe, et même plus
  précisément un morphisme de $(\R,+)$ dans le groupe $\R^*,\times)$.
\end{remarque}

En utilisant explicitement cette propriété avec des nombres négatifs, on
a le corollaire suivant.

\begin{corollaire}
  Pour tout $a$ et $b$ réels, \[ \exp(a-b) = \frac{\exp(a)}{\exp(b)} \]
\end{corollaire}

On a aussi la propriété suivante :
\begin{propriete}
  Pour tout entier naturel $n$ et pour tout nombre réel $a$, \[\exp(na)
  = (\exp(a))^n.\]
  \begin{preuve}
    Par récurrence en utilisant la propriété de transformation de somme
    en produit.
  \end{preuve}
\end{propriete}

Cette propriété s'étend assez naturellement aux entiers relatifs.

\begin{notation}
  Si on pose $\exp(1) = e$, on obtient pour tout entier relatif que
  $\exp(n) = e^n$. Cette notation étant très commode, on l'adopte
  souvent pour tous les nombres réels.
\end{notation}

\section{Étude «classique» de la fonction exponentielle}

\subsection{Variations de la fonction exponentielle}

Par définition de la fonction exponentielle, $\exp' = \exp$. Les
variations de $\exp$ sont données par son signe.

\begin{propriete}
  Pour tout $x$ réel, $e^x > 0$.
  \begin{preuve}
    Soit $x$ un nombre réel. $e^x$ s'écrit $e^{2\frac{x}2} =
    \left(e^{\frac{x}2}\right)$ qui est positif (c'est un carré).
  \end{preuve}
\end{propriete}

On en déduit que la fonction exponentielle est strictement croissante
sur $\R$.

\begin{corollaire}
  Pour tous nombres réels $a$ et $b$, \[ a \leqslant b \iff e^a
  \leqslant e^b.\]
\end{corollaire}

Une autre conséquence, en utilisant le théorème des valeurs
intermédiaires, est que l'équation $e^x = a$ avec $a$ strictement
positif possède une unique solution sur $\R$.

\subsection{Limites de $\exp$}

\begin{propriete}~\\[-7ex]
  \begin{multicols}{2}
    \begin{itemize}
      \item $\lim_{x\to+\infty}e^x = +\infty$
      \item $\lim_{x\to-\infty}e^{x} = 0$
    \end{itemize}
  \end{multicols}
\end{propriete}

Une activité a permis de démontrer la première de ces limites, la
deuxième se déduisant aisément de la précédente.

On peut également démontrer que $\exp$ «croit plus vite que n'importe
quelle puissance» et écrire les limites suivantes :

\begin{propriete}~\\[-7ex]
  \begin{multicols}{3}
    \begin{itemize}
      \item $\lim_{x\to0}\frac{e^x-1}{x} = 1$
      \item $\lim_{x\to+\infty}\frac{e^{x}}x = +\infty$
      \item $\lim_{x\to-\infty}xe^{x} = 0$
    \end{itemize}
  \end{multicols}
\end{propriete}

La propriété 5, telle qu'écrite ci-dessus, est la seule démontrée en
Terminale, mais elle possède en puissance la force de sa généralisation.
Voir pour cela l'activité sur les limites de la fonction exponentielle.

\subsection{Tableau de variation et courbe représentative}

On a le tableau de variation suivant pour la fonction exponentielle :

\begin{center}
  \begin{tikzpicture}
    \tkzTabInit[espcl=9]
    {$x$ / 1 , $\exp(x)$ / 1, $\exp$ / 3}
    {$-\infty$, $+\infty$}
    \tkzTabLine{, + , }
    \tkzTabVar{-/0,+/$+\infty$}
    \tkzTabVal{1}{2}{0.33}{0}{1}
    \tkzTabVal{1}{2}{0.67}{1}{$e$}
  \end{tikzpicture}
\end{center}

Ce tableau de variation nous permet de déduire la représentation
graphique suivante :

\begin{center}
  \begin{tikzpicture}[>=latex]
    \tkzInit[xmin=-5,xmax=5,ymin=-0.5,ymax=16]
    \tkzGrid
    \tkzAxeXY

    \foreach \a in {-1,0,1,2} {
      \draw[thick,red,<->] plot[domain={\a-1}:{\a+1}] (\x, {exp(\a)*(\x-\a)
      + exp(\a)}) ;
    }

    \draw[very thick] plot[smooth,domain=-5:0] (\x,{exp(\x)}) ;
    \draw[very thick] plot[smooth,domain=0:2] (\x,{exp(\x)}) ;
    \draw[very thick] plot[smooth,domain=2:{exp(1)}] (\x,{exp(\x)}) ;

  \end{tikzpicture}
\end{center}

\end{document}
