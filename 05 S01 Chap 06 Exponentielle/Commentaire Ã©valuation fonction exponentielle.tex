\documentclass[12pt,a4paper,french]{article}
\usepackage[utf8]{inputenc}
\usepackage[T1]{fontenc}
\usepackage{babel}
\usepackage[thmmarks]{ntheorem}
\usepackage{amsmath}
\usepackage{amsfonts}
\usepackage{amssymb}

\usepackage{array}

\usepackage{lmodern}
\usepackage{kpfonts}

\usepackage[bookmarks=false,colorlinks,linkcolor=blue]{hyperref}

\pdfminorversion 7
\pdfobjcompresslevel 3

\usepackage{tabularx}
\usepackage[autolanguage,np]{numprint}
\usepackage{enumitem}

\usepackage{tipfr}
\usepackage{pgf}
\usepackage{tikz}
\usepackage{tkz-euclide}
\usetkzobj{all}
\usetikzlibrary{hobby}
\usepackage{tkz-tab}

\usepackage[top=1.7cm,bottom=2cm,left=2cm,right=2cm]{geometry}

\usepackage{lastpage}

\usepackage{esvect}
\usepackage{marginnote}

\usepackage{wrapfig}

\usepackage[defaultlines=5,all]{nowidow}


\makeatletter
\renewcommand{\@evenfoot}%
        {\hfil \upshape \small page {\thepage} de \pageref{LastPage}}
\renewcommand{\@oddfoot}{\@evenfoot}

\renewcommand{\maketitle}%
{\framebox{%
    \begin{minipage}{1.0\linewidth}%
      \begin{center}%
        \Large \@title ~-- \@author \\%
        \@date%
      \end{center}%
    \end{minipage}}%
  \normalsize%
  %\vspace{1cm}%
}

\pgfdeclarepatternformonly{mes_hachures}
{\pgfpoint{-0.1cm}{-0.1cm}}
{\pgfpoint{0.9cm}{0.5cm}}
{\pgfpoint{0.8cm}{0.4cm}}
{\pgfpathmoveto{\pgfpointorigin}
  \pgfpathlineto{\pgfpoint{0.8cm}{0.4cm}}
\pgfusepath{stroke}}

%Des macros pour les noms d'ensmbles
\newcommand{\R}{\mathbf{R}}
\newcommand{\Q}{\mathbf{Q}}
\newcommand{\Z}{\mathbf{Z}}
\newcommand{\C}{\mathbf{C}}
\newcommand{\N}{\mathbf{N}}

\newcommand{\norme}[1]{\left\lVert #1 \right\rVert}
\newcommand{\abs}[1]{\left\lvert #1 \right\rvert}

%Une macro récursive pour l'intérieru des vecteurs
%http://tex.stackexchange.com/questions/19693/arguments-of-custom-commands-as-comma-separated-list

\newcommand\vecteur[2][\\]{%
    \global\def\my@delim{#1}%
    \left(\negthinspace\begin{matrix}
        \my@vector #2,\relax\noexpand\@eolst%
    \end{matrix}\right)}

%Une macro pour les vecteurs
\def\my@vector #1,#2\@eolst{%
   \ifx\relax#2\relax
      #1
   \else
      #1\my@delim
      \my@vector #2\@eolst%
   \fi}

%Une macro récursive pour mettre formater l'intérieur des intervalles
\def\my@intervalle #1;#2\@eolst{%
  \ifx\relax#2\relax
    #1
  \else
    \my@intervalle #2\@eolst%
  \fi}

%Quatre macros pour les quatres types d'intervalles
\newcommand{\interff}[1]{%
  \left[\my@intervalle #1;\relax\noexpand\@eolst%
  \right]
}
\newcommand{\interfo}[1]{%
  \left[\my@intervalle #1;\relax\noexpand\@eolst%
  \right[}
\newcommand{\interof}[1]{%
  \left]\my@intervalle #1;\relax\noexpand\@eolst%
  \right]}
\newcommand{\interoo}[1]{%
  \left]\my@intervalle #1;\relax\noexpand\@eolst%
  \right[}

\makeatother


\usepackage{framed}

\theoremstyle{break}
\newtheorem{definition}{Définition}
\newtheorem{propriete}{Propriété}
\newtheorem{corollaire}{Corollaire}
\newtheorem{propdef}{Propriété - Définition}
\newtheorem{theoreme}{Théorème}
\theoremstyle{plain}
\theorembodyfont{\normalfont}
\newtheorem{exerciceT}{Exercice}
\theoremstyle{nonumberplain}
\newtheorem{remarque}{Remarque}
\newtheorem{notation}{Notation}
\newtheorem{probleme}{Problème}
\theoremsymbol{\ensuremath{\blacksquare}}
\newtheorem{preuve}{Preuve}
\theoremsymbol{}
\theoremstyle{nonumberbreak}
\newtheorem{exemple}{Exemple}

\newenvironment{exercice}{\begin{framed}\begin{exerciceT}}{\end{exerciceT}\end{framed}}

\setlength{\parsep}{0pt}
\setlength{\parskip}{5pt}
\setlength{\parindent}{0pt}
\setlength{\itemsep}{7pt}

\setlist{noitemsep}
%\setlist[1]{\labelindent=\parindent} % < Usually a good idea
\setlist[itemize]{leftmargin=*}
\setlist[itemize,1]{label=$\triangleright$}
\setlist[enumerate]{labelsep=*, leftmargin=1.5pc}
\setlist[enumerate,1]{label=\arabic*., ref=\arabic*}
\setlist[enumerate,2]{label=\emph{\alph*}),
ref=\theenumi.\emph{\alph*}}
\setlist[enumerate,3]{label=\roman*), ref=\theenumii.\roman*}
\setlist[description]{font=\sffamily\bfseries}

\usepackage{multicol}
\setlength{\columnseprule}{0pt}

\usepackage[]{exsheets}
\SetupExSheets{headings=block}

\everymath{\displaystyle\everymath{}}

\title{Commentaires sur évaluation \no{6} : exponentielle}
\author{\bsc{Ts 3}}
\date{22 janvier 2015}

\begin{document}

\maketitle

L'évaluation a été plutôt bien réussie dans l'ensemble et les élèves qui
ont appris leur cours ont su reconnaître les questions de cours et y
répondre directement. Cependant quelques erreurs plus oun moins graves
se sont glissées dans les copies.

Attention toutefois, dans le premier exercice, peu ont correctement
calculé la dérivée en question et encore moins ont penser à justifier du
signe de la dérivée. À titre d'exercice : étudiez les fonctions définies
pour tout $x$ réel par $\sinh(x) = \frac{e^x - e^{-x}}2$ et $\cosh(x) =
\frac{e^x + e^{-x}}2$.

\section{Simplification abusive}

Heureusement présente qu'une seule fois dans les copies, mais
intolérable en classe de Terminale S. En effet, même avec des
exponetielles (ou des radicaux), on ne peut pas simplifier une fraction
de la forme $\frac{ab + c}{ad}$ et ce résultat est en général faux.

À titre d'exercice, étudier les valeurs de $a$, $b$, $c$ et $d$ qui
réalisent $\frac{ab + c}{ad} = \frac{b+c}{d}$.

\section{Cohérence des résultats ou avec les questions}

Une erreur assez grave en Terminale est celle de la cohérence de vos
résultats entre eux. Prétendre qu'une fonction est croissante alors que
le signe de la dérivée n'a pas été étudié ou pire encore qu'il est
manifestement négatif est une faute qui est sanctionnée de façon
implicite par le correcteur, qui se met à douter de toutes vos
affirmations. Soyez vigilants et relisez-vous.

\section{Confusion entre égalité et équivalence}

On trouve dans des copies, souvent dans un souci de bien faire, des
confusions entre l'équivalence ($\iff$) et l'égalité ($=$). Il s'agit de
deux notions distinctes.

L'égalité est une relation binaire entre deux expressions mathématiques,
éventuellement numériques. Écrire que $a = b$ signifie qu'on pourra
remplacer $a$ par $b$ dans une expression à venir.

En toute rigueur, on ne devrait pas écrire $a = b = c$, mais $a = b$ et
$b = c$ donc $a = c$.

L'équivalence est une relation de la logique, qui porte sur des
assertions logiques (des propositions logiques, c'est-à-dire des
phrases) parfois écrites avec le symbolisme mathématique. Si $P$ et $Q$
sont deux assertions logiques, $P\iff Q$ signifie que $P$ et $Q$ ont
même valeurs de vérité, c'est-à-dire que si $P$ est vraie, alors $Q$ est
vraie et si $Q$ est vraie, alors $P$ est vraie. On peut résumer ce fait
par le tableau de vérité suivant :
\begin{center}
  \begin{tabular}{|*{3}{c|}}\hline
    $P$ & $Q$ & $P \iff Q$ \\ \hline
    V & V & V \\ \hline
    F & F & V \\ \hline
    V & F & F \\ \hline
    F & V & F \\ \hline
  \end{tabular}
\end{center}

L'équivalence $P \iff Q$ est «syntaxiquement équivalente» à $P \implies
Q$ et $Q \implies P$, ce qui est souvent utile dans les démonstrations.

\section{Confusion entre fonction et valeur de la fonction}

Il faut noter qu'une fonction est en quelque sorte un processus
(mathématiquement ça n'est pas tout à fait juste, mais ça permet de
comprendre) qui associe à un élément d'un ensemble de départ un unique
élement d'un ensemble d'arrivée. La fonction est donc l'ensemble de
toutes ces associations et de comment on y arrive. On note la fonction
par une lettre minuscule $f$ par exemple.

Pour une fonction numérique, c'est à dire d'une partie de $\R$ dans une
partie de $\R$, il y a généralement une méthode calculatoire explicite,
c'est ce qu'on appelle son expression, qu'on note $f(x)$, qui est en
fait la réalisation du calcul pour la valeur $x$, mais qui n'est pas la
fonction.

On peut le voir dans la notation suivante : «Soit $f$ la fonction qui à
$x$ associe $f(x)$ (noté $f:x\mapsto f(x)$)» où $f(x)$ est généralement
une formule de calcul.

Ainsi, on parle de la fonction dérivée $f'$ qui existe et de $f'(x)$ le
\emph{nombre} dérivé de $f$ en $x$.

Pour conclure cette remarque, on peut considérer la fonction suivante,
qui ne possède pas d'expression : $\mathbf{1}_{\mathbf{Q}}:x\mapsto
\left\lbrace\begin{matrix} 1\text{ si }x\in\mathbf{Q}\\ 0\text{ sinon
}\end{matrix}\right.$.

\end{document}
