\documentclass[12pt,a4paper,french]{article}
\usepackage[utf8]{inputenc}
\usepackage[T1]{fontenc}
\usepackage{babel}
\usepackage{ntheorem}
\usepackage{amsmath}
\usepackage{amsfonts}
\usepackage{amssymb}

\usepackage{array}

\usepackage{kpfonts}

\usepackage[bookmarks=false,colorlinks,linkcolor=blue]{hyperref}

\pdfminorversion 7
\pdfobjcompresslevel 3

\usepackage{tabularx}
\usepackage[autolanguage,np]{numprint}
\usepackage{enumitem}

\usepackage{tipfr}
\usepackage{pgf}
\usepackage{tikz}
\usepackage{tkz-euclide}
\usetkzobj{all}
\usetikzlibrary{hobby}

\usepackage[top=1.7cm,bottom=2cm,left=2cm,right=2cm]{geometry}

\usepackage{lastpage}

\usepackage{esvect}
\usepackage{marginnote}

\usepackage{wrapfig}

\usepackage[defaultlines=5,all]{nowidow}


\makeatletter
\renewcommand{\@evenfoot}%
        {\hfil \upshape \small page {\thepage} de \pageref{LastPage}}
\renewcommand{\@oddfoot}{\@evenfoot}

\renewcommand{\maketitle}%
{\framebox{%
    \begin{minipage}{1.0\linewidth}%
      \begin{center}%
        \Large \@title ~-- \@author \\%
        \@date%
      \end{center}%
    \end{minipage}}%
  \normalsize%
  %\vspace{1cm}%
}

\pgfdeclarepatternformonly{mes_hachures}
{\pgfpoint{-0.1cm}{-0.1cm}}
{\pgfpoint{0.9cm}{0.5cm}}
{\pgfpoint{0.8cm}{0.4cm}}
{\pgfpathmoveto{\pgfpointorigin}
  \pgfpathlineto{\pgfpoint{0.8cm}{0.4cm}}
\pgfusepath{stroke}}

%Des macros pour les noms d'ensmbles
\newcommand{\R}{\mathbf{R}}
\newcommand{\Q}{\mathbf{Q}}
\newcommand{\Z}{\mathbf{Z}}
\newcommand{\C}{\mathbf{C}}
\newcommand{\N}{\mathbf{N}}

\newcommand{\norme}[1]{\left\lVert #1 \right\rVert}
\newcommand{\abs}[1]{\left\lvert #1 \right\rvert}

%Une macro récursive pour l'intérieru des vecteurs
%http://tex.stackexchange.com/questions/19693/arguments-of-custom-commands-as-comma-separated-list

\newcommand\vecteur[2][\\]{%
    \global\def\my@delim{#1}%
    \left(\negthinspace\begin{matrix}
        \my@vector #2,\relax\noexpand\@eolst%
    \end{matrix}\right)}

%Une macro pour les vecteurs
\def\my@vector #1,#2\@eolst{%
   \ifx\relax#2\relax
      #1
   \else
      #1\my@delim
      \my@vector #2\@eolst%
   \fi}

%Une macro récursive pour mettre formater l'intérieur des intervalles
\def\my@intervalle #1;#2\@eolst{%
  \ifx\relax#2\relax
    #1
  \else
    \my@intervalle #2\@eolst%
  \fi}

%Quatre macros pour les quatres types d'intervalles
\newcommand{\interff}[1]{%
  \left[\my@intervalle #1;\relax\noexpand\@eolst%
  \right]
}
\newcommand{\interfo}[1]{%
  \left[\my@intervalle #1;\relax\noexpand\@eolst%
  \right[}
\newcommand{\interof}[1]{%
  \left]\my@intervalle #1;\relax\noexpand\@eolst%
  \right]}
\newcommand{\interoo}[1]{%
  \left]\my@intervalle #1;\relax\noexpand\@eolst%
  \right[}

\makeatother


\usepackage{framed}

\theoremstyle{break}
\newtheorem{definition}{Définition}
\newtheorem{propriete}{Propriété}
\newtheorem{propdef}{Propriété - Définition}
\newtheorem{theoreme}{Théorème}
\theoremstyle{plain}
\theorembodyfont{\normalfont}
\newtheorem{exerciceT}{Exercice}
\theoremstyle{nonumberplain}
\newtheorem{remarque}{Remarque}
\newtheorem{probleme}{Problème}
\newtheorem{preuve}{Preuve}
\theoremstyle{nonumberbreak}
\newtheorem{exemple}{Exemple}

\newenvironment{exercice}{\begin{framed}\begin{exerciceT}}{\end{exerciceT}\end{framed}}

\setlength{\parsep}{0pt}
\setlength{\parskip}{5pt}
\setlength{\parindent}{0pt}
\setlength{\itemsep}{7pt}

\setlist{noitemsep}
%\setlist[1]{\labelindent=\parindent} % < Usually a good idea
\setlist[itemize]{leftmargin=*}
\setlist[itemize,1]{label=$\triangleright$}
\setlist[enumerate]{labelsep=*, leftmargin=1.5pc}
\setlist[enumerate,1]{label=\arabic*., ref=\arabic*}
\setlist[enumerate,2]{label=\emph{\alph*}),
ref=\theenumi.\emph{\alph*}}
\setlist[enumerate,3]{label=\roman*), ref=\theenumii.\roman*}
\setlist[description]{font=\sffamily\bfseries}

\usepackage{multicol}
\setlength{\columnseprule}{0pt}

\everymath{\displaystyle\everymath{}}

\title{Limites de la fonction exponentielle}
\author{\bsc{Jumel}}
\date{janvier 2017}

\begin{document}

\maketitle

\section{Limites infinies}

\subsection{Limite en $+\infty$}

\begin{exercice}
  On considère la fonction
  $f:\left\lbrace\begin{matrix}\R\to\R\\x\mapsto e^x -
  x\end{matrix}\right.$.
  \begin{enumerate}
    \item
      \begin{enumerate}
        \item Justifier que cette fonction est dérivable.
        \item Calculer la dérivée $f'$ de cette fonction.
        \item Dresser son tableau de variation.
      \end{enumerate}
    \item Avec les variations, que peut-on en déduire concernant l'image
      de 0 ?
    \item Reformuler l'inégalité obtenue.
  \end{enumerate}
\end{exercice}

\begin{exercice}
  On considère l'inégalité $\forall x\in \R, e^x > x$ (qu'on admet
  éventuellement.)

  \begin{enumerate}
    \item Donner la limite de la fonction $x\mapsto x$ en $+\infty$.
    \item Quel théorème sur les limites peut-on utiliser ?
    \item Conclure.
  \end{enumerate}
\end{exercice}

\subsection{Limite en $-\infty$}

\begin{exercice}
  Démontrer que $\lim_{x\to-\infty}e^x = 0$. On pourra poser $Y = -x$.
\end{exercice}

\section{Croissances comparées}

\subsection{Développement limité au premier ordre}

\begin{exercice}
  Une première limite, qui correspond au développement limité de
  l'exponentielle en 0.
  \begin{enumerate}
    \item
      \begin{enumerate}
        \item Rappeler la définition du nombre dérivé d'une fonction en
          $a$.
        \item Mettre en œuvre cette définition pour exprimer le nombre
          dérivée de la fonction $\exp$ en 0.
        \item Calculer ce nombre dérivé en utilisant la fonction
          dérivée d'$\exp$.
      \end{enumerate}
    \item Donner la limite \\$\lim_{x\to0}\frac{e^x - 1}{x}$.
  \end{enumerate}
\end{exercice}

\subsection{Des équivalents}

On va utiliser un théorème de comparaison.
\begin{exercice}
  Une deuxième limite.
  \begin{enumerate}
    \item Étudier la différence $e^x - \frac{x^2}2$.
    \item Démontrer que $f(x) > 1 > 0$.
    \item Soit $x$ un réel, comparer $\frac{e^x}x$ et $\frac{x}2$
    \item En déduire la limite de $\frac{e^x}x$ en $+\infty$.
  \end{enumerate}
\end{exercice}

\begin{exercice}
  En utilisant une remarque analogue à l'exercice 3, démontrer que
  $\lim_{x\to-\infty}xe^x = 0$
\end{exercice}


\end{document}
