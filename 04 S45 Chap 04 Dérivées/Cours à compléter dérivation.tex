\documentclass[12pt,a4paper,french]{article}
\usepackage[utf8]{inputenc}
\usepackage[T1]{fontenc}
\usepackage{babel}
\usepackage{ntheorem}
\usepackage{amsmath}
\usepackage{amsfonts}
\usepackage{amssymb}

\usepackage{array}

\usepackage{kpfonts}

\usepackage[bookmarks=false,colorlinks,linkcolor=blue,pdfusetitle]{hyperref}

\pdfminorversion 7
\pdfobjcompresslevel 3

\usepackage{tabularx}
\usepackage[autolanguage,np]{numprint}
\usepackage{enumitem}

\usepackage{tipfr}
\usepackage{pgf}
\usepackage{tikz}
\usepackage{tkz-euclide}
\usetkzobj{all}
\usetikzlibrary{hobby}

\usepackage[top=1.7cm,bottom=2cm,left=2cm,right=2cm]{geometry}

\usepackage{lastpage}

\usepackage{esvect}
\usepackage{marginnote}

\usepackage{wrapfig}

\usepackage[defaultlines=5,all]{nowidow}


\makeatletter
\renewcommand{\@evenfoot}%
        {\hfil \upshape \small page {\thepage} de \pageref{LastPage}}
\renewcommand{\@oddfoot}{\@evenfoot}

\renewcommand{\maketitle}%
{\framebox{%
    \begin{minipage}{1.0\linewidth}%
      \begin{center}%
        \Large \@title ~-- \@author \\%
        \@date%
      \end{center}%
    \end{minipage}}%
  \normalsize%
  %\vspace{1cm}%
}

\pgfdeclarepatternformonly{mes_hachures}
{\pgfpoint{-0.1cm}{-0.1cm}}
{\pgfpoint{0.9cm}{0.5cm}}
{\pgfpoint{0.8cm}{0.4cm}}
{\pgfpathmoveto{\pgfpointorigin}
  \pgfpathlineto{\pgfpoint{0.8cm}{0.4cm}}
\pgfusepath{stroke}}

%Des macros pour les noms d'ensmbles
\newcommand{\R}{\mathbf{R}}
\newcommand{\Q}{\mathbf{Q}}
\newcommand{\Z}{\mathbf{Z}}
\newcommand{\C}{\mathbf{C}}
\newcommand{\N}{\mathbf{N}}

\newcommand{\norme}[1]{\left\lVert #1 \right\rVert}
\newcommand{\abs}[1]{\left\lvert #1 \right\rvert}

%Une macro récursive pour l'intérieru des vecteurs
%http://tex.stackexchange.com/questions/19693/arguments-of-custom-commands-as-comma-separated-list

\newcommand\vecteur[2][\\]{%
    \global\def\my@delim{#1}%
    \left(\negthinspace\begin{smallmatrix}
        \my@vector #2,\relax\noexpand\@eolst%
    \end{smallmatrix}\right)}

%Une macro pour les vecteurs
\def\my@vector #1,#2\@eolst{%
   \ifx\relax#2\relax
      #1
   \else
      #1\my@delim
      \my@vector #2\@eolst%
   \fi}

%Une macro récursive pour mettre formater l'intérieur des intervalles
\def\my@intervalle #1;#2\@eolst{%
  \ifx\relax#2\relax
    #1
  \else
    \my@intervalle #2\@eolst%
  \fi}

%Quatre macros pour les quatres types d'intervalles
\newcommand{\interff}[1]{%
  \left[\my@intervalle #1;\relax\noexpand\@eolst%
  \right]
}
\newcommand{\interfo}[1]{%
  \left[\my@intervalle #1;\relax\noexpand\@eolst%
  \right[}
\newcommand{\interof}[1]{%
  \left]\my@intervalle #1;\relax\noexpand\@eolst%
  \right]}
\newcommand{\interoo}[1]{%
  \left]\my@intervalle #1;\relax\noexpand\@eolst%
  \right[}

\newcommand*{\versioncomplet}{}
\newcommand{\complet}[2]{%
  \ifdefined\versioncomplet
    {\color{blue} #2}
  \else
   ~\\\hspace{0.98\linewidth}\vspace{#1}
  \fi
}


\makeatother


\theoremstyle{break}
\newtheorem{definition}{Définition}
\newtheorem{propriete}{Propriété}
\newtheorem{propdef}{Propriété - Définition}
\newtheorem{theoreme}{Théorème}
\theoremstyle{plain}
\theorembodyfont{\normalfont}
\newtheorem{exercice}{Exercice}
\theoremstyle{nonumberplain}
\newtheorem{remarque}{Remarque}
\newtheorem{probleme}{Problème}
\newtheorem{preuve}{Preuve}
\theoremstyle{nonumberbreak}
\newtheorem{exemple}{Exemple}

\setlength{\parsep}{0pt}
\setlength{\parskip}{5pt}
\setlength{\parindent}{0pt}
\setlength{\itemsep}{7pt}

\setlist{noitemsep}
%\setlist[1]{\labelindent=\parindent} % < Usually a good idea
\setlist[itemize]{leftmargin=*}
\setlist[itemize,1]{label=$\triangleright$}
\setlist[enumerate]{labelsep=*, leftmargin=1.5pc}
\setlist[enumerate,1]{label=\arabic*., ref=\arabic*}
\setlist[enumerate,2]{label=\emph{\alph*}),
ref=\theenumi.\emph{\alph*}}
\setlist[enumerate,3]{label=\roman*), ref=\theenumii.\roman*}
\setlist[description]{font=\sffamily\bfseries}

\usepackage{multicol}
\setlength{\columnseprule}{0pt}

\everymath{\displaystyle\everymath{}}

\title{Dérivation des fonctions}
\author{Terminale S}
\date{novembre 2017}

\begin{document}

\maketitle

Dans tout ce document, $f$ désigne une fonction continue et
$\interff{a,b}$ un intervalle de $\R$.

\section{Nombre dérivé -- fonction dérivée}

En classe de première, le nombre dérivé est défini comme limite du taux
d'accroissement en un point, sachant que la limite est une notion encore
mal définie.

On peut désormais poser la définition suivante :

\framebox{
  \begin{minipage}{0.99\linewidth}
    \complet{4cm}{%
      \begin{definition}
        $f$ est dérivable en $x_0\in\interoo{a,b}$ si
        $\lim_{\substack{h\to x_0 \\ h<x_0}} \frac{f(x_0 + h) - f(x_0)}h
        = \lim_{\substack{h\to x_0 \\ h>x_0}} \frac{f(x_0 + h) -
        f(x_0)}h = d$, un nombre fini.

        Cette limite s'appelle nombre dérivé de $f$ en $x_0$ et se note
        $f'(x_0)$
      \end{definition}%
    }
  \end{minipage}
}

\begin{center}
  \begin{tikzpicture}[scale=1,
      use Hobby shortcut,
      tangent/.style={%
        in angle={(180+#1)},
        Hobby finish,
        designated Hobby path=next,
        out angle=#1,
      },>=latex,
    ]
    \draw [dotted, help lines, line width=0.6pt] (-6,-1) grid (6,6) ;
    \draw [dotted, help lines, step=0.5] (-6,-1) grid (6,6) ;
    \draw [very thick, ->] (0,-1) -- (0,6) ;
    \draw [very thick, ->] (-6,0) -- (6,0) ;
    \draw [thick] ([tangent=-45]-4,5) .. ([tangent=-45]-2.5,1.5) ..
    ([tangent=0]-1,1) .. ([tangent=45]0,1.5) .. ([tangent=0]1,2) .. (2,1.5) ..
    ([tangent=0]4,0) ;
    \draw[thick,->] (-2.5,1.5) -- +(-45:1) ;
    \draw[thick,->] (-2.5,1.5) -- +(-45:-1) ;
    \draw[thick,->] (0,1.5) -- +(45:1) ;
    \draw[thick,->] (0,1.5) -- +(45:-1) ;
     \foreach \x/\y in {-4/5,-2.5/1.5,0/1.5,2/1.5,4/0}
    { \draw (\x,\y) node [circle,fill,inner sep=1pt] {} ; }
    \draw (-4,5) node [above right] {$\mathscr{C}_f$} ;
  \end{tikzpicture}
\end{center}

\framebox{
  \begin{minipage}{0.99\linewidth}
    \complet{3cm}{%
      \begin{definition}
        La fonction dérivée de $f$ sur $\interff{a,b}$ est la fonction
        qui à tout $x$ de $\interff{a,b}$ associe le nombre dérivé en
        $x$.

        On note cette fonction $f'$.
      \end{definition}%
    }
  \end{minipage}
}

\begin{remarque}
  Dans certains cas, on ne peut associer de nombre dérivé en un point.

  Dans ce cas, la dérivée présente une discontinuité.
\end{remarque}

\framebox{
  \begin{minipage}{0.99\linewidth}
    \begin{theoreme}
      \complet{2cm}{%
        \begin{itemize}
          \item Si $f$ est dérivable en $a$, alors $f$ est continue en
            $a$.
          \item Si $f$ est dérivable sur $I$ intervalle de $\R$, alors
            $f$ est continue sur $I$.
        \end{itemize}
      }
    \end{theoreme}
  \end{minipage}
}

\begin{remarque}
  La réciproque de ce théorème est fausse en général.
\end{remarque}

Les opérations sur les limites nous permettent d'écrire la propriété
suivante :

\framebox{
  \begin{minipage}{0.99\linewidth}
    \begin{propriete}
      \complet{2cm}{%
        \begin{itemize}
          \item $(f+g)' = f'+g'$
          \item $(\lambda f)' = \lambda f'$
        \end{itemize}
      }
    \end{propriete}
  \end{minipage}
}

On a aussi la propriété suivante sur la dérivée du produit de deux
fonctions dérivables.
\framebox{
  \begin{minipage}{0.99\linewidth}
    \begin{propriete}
      \complet{2cm}{%
        \begin{itemize}
          \item $(fg)' = f'g+fg'$
        \end{itemize}
      }
    \end{propriete}
  \end{minipage}
}


Cette proriété justifie que l'essentiel soit de construire le tableau
des dérivées.

\begin{center}
  \begin{tabular}{|p{5cm}|p{5cm}|}\hline
    $f:\R\to\R,\ x\mapsto k,\ k\in \R$ & \complet{1cm}{$f':\R\to\R,\ x\to 0$} \\ \hline
    $f:\R\to\R,\ x\mapsto x$ & \complet{1cm}{$f':\R\to\R,\ x\to 1$} \\ \hline
    $f:\R\to\R,\ x\mapsto x^2$ & \complet{1cm}{$f':\R\to\R,\ x\to 2x$} \\ \hline
    $f:\R^*\to\R^*,\ x\mapsto \frac1x $ & \complet{1cm}{$f':\R^*\to\R^*,\ x\to -\frac1{x^2}$} \\ \hline
    $f:\R_+\to\R_+,\ x\mapsto \sqrt{x}$ & \complet{1cm}{$f':\R_+^* \to \R_+^*,\ x\to \frac1{2\sqrt{x}}$} \\ \hline
  \end{tabular}
\end{center}

\section{Dérivation d'une fonction composée}

Soit $u$ une fonction définie et dérivable sur $\interff{a,b}$.

\framebox{
  \begin{minipage}{0.99\linewidth}
    \begin{propriete}
      \complet{2cm}{%
        $u^n$ est dérivable et sa dérivée est $x\mapsto nu'(x)u^{n-1}(x)$
    }
    \end{propriete}
  \end{minipage}
}

\begin{exemple}
  Sans développer dériver la fonction définie pour tout réel par
  $f(x)=(x^2 + 2x -3)^3$.
\end{exemple}

On suppose que $u(x)$ ne s'annule pas $\interff{a,b}$

\framebox{
  \begin{minipage}{0.99\linewidth}
    \begin{propriete}
      \complet{2.5cm}{%
        La fonction
        $x\mapsto \frac1{u(x)}$ est dérivable et sa dérivée est
        $x\mapsto -\frac{u'(x)}{(u(x))^2}$.
    }
    \end{propriete}
  \end{minipage}
}

\begin{exemple}
  Dériver $x\mapsto \frac1{x^2 +1}$
\end{exemple}

On suppose que pour tout $x$ de $\interff{a,b}$, $u(x) > 0$.

\framebox{
  \begin{minipage}{0.99\linewidth}
    \begin{propriete}
      \complet{2cm}{%
        $x\mapsto\sqrt{u(x)}$ est dérivable et sa dérivée est $x\mapsto
        \frac{u'(x)}{2\sqrt{u(x)}}$
    }
    \end{propriete}
  \end{minipage}
}

\begin{exemple}
  Dériver la fonction définie sur $\R$ par $f(x) = \sqrt{x^2 + 2x + 2}$
\end{exemple}

\framebox{
  \begin{minipage}{0.99\linewidth}
    \begin{theoreme}
      \complet{2cm}{%
        Sous réserve d'existence, la dérivée de $f(u(x))$, notée $f\circ
        u$ est $u'\times f'(u(x)) = u'\times (f'\circ u)$
   }
    \end{theoreme}
  \end{minipage}
}



\end{document}
