\documentclass[11pt,a4paper,french]{article}
\usepackage[utf8]{inputenc}
\usepackage[T1]{fontenc}
\usepackage{babel}
\usepackage[thmmarks]{ntheorem}
\usepackage{amsmath}
\usepackage{amsfonts}
\usepackage{amssymb}

\usepackage{array}

\usepackage{kpfonts}

\usepackage[bookmarks=false,colorlinks,linkcolor=blue]{hyperref}

\pdfminorversion 7
\pdfobjcompresslevel 3

\usepackage{tabularx}
\usepackage[autolanguage,np]{numprint}
\usepackage{enumitem}

\usepackage{tipfr}
\usepackage{pgf}
\usepackage{tikz}
\usepackage{tkz-base}
\usepackage{tkz-euclide}
\usetkzobj{all}
\usetikzlibrary{hobby}
\usepackage{tkz-tab}

\usepackage[top=1.7cm,bottom=2cm,left=2cm,right=2cm]{geometry}

\usepackage{lastpage}

\usepackage{esvect}
\usepackage{marginnote}

\usepackage{wrapfig}

\usepackage[defaultlines=5,all]{nowidow}


\makeatletter
\renewcommand{\@evenfoot}%
        {\hfil \upshape \small page {\thepage} de \pageref{LastPage}}
\renewcommand{\@oddfoot}{\@evenfoot}

\renewcommand{\maketitle}%
{\framebox{%
    \begin{minipage}{1.0\linewidth}%
      \begin{center}%
        \Large \@title ~-- \@author \\%
        \@date%
      \end{center}%
    \end{minipage}}%
  \normalsize%
  %\vspace{1cm}%
}

%Des macros pour les noms d'ensmbles
\newcommand{\R}{\mathbf{R}}
\newcommand{\Q}{\mathbf{Q}}
\newcommand{\Z}{\mathbf{Z}}
\newcommand{\C}{\mathbf{C}}
\newcommand{\N}{\mathbf{N}}

\newcommand{\Cf}{\mathscr{C}}

\newcommand{\norme}[1]{\left\lVert #1 \right\rVert}
\newcommand{\abs}[1]{\left\lvert #1 \right\rvert}
\newcommand{\diff}{\mathop{}\mathopen{}\mathrm{d}}

%Une macro récursive pour l'intérieru des vecteurs
%http://tex.stackexchange.com/questions/19693/arguments-of-custom-commands-as-comma-separated-list

\newcommand\vecteur[2][\\]{%
    \global\def\my@delim{#1}%
    \left(\negthinspace\begin{matrix}
        \my@vector #2,\relax\noexpand\@eolst%
    \end{matrix}\right)}

%Une macro pour les vecteurs
\def\my@vector #1,#2\@eolst{%
   \ifx\relax#2\relax
      #1
   \else
      #1\my@delim
      \my@vector #2\@eolst%
   \fi}

%Une macro récursive pour mettre formater l'intérieur des intervalles
\def\my@intervalle #1;#2\@eolst{%
  \ifx\relax#2\relax
    #1
  \else
    \my@intervalle #2\@eolst%
  \fi}

%Quatre macros pour les quatres types d'intervalles
\newcommand{\interff}[1]{%
  \left[\my@intervalle #1;\relax\noexpand\@eolst%
  \right]
}
\newcommand{\interfo}[1]{%
  \left[\my@intervalle #1;\relax\noexpand\@eolst%
  \right[}
\newcommand{\interof}[1]{%
  \left]\my@intervalle #1;\relax\noexpand\@eolst%
  \right]}
\newcommand{\interoo}[1]{%
  \left]\my@intervalle #1;\relax\noexpand\@eolst%
  \right[}

\makeatother


\usepackage{mdframed}

\theoremstyle{break}
\newtheorem{definition}{Définition}
\newtheorem{propriete}{Propriété}
\newtheorem{proposition}{Proposition}
\newtheorem{corollaire}{Corollaire}
\newtheorem{propdef}{Propriété - Définition}
\newtheorem{theoreme}{Théorème}
\theoremstyle{plain}
\theorembodyfont{\normalfont}
\newtheorem{exerciceT}{Exercice}
\theoremstyle{nonumberplain}
\newtheorem{remarque}{Remarque}
\newtheorem{notation}{Notation}
\newtheorem{probleme}{Problème}
\theoremsymbol{\ensuremath{\blacksquare}}
\newtheorem{preuve}{Preuve}
\theoremsymbol{}
\theoremstyle{nonumberbreak}
\newtheorem{exemple}{Exemple}


\setlength{\parsep}{0pt}
\setlength{\parskip}{5pt}
%\setlength{\parindent}{0pt}
\setlength{\itemsep}{7pt}

\setlist{noitemsep}
%\setlist[1]{\labelindent=\parindent} % < Usually a good idea
\setlist[itemize]{leftmargin=*}
\setlist[itemize,1]{label=$\triangleright$}
\setlist[enumerate]{labelsep=*, leftmargin=1.5pc}
\setlist[enumerate,1]{label=\arabic*., ref=\arabic*}
\setlist[enumerate,2]{label=\emph{\alph*}),
ref=\theenumi.\emph{\alph*}}
\setlist[enumerate,3]{label=\roman*), ref=\theenumii.\roman*}
\setlist[description]{font=\sffamily\bfseries}

\usepackage{multicol}
\setlength{\columnseprule}{0pt}

\usepackage[]{exsheets}
\SetupExSheets{
  headings = block,
  question/pre-hook = \mdframed,
  question/post-hook = \endmdframed,
}


\everymath{\displaystyle\everymath{}}

\title{Devoir maison \no 4 : Intégration}
\author{\bsc{Jumel}}
\date{pour le 20 avril 2017}

\begin{document}

\noindent\maketitle

\bigskip

Le but de ce devoir est d'approfondir la notion d'intégrale dans
différents cas de figures. Les parties de ce devoir sont largement
indépendantes les unes des autres.

\section{Intégration par parties}

On a vu en cours que l'intégrale d'un produit et le produit d'intégrale
différent. On a cependant vu que si $f$ s'écrit sous certaines formes,
alors $f$ possède une primitive.

\begin{question}
  Donner les cas où $f$ est un produit de deux fonctions et $f$ possède
  une primitive.
\end{question}
\begin{solution}
  On peut citer les cas suivants :
  \begin{itemize}
    \item le produit d'une fonction polynomiale et d'une exponentielle
      ou d'un (co)sinus ;
    \item les formes $u'e^u$ ;
    \item les formes $u' \times \frac1u$.
  \end{itemize}
\end{solution}

Le but de l'exercice qui vient est de voir plus précisement dans quels
cas on peut intégrer un produit et comment procéder.

\begin{question} Dans cet exercice, $u$ et $v$ sont deux fonctions.
  \begin{enumerate}
    \item Écrire la formule de dérivation du produit $uv$.
    \item Exprimer $uv'$ en fonction de $(uv)'$ et de $u'v$.
    \item À quelle condition $uv'$ est elle intégrable ? Même question
      pour $u'v$. En déduire une condition sur $u$ et $v$.
    \item Que vaut $\int_a^b (uv)'(x) \diff x$.
    \item Exprimer $\int_a^b u(x)v'(x) \diff x$. Cette dernière égalité
      est connue sous le nom de \emph{formule d'intégration par parties}
      sous réserve d'existence dans les conditions définies au 3.
  \end{enumerate}
\end{question}
\begin{solution}
  \begin{enumerate}
    \item $(uv)' = u'v + uv'$
    \item $uv' = (uv)' - u'v$
    \item $uv'$ est intégrable si et seulement si $(uv)'$ et $u'v$ sont
      continues. Par symétrie, on en déduit que $u$ et $v$ doivent être
      dérivables à dérivée continue.
    \item $\int_a^b (uv)'(x) \diff x = \big[ (uv)(x) \big]_a^b$
    \item $\int_a^b u(x)v'(x) \diff x = \big[ (uv)(x) \big]_a^b -
      \int_a^b u'(x)v(x) \diff x$. Cette dernière égalité
      est connue sous le nom de \emph{formule d'intégration par parties}
      sous réserve d'existence dans les conditions définies au 3.
  \end{enumerate}
\end{solution}

On peut désormais, sous de nouvelles conditions, intégrer des produits.
L'exercice suivant permettra de préciser ces conditions et cette mise en
œuvre.

\begin{question}
  À l'aide d'une intégration par parties, calculer :
  \begin{itemize}
    \item $\int_0^1 xe^x \diff x$
    \item $\int_0^1 x^2e^x \diff x$
    \item $\int_1^e x\ln x \diff x$
  \end{itemize}
\end{question}
\begin{solution}
  \begin{itemize}
    \item On pose $u(x) = x$ et $v'(x) = e^x$.  $u'(x) = 1$ et $v(x) =
      e^x$. $\int_0^1 xe^x \diff x = [xe^x]_0^1 - \int_0^1 e^x \diff x =
      e - e + 1 = 1$
    \item On pose $u(x) = x^2$ et $v'(x) = e^x$.  $u'(x) = 2x$ et et
      $v(x) = e^x$. on trouve, de la mêm façon que précédement $\int_0^1
      x^2e^x \diff x = e - 2$.
    \item On pose $u(x) = \ln(x) $ et $v'(x) = x^2$, ainsi $u'(x) =
      \frac1x$ et $v(x) = x^3/3$, mais $u'(x)v(x) = x^2/3$ et donc on
      obtient $\int_1^e x\ln x \diff x = [x^3/3\ln x]_1^e - \int_1^e
      x^2/3 \diff x = \frac{e^3}3 + \frac19 - \frac{e^3}9$.
  \end{itemize}
\end{solution}

On peut aussi appliquer cette formule de calcul à la recherche de
primitives. Démontrons ainsi qu'une primitive de $\ln$ est $x \mapsto
x\ln x - x$.

\pagebreak

\begin{question}
  ~\\[-5mm]
  \begin{enumerate}
    \item Démontrer, en dérivant, l'affirmation précédente.
    \item
      \begin{enumerate}
        \item On pose $u(x) = \ln x$ et $v'(x) = 1$. Donner une
          primivite de $v'$ notée $v$ et $u'$ la dérivée de $u$.
        \item Calculer, à l'aide d'une intégration par parties,
          $\int_e^t 1\times \ln x \diff x$
        \item En déduire la primitive de $\ln$. En quelle valeur
          s'annule-t-elle ?
      \end{enumerate}
  \end{enumerate}
\end{question}
\begin{solution}
  \begin{enumerate}
    \item $x\ln x - x $ se dérive en $\ln x + x\frac1x - 1 = \ln x$
    \item
      \begin{enumerate}
        \item $v(x) = x$, $u'(x) = \frac1x$
        \item On applique la formule
        \item La primitive s'annule en $e$.
      \end{enumerate}
  \end{enumerate}
\end{solution}

\section{Changement de variable}

Dans certains cas, et c'est d'ailleurs ce qui se passe avec les formes
du «deuxième» type dans le cours, il est pratique de passer par une
fonction intermédiaire. Donnons sans plus attendre le théorème en
question. Le premier exercice de cette partie constituera la preuve de
ce théorème et les exercices suivants en seront des exemples
d'applications.

\begin{definition}
  Une application $\varphi$ continue bijective, à dérivée continue et
  bijective s'appelle un $\mathcal{C}^1$-difféomorphisme.
\end{definition}

\begin{theoreme}
  Soit $f$ une fonction continue et $\varphi$ un
  $\mathcal{C}^1$-difféomorphisme telle que $\varphi(\interff{a;b})
  \subset \mathcal{D}_f$. \[ \int_{\varphi(a)}^{\varphi(b)}f(x) \diff x
  = \int_a^bf(\varphi(t))\varphi'(t) \diff t.\]
\end{theoreme}

Commençons par la démonstration de ce théorème.

\begin{question}
  Soit $F$ une primitive de $f$ sur $\mathcal{D}_f$.
  \begin{enumerate}
    \item Justifier de l'existence de $F$.
    \item Rappeller la formule de calcul de la dérivée d'une fonction
      composée.

      L'appliquer à la fonction $F\circ \varphi$.
    \item Intégrer l'égalité précédente à droite et à gauche entre $a$
      et $b$.
    \item Écrire le membre de droite sous la forme de la différence des
      valeurs extrémales de la primitive sur l'intervalle
      $\interff{a;b}$.
    \item Écrire cette dernière égalité sous la forme d'une nouvelle
      intégrale.
  \end{enumerate}
\end{question}
\begin{solution}
  \begin{enumerate}
    \item $F$ existe car on a supposé $f$ continue.
    \item $[f(u(x))]' = u'(x)f'(u(x))$

      Avec la notation $F\circ \varphi$, on a $(F\circ \varphi)' = \varphi'
      \times F'\circ \varphi$.
    \item On peut donc écrire $\int_a^b (F\circ \varphi)'(x) \diff x =
      \int_a^b \varphi'(x) \times F'\circ \varphi(x) \diff x$
    \item $\int_a^b \varphi'(x) \times F'\circ \varphi(x) \diff x =
      F\circ \varphi(b) - F\circ \varphi(a) =
      [F(x)]_{\varphi(a)}^{\varphi(b)}$
    \item $\int_a^b \varphi'(x) \times F'\circ \varphi(x) \diff x =
      \int_{\varphi(a)}^{\varphi(b)} F(x) \diff x$.
  \end{enumerate}
\end{solution}

Ce résultat une fois acquis, les exercices suivants de difficulté
croissante permettront de mettre en œuvre cette méthode de calcul.

\begin{question}
  Calculer $\int_0^1 \frac1{1+\sqrt{x}}\diff x$.

  \emph{On pourra poser $y = \sqrt{x}$.}
\end{question}
\begin{solution}
  On pose $y = \sqrt{x}$, d'où $y^2 = x$ sur $[0;1]$ et $\diff x = 2 y
  \diff y$. $\frac1{1+\sqrt{x}}\diff x$ devient donc $\frac{2y}{1+y}
  \diff y$
  
  On peut ensuite intégrer par parties et on trouve $2 - \ln(2)$.
\end{solution}

\pagebreak

\begin{question}~\\[-5mm]
  \begin{enumerate}
    \item \begin{enumerate}
        \item Justifier par un raisonnement géométrique que $\int_0^1
          \sqrt{1 - x^2} \diff x = \frac{\pi}4$.
        \item En utilisant le changement de variable $x = \sin y$,
          retrouver ce résultat par le calcul.

          \emph{$\sin' = \cos$}
      \end{enumerate}
    \item \begin{enumerate}
        \item On définit, pour tout $x$ réel les fonctions $\cosh x =
          \frac{e^x + e^{-x}}2$ et $\sinh x = \frac{e^x - e^{-x}}2$.

          Calculer $\cosh'$ et $\sinh'$. En déduire les primitives de
          $\cosh$ et $\sinh$.
        \item Calculer $\int_0^1 \sqrt{1 + x^2} \diff x$
      \end{enumerate}
  \end{enumerate}
\end{question}
\begin{solution}
  \begin{enumerate}
    \item
      \begin{enumerate}
        \item La courbe est un quart de cercle et l'intégrale est l'aire
          sous la courbe.
        \item On pose $x = \sin t$. $\sqrt{1 - x^2} \diff x = \sqrt{1 -
          \sin^2 t} \diff \sin t = \cos t \cos t \diff t $ Pour $x = 0$,
          on ttrouve $t = 0$, et pour $x = 1$, on trouve $t = \pi$. On a
          donc $\int_0^1 \sqrt{1 - x^2} \diff x = \int_0^{\pi} \cos^2 t
          \diff t$. On en déduit alors le résultat.
      \end{enumerate}
    \item
      \begin{enumerate}
        \item $\cosh' = \sinh$ et $\sinh' = \cosh$
        \item Le calcul est très similaire au précédent.
      \end{enumerate}
  \end{enumerate}
\end{solution}

On peut ensuite retrouver un résultat intéressant : l'aire d'un disque.

\begin{question}
  L'équation d'un cercle de rayon $r$ est donnée par la relation $x^2 +
  y^2 = r^2$.

  \begin{enumerate}
    \item Exprimer $y$ en fonction de $x$.
    \item Justifier que $\mathcal{A} = 2\int_{-r}^r \sqrt{r^2 - x^2}
      \diff x$ est l'aire d'un cercle.
    \item Calculer $\mathcal{A}$ à l'aide du changement de variable $x =
      r\cos y$.
    \item Conclure.
  \end{enumerate}
\end{question}
\begin{solution}
  \begin{enumerate}
    \item On trouve $y^2 = \sqrt{r^2 - x^2}$ soit encore $y = \pm
      \sqrt{r^2 - x^2}$.
    \item On s'intéresse donc à la partie positive et on doublera le
      résultat ainsi obtenu.
    \item Le changement de variable proposé donne $2r^2 \int_{-\pi}^{\pi}
      \cos y \sin y \diff y$
    \item On trouve bien $\mathcal{A} = \pi r^2$.
  \end{enumerate}
\end{solution}

Ce dernier exercice est de loin le plus dur qu'il comporte une intégrale
impropre ou intégrale généralisée. On admettra la proposition-définition
suivante :
\begin{proposition}
  Soit $f$ une fonction intégrable sur $\interfo{1;+\infty}$. \[
    \int_1^{+\infty} f(x) \diff x = \lim_{L\to +\infty} \int_1^L f(x)
  \diff x.\] sous réserve d'existence dans $\R$ de cette limite.
\end{proposition}

\begin{question}
  Soit à calculer $\int_1^{+\infty} \frac{\diff x}{x\sqrt{1+x^2}}$.

  \begin{enumerate}
    \item Justifier que $y = \sqrt{1 + x^2}$ est un changement de
      variable.
    \item Effectuer ce changement de variable.
    \item Vérifier que $\frac1{y^2 - 1} = \frac12 \left(\frac1{y - 1} -
      \frac1{y + 1}\right)$
    \item Écrire $\int_1^L \frac{\diff x}{x\sqrt{1+x^2}}$ à l'aide d'une
      primitive.
    \item Conclure en donnant la valeur exacte de l'intégrale.
  \end{enumerate}
\end{question}
\begin{solution}
  \begin{enumerate}
    \item $y = \sqrt{1 + x^2}$ est un changement de variable car $x
      \mapsto \sqrt{1 + x^2}$ est bien une fonction continue, dérivable
      à dérivée continue.
    \item Le changement de variable donne 
    \item $\frac12 \left(\frac1{y - 1} - \frac1{y + 1}\right) = \frac12
      \left( \frac{y + 1}{y^2 - 1} - \frac{y - 1}{y^2 - 1} \right) =
      \frac1{y^2 - 1}$
  \end{enumerate}
\end{solution}

\section{Intégration dans $\R^n$}

Faire le T.D. 46 p 211.

\pagebreak

\printsolutions
\end{document}

Les sources

https://fr.wikipedia.org/wiki/Int%C3%A9gration_par_changement_de_variable
https://fr.wikiversity.org/wiki/Changement_de_variable_en_calcul_int%C3%A9gral/Exercices/Changement_de_variable_tr%C3%A8s_facile
https://fr.wikiversity.org/wiki/Changement_de_variable_en_calcul_int%C3%A9gral/Exercices/Changement_de_variable_facileo
http://gecif.net/articles/mathematiques/integrale/#1


