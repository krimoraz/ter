\documentclass[12pt,a4paper,french]{article}

\input{../../commons.tex.inc}

\title{Devoir à la maison \no{7}}
\author{\bsc{Jumel}}
\date{pour le 9 mai 2018}

\begin{document}

\noindent\maketitle

\bigskip

\section*{Activités}

\subsection*{Le pêcheur de Poisson}

Dans cette activité, on se pose la question d'obtenir la probabilité
qu'un pêcheur attrape un poisson lors d'une partie de pêche.

\textbf{Partie A}

\begin{enumerate}
  \item Si on considère cette partie de pêche sur l'ensemble de la
    journée du pêcher, quelle est le type de loi de probabilité
    rencontrée ici ?
  \item La rivière où se trouve notre pêcheur est extrèmement
    poissonneuse. Pour fixer les idées, il y'a plus de \np{10000}
    poissons à proximité de notre pécheur.

    Dresser un arbre de probabilité permettant de modéliser trois
    lancers successifs.
  \item Quelle loi permet de modéliser ce genre de situation ?
  \item Justifier alors qu'on peut fixer la probabilité.

    Quelle est l'espérance d'une telle loi de probabilité ?
\end{enumerate}

\textbf{Partie B}

On va estimer qu'en moyenne notre pécheur rapporte \np{1.6} poissons par
journée de pêche. Sa journée de pêche dure en moyenne 8h.

\begin{enumerate}
  \item Quelle est, en moyenne, la probabilité qu'il pêche un poisson
    pendant une heure donnée ?
  \item Même question avec une minute ?
\end{enumerate}

On admet que si on découpe la journée en $n$ unités de temps, la
probabilité de pêcher un poisson dans cet intervalle de temps est
$p = \frac{\lambda}{n}$ et que cette expérience aléatoire est
indépendante des autres, et qu'elle sont répétées $n$ fois.

Ainsi, $\mathcal{P}(X = k)$, la probabilité que le pêcheur
rapporte exactement $k$ poissons est donnée par $\mathcal{P}(X = k)
= \binom{n}{k}p^k(1-p)^{n-k}$

\begin{enumerate}[resume*]
  \item Réécrire la dernière égalité en fonction de $\lambda$ et de $n$.
\end{enumerate}

\textbf{Partie C}

On image qu'on peut découper notre intervalle de temps de plus en plus.

\begin{enumerate}
  \item Traduire cette condition en terme mathématiques.
  \item Justifier que $\binom{n}{k} = \frac{n(n-1)(n-2)\cdots
    (n-k+1)}{k!}$.
  \item Regrouper les termes en $n$ et les autres dans l'expression
    $\binom{n}{k} \left(\frac{\lambda}{n}\right)^k$ et justifier que \\
    $\lim_{n\to+\infty}\frac{n(n-1)(n-2)\cdots (n-k+1)}{n^k} = 1$
  \item Écrire le terme $\left(1 - \frac{\lambda}n \right)^{n-k}$ sous
    la forme d'un produit de deux termes et donner la limite de $\left(1
    - \frac{\lambda}n \right)^{-k}$ quand $n$ tend vers $+\infty$.
  \item On admet que $\left(1 - \frac{\lambda}n \right)^{n} =
    e^{-\lambda}$. Donner les arguments qui permettent de conclure.
\end{enumerate}

On a obtenu, par passage à la limite que $\mathcal{P}(X = k) =
\frac{e^{-\lambda}\lambda^k}{k!}$. On peut montrer que l'espérance de
cette loi est la même que pour la loi binomiale dont elle est issue.

Cette loi s'appelle la loi de Poisson (du mathématicien Simeon-Denis
Poisson, rien à voir avec le milieu aquatique) et permet de modéliser le
passage («rare») d'un objet : on peut ainsi modéliser en probabilité le
passage d'une voiture dans une rue, mais pas sur le périphérique
parisien. On note souvent $X\leadsto \mathcal{P}(\lambda)$ cette loi
d'espérance $\lambda$.

\subsection*{Avec les calculatrices}

Dans cette activité, on ne dispose que d'une calculatrice permettant de
générer un nombre aléatoire entre 0 et 1. C'est en fait le cas de la
plus part des calculatrices, de la fonction \texttt{rand()} du module
\texttt{random} de Python ou de la fonction \texttt{=ALEA()} des
tableurs.

\textbf{Partie A.} Modélisation du jeu de pile ou face

Proposer un algorithme qui renvoie Pile ou Face en n'utilisant que la
fonction décrite ci-dessus.

Mettre en œuvre cet algorithme sur Algobox, sur la calculatrice puis sur
Python.

On désire écrire une boucle qui permet de répéter 1000 fois l'expérience
et de compter le nombre de Pile. Mettre en œuvre une telle boucle sur
Python.

\emph{Rappels}
\begin{itemize}
  \item Un test s'écrit \texttt{if <condition>:} suivi des actions à
    exécuter dans un bloc indenté.
  \item Une boucle «Pour» s'écrit \texttt{for i in range(k):} suivi des
    actions à exécuter dans un bloc indenté.
  \item Une boucle «Tant que» s'écrit \texttt{while <condition>:} suivi
    des actions à exécuter dans un bloc indenté.
  \item Une fonction se définit par \texttt{def <name>(<variables>):} et
    son bloc d'exécution (indenté) se termine impérativement par
    \texttt{return <variable>|<expression>}.
\end{itemize}

\textbf{Partie B.} Étude approfondie de la fonction qui renvoie un
nombre au hasard entre 0 et 1.

Activité 1 page 378.

\end{document}


