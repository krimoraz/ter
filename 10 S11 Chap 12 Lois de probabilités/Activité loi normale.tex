%vim: ft=tex
%!TEX encoding = UTF-8 Unicode
\documentclass[12pt,french]{article}

\input{../../commons.tex.inc}

\title{Activité loi normale}
\date{mai \the\year}
\author{\bsc{Jumel}}

\begin{document}

\maketitle

\section*{Activité de découverte d'une nouvelle loi : Étude statistiques
des notes du bac de 1000 élèves à la session 2015}

On désire représenter les notes moyennes du bac de 1000 élèves pris au
hasard parmi les élèves qui ont passé le bac l'année dernière.

On dispose pour cela du fichier de données qui est à récupérer à
l'adresse suivante : \url{fichier://a.recuperer}. Ce fichier ressemble à
cela :
\begin{verbatim}
l = [..]

import numpy as np
from matplotlib import pyplot as plt
\end{verbatim}

Éxécutez le et nous continuerons ensuite à travailler dans la console
interactive.

Pour obtenir un histogramme des données, il suffit de taper
successivement \texttt{plt.hist(l)} puis \texttt{plt.show()}.

\begin{enumerate}
  \item Préciser le rôle de chacune de ces deux commandes.
\end{enumerate}

Recommencer l'opération avec \texttt{plt.hist(l,50)} puis 100.

\begin{enumerate}[resume*]
  \item Préciser ce qui se passe. À quoi sert ce deuxième argument ?
\end{enumerate}

On veut désormais trouver une fonction qui approche cet histogramme.

Pour tracer une fonction, il faut d'abord se donner un intervalle $x$
avec la fonction \\
\texttt{x = np.linspace(<début>,<fin>)}, puis calculer
les images avec \texttt{y = f(x)} et enfin tracer et afficher le
résultat avec \texttt{plt.plot(x,y)} et \texttt{plt.show()}.

\begin{enumerate}[resume*]
  \item Tracer la courbe représentative de la fonction racine sur
    $\intv{2}{20}$. On donne la fonction racine : \texttt{np.sqrt()}.
  \item Que permet de faire l'enchainement \texttt{plt.hist(l,50)} ;
    \texttt{plt.plot(x,y)} ; \texttt{plt.show()} ?
\end{enumerate}

Définissons désormais nos propres fonctions. Pour cela, retournons
temporairement dans l'éditeur. Ajoutons y le code suivant :
\begin{verbatim}
def f(x,a,b,c):
    return a*x**2 + b*x + c
\end{verbatim}

Éxécutons de nouveau le script et considérons la fonction $f$.

\begin{enumerate}[resume*]
  \item Combien d'argument prend cette fonction ?
  \item Que renvoie-t-elle ?
  \item Quelles valeurs donner à $a$,$b$ et $c$ de façon à avoir le même
    maximum que l'histogramme et essayer d'approcher l'histogramme par
    une telle courbe.
  \item Cette approche semble-t-elle valide ?
  \item Par essais-successifs sur les valeurs $m$ et $s$, essayer
    d'approcher l'histogramme par la fonction définie par $\phi : x
    \mapsto \frac1{s\sqrt{2\pi}} e^{-\frac{(x - m)^2}{2s^2}}$.
\end{enumerate}

\emph{Indication :}
\begin{itemize}
  \item $\exp \to$ \texttt{np.exp} ;
  \item $\pi \to$ \texttt{np.pi}.
\end{itemize}

Cette dernière fonction s'appelle \textbf{densité de probabilité de
Laplace-Gauss}.


\end{document}
