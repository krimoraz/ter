% vim: set ft=tex:
\documentclass[12pt,french,a4paper]{article}

\input{../commons.tex.inc}

\title{Entraînement sujet type bac}
\date{3 octobre 2017}
\author{}

\begin{document}

\maketitle
\thispagestyle{fancy}

Louis et Arthur jouent à un jeu de société dans lequel il n’y a pas d’égalité.

Les deux joueurs ont la même probabilité de gagner la première partie.

En revanche, si Louis gagne une partie, la probabilité qu’il gagne la suivante est 0,7 ; s’il perd, la probabilité qu’il perde la suivante est 0,9.

$n$ étant un entier naturel non nul, on note $G_n$ l'événement «Louis
gagne la $n$-ième partie».

\textbf{Partie A : Deux parties}

On suppose, ici, que Louis et Arthur font deux parties.
\begin{enumerate}
  \item Écrire l'énoncé à l'aide d'un arbre de probabilités.
  \item Calculer la probabilité que Louis gagne les deux parties.
  \item Démontrer que $P(G_2) = 0,4$.
  \item Sachant que Louis a gagné la deuxième partie, quelle est la
    probabilité qu'il ait gagné la première.
  \item Les événements $G_1$ et $G_2$ sont ils indépendants ?
\end{enumerate}

\textbf{Partie B : plusieurs parties}

On suppose, ici, que les joueurs font plusieurs parties. $n$ étant un
entier naturel non nul, on pose $p_n = P(G_n)$.
\begin{enumerate}
  \item À l'aide de l'énoncé, donner les valeurs de $p_1,
    P_{G_n}(G_{n+1})$ et $P_{\overline{G_n}}(G_{n+1})$.
  \item Montrer que, pour tout entier naturel $n$ non nul, $p_{n+1} =
    0,6p_n +0,1$.
  \item Pour tout entier naturel $n$ non nul, on pose $v_n = p_n -
    0,25$.

    Démontrer que la suite $(v_n)$ est géométrique de raison 0,6.
  \item Exprimer $v_n$ en fonction de $n$, puis $p_n$ en fonction de
    $n$.
  \item Calculer la limite de la suite $(p_n)$ et interpréter ce
    résultat.
\end{enumerate}
\end{document}
