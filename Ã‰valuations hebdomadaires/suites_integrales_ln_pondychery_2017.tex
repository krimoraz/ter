%!TEX encoding = UTF-8 Unicode
\documentclass[10pt]{article}
\usepackage[T1]{fontenc}
\usepackage[utf8]{inputenc}
\usepackage{fourier}
\usepackage[scaled=0.875]{helvet}
\renewcommand{\ttdefault}{lmtt}
\usepackage{makeidx}
\usepackage{amsmath,amssymb}
\usepackage{fancybox}
\usepackage[normalem]{ulem}
\usepackage{pifont}
\usepackage{lscape}
\usepackage{multicol}
\usepackage{mathrsfs}
\usepackage{tabularx}
\usepackage{multirow}
\usepackage{textcomp} 
\newcommand{\euro}{\eurologo{}}
%Tapuscrit : Denis Vergès
\usepackage{pst-plot,pst-tree,pstricks,pst-node,pst-text}
\usepackage{pst-eucl}
\usepackage{pstricks-add}
\newcommand{\R}{\mathbb{R}}
\newcommand{\N}{\mathbb{N}}
\newcommand{\D}{\mathbb{D}}
\newcommand{\Z}{\mathbb{Z}}
\newcommand{\Q}{\mathbb{Q}}
\newcommand{\C}{\mathbb{C}}
\setlength{\textheight}{23.5cm}
\setlength{\voffset}{-1.5cm}
\newcommand{\vect}[1]{\mathchoice%
{\overrightarrow{\displaystyle\mathstrut#1\,\,}}%
{\overrightarrow{\textstyle\mathstrut#1\,\,}}%
{\overrightarrow{\scriptstyle\mathstrut#1\,\,}}%
{\overrightarrow{\scriptscriptstyle\mathstrut#1\,\,}}}
\renewcommand{\theenumi}{\textbf{\arabic{enumi}}}
\renewcommand{\labelenumi}{\textbf{\theenumi.}}
\renewcommand{\theenumii}{\textbf{\alph{enumii}}}
\renewcommand{\labelenumii}{\textbf{\theenumii.}}
\def\Oij{$\left(\text{O},~\vect{\imath},~\vect{\jmath}\right)$}
\def\Oijk{$\left(\text{O},~\vect{\imath},~\vect{\jmath},~\vect{k}\right)$}
\def\Ouv{$\left(\text{O},~\vect{u},~\vect{v}\right)$}
\makeindex
\usepackage{fancyhdr}
\usepackage[dvips]{hyperref}

\usepackage[frenchb]{babel}
\usepackage[np]{numprint}
\begin{document}
\setlength\parindent{0mm}
%\rhead{\textbf{A. P{}. M. E. P{}.}}
%\lhead{\small Baccalauréat S}
%\lfoot{\small{Pondichéry}}
%\rfoot{\small{22 avril 2016}}
\renewcommand \footrulewidth{.2pt}
\pagestyle{fancy}
\thispagestyle{empty}
Une entreprise spécialisée dans les travaux de construction a été mandatée pour percer un tunnel à flanc de montagne. 

Après étude géologique, l'entreprise représente dans le plan la situation de la façon suivante : dans un repère orthonormal, d'unité 2~m, la zone de creusement est la surface délimitée par l'axe des abscisses et la courbe 
$\mathcal{C}$. 

\begin{center}
\psset{unit=2cm}
\begin{pspicture}(-3,-0.2)(3,3.2)

\psframe[fillstyle=vlines](-3,0)(3,3)
\pscustom[fillstyle=solid,fillcolor=lightgray]{
\psplot[plotpoints=3000,linewidth=1.25pt,linecolor=blue]{-2.5}{2.5}{13.5 x dup mul 2 mul sub ln}
\psline(2.5,0)(-2.5,0)}
\rput(-2.1,2.5){montagne}\rput(0,1.6){zone de creusement}
\uput[ur](1,2.1){$\mathcal{C}$}\uput[dl](0,0){O}
\uput[d](0.5,0){$\vect{u}$}\uput[l](0,0.5){$\vect{v}$}
\psaxes[linewidth=1.25pt,Dx=10,Dy=10](0,0)(-3,-0.2)(3,3.2)
\psaxes[linewidth=1.25pt,Dx=10,Dy=10]{->}(0,0)(0,0)(1,1)
\end{pspicture}
\end{center} 

On admet que $\mathcal{C}$ est la courbe représentative de la fonction $f$ définie sur l'intervalle $[- 2,5~;~2,5]$ par: 

\[f(x) = \ln \left(- 2x^2 + 13,5\right).\] 

L'objectif est de déterminer une valeur approchée, au mètre carré près, de l'aire de la zone de creusement. 

\bigskip

\textbf{Partie A : Étude de la fonction } \boldmath $f$ \unboldmath 

\medskip

\begin{enumerate}
\item Calculer $f'(x)$ pour $x \in  [- 2,5~;~2,5]$.
\item Dresser, en justifiant, le tableau de variation de la fonction $f$ sur 
$[- 2,5~;~2,5]$. 

En déduire le signe de $f$ sur $[- 2,5~;~2,5]$. 
\end{enumerate}

\bigskip

\textbf{Partie B : Aire de la zone de creusement}

\medskip 

On admet que la courbe $\mathcal{C}$ est symétrique par rapport à l'axe des ordonnées du repère. 

\medskip

\begin{enumerate}
\item La courbe $\mathcal{C}$ est-elle un arc de cercle de centre O ? Justifier la réponse.
\item Justifier que l'aire, en mètre carré, de la zone de creusement est 

$\mathcal{A} = 8\displaystyle\int_0^{2,5}  f(x)\:\text{d}x$. 
\item L'algorithme, donné en annexe, permet de calculer une valeur approchée par défaut de $I = \displaystyle\int_0^{2,5}  f(x)\:\text{d}x$, notée $a$. 

On admet que : $a \leqslant  I \leqslant a + \dfrac{f(0) - f(2,5)}{n}\times  2,5$. 

	\begin{enumerate}
		\item Le tableau fourni en annexe, donne différentes valeurs obtenues pour $R$ et $S$ lors de l'exécution de l'algorithme pour $n = 50$. 

Compléter ce tableau en calculant les six valeurs manquantes. 
		\item En déduire une valeur approchée, au mètre carré près, de l'aire de la zone de creusement. 
	\end{enumerate}
\end{enumerate}
\end{document}