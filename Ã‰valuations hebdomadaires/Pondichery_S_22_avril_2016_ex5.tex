%!TEX encoding = UTF-8 Unicode
\documentclass[10pt]{article}
\usepackage[T1]{fontenc}
\usepackage[utf8]{inputenc}
\usepackage{fourier}
\usepackage[scaled=0.875]{helvet}
\renewcommand{\ttdefault}{lmtt}
\usepackage{makeidx}
\usepackage{amsmath,amssymb}
\usepackage{fancybox}
\usepackage[normalem]{ulem}
\usepackage{pifont}
\usepackage{lscape}
\usepackage{multicol}
\usepackage{mathrsfs}
\usepackage{tabularx}
\usepackage{multirow}
\usepackage{textcomp} 
\newcommand{\euro}{\eurologo{}}
%Tapuscrit : Denis Vergès
\usepackage{pst-plot,pst-tree,pstricks,pst-node,pst-text}
\usepackage{pst-eucl}
\usepackage{pstricks-add}
\newcommand{\R}{\mathbb{R}}
\newcommand{\N}{\mathbb{N}}
\newcommand{\D}{\mathbb{D}}
\newcommand{\Z}{\mathbb{Z}}
\newcommand{\Q}{\mathbb{Q}}
\newcommand{\C}{\mathbb{C}}
\setlength{\textheight}{23.5cm}
\setlength{\voffset}{-1.5cm}
\newcommand{\vect}[1]{\mathchoice%
{\overrightarrow{\displaystyle\mathstrut#1\,\,}}%
{\overrightarrow{\textstyle\mathstrut#1\,\,}}%
{\overrightarrow{\scriptstyle\mathstrut#1\,\,}}%
{\overrightarrow{\scriptscriptstyle\mathstrut#1\,\,}}}
\renewcommand{\theenumi}{\textbf{\arabic{enumi}}}
\renewcommand{\labelenumi}{\textbf{\theenumi.}}
\renewcommand{\theenumii}{\textbf{\alph{enumii}}}
\renewcommand{\labelenumii}{\textbf{\theenumii.}}
\def\Oij{$\left(\text{O},~\vect{\imath},~\vect{\jmath}\right)$}
\def\Oijk{$\left(\text{O},~\vect{\imath},~\vect{\jmath},~\vect{k}\right)$}
\def\Ouv{$\left(\text{O},~\vect{u},~\vect{v}\right)$}
\makeindex
\usepackage{fancyhdr}
\usepackage[dvips]{hyperref}
\hypersetup{%
pdfauthor = {APMEP},
pdfsubject = {Baccalauréat S},
pdftitle = {Pondichéry  22 avril 2016},
allbordercolors = white,
pdfstartview=FitH} 
\usepackage[frenchb]{babel}
\usepackage[np]{numprint}
\begin{document}
\setlength\parindent{0mm}
%\rhead{\textbf{A. P{}. M. E. P{}.}}
%\lhead{\small Baccalauréat S}
%\lfoot{\small{Pondichéry}}
%\rfoot{\small{22 avril 2016}}
\renewcommand \footrulewidth{.2pt}
\pagestyle{fancy}
\thispagestyle{empty}

\textbf{\textsc{Exercice 1} \hfill 5 points}
 
\textbf{Commun  à tous les candidats}

\medskip

On souhaite stériliser une boîte de conserve.

Pour cela, on la prend à la température ambiante $T_0 = 25$\,\degres C et on la place dans un four à
température constante $T_F = 100$\,\degres C.

La stérilisation débute dès lors que la température de la boîte est supérieure à 85\,\degres C.

\smallskip

\emph{Les deux parties de cet exercice sont indépendantes}

\bigskip

\textbf{Partie A : Modélisation discrète}

\medskip

Pour $n$ entier naturel, on note $T_n$ la température en degré Celsius de la boîte au bout de $n$
minutes. On a donc $T_0 = 25$.

Pour $n$ non nul, la valeur $T_n$ est calculée puis affichée par l'algorithme suivant :

\begin{center}
\begin{tabularx}{0.7\linewidth}{|l|X|}\hline
Initialisation : 	&$T$ prend la valeur 25\\ \hline
Traitement :		& Demander la valeur de $n$\\
					&Pour $i$ allant de 1 à $n$ faire\\
					&\hspace{0,5cm}$T$ prend la valeur $0,85 \times  T + 15$\\
					&Fin Pour\\ \hline
Sortie : 			&Afficher T\\ \hline
\end{tabularx}
\end{center}

\begin{enumerate}
\item Déterminer la température de la boîte de conserve au bout de 3 minutes.

Arrondir à l'unité.
\item Démontrer que, pour tout entier naturel $n$, on a $T_n = 100 - 75 \times 0,85^n$.
\item Au bout de combien de minutes la stérilisation débute-elle ?
\end{enumerate}

\bigskip

\textbf{Partie B : Modélisation continue}

\medskip

Dans cette partie, $t$ désigne un réel positif.

On suppose désormais qu'à l'instant $t$ (exprimé en minutes), la température de la boîte est
donnée par $f(t)$ (exprimée en degré Celsius) avec :

\[f(t) = 100 - 75\text{e}^{- \frac{\ln 5}{10}t}.\]

\begin{enumerate}
\item 
	\begin{enumerate}
		\item Étudier le sens de variations de $f$ sur $[0~;~+ \infty[$.
		\item Justifier que si $t \geqslant 10$ alors $f(t) \geqslant 85$.
	\end{enumerate}
\item Soit $\theta$ un réel supérieur ou égal à 10.
		
On note $\mathcal{A}(\theta)$ le domaine délimité par les droites d'équation $t =  10,\: t = \theta,\:$
		 
$y = 85$ et la courbe représentative $\mathcal{C}_f$ de $f$.
		
On considère que la stérilisation est finie au bout d'un temps $\theta$, si l'aire, exprimée en unité
d'aire du domaine $\mathcal{A}(\theta)$ est supérieure à $80$.

\begin{center}
\psset{xunit=0.35cm,yunit=0.05cm}
\begin{pspicture}(-0.5,-5)(32,110)
\multido{\n=0+5}{7}{\psline[linestyle=dashed,linewidth=0.3pt](\n,0)(\n,110)}
\multido{\n=0+5}{21}{\psline[linestyle=dashed,linewidth=0.3pt](0,\n)(32,\n)}
\psaxes[linewidth=1.25pt,Dx=5,Dy=10]{->}(0,0)(-0.5,-5)(32,110)
\psaxes[linewidth=1.25pt,Dx=5,Dy=10](0,0)(0,0)(32,110)
\psplot[plotpoints=3000,linewidth=1.25pt,linecolor=blue]{0}{32}{100 2.71828 5 ln 10 div x mul  neg exp 75 mul sub}
\uput[u](29,0){temps (en minutes)}
\uput[r](0,105){température (en degré Celsius)}
\uput[u](2,46){\blue $\mathcal{C}_f$}
\psline[linestyle=dashed,linewidth=1.5pt](0,85)(32,85)
\uput[d](27.5,85){$y = 85$}
\end{pspicture}
\end{center}
\medskip
	\begin{enumerate}
		\item Justifier, à l'aide du graphique donné en annexe , que l'on a $\mathcal{A}(25) > 80$.
		%\item Justifier que, pour $\theta \geqslant 10$, on a $\mathcal{A}(\theta) = 15(\theta - 10) - 75 \displaystyle\int_{10}^{\theta} \text{e}^{- \frac{\ln 5}{10}t}\:\text{d}t$.
		%\item La stérilisation est-elle finie au bout de 20 minutes ?
	\end{enumerate}
\end{enumerate}

\vspace{1cm}

\begin{center}
\begin{flushleft}\textbf{EXERCICE 1}\end{flushleft}

\bigskip

\psset{xunit=0.35cm,yunit=0.075cm}
\begin{pspicture}(-0.5,-5)(32,110)
\multido{\n=0+5}{7}{\psline[linestyle=dashed,linewidth=0.5pt](\n,0)(\n,110)}
\multido{\n=0+5}{21}{\psline[linestyle=dashed,linewidth=0.5pt](0,\n)(32,\n)}
\psaxes[linewidth=1.25pt,Dx=5,Dy=10]{->}(0,0)(-0.5,-5)(32,110)
\psplot[plotpoints=3000,linewidth=1.25pt,linecolor=blue]{0}{32}{100 2.71828 5 ln 10 div x mul  neg exp 75 mul sub}
\uput[u](28,0){temps (en minutes)}
\uput[r](0,105){température (en degré Celsius)}
\uput[u](2,48){\blue $\mathcal{C}_f$}
\psline[linestyle=dashed,linewidth=1.5pt](0,85)(32,85)
\uput[d](27.5,86){$y = 85$}
\uput[dl](0,0){O}
\end{pspicture}
\end{center}
\end{document}
