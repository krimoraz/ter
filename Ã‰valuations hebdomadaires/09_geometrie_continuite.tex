\documentclass[a4paper,12pt,frenchb]{article}

\input{../../commons.tex.inc}

\title{Évaluation : géométrie et continuité}
\author{semaine \no{45} -- 9\up{ieme} semaine de cours}
\date{13 novembre 2017}

\SetWatermarkText{}
\parindent0pt

\begin{document}


\maketitle

\thispagestyle{fancy}

\begin{question}[subtitle={13 juin 2017, centres étrangers, 2 points}]
  Le plan est muni d'un repère $\brk*{O ; \vv{u}; \vv{v}}$.

  Pour tout entier $n ≥ 4$, on considère $P_n$ un polygone régulier à $n$
  côtés, de centre $O$ et dont l'aire est égale à 1. On admet qu'un tel
  polygone est constitué de $n$ triangles superposables à un triangle
  $OA_nA_{n+1}$ donné en, isocèle en $O$. Les sommets du polygone sont
  numérotés de 0 à $n-1$ avec la convention $A_6 = A_0$.

  On note $r = OA_n$ la distance entre le centre $O$ et le sommet $A_n$ d'un
  tel polygone.

  \bigskip
  \textbf{Étude du cas particulier $n=6$}
  \medskip

  \begin{enumerate}
    \item Justifier le fait que le triangle $OA_0A_1$ est équilatéral, et
      que sont aire est $\dfrac16$.
    \item Exprimer en fonction $r_6$ la hauteur du triangle $OA_1A_2$ issue
      du sommet $A_1$.
    \item En déduire que $r_6 = \sqrt{\dfrac{2}{3\sqrt{3}}}$
  \end{enumerate}
\end{question}

\begin{question}
  Citer le théorème des valeurs intermédiaires et son corollaire.
\end{question}

\end{document}
