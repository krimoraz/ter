\documentclass[12pt,a4paper,french]{article}
\usepackage[utf8]{inputenc}
\usepackage[T1]{fontenc}
\usepackage{babel}
\usepackage[thmmarks]{ntheorem}
\usepackage{amsmath}
\usepackage{amsfonts}
\usepackage{amssymb}

\usepackage{array}

\usepackage{lmodern}
\usepackage{kpfonts}
\usepackage{cmbright}

\usepackage[bookmarks=false,colorlinks,linkcolor=blue,pdfusetitle]{hyperref}

\pdfminorversion 7
\pdfobjcompresslevel 3

\usepackage{tabularx}
\usepackage[autolanguage,np]{numprint}
\usepackage{enumitem}

\usepackage{tipfr}
\usepackage{pgf}
\usepackage{tikz}
\usepackage{tkz-euclide}
\usetkzobj{all}
%\usetikzlibrary{hobby}
\usepackage{tkz-tab}

\usepackage[top=1.7cm,bottom=2cm,left=2cm,right=2cm]{geometry}

\usepackage{lastpage}

\usepackage{esvect}
\usepackage{marginnote}

\usepackage{wrapfig}

\usepackage[defaultlines=5,all]{nowidow}


\makeatletter
\renewcommand{\@evenfoot}%
        {\hfil \upshape \small page {\thepage} de \pageref{LastPage}}
\renewcommand{\@oddfoot}{\@evenfoot}

\renewcommand{\maketitle}%
{\framebox{%
    \begin{minipage}{1.0\linewidth}%
      \begin{center}%
        \Large \@title ~-- \@author \\%
        \@date%
      \end{center}%
    \end{minipage}}%
  \normalsize%
  %\vspace{1cm}%
}

\pgfdeclarepatternformonly{mes_hachures}
{\pgfpoint{-0.1cm}{-0.1cm}}
{\pgfpoint{0.9cm}{0.5cm}}
{\pgfpoint{0.8cm}{0.4cm}}
{\pgfpathmoveto{\pgfpointorigin}
  \pgfpathlineto{\pgfpoint{0.8cm}{0.4cm}}
\pgfusepath{stroke}}

%Des macros pour les noms d'ensmbles
\newcommand{\R}{\mathbf{R}}
\newcommand{\Q}{\mathbf{Q}}
\newcommand{\Z}{\mathbf{Z}}
\newcommand{\C}{\mathbf{C}}
\newcommand{\N}{\mathbf{N}}

\newcommand{\norme}[1]{\left\lVert #1 \right\rVert}
\newcommand{\abs}[1]{\left\lvert #1 \right\rvert}

%Une macro récursive pour l'intérieru des vecteurs
%http://tex.stackexchange.com/questions/19693/arguments-of-custom-commands-as-comma-separated-list

\newcommand\vecteur[2][\\]{%
    \global\def\my@delim{#1}%
    \left(\negthinspace\begin{matrix}
        \my@vector #2,\relax\noexpand\@eolst%
    \end{matrix}\right)}

%Une macro pour les vecteurs
\def\my@vector #1,#2\@eolst{%
   \ifx\relax#2\relax
      #1
   \else
      #1\my@delim
      \my@vector #2\@eolst%
   \fi}

%Une macro récursive pour mettre formater l'intérieur des intervalles
\def\my@intervalle #1;#2\@eolst{%
  \ifx\relax#2\relax
    #1
  \else
    \my@intervalle #2\@eolst%
  \fi}

%Quatre macros pour les quatres types d'intervalles
\newcommand{\interff}[1]{%
  \left[\my@intervalle #1;\relax\noexpand\@eolst%
  \right]
}
\newcommand{\interfo}[1]{%
  \left[\my@intervalle #1;\relax\noexpand\@eolst%
  \right[}
\newcommand{\interof}[1]{%
  \left]\my@intervalle #1;\relax\noexpand\@eolst%
  \right]}
\newcommand{\interoo}[1]{%
  \left]\my@intervalle #1;\relax\noexpand\@eolst%
  \right[}

\makeatother


\usepackage{mdframed}

\theoremstyle{break}
\newtheorem{definition}{Définition}
\newtheorem{propriete}{Propriété}
\newtheorem{corollaire}{Corollaire}
\newtheorem{propdef}{Propriété - Définition}
\newtheorem{theoreme}{Théorème}
\theoremstyle{plain}
\theorembodyfont{\normalfont}
\newtheorem{exerciceT}{Exercice}
\theoremstyle{nonumberplain}
\newtheorem{remarque}{Remarque}
\newtheorem{notation}{Notation}
\newtheorem{probleme}{Problème}
\theoremsymbol{\ensuremath{\blacksquare}}
\newtheorem{preuve}{Preuve}
\theoremsymbol{}
\theoremstyle{nonumberbreak}
\newtheorem{exemple}{Exemple}

\newenvironment{exercice}{\begin{framed}\begin{exerciceT}}{\end{exerciceT}\end{framed}}

\setlength{\parsep}{0pt}
\setlength{\parskip}{5pt}
\setlength{\parindent}{0pt}
\setlength{\itemsep}{7pt}

\setlist{noitemsep}
%\setlist[1]{\labelindent=\parindent} % < Usually a good idea
\setlist[itemize]{leftmargin=*}
\setlist[itemize,1]{label=$\triangleright$}
\setlist[enumerate]{labelsep=*, leftmargin=2.5pc}
\setlist[enumerate,1]{label=\arabic*., ref=\arabic*}
\setlist[enumerate,2]{label=\emph{\alph*}),
ref=\theenumi.\emph{\alph*}}
\setlist[enumerate,3]{label=\roman*), ref=\theenumii.\roman*}
\setlist[description]{font=\sffamily\bfseries}

\usepackage{multicol}
\setlength{\columnseprule}{0pt}

\usepackage[]{exsheets}
\SetupExSheets{
  headings = block,
  question/pre-hook = \mdframed,
  question/post-hook = \endmdframed,
}
\usepackage{exsheets-listings}

\everymath{\displaystyle\everymath{}}

\title{Devoir à la maison \no{2} : Analyse : continuité et dérivabilité}
\author{\bsc{Ts 3}}
\date{pour le 17 novembre 2016}

\begin{document}

\maketitle

\bigskip

Ce devoir à la maison se propose de démontrer quelques théorèmes qui
sont à la limite du programme : une partie d'entre-eux vous sont
présentés en cours, mais sont admis. Une partie des théorèmes ne sont
ici présents que pour démontrer les théorèmes suivants et donc ceux du
cours.

En revanche, comme il faut bien partir de quelque part, la première
propriété sera admise comme \emph{axiome}. Il est d'ailleurs possible de
la démontrer en partant d'un autre axiome, c'est ce qui est fait dans
l'enseignement supérieur.

\section{Sur les suites}

Comme annoncé précédement, on admet la proposition suivante :

\begin{propriete}
  Toute partie non-vide majorée de $\R$ admet une borne supérieure.
\end{propriete}

\begin{theoreme}
  Toute suite croissante majorée est convergente.
\end{theoreme}


\begin{question}[name=Démonstration,subtitle={Toute suite croissante
  majorée est convergente}]

  Soit $(u_n)$ une suite croissante majorée par un réel $A$.

  \begin{enumerate}
    \item Pourquoi peut on dire que $U = \{ u_n | n\in \N \}$ est majoré
      ?
    \item Notons $\ell$ la borne supérieure de $U$.

      \begin{enumerate}
        \item Soit $\varepsilon > 0$. En utilisant le fait que la borne
          supérieure de $U$ est un élément de $U$ tel qu'il n'y en ait
          pas de plus grand que lui, caractériser le fait que pour un
          rang donné $N$, on peut trouver une valeur proche de la borne
          supérieure.
        \item En utilisant ensuite la croissance de la suite $(u_n)$
          démontrer que pour tout $n$ supérieur à $N$, $\ell
          -\varepsilon < u_N \leqslant u_n \leqslant \ell$.
        \item Conclure.
      \end{enumerate}
  \end{enumerate}
\end{question}
\begin{solution}
  Soit $(u_n)$ une suite croissante majorée par un réel $A$.

  \begin{enumerate}
    \item $U = \{ u_n | n\in \N \}$ est majoré par hypothèse (voir P1)
    \item
      \begin{enumerate}
        \item Notons $\ell$ la borne supérieure de $U$. On a d'une part,
          pour tout $N\in \N$, $u_N \leqslant \ell$. D'autre part, pour
          $\varepsilon > 0$ fixé, on peut trouver $u_N$ tel que $\ell -
          \varepsilon < u_N$.
        \item $(u_n)$ est croissante, on peut donc dire que pour tout
          $n \geqslant N$, on a $\ell - \varepsilon < u_N
          \leqslant u_n \leqslant \ell < \ell + \varepsilon$.
        \item L'implication précédente étant vraie pour tout
          $\varepsilon > 0$ et pour tout entier $n \geqslant N$ on a
          bien la définition d'une suite convergente.
      \end{enumerate}
  \end{enumerate}
\end{solution}


\begin{definition}[Suites adjacentes]
  Soient $(u_n)$ et $(v_n)$ deux suites réelles vérifiant :
  \begin{enumerate}[label=(\roman*)]
    \item $(u_n)$ est croissante et $(v_n)$ décroissante ;
    \item $(v_n - u_n)$ converge vers 0.
  \end{enumerate}

  On dit alors que ces deux suites sont \emph{adjacentes}
\end{definition}

\begin{theoreme}
  Si deux suites sont adjacentes, alors
  \begin{enumerate}[label=\roman*)]
    \item les suites sont convergentes ;
    \item la limite de ces deux suites est la même.
  \end{enumerate}
\end{theoreme}

\begin{question}[name=Démonstration,subtitle={suites adjacentes}]
  Soient $(u_n)$ $(v_n)$ deux suites adjacentes.
  \begin{enumerate}
    \item Convergences des suites.
      \begin{enumerate}
        \item Quel est le sens de variation de $(v_n - u_n)$ ?
        \item Quel est le signe des termes de $(v_n - u_n)$ ?
        \item Démontrer que pour tout $n\in \N$, \[ a_0 \leqslant a_n
          \leqslant b_n \leqslant b_0 \]
        \item Justifier ainsi qu'elles convergent.
      \end{enumerate}
    \item Unicité de la limite. On suppose désormais que $u_n$ converge
      vers $\ell$ et $(v_n)$ vers $\ell'$ deux limites finies.
      \begin{enumerate}
        \item Justifier que $\ell - \ell' = 0$.
        \item Conclure.
      \end{enumerate}
  \end{enumerate}
\end{question}
\begin{solution}
  Soient $(u_n)$ $(v_n)$ deux suites adjacentes.
  \begin{enumerate}
    \item Convergences des suites.
      \begin{enumerate}
        \item $(v_n - u_n)$ est décroissante par (i).
        \item Comme $(v_n - u_n)$ converge vers 0 en étant décroissante
          (ii) + point précédent, on a $\forall n \in \N,\ v_n - u_n
          \geqslant 0$
        \item Le point (i) de la définition entraîne que $a_0 \leqslant
          a_n$ et $b_n \leqslant b_0$. L'inégalité centrale est le
          résultat du point précédent.
        \item $(a_n)$ est croissante majorée par $b_0$ donc elle
          converge ; $(b_n)$ est décroissante minorée par $a_0$ donc
          elle converge.
      \end{enumerate}
    \item Unicité de la limite. On suppose désormais que $u_n$ converge
      vers $\ell$ et $(v_n)$ vers $\ell'$ deux limites finies.
      \begin{enumerate}
        \item (ii) entraine que $\ell - \ell' = 0$.
        \item Les deux suites ont donc la même limite.
      \end{enumerate}
  \end{enumerate}
\end{solution}

\textsc{Remarque :} On peut montrer que la propriété 1 est équivalente à
l'un ou l'autre de ces énoncés. Il est plus simple de l'admettre.

Le théorème qui suite est un théorème assez fondamental de l'analyse et
plus particulièrement de la topologie. On peut le démontrer de bien des
manières, certaines faisant appel à des notions largement hors du
programme. La démonstration guidée laissée en exercice peut se faire dès
à présent avec les «outils» de la classe de Terminale.

Signalons que l'énoncé de ce théorème est assez bref, et qu'il sera
précisé ensuite.

\begin{theoreme}[Bolzano-Weierstrass]
  De toute suite bornée, on peut extraire une sous-suite convergente.
\end{theoreme}

\begin{definition}
  \begin{itemize}
    \item On appelle \emph{fonction d'extraction} une fonction $\varphi
      : \N \to \N$ tel que pour tout $n \in \N$, $\varphi(n) \geqslant
      n$.
    \item On appelle \emph{suite extraite} ou \emph{sous-suite} de
      $(u_n)$ une suite de la forme $(u_{\varphi(n)})$ où $\varphi$ est
      une fonction d'extraction.
  \end{itemize}
\end{definition}

\begin{question}[name=Démonstration,subtitle={BW par dichotomie}]
  Soit $(u_n)$ une suite bornée.
  \begin{enumerate}
    \item Une construction par dichotomie qui sera utile plus tard.
      \begin{enumerate}

        \item Traduire le fait que $(u_n)$ est bornée.
        \item On définit les suites $(a_n)$ et $(b_n)$ par récurrence de
          la façon suivante : \[ a_0 = a\ \ ; \ \ b_0 = b \] et pour
          tout $p$ entier naturel en prenant l'un des deux choix :
          \begin{itemize}
            \item $a_{p+1} = a_p$ et $b_{p+1} = \frac{a_p + b_p}2$
            \item $a_{p+1} = \frac{a_p + b_p}2$ et $b_{p+1} = b_p$
          \end{itemize}
          de façon à ce qu'il y ait toujours une infinité de termes dans
          $\interff{a_p,b_p}$.

          Faire un schéma de la situation.
        \item Quelle est la nature des deux suites $(a_n)$ et $(b_n)$ ?
        \item Conclure.
      \end{enumerate}
    \item Construction de la suite extraite.
      \begin{enumerate}
        \item Combien de termes sont dans les intervalles
          $\interff{a_p,b_p}$ ?
        \item Peut-on choisir des termes croissants dans les intervalles
          de la forme $\interff{a_{p+1},b_{p+1}}$ ?
        \item Peut-on ainsi définir une suite extraite ?
        \item En notant $\varphi(n) = p$, encadrer cette suite extraite
          pour tout $p$ entier naturel.
        \item Conclure.
      \end{enumerate}
  \end{enumerate}
\end{question}
\begin{solution}
  \begin{enumerate}
    \item
      \begin{enumerate}
        \item $(u_n)$ bornée $\iff \exists (a,b) \in \R^2,\ \forall n
          \in \N,\ a \leqslant u_n \leqslant b$
        \item
          \begin{center}
            \begin{tikzpicture}
              \draw (-2.5,0) -- (-2,0) node [circle] {} -- (0,0) node
              [circle] {} -- (2,0) node [circle] {} -- (2.5,0) ;
              \draw (-2,0) node [below] {$a_n$} ;
              \draw (+2,0) node [below] {$b_n$} ;
              \draw (0,0) node [below] {$\frac{a_n+b_n}2$} ;
            \end{tikzpicture}
          \end{center}
        \item $(a_n)$ et $(b_n)$ sont deux suites adjacentes.
        \item $(a_n)$ et $(b_n)$ sont convergentes.
      \end{enumerate}
    \item
      \begin{enumerate}
        \item Il y a une infinité de termes dans les intervalles
          $\interff{a_p,b_p}$.
        \item Oui, car on a toujours $\interff{a_p,b_p} \subset
          \interff{a_{p+1},b_{p+1}}$.
        \item Le choix des termes précédents définit précisément une
          suite extraite.
        \item On a, pour tout $p$ entier naturel, $a_p \leqslant u_p
          \leqslant b_p$.
        \item En utilisant le théorème d'encadrement des suites, on a le
          résultat attendu.
      \end{enumerate}
  \end{enumerate}
\end{solution}

On pourrait, pour terminer cette partie, rappeler la définition d'une
suite de Cauchy et montrer l'équivalence entre suites de Cauchy et
suites convergentes, mais cette notion est hors-progamme et n'apporte
pas grand chose en Terminale.

\section{Continuité des fonctions}

Commençons pas redonner une caractérisation des fonctions continues.

\begin{theoreme}
  Soit $f$ une fonction continue sur $I$ et $(u_n)$ une suite d'éléments
  de $I$ de limite $\ell$

  Alors $\lim_{n\to +\infty}f(u_n) = f(\ell)$
\end{theoreme}

\begin{theoreme}[des valeurs intermédiaires]
  Soit $f$ une fonction continue sur $\interff{a,b}$.

  Alors pour tout réel $y $ compris entre $f(a)$ et $f(b)$ il existe
  $c\in \interff{a,b}$ tel que $f(c) = y$.
\end{theoreme}

\begin{question}[name=Démonstration,subtitle={TVI par dichotomie}]
  Supposons (par exemple) que $f(a) < f(b)$.
  \begin{enumerate}
    \item Écrire l'hypothèse $y $ compris entre $f(a)$ et $f(b)$ sous
      forme d'une inégalité.
    \item Proposer deux suites définies par récurrence pour découper en
      deux l'intervalle $\interff{a,b}$.
    \item Caractériser les suites ainsi définies.
    \item Justifier que \[ f(a_n) \leqslant y \leqslant
      f(b_n) \] où $(a_n)$ et $(b_n)$ sont les suites définies plus
      haut.
    \item Étudier la convergence des suites $(a_n)$ et $(b_n)$.
    \item En utilisant le théorème 4, en déduire que $f(c) = y$.
  \end{enumerate}
\end{question}
\begin{solution}
  \begin{enumerate}
    \item $f(a) \leqslant y \leqslant f(b)$
    \item On peut prendre les suites $(a_n)$ et $(b_n)$ définies par
      $a_0 = a$, $b_0 = b$ et pour tout $n$ entier naturel,
      \begin{itemize}
        \item si $f\left(\frac{a_n + b_n}{2}\right) \leqslant y$,
          $a_{n+1} = a_n$ et $b_{n+1} = \frac{a_n + b_n}2$ ;
        \item sinon $a_{n+1} = \frac{a_n + b_n}2$ et $b_{n+1} = b_n$
      \end{itemize}
    \item Les suites ainsi définies sont des suites adjacentes. En
      effet, $(a_n)$ est croissante, $(b_n)$ décroissante et pour tout
      $n$, $b_{n+1} - a_{n+1} = \frac{b_n - a_n}2$ distance qui
      peut-être rendue aussi petite que l'on veut.
    \item Cette inégalité est justifiée par construction des suites.
    \item $(a_n)$ et $(b_n)$ sont adjacentes donc elles convergent vers
      une limite commune $c$.
    \item $f$ étant continue, on obtient, par passage à la limite dans
      l'inégalité du 4. $f(c) \leqslant y \leqslant f(c)$ donc $c =
      f(y)$.
  \end{enumerate}
\end{solution}

\begin{question}
  Proposer un algorithme d'encadrement d'une solution, utilisant la
  méthode par dichotomie, vue dans les preuves précédentes, dans le cas
  d'une fonction strictement monotone sur un intervalle $\interff{a,b}$.

  On écrira l'algorithme en langage calculatrice, en \textsc{Python} ou
  en pseudo-code.
\end{question}
\begin{lstsolution}[listings={language=Python},
  pre={Voici une correction en Python pour le calcul de racine de 2.}]
  def f(x):
      return x*x

  a,b = 1,2
  y = 2

  def g(x):
      return f(x) - y

  while (b-a) > 10**(-3):
      c = (a+b)/2
      if g(a) * g(c) <= 0:
              b=c
      else:
              a=c

  print(a,b)
\end{lstsolution}

On peut déduire de ce théorème en utilisant le théorème de
Bolzano-Weierstrass le théorème des bornes.

\begin{corollaire}
  L'image par une fonction continue d'un intervalle est un intervalle.
\end{corollaire}

\begin{corollaire}[des bornes]
  L'image par une fonction continue d'un segment est un segment.
\end{corollaire}

\section{Des conséquences de la dérivabilité des fonctions}

Une première conséquence, assez évidente est la proposition suivante.

\begin{propriete}
  Une fonction dérivable sur un intervalle $I$ est continue sur $I$.
\end{propriete}

\begin{question}[name=Démonstration]
  \begin{enumerate}
    \item Formuler la version «en un point» de la propriété.
    \item Justifier que cette formulation est équivalente à la
      proposition \[ \forall \varepsilon > 0 \exists \eta > 0 \, \abs{x
        - c} < \eta \implies \abs{\frac{f(x) - f(c)}{x - c} - f'(c)} <
      \varepsilon \]
    \item Fixons un $\varepsilon > 0$
      \begin{enumerate}
        \item Écrire l'inégalité finale de l'expression de la limite
          sous la forme d'un encadrement.
        \item Justifier que $\abs{\frac{f(x) - f(c)}{x - c}} \leqslant
          \max\left(\abs{f'(c) + \varepsilon},\abs{f'(c) -
          \varepsilon}\right)$.

          En déduire que $\abs{\frac{f(x) - f(c)}{x - c}} \leqslant
          \abs{f'(c)} + \varepsilon$
        \item Proposer une majoration de $\abs{f(x) - f(c)}$.
        \item Que se passe-t-il si on prend $x$ tel que $\abs{x-c} <
          \min\left(\eta,\frac{\varepsilon}{\abs{f'(c)} + \varepsilon}
            \right)$ ?
        \item Conclure.
      \end{enumerate}
  \end{enumerate}
\end{question}
\begin{solution}
  \begin{enumerate}
    \item On peut dire que $f$ dérivable en $c$ entraîne $f$ continue en
      $c$.
    \item Il s'agit ni plus ni moins que du cumul de la définition de la
      limite et de la définition de la dérivée.
      \begin{enumerate}
        \item On obtient $f'(c) - \varepsilon < \frac{f(x) - f(c)}{x -
          c} < f'(c) + \varepsilon$.
        \item On obtient ceci par passage à la valeur absolue. On en
          déduit le deuxième point en utilisant le fait qu'il s'agisse
          d'un $\max$ et l'inégalité triangulaire : $\forall (x,y) \in
          \R^2,\ \abs{x+y} \leqslant \abs{x} + \abs{y}$.
        \item On a la majoration suivante : $\abs{f(x) - f(c)} \leqslant
          \abs{x - c}\left(\abs{f'(c)}+\varepsilon\right)$.
        \item On a alors $\abs{f(x) - f(c)} \leqslant \varepsilon$
        \item $f$ est continue en $c$ et puisqu'on a raisonné sur un
          point $c$ quelconque, c'est vrai pour tout $c\in I$.
      \end{enumerate}
  \end{enumerate}
\end{solution}

Attention, la réciproque n'est pas vraie.

\begin{question}[subtitle={Réciproque de l'implication dérivable
  entraîne continue}]
  \begin{enumerate}
    \item Formuler la proposition réciproque.
    \item Exhiber un contre-exemple.
    \item Que peut-on dire d'une fonction non dérivable ? Est-ce une
      condition nécessaire ou suffisante ?
  \end{enumerate}
\end{question}

\begin{theoreme}[de Rolle]
  Soit $f$ une fonction continue sur $\interff{a,b}$, dérivable sur
  $\interoo{a,b}$ telle que $f(a) = f(b)$.

  Alors il existe (au moins) un point $c \in \interoo{a,b}$ tel que
  $f'(c) = 0$.
\end{theoreme}

\begin{question}[name=Démonstration,subtitle={du théorème de Rolle}]
  Soit $f$ une fonction continue sur $\interff{a,b}$, dérivable sur
  $\interoo{a,b}$ telle que $f(a) = f(b)$.

  \textsc{Premier cas} : Si $f$ est constante, que peut-on dire de sa
  dérivée ?

  \textsc{Deuxième cas} : Supposons $f$ non constante. Elle admet un
  maximum global (théorème des bornes) différent du minimum global.
  Nommons ce maximum $c$.
  \begin{enumerate}
    \item Pourquoi $c$ ne peut-il être égal à $a$ ou à $b$.
    \item
      \begin{enumerate}
        \item pour $h$ strictement positif et tel que $c + h$
          appartienne à l'intervalle $\interff{a,b}$, quel est le signe
          de $\frac{f(c + h) - f(c)}{h}$.
        \item En déduire que la limite du nombre dérivée à droite est
          négative.
      \end{enumerate}
    \item Reproduire le raisonnement à gauche.
    \item que dire d'un nombre à la fois positif et négatif.
    \item Conclure.
  \end{enumerate}
\end{question}
\begin{solution}
  \textsc{Premier cas} : Si $f$ est constante, sa dérivée est nulle et
  tout $c$ convient.

  \textsc{Deuxième cas} :
  \begin{enumerate}
    \item $c$ est un maximum global et donc $f(c) \neq f(a)$ et $f(c)
      \neq f(b)$ qui sont égaux. Donc $c$ est différent de $a$ et $b$.
    \item
      \begin{enumerate}
        \item pour $h$ strictement positif et tel que $c + h$
          appartienne à l'intervalle $\interff{a,b}$, on a $\frac{f(c +
          h) - f(c)}{h} \leqslant 0$.
        \item On en déduit $f'_d(c) \leqslant 0$.
      \end{enumerate}
    \item En prenant $h$ négatif, on obtient $f'_g(c) \geqslant 0$.
    \item Un nombre à la fois positif et négatif est nul.
    \item $f'(c) = 0$.
  \end{enumerate}
\end{solution}

\begin{theoreme}[des accroissements finis]
  Soit $f$ une fonction continue sur $\interff{a,b}$, dérivable sur
  $\interoo{a,b}$.

  Alors il existe $c \in \interoo{a,b}$ tel que \[ f'(c) = \frac{f(b) -
  f(a)}{b - a}\]
\end{theoreme}

\begin{question}[name=Démonstration,subtitle={des accroissements finis}]
  Quelles hypothèses sont vérifiées par la fonction $\varphi :
  \interff{a,b} \to \R,\ x\mapsto f(x) - \frac{f(b) - f(a)}{b - a}x$.

  Conclure.
\end{question}
\begin{solution}
  $\varphi$ vérifie précisément les hypothèses du théorème de Rolle, il
  existe donc $c \in \interoo{a,b}$ tel que $\varphi'(c) = 0$ d'où on
  déduit que $f'(c) - \frac{f(b) - f(a)}{b - a} = 0$.
\end{solution}

\begin{question}
  À l'aide du théorème des accroissements finis, démontrer l'équivalence
  \begin{enumerate}[label=(\roman*)]
    \item $f$ croissante sur $\interff{a,b} \implies \forall x \in
      \interoo{a,b},\ f'(x) \geqslant 0$
    \item $f$ décroissante sur $\interff{a,b} \implies \forall x \in
      \interoo{a,b},\ f'(x) \leqslant 0$
    \item $f$ constante sur $\interff{a,b} \implies \forall x \in
      \interoo{a,b},\ f'(x) = 0$
  \end{enumerate}
\end{question}
\begin{solution}
  La solution n'est donnée que pour le premier point.

  $\Rightarrow$ : Il suffit de reprendre la définition d'une fonction
  croissante et d'étudier le signe de $\frac{f(b) - f(a)}{b - a}$.

  $\Leftarrow$ : On considère $(x,y) \in \interff{a,b}^2$ tel que $x <
  y$. Graçe au théorème précédent, on peut trouver $c \in \interoo{x,y}
  \subset \interoo{a,b}$ tel que $f(y) - f(x) = f'(c)(y - x)$. Comme
  $f'(c) \geqslant 0$, on en déduit que $f(y) - f(x) \geqslant 0$ et
  donc $f$ est croissante.
\end{solution}

On peut également citer la formule de Taylor à l'ordre 1 pour les
fonctions continues dérivables.

\begin{theoreme}[de Taylor]
 Pour une fonction $f$ dérivable sur $\interff{a,b}$ telle que $f'$ soit
 dérivable sur $\interoo{a,b}$ on a \[ f(b) = f(a) + f'(a)(b-a) +
 \frac{f"(c)}{2}(b-a)^2 \] où $c\in \interoo{a,b}$
\end{theoreme}

\begin{question}[name=Démonstration,subtitle={Taylor}]
  Démontrer ce résultat.
\end{question}
\begin{solution}
  On considère la fonction $\psi$ qui va bien.
\end{solution}

\begin{propriete}[inégalité des accroissements finis]
  Soit $f$ une fonction continue sur $\interff{a,b}$, dérivable sur
  $\interoo{a,b}$ telle que pour tout $x \in \interoo{a,b},\ \abs{f'(x)}
  < k$

  Alors \[ \abs{f(b) - f(a)} < k \abs{b - a}\]
\end{propriete}

\begin{question}[name=Démonstration,subtitle={Inégalité des
  accroissements finis}]
  En appliquant le théorème des accroissements finis, proposer une
  démonstration.
\end{question}

\section*{Références}
\url{http://www.math-info.univ-paris5.fr/~avner/MC1/L1_S1/cours/df/node5.html}

\url{http://www-irma.u-strasbg.fr/~bopp/CAPES/cours/demo-analyse.pdf}

Wikipedia
%\url{http://www.capes-de-maths.com/lecons/lecon66.pdf}

%\end{document}

\pagebreak

\section*{Solutions des exercices}

\vspace{-7mm}

\printsolutions

\end{document}
