\documentclass[12pt,a4paper,french]{article}
\usepackage[utf8]{inputenc}
\usepackage[T1]{fontenc}
\usepackage{babel}
\usepackage{ntheorem}
\usepackage{amsmath}
\usepackage{amsfonts}
\usepackage{amssymb}

\usepackage{array}

\usepackage{kpfonts}

\usepackage[bookmarks=false,colorlinks,linkcolor=blue]{hyperref}

\pdfminorversion 7
\pdfobjcompresslevel 3

\usepackage{tabularx}
\usepackage[autolanguage,np]{numprint}
\usepackage{enumitem}

\usepackage{tipfr}
\usepackage{pgf}
\usepackage{tikz}
\usepackage{tkz-euclide}
\usetkzobj{all}
\usetikzlibrary{hobby}

\usepackage[top=1.7cm,bottom=2cm,left=2cm,right=2cm]{geometry}

\usepackage{lastpage}

\usepackage{esvect}
\usepackage{marginnote}

\usepackage{wrapfig}

\usepackage[defaultlines=5,all]{nowidow}


\makeatletter
\renewcommand{\@evenfoot}%
        {\hfil \upshape \small page {\thepage} de \pageref{LastPage}}
\renewcommand{\@oddfoot}{\@evenfoot}

\renewcommand{\maketitle}%
{\framebox{%
    \begin{minipage}{1.0\linewidth}%
      \begin{center}%
        \Large \@title ~-- \@author \\%
        \@date%
      \end{center}%
    \end{minipage}}%
  \normalsize%
  %\vspace{1cm}%
}

\pgfdeclarepatternformonly{mes_hachures}
{\pgfpoint{-0.1cm}{-0.1cm}}
{\pgfpoint{0.9cm}{0.5cm}}
{\pgfpoint{0.8cm}{0.4cm}}
{\pgfpathmoveto{\pgfpointorigin}
  \pgfpathlineto{\pgfpoint{0.8cm}{0.4cm}}
\pgfusepath{stroke}}

%Des macros pour les noms d'ensmbles
\newcommand{\R}{\mathbf{R}}
\newcommand{\Q}{\mathbf{Q}}
\newcommand{\Z}{\mathbf{Z}}
\newcommand{\C}{\mathbf{C}}
\newcommand{\N}{\mathbf{N}}

\newcommand{\norme}[1]{\left\lVert #1 \right\rVert}
\newcommand{\abs}[1]{\left\lvert #1 \right\rvert}

%Une macro récursive pour l'intérieru des vecteurs
%http://tex.stackexchange.com/questions/19693/arguments-of-custom-commands-as-comma-separated-list

\newcommand\vecteur[2][\\]{%
    \global\def\my@delim{#1}%
    \left(\negthinspace\begin{matrix}
        \my@vector #2;\relax\noexpand\@eolst%
    \end{matrix}\right)}

%Une macro pour les vecteurs
\def\my@vector #1;#2\@eolst{%
   \ifx\relax#2\relax
      #1
   \else
      #1\my@delim
      \my@vector #2\@eolst%
   \fi}

%Une macro récursive pour mettre formater l'intérieur des intervalles
\def\my@intervalle #1;#2\@eolst{%
  \ifx\relax#2\relax
    #1
  \else
    \my@intervalle #2\@eolst%
  \fi}

%Quatre macros pour les quatres types d'intervalles
\newcommand{\interff}[1]{%
  \left[\my@intervalle #1;\relax\noexpand\@eolst%
  \right]
}
\newcommand{\interfo}[1]{%
  \left[\my@intervalle #1;\relax\noexpand\@eolst%
  \right[}
\newcommand{\interof}[1]{%
  \left]\my@intervalle #1;\relax\noexpand\@eolst%
  \right]}
\newcommand{\interoo}[1]{%
  \left]\my@intervalle #1;\relax\noexpand\@eolst%
  \right[}

\makeatother


\theoremstyle{break}
\newtheorem{definition}{Définition}
\newtheorem{propriete}{Propriété}
\newtheorem{propdef}{Propriété - Définition}
\newtheorem{theoreme}{Théorème}
%\theoremstyle{plain}
\theorembodyfont{\normalfont}
\newtheorem{exercice}{Exercice}
\theoremstyle{nonumberplain}
\newtheorem{remarque}{Remarque}
\newtheorem{probleme}{Problème}
\newtheorem{preuve}{Preuve}
\theoremstyle{nonumberbreak}
\newtheorem{exemple}{Exemple}
\newcommand{\qed}{\hfill$\Box$}

\setlength{\parsep}{0pt}
\setlength{\parskip}{5pt}
\setlength{\parindent}{0pt}
\setlength{\itemsep}{7pt}

\setlist{noitemsep}
%\setlist[1]{\labelindent=\parindent} % < Usually a good idea
\setlist[itemize]{leftmargin=*}
\setlist[itemize,1]{label=$\triangleright$}
\setlist[enumerate]{labelsep=*, leftmargin=1.5pc}
\setlist[enumerate,1]{label=\arabic*., ref=\arabic*}
\setlist[enumerate,2]{label=\emph{\alph*}),
ref=\theenumi.\emph{\alph*}}
\setlist[enumerate,3]{label=\roman*), ref=\theenumii.\roman*}
\setlist[description]{font=\sffamily\bfseries}

\usepackage{multicol}
\setlength{\columnseprule}{0pt}

\everymath{\displaystyle\everymath{}}

\title{Correction entrainement bac blanc}
\author{\bsc{Jumel}}
\date{7 décembre 2015}

\begin{document}

\maketitle

\begin{exercice}
  On considère une fonction $f$ définie par l'expression $f(x) =
  \frac{x}{\sqrt{x^2 -1}}$.

  \begin{enumerate}
    \item $f$ est définie sur $\interoo{-\infty;-1} \cup
      \interoo{1;+\infty}$
  \end{enumerate}
  Soit $\mathscr{C}$ sa courbe représentative dans un repère orthonormé
  $\vecteur[&;]{O;\vv{\imath};\vv{\jmath}}$.

  \begin{enumerate}[resume]
    \item \begin{enumerate}
        \item $\lim_{\substack{x\to1 \\ x>1}}f(x) = +\infty$ et
          $\lim_{x\to+\infty}f(x) = 1.$
        \item $\mathscr{C}$ admet deux asymptotes
          $\Delta_1: x = 1$ et $\Delta_2 : y = 1$.
      \end{enumerate}
    \item \begin{enumerate}
        \item Soit $g$ la fonction définie pour tout réel $x$ de
          $\interoo{1;+\infty}$, par : $g(x) = \sqrt{x^2 -1}$. Justifier
          que sa dérivée $g'$ vérifie, pour tout réel $x$ de
          $\interoo{1;+\infty}\ :\  g'(x) = f(x)$.

          Posons $u(x) = x^2 -1 \forall x\in D_f$. $(\sqrt{u}=' =
          \frac{u'}{2\sqrt{u}}$ et $u'(x) = 2x \forall x\in D_f \implies
          g'(x) = f(x)$
        \item $f'$ désigne la dérivée de $f$. Justifier que, pour tout
          réel $x$ de $\interoo{1;+\infty}$, on a : \[ f'(x) =
          \frac{-1}{\left(x^2 -1\right)\sqrt{x^2-1}} .\]
          Soit $x$ un réel de $\interoo{1;+\infty}$ $f(x) =
          \frac{x}{g(x)}$ et donc $f'(x) = \frac{1\times g(x) - x\times
          g'(x)}{g^2(x)} = \frac{\sqrt{x^2 - 1} - \frac{x^2}{\sqrt{x^2 -
          1}}}{x^2 -1} $ $f'(x) = \frac{1}{\sqrt{x^2 -1}}\times\frac{x^2
          - 1 - x^2}{x^2 -1}$ \qed
        \item Dresser le tableau de variation de $f$.
        \item Compléter le tableau suivant, en donnant des valeurs
          approchées à \np{0.01} des images $f(x)$.
          \begin{center}
            \begin{tabular}{|*{7}{>{\hfill}p{3em}<{\hfill~}|}} \hline
              $x$    & 1,1  & 1,25 & 1,5  & 1,75 & 2    & 4    \\ \hline
              $f(x)$ & 2,40 & 1,67 & 1,34 & 1,22 & 1,15 & 1,03 \\ \hline
            \end{tabular}
          \end{center}
        \item Tracer la courbe $\mathscr{C}$ et les droites $\Delta_1$
          et $\Delta_2$ dans le repère $\vecteur[&;]{0; \vv{\imath};
        \vv{\jmath}}$
      \end{enumerate}
    \item Soient deux réels $a$ et $b$ tels que $1 < a < b$.

      \begin{enumerate}
        \item $I(b) = \lim_{a\to1^+}\sqrt{b^2 - 1} - \sqrt{a^2 - 1} - b
          + a = \sqrt{b^2 - 1} - b + 1$
        \item Soit $b$ un réel $b > 1$

          $\sqrt{b^2 - 1} - b = \frac{\left(\sqrt{b^2 - 1} - b\right)
          \left(\sqrt{b^2 - 1} + b\right)}{\sqrt{b^2 - 1} + b} =
          \frac{b^2 - 1 - b^2}{\sqrt{b^2 - 1} + b} = \frac{-1}{\sqrt{b^2
          - 1} + b}.$
        \item $K = \lim_{b\to+\infty} I(b) = \frac{-1}{\sqrt{b^2 - 1} +
          b} +1 = 1$.
      \end{enumerate}
  \end{enumerate}
\end{exercice}

\begin{exercice}
  Soit $f$ une fonction définie sur $\R$, strictement positive,
  dérivable et dont la dérivée est strictement positive. Pour tout point
  $M$ d'abscisse $t$ appartenant à la courbe représentative de $f$, on
  considère le point $P$ de coordonnées $(t,~ 0)$ et le point $N$, point
  d'intersection de la tangente en $M$ à la courbe représentative de $f$
  avec l'axe des abscisses.
  \begin{enumerate}
    \item Calculer la distance $PN$ en fonction de $f(t)$ et de $f'(t)$.
    \item Proposer une relation qui lie $f$ et $f'$ lorsque :
      \begin{enumerate}
        \item $PN$ est constante ;
        \item $PN$ varie linéairement avec $P$
      \end{enumerate}
  \end{enumerate}
\end{exercice}

\pagebreak

\begin{exercice}
  ~\\[-\baselineskip]
  \parbox[c]{0.5\textwidth}{Soit $f$ la fonction définie sur
    l'intervalle [0~;~ 1] par $f(x) = x - 2\sqrt{x} + 1$. Cette fonction
    est dérivable sur [0~;~1] et sa dérivée $f'$ vérifie $f'(1) = 0$. La
    courbe représentative
    $\Gamma$ de la fonction $f$ dans un repère orthonormal est donnée
    ci-contre.

    \textbf{1. a.}  Montrer que le point $M$ de coordonnées $(x,~y)$
    appartient à $\Gamma$ si et seulement si $x \geqslant 0,~ y
    \geqslant 0$ et $\sqrt{x} + \sqrt{y} = 1$.

    \textbf{b.} Montrer que $\Gamma$ est symétrique par rapport à la
  droite d'équation $y = x$.} \hfill
  \parbox[c]{0.5\textwidth}{\begin{center}
      \begin{tikzpicture}[scale=4,>=latex]
        \draw [->] (-0.25,0) -- (1.25,0) ;
        \draw [->] (0,-0.25) -- (0,1.25) ;
        \draw plot [domain=0:1] (\x,{\x - 2*sqrt(\x) + 1});
    \end{tikzpicture}
  \end{center}}

  \textbf{2.} \textbf{a.}  Si $\Gamma$ était un arc de cercle, quel
  serait son centre ? Quel serait son rayon ?

  \textbf{b.}  La courbe $\Gamma$ est-elle un arc de cercle ?
\end{exercice}

\begin{exercice}

  Soit I l'intervalle [0 ; 1]. On considère la fonction $f$ définie sur
  I par $f(x) = \dfrac{3x +2}{x+4}$.

  \begin{enumerate}
    \item Les variations de la fonctions sont données par le signe de
      $3(x +2) - 3x -2 = 10 >0$ La fonction est donc strictement
      croissante.

      $f$ est continue (car dérivable) et donc on peut utiliser le
      théorème des valeurs intermédiaires : l'image d'un intervalle par
      $f$ est un intervalle et on peut même écrire $f(I) = \interff{f(0)
      = \frac12 ; f(1) = 1}\subset\interff{0;1}$
    \item On considère la suite $\left(u_{n}\right)$ définie par
      \[u_{0} = 0\quad  \text{et} \quad u_{n +1} =
      \dfrac{3u_{n}+2}{u_{n} + 4}.\]

      Montrons que, pour tout $n,~ u_{n}$ appartient à I.

      La suite est ici définie par récurrence, il faut démontrer ce
      point par récurrence. On a les points suivants :
      \begin{itemize}
        \item $u_0\in I$
        \item Soit $n\in\N$, supposons que $u_n\in I$. D'après ce qui
          précède, $u_{n+1} = f(u_n) \in I$ \qed
        \item $\forall n\in\N,\ u_n\in I$.
      \end{itemize}

      On se propose d'étudier la suite $\left(u_{n}\right)$ par deux
      méthodes différentes.
      \vspace{0,8cm}

      \textbf{Première méthode :}

    \item \begin{enumerate} \item Représenter graphiquement $f$ dans un
          repère orthonormal d'unité graphique 10 cm.

        \item En utilisant le graphique précédent, placer les points
          A$_{0}$,~ A$_{1}$,~ A$_{2}$ et A$_{3}$ d'ordonnée nulle et
          d'abscisses respectives $u_{0},~ u_{1},~ u_{2}$ et $u_{3}$.

          Que suggère le graphique concernant le sens de variation de
          $\left(u_{n}\right)$ et sa convergence ?

          À propos de la convergence, le graphique suggère que la
          suite converge vers 1.

        \item $u_{n+1} - u_n = \dfrac{3u_{n}+2}{u_{n} + 4} - u_n =
          \dfrac{3u_{n}+2 - (u_{n} + 4)u_n}{u_{n} + 4} \dfrac{-u_n^2 -
          u_{n}+2}{u_{n} + 4}$

          Considérons le polynome en $X$ $-X^2 - X +2$ dont les racines
          sont 1 et -2. On obtient donc la factorisation suivante :
          $-u_n^2 - u_{n}+2 = (1- u_n)(u_n+2)$ et donc $u_{n+1} - u_n =
          \dfrac{(1- u_n)(u_n+2)}{u_n+4}$

          $u_{n+1} - u_n > 0 \implies u_n$ croissante. (signe du
          polynome)
        \item $u$ est une suite croissante majorée, donc convergente.
        \item On a que $\lim_{n\to+\infty}u_n =
          \lim_{n\to+\infty}u_{n+1} = \lim_{n\to+\infty}f(u_n)$

          $f$ étant continue, on en déduit que si $\lim_{n\to+\infty}u_n
          = l$, alors $l = f(l)$.

          Résolvons l'équation $ l = f(l) \iff \dfrac{(1-l)(l+2)}{l+4} =
          0$ dont la solution sur $\interff{0;1}$ est 1. \qed
      \end{enumerate}
      \vspace{0,8cm}

      \textbf{Deuxième méthode :}

      On considère la suite $\left(v_{n}\right)$ definie par $v_{n} =
      \dfrac{u_{n} - 1}{u_{n} + 2}$.

    \item
      \begin{enumerate}
        \item $v_{n+1} = \dfrac{u_{n} - 1}{u_{n} + 2} =
          \dfrac{\dfrac{3u_n+2}{u_n+4} - 1}{\dfrac{3u_n+2}{u_n+4} + 2} =
          \dfrac{3u_n+2 - u_n -4}{3u_n+2 + 2u_n+8} = \frac{3(u_n -
          1)}{5(u_n+2)} = \frac25v_n$

        \item $v_{0} = -\frac12$ et $v_{n} = -\frac12\left( \frac25
          \right)^n$.
        \item $v_nu_{n} + 2 v_n = u_n - 1 \iff u_n \left(v_n - 1\right)
          = -1 -2 v_n \iff u_n = - \dfrac{1+2v_n}{v_n -1}$

          $u_n = \frac{1 - 2\frac12\left( \frac25 \right)^n}{1 +
          \frac12\left( \frac25 \right)^n}$.
        \item Posons $g:x\mapsto \frac{1 - 2x}{1+x}$. On a $u_n =
          g(v_n) \forall n\in\N$ et $\lim_{\to\infty}v_n = 0$. De plus,
          $g$ est continue en 0 et $g(0) = 1$.

          $\lim_{n\to\infty}u_{n}$
      \end{enumerate}
  \end{enumerate}
\end{exercice}

\pagebreak

\begin{exercice}
  On sait tous qu'il y a des années à coccinelles et d'autres sans !

  On se propose d'étudier l'évolution d'une population de coccinelles à
  à l'aide d'un modèle utilisant la fonction numérique $f$ définie par
  $f(x) = kx(1 - x),~ k$ étant un paramètre qui dépend de
  l'environnement $(k \in \R)$.

  Dans le modèle choisi, on admet que le nombre des coccinelles reste
  inférieur àà un million. L'effectif des coccinelles, exprimé en
  millions d'individus, est approché pour l'année $n$ par un nombre réel
  $u_{n}$, avec $u_{n}$ compris entre 0 et 1. Par exemple, si pour
  l'année zéro il y a \nombre{300000}~coccinelles, on prendra $u_{0} =
  0,3$.

  On admet que l'évolution d'une année sur l'autre obéit à la relation
  $u_{n+1} = f\left(u_{n}\right),~ f$ étant la fonction définie
  ci-dessus.

  Le but de l'exercice est d'étudier le comportement de la suite
  $\left(u_{n}\right)$ pour différentes valeurs de la population
  initiale $u_{0}$ et du paramètre $k$.

  \begin{enumerate}
    \item Démontrer que si la suite $\left(u_{n}\right)$
      converge, alors sa limite $l$ vérifie la relation $f(l) = l$.
    \item Supposons $u_{0} = 0,4$ et $k = 1$.
      \begin{enumerate}
        \item Étudier le sens de variation de la suite
          $\left(u_{n}\right)$.
        \item Montrer par récurrence que, pour tout entier $n,~ 0
          \leqslant u_{n} \leqslant 1$.
        \item La suite $\left(u_{n}\right)$ est-elle convergente ? Si
          oui, quelle est sa limite ?
        \item Que peut-on dire de l'évolution à long terme de la
          population de coccinelles avec ces hypothèses ?
      \end{enumerate}
    \item Supposons maintenant $u_{0} = 0,3$ et $k = 1,8$.
      \begin{enumerate}
        \item Étudier les variations de la fonction $f$
          sur $[0~;~ 1]$ et montrer que $f\left(\dfrac{1}{2}\right) \in
          \left[0~;~\dfrac{1}{2}\right]$.
        \item En utilisant éventuellement un raisonnement par
          récurrence,
          \begin{itemize}
            \item  montrer que, pour tout entier naturel $n,~ 0
              \leqslant u_{n} \leqslant \dfrac{1}{2}$
          \item  établir que, pour tout entier naturel $n,~ u_{n+1}
            \geqslant u_{n}$.
        \end{itemize}
      \item La suite $\left(u_{n}\right)$ est-elle convergente ? Si
        oui, quelle est sa limite ?
      \item Que peut-on dire de l'évolution à long terme de la
        population de coccinelles avec ces hypothèses ?
    \end{enumerate}
  \item On a représenté sur les feuilles annexes la fonction $f$ dans
    les deux cas étudiés ci-dessus ainsi que la droite d'équation $y =
    x$. Le troisième graphique correspond au cas où\`u $u_{0} = 0,8$
    et $k = 3,2$.

    Illustrer sur les deux premiers graphiques les résultats trouvés
    en \textbf{1.} et \textbf{2.} en laissant les traits de
    construction et en faisant apparaî\^{\i}tre en abscisse les
    valeurs successives $u_{0},~u_{1},~u_{2},~\ldots$

    En utilisant la même méthode, formuler une conjecture sur
    l'évolution de la population dans le troisième cas.
  \end{enumerate}
\end{exercice}

\begin{exercice}
  Cet exercice se présente comme un questionnaire à choix multiples
  (QCM). les quatre questions posées sont indépendantes.

  Pour chaque question il y a deux conclusions correctes. Le candidat
  doit cocher \textbf{au plus deux cases} (celles qu'il juge correctes).

  Aucune justification n'est demandée.

  À chaque question est affecté  un certain nombre de points. Chaque
  réponse exacte rapporte la moitié des  points affectés ; chaque
  réponse fausse enlève le quart des points affectés. Cocher trois cases
  ou plus à d'une question, ou n'en cocher aucune, rapporte zéro point à
  cette question.

  Si, par aplication de ce barème, le total des points de l'exercice est
  négatif, il est ramené à zéro.


  On considère trois suites $\left(u_{n}\right),~ \left(v_{n}\right)$ et
  $\left(w_{n}\right)$ qui vérifient la propriété suivante :

  \begin{center}
    \og Pour tout entier naturel~ $n$~ strictement positif :~
    $\left(u_{n}\right) \leqslant \left(v_{n}\right) \leqslant
    \left(w_{n}\right)$~\fg.
\end{center}

  \begin{enumerate}
    \item Si la suite $\left(v_{n}\right)$ tend vers $- \infty$, alors :

      $\Box$ La suite $\left(w_{n}\right)$ tend vers $- \infty$

      $\blacksquare$ la suite $\left(u_{n}\right)$ est majorée

      $\blacksquare$ la suite $\left(u_{n}\right)$ tend vers $- \infty$

      $\Box$ la suite $\left(w_{n}\right)$ n'a pas de limite.

    \item Si $\left(u_{n}\right) \geqslant 1,~\left(w_{n}\right) =
      2\left(u_{n}\right)$ et $\lim \left(u_{n}\right) = l$, alors

      $\Box$ $\lim  \left(v_{n}\right) = l$

      $\Box$ La suite $\left(w_{n}\right)$ tend vers $+ \infty$

      $\blacksquare$ $\lim (w_{n} - u_{n})= l$

      $\blacksquare$ On ne sait pas dire si la suite $\left(v_{n}\right)$ a une
      limite ou non.

    \item Si $lim \left(u_{n}\right) = - 2$ et $\lim \left(w_{n}\right)
      = 2$, alors :

      $\blacksquare$ La suite $\left(v_{n}\right)$ est majorée

      $\Box$ $\lim \left(v_{n}\right) = 0$

      $\Box$ la suite $\left(v_{n}\right)$ n'a pas de limite

      $\blacksquare$ On ne sait pas dire si la suite $\left(v_{n}\right)$ a une
      limite ou non.

    \item Si $u_{n} =\dfrac{2n^2-1}{n^2}$ et  $w_{n} =
      \dfrac{2n^2+3}{n^2}$2nalors :

      $\Box~ \lim \left(w_{n}\right) = 0$

      $\blacksquare~ \lim \left(v_{n}\right) = 2$

      $\blacksquare~  \lim \left(u_{n}\right) = 2$

      $\Box$ la suite $\left(v_{n}\right)$ n'a pas de limite.

  \end{enumerate}
\end{exercice}

\pagebreak

\begin{exercice}
  On se propose d'étudier les fonctions $f$ dérivables sur $\interfo{0;+
  \infty}$ vérifiant la condition \[(1) \quad\left\{\begin{array}{l}
    \text{pour tout} \quad  x \in [0~;~ + \infty[,~ f(x)f'(x) = 1\\
  f(0) = 1\\ \end{array}\right.\]

  \textbf{ Partie A}

  On suppose qu'il existe une fonction $f$ qui vérifie (1).

  La méthode d'\textsc{euler} permet de construire une suite de points
  $\left(M_{n}\right)$ proches de la courbe représentative de la
  fonction $f$.

  On choisit le pas $h = 0,1$.

  On admet que les coordonnées $\left(x_{n},~ y_{n}\right)$ des points
  $M_{n}$ obtenus en appliquant cette méthode avec ce pas vérifient :
  \[\left\{\begin{array}{l c l}
        x_{0} & = & 0\\
        y_{0} & = & 1\\
      \end{array}\right. \qquad
      \left\{\begin{array}{l c l}
          x_{n+1} & = & x_{n} + 0,1\\
          y_{n+1} & = & y_{n} + \dfrac{0,1}{y_{n}}\\
  \end{array} \right. \quad \text{pour tout entier naturel}~ n.\]
  Calculer les coordonnées des points M$_{1}$,~ M$_{2}$,~ M$_{3}$,~
  M$_{4}$,~ M$_{5}$ (on arrondira au millième les valeurs trouvées).

  \textbf{ Partie B}

  On se propose de démontrer qu'une fonction vérifiant (1) est
  nécessairement strictement positive sur $\interfo{0;+ \infty}$.

  \begin{enumerate}
    \item Montrer que si la fonction $f$ vérifie (1)
      alors $f$ ne s'annule pas sur $\interfo{0;+\infty}$.
    \item On suppose que la fonction $f$ vérifie la condition (1) et qu'il
      existe un réel a strictement positif tel que $f(a) < 0$.

      En déduire que l'équation $f (x) = 0$ admet au
      moins une solution dans l'intervalle $[0~;~ a]$.

    \item Conclure.

  \end{enumerate}

  \textbf{ Partie C}

  \textbf{Existence et unicité de la fonction} \boldmath $f$
  \unboldmath.

  \begin{enumerate}
    \item Soit $u$ une fonction dérivable sur un intervalle I. Quelles
      sont les fonctions telles que leurs dérivées soient $uu'$ ?
    \item En déduire que si $f$ est telle que,
      \[\text{pour tout} \quad x \in \interfo{0; + \infty},~ f(x)f'(x) =
      1,\] alors il existe une constante $C$ telle que :
      \[\text{pour tout} \quad x \in \interfo{0; + \infty},
      \left[f(x)\right]^2 = 2x + C.\]
    \item On rappelle que $f(0) = 1$. Déterminer l'expression de $f(x)$
      pour $x$ réel positif.
    \item  En déduire les valeurs arrondies au millième de $f(0,1),~
      f(0,2),~ f(0,3),~ f(0,4),~f(0,5)$, puis les comparer avec les
      valeurs obtenues par la méthode d'\textsc{euler}.
  \end{enumerate}
\end{exercice}

\pagebreak

\begin{exercice}

  Un certain concours d'entrée dans une école d'ingénieurs consiste en
  plusieurs épreuves:

  \setlength\parindent{5mm}
  \begin{itemize}
    \item Après examen de leur dossier scolaire, 15\,\% des candidats
      (les meilleurs) sont admis directement sans passer d'autres
      épreuves.
    \item Les autres candidats, non admis sur dossier, passent
      une épreuve écrite. On estime que 60\,\% des candidats réussissent
      cette épreuve écrite et les autres sont recalés.
    \item Les candidats ayant réussi l'épreuve écrite sont alors
      convoqués pour passer une épreuve orale. Les candidats réussissant
      l'épreuve orale sont alors admis. On estime que les candidats ont
      une chance sur trois de réussir l'épreuve orale.
  \end{itemize}
  \setlength\parindent{0mm}

  On considère les évènements suivants :

  \setlength\parindent{5mm}
  \begin{itemize}
    \item $D$ : \og Le candidat est admis sur dossier \fg
    \item $E$ : \og Le candidat passe et réussit l'épreuve écrite \fg
    \item $O$ : \og Le candidat passe et réussit l'épreuve orale \fg
    \item $A$ : \og Le candidat est admis \fg.
  \end{itemize}
  \setlength\parindent{0mm}

  On note $\overline{E}$ l'évènement contraire de $E$.

  On note $P(D)$ la probabilité de l'évènement $D$ et $P_{E}(O)$ la
  probabilité de l'évènement $O$ sachant que l'évènement $E$ est
  réalisé.

  \begin{enumerate}
    \item Établir l'arbre des probabilités.
    \item \begin{enumerate}
        \item Compléter à l'aide des hypothèses :

          $P(D) =  \ldots \qquad P_{\overline{D}}(E) = \ldots \qquad
          P_{R}(O) = \ldots $
        \item Déterminer la probabilité $P(E)$ qu'un candidat passe et
          réussisse l'épreuve écrite et la probabilité $P(O)$ qu'un
          candidat passe et réussisse l'épreuve orale.
        \item On note $p$ la probabilité $P(A)$ qu'un candidat soit
          admis dans cette école d'ingénieurs. Justifier que $p$ vaut
          $0,32$.
      \end{enumerate}
    \item Cinq amis décident de passer ce concours (les résultats
      obtenus par chaque candidat sont indépendants les uns des autres).
      \begin{enumerate}
        \item Exprimer, en fonction de $p$, la probabilité $P_{1}$ que
          les cinq soient admis.

          Puis donner une valeur approchée de $P_{1}$ à $10^{-4}$ près.
        \item  Exprimer, en fonction de $p$, la probabilité $P_{2}$
          qu'au moins un des cinq soit recalé.

          Puis donner une valeur approchée de $P_{2}$ à $10^{-4}$ près.
        \item   Exprimer, en fonction de $p$, la probabilité $P_{3}$
          qu'au moins un des cinq soit admis.

          Puis donner une valeur approchée de $P_{3}$ à $10^{-4}$ près.
        \item  Exprimer, en fonction de $p$, la probabilité $P_{4}$ que
          trois exactement soient admis.

          Puis donner une valeur approchée de $P_{4}$ à $10^{-4}$ près.
      \end{enumerate}
    \item Par hasard, je rencontre un candidat qui me dit avoir été
      admis dans cette école d'ingénieurs. Quelle est la probabilité
      $P_{A}(D)$ qu'il ait été admis sur dossier ?
  \end{enumerate}
\end{exercice}

\pagebreak

\begin{exercice}
  Soit $f$ la fonction définie sur l'intervalle $\interfo{0;+\infty}$
  par : \[f(x) = 6 -  \dfrac{5}{x + 1}.\]

  Le but de cet exercice est d'étudier des suites $\left(u_{n}\right)$
  définies par un premier terme positif ou nul $u_{0}$ et vérifiant pour
  tout entier naturel $n$ : \[u_{n+1} = f\left(u_{n}\right).\]

  \begin{enumerate}
    \item  Étude de propriétés de la fonction $f$
      \begin{enumerate}
        \item Étudier le sens de variation de la fonction $f$ sur
          l'intervalle $\interfo{0;+\infty}$.
        \item Résoudre dans l'intervalle $\interfo{0;+\infty}$
          l'équation $f(x) = x$.

          On note $\alpha$ la solution.
        \item Montrer que si $x$ appartient à l'intervalle
          $[0~;~\alpha]$, alors $f(x)$ appartient à l'intervalle
          $[0~;~\alpha]$.

          De même, montrer que si $x$ appartient à l'intervalle
          $\interfo{\alpha ;+\infty}$ alors $f(x)$ appartient à
          l'intervalle $\interfo{\alpha~;~ + \infty}$.
      \end{enumerate}
    \item Étude de la suite $\left(u_{n}\right)$ pour $u_{0} = 0$

      Dans cette question, on considère la suite $\left(u_{n}\right)$
      définie par $u_{0} = 0$ et pour tout entier naturel $n$ :
      \[u_{n + 1} = f\left(u_{n}\right) = 6 - \dfrac{5}{u_{n} + 1}.\]

      \begin{enumerate}
        \item Sur le graphique représenté dans l'annexe 2, sont
          représentées les courbes d'équations $y = x$ et $y=f(x)$.

          Placer le point $A_{0}$ de coordonnées
          $\left(u_{0}~;~0\right)$, et, en utilisant ces courbes,
          construire à partir de $A_{0}$ les points
          $A_{1}$,~$A_{2}$,~$A_{3}$  et $A_{4}$ d'ordonnée nulle et
          d'abscisses respectives $u_{1},~u_{2},~u_{3}$ et $u_{4}$.

          Quelles conjectures peut-on émettre quant au sens de variation
          et à la convergence de la suite $\left(u_{n}\right)$ ?
        \item Démontrer, par récurrence, que, pour tout entier naturel
          $n,~ 0 \leqslant u_{n} \leqslant  u_{n+1} \leqslant \alpha$.
        \item En déduire que la suite $\left(u_{n}\right)$ est
          convergente et déterminer sa limite.
      \end{enumerate}
    \item Étude des suites $\left(u_{n}\right)$ selon les valeurs du
      réel positif ou nul $u_{0}$

      \emph{Dans cette question, toute trace d'argumentation, même
        incomplète, ou d'initiative, même non fructueuse, sera prise en
      compte dans l'évaluation.}

\medskip

      Que peut-on dire du sens de variation et de la convergence de la
      suite $\left(u_{n}\right)$ suivant les valeurs du réel positif ou
      nul $u_{0}$ ?
  \end{enumerate}

  \pagebreak

  \begin{center}
    \begin{tikzpicture}[>=latex]
      \draw [help lines] (0,0) grid (-0.15,7.9) ;
      \draw [help lines] (0,0) grid (7.9,-0.15) ;
      \draw [thick,red] plot [smooth,domain=0:8] (\x,{6 - 5/(\x+1)}) ;
      \draw [thick,blue] plot [domain=0:8] (\x,\x) ;
      \draw [very thick,->] (0,0) -- (8.2,0) ;
      \draw [very thick,->] (0,-0.5) -- (0,8.2) ;

    \end{tikzpicture}
  \end{center}

\end{exercice}

\end{document}
