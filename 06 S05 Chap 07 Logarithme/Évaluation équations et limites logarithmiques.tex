\documentclass[12pt,a4paper,french]{article}
\usepackage[utf8]{inputenc}
\usepackage[T1]{fontenc}
\usepackage{babel}
\usepackage[thmmarks]{ntheorem}
\usepackage{amsmath}
\usepackage{amsfonts}
\usepackage{amssymb}

\usepackage{array}

\usepackage{lmodern}
\usepackage{kpfonts}

\usepackage[bookmarks=false,colorlinks,linkcolor=blue,pdfusetitle]{hyperref}

\pdfminorversion 7
\pdfobjcompresslevel 3

\usepackage{tabularx}
\usepackage[autolanguage,np]{numprint}
\usepackage{enumitem}

\usepackage{tipfr}
\usepackage{pgf}
\usepackage{tikz}
\usepackage{tkz-euclide}
\usetkzobj{all}
\usetikzlibrary{hobby}
\usepackage{tkz-tab}

\usepackage[top=1.9cm,bottom=2cm,left=2cm,right=2cm]{geometry}

\usepackage{lastpage}

\usepackage{esvect}
\usepackage{marginnote}

\usepackage{wrapfig}

\usepackage[defaultlines=5,all]{nowidow}


\usepackage[]{algorithm2e}

\usepackage{ifthen}
\usepackage{fancyhdr}
\pagestyle{fancy}


\makeatletter

\count1=\year \count2=\year
\ifnum\month<8\advance\count1by-1\else\advance\count2by1\fi

\setlength{\headheight}{14.5pt}
\renewcommand{\headrulewidth}{0pt}
\renewcommand{\footrulewidth}{0pt}
\cfoot{\textsl{\footnotesize{Année \number\count1/\number\count2}}}

\rfoot{%
  \ifthenelse{\value{page}=1}{%
  }
  {%
    \footnotesize{Page \thepage/ \pageref{LastPage}}
  }
}

\rhead{}

\lhead{%
  \ifthenelse{\value{page}=1}{%
    Nom:\dotfill\hfill Prénom: \dotfill \hfill
    Classe: \@author \dots%
  }
  { }
}

\renewcommand{\maketitle}%
{\framebox{%
    \begin{minipage}{1.0\linewidth}%
      \begin{center}%
        \Large \@title ~-- \@author \\%
        \@date%
      \end{center}%
    \end{minipage}}%
  \normalsize%
  %\vspace{1cm}%
}

%Des macros pour les noms d'ensmbles
\newcommand{\R}{\mathbf{R}}
\newcommand{\Q}{\mathbf{Q}}
\newcommand{\Z}{\mathbf{Z}}
\newcommand{\C}{\mathbf{C}}
\newcommand{\N}{\mathbf{N}}

\newcommand{\norme}[1]{\left\lVert #1 \right\rVert}
\newcommand{\abs}[1]{\left\lvert #1 \right\rvert}

%Une macro récursive pour l'intérieru des vecteurs
%http://tex.stackexchange.com/questions/19693/arguments-of-custom-commands-as-comma-separated-list

\newcommand\vecteur[2][\\]{%
    \global\def\my@delim{#1}%
    \left(\negthinspace\begin{matrix}
        \my@vector #2,\relax\noexpand\@eolst%
    \end{matrix}\right)}

%Une macro pour les vecteurs
\def\my@vector #1,#2\@eolst{%
   \ifx\relax#2\relax
      #1
   \else
      #1\my@delim
      \my@vector #2\@eolst%
   \fi}

%Une macro récursive pour mettre formater l'intérieur des intervalles
\def\my@intervalle #1;#2\@eolst{%
  \ifx\relax#2\relax
    #1
  \else
    \my@intervalle #2\@eolst%
  \fi}

%Quatre macros pour les quatres types d'intervalles
\newcommand{\interff}[1]{%
  \left[\my@intervalle #1;\relax\noexpand\@eolst%
  \right]
}
\newcommand{\interfo}[1]{%
  \left[\my@intervalle #1;\relax\noexpand\@eolst%
  \right[}
\newcommand{\interof}[1]{%
  \left]\my@intervalle #1;\relax\noexpand\@eolst%
  \right]}
\newcommand{\interoo}[1]{%
  \left]\my@intervalle #1;\relax\noexpand\@eolst%
  \right[}

\makeatother


\usepackage{framed}

\theoremstyle{break}
\newtheorem{definition}{Définition}
\newtheorem{propriete}{Propriété}
\newtheorem{corollaire}{Corollaire}
\newtheorem{propdef}{Propriété - Définition}
\newtheorem{theoreme}{Théorème}
\theoremstyle{plain}
\theorembodyfont{\normalfont}
\newtheorem{exerciceT}{Exercice}
\theoremstyle{nonumberplain}
\newtheorem{remarque}{Remarque}
\newtheorem{notation}{Notation}
\newtheorem{probleme}{Problème}
\theoremsymbol{\ensuremath{\blacksquare}}
\newtheorem{preuve}{Preuve}
\theoremsymbol{}
\theoremstyle{nonumberbreak}
\newtheorem{exemple}{Exemple}

\newenvironment{exercice}{\begin{framed}\begin{exerciceT}}{\end{exerciceT}\end{framed}}

\setlength{\parsep}{0pt}
\setlength{\parskip}{5pt}
\setlength{\parindent}{0pt}
\setlength{\itemsep}{7pt}

\setlist{noitemsep}
%\setlist[1]{\labelindent=\parindent} % < Usually a good idea
\setlist[itemize]{leftmargin=*}
\setlist[itemize,1]{label=$\triangleright$}
\setlist[enumerate]{labelsep=*, leftmargin=1.5pc}
\setlist[enumerate,1]{label=\arabic*., ref=\arabic*}
\setlist[enumerate,2]{label=\emph{\alph*}),
ref=\theenumi.\emph{\alph*}}
\setlist[enumerate,3]{label=\roman*), ref=\theenumii.\roman*}
\setlist[description]{font=\sffamily\bfseries}

\usepackage{multicol}
\setlength{\columnseprule}{0pt}

\usepackage[]{exsheets}
\SetupExSheets{headings=block}

\everymath{\displaystyle\everymath{}}

\title{Évaluation \no{8} : logarithmes}
\author{\bsc{Ts}}
\date{28 février 2017}

\begin{document}

\maketitle

\begin{tabular}{|p{6em}|p{26em}|p{6em}|}\hline
   & & \\
   & & \\
   \hfill\Huge /\totalpoints* & & \\
   & & \\
   & & \\ \hline
\end{tabular}


\begin{question}[ID=equation.log.transformation]
  ~\\[-6ex]
  \phantom{a}\hfill\textbf{(\GetQuestionProperty{points}{\CurrentQuestionID} points)}\\
  \begin{enumerate}
    \item Résoudre l'équation $\ln(x-2) + \ln(x-6) = \ln(x+4)$
      \addpoints*{2}
    \item En déduire les solutions de l'inéquation $\ln(x-6) < \ln(x+4)
      - \ln(x-2)$ \addpoints*{1}
  \end{enumerate}

  \blank[style=dotted,width=11\linewidth,linespread=1.7]{}
\end{question}
\begin{solution}
  \begin{enumerate}
    \item L'équation est a priori définie sur $I=\interoo{6;+\infty}$.
      Sur cet intervalle $I$, elle est successivement équivalente à
      \begin{align}
        \ln\left[(x-2)(x-6)\right] = \ln(x+4) \\
        (x-2)(x-6) = (x+4) \\
        x^2 - 8x + 12 = x + 4 \\
        x^2 - 9x + 8 = 0 \\
        (x - 1)(x - 8) = 0
      \end{align}
      La dernière ligne se trouve en remarquant que 1 est une racine
      évidente et en utilisant somme et produit des racines.

      Parmi les solutions de cette dernière équation, $1 \notin I$ et $8
      \in I$.
    \item L'inéquation proposée se ramène à l'inéquation $\ln(x-2) +
      \ln(x-6) < \ln(x+4)$. Ce qui est vrai pour $\ x \in
      \interoo{6;8}$.
  \end{enumerate}
\end{solution}

\begin{question}[ID=etude.log,use=true]
  ~\\[-6ex]
  \phantom{a}\hfill\textbf{(\GetQuestionProperty{points}{\CurrentQuestionID} points)}\\
  \begin{enumerate}
    \item $g$ est la fonction définie sur $\R$ par $g(x) = xe^x + 1$.
      Démontrer que pour tout $x$ de $\R$, $g(x) > 0$. \addpoints*{2}

    \item $f$ est la fonction définie sur $I = \interoo{0;+\infty}$ par
      : \[ f(x) = e^x + \ln(x) .\]
      \begin{enumerate}
        \item Démontrer que pour tout $x$ de $\R$, $f'(x) =
          \frac{g(x)}x$. \addpoints*{2}
        \item Dresser le tableau de variation de $f$. \addpoints*{1}
        \item Déduisez-en que pour tout $m$, l'équation $f(x) = m$
          possède une unique solution. \addpoints*{2}
      \end{enumerate}
  \end{enumerate}
  \blank[style=dotted,width=5\linewidth,linespread=1.7]{}

  \blank[style=dotted,width=15\linewidth,linespread=1.7]{}
\end{question}

\begin{question}[ID=equation.logarithme.chgt.var,use=false]
  ~\\[-6ex]
  \phantom{a}\hfill\textbf{(\GetQuestionProperty{points}{\CurrentQuestionID} points)}\\
   Résoudre l'équation $\left[-\ln(x)\right]^2 + 4\ln(x) + 5 = 0$.
  \addpoints*{2}

  \blank[style=dotted,width=8\linewidth,linespread=1.7]{}
\end{question}
\begin{solution}
  Procédons par changement de variable : on pose $X = \ln x$. L'équation
  devient $-X^2 + 4X + 5 = 0$ dont une racine évidente est $-1$. On en
  déduit l'autre racine 5 avec le produit.

  Ainsi, $x = e^{-1}$ ou $x = e^5$ sont les solutions de cette équation.
\end{solution}

\end{document}
\newpage
\section*{Correction}
\printsolutions
\vfill
\hrule
\vfill
\section*{Correction}
\printsolutions
\vfill
\hrule
\vfill
\end{document}
