\documentclass[11pt,a4paper,french]{article}
\usepackage[utf8]{inputenc}
\usepackage[T1]{fontenc}
\usepackage{babel}
\usepackage[thmmarks]{ntheorem}
\usepackage{amsmath}
\usepackage{amsfonts}
\usepackage{amssymb}

\usepackage{array}

\usepackage{kpfonts}

\usepackage[bookmarks=false,colorlinks,linkcolor=blue,pdfusetitle]{hyperref}

\pdfminorversion 7
\pdfobjcompresslevel 3

\usepackage{tabularx}
\usepackage[autolanguage,np]{numprint}
\usepackage{enumitem}

\usepackage{tipfr}
\usepackage{pgf}
\usepackage{tikz}
\usepackage{tkz-euclide}
\usetkzobj{all}
\usetikzlibrary{hobby}
\usepackage{tkz-tab}

\usepackage[top=1.7cm,bottom=2cm,left=2cm,right=2cm]{geometry}

\usepackage{lastpage}

\usepackage{esvect}
\usepackage{marginnote}

\usepackage{wrapfig}

\usepackage[defaultlines=5,all]{nowidow}


\makeatletter
\renewcommand{\@evenfoot}%
        {\hfil \upshape \small page {\thepage} de \pageref{LastPage}}
\renewcommand{\@oddfoot}{\@evenfoot}

\renewcommand{\maketitle}%
{\framebox{%
    \begin{minipage}{1.0\linewidth}%
      \begin{center}%
        \Large \@title ~-- \@author \\%
        \@date%
      \end{center}%
    \end{minipage}}%
  \normalsize%
  %\vspace{1cm}%
}

\pgfdeclarepatternformonly{mes_hachures}
{\pgfpoint{-0.1cm}{-0.1cm}}
{\pgfpoint{0.9cm}{0.5cm}}
{\pgfpoint{0.8cm}{0.4cm}}
{\pgfpathmoveto{\pgfpointorigin}
  \pgfpathlineto{\pgfpoint{0.8cm}{0.4cm}}
\pgfusepath{stroke}}

%Des macros pour les noms d'ensmbles
\newcommand{\R}{\mathbf{R}}
\newcommand{\Q}{\mathbf{Q}}
\newcommand{\Z}{\mathbf{Z}}
\newcommand{\C}{\mathbf{C}}
\newcommand{\N}{\mathbf{N}}

\newcommand{\norme}[1]{\left\lVert #1 \right\rVert}
\newcommand{\abs}[1]{\left\lvert #1 \right\rvert}

%Une macro récursive pour l'intérieru des vecteurs
%http://tex.stackexchange.com/questions/19693/arguments-of-custom-commands-as-comma-separated-list

\newcommand\vecteur[2][\\]{%
    \global\def\my@delim{#1}%
    \left(\negthinspace\begin{matrix}
        \my@vector #2,\relax\noexpand\@eolst%
    \end{matrix}\right)}

%Une macro pour les vecteurs
\def\my@vector #1,#2\@eolst{%
   \ifx\relax#2\relax
      #1
   \else
      #1\my@delim
      \my@vector #2\@eolst%
   \fi}

%Une macro récursive pour mettre formater l'intérieur des intervalles
\def\my@intervalle #1;#2\@eolst{%
  \ifx\relax#2\relax
    #1
  \else
    \my@intervalle #2\@eolst%
  \fi}

%Quatre macros pour les quatres types d'intervalles
\newcommand{\interff}[1]{%
  \left[\my@intervalle #1;\relax\noexpand\@eolst%
  \right]
}
\newcommand{\interfo}[1]{%
  \left[\my@intervalle #1;\relax\noexpand\@eolst%
  \right[}
\newcommand{\interof}[1]{%
  \left]\my@intervalle #1;\relax\noexpand\@eolst%
  \right]}
\newcommand{\interoo}[1]{%
  \left]\my@intervalle #1;\relax\noexpand\@eolst%
  \right[}

\makeatother


\usepackage{framed}

\theoremstyle{break}
\newtheorem{definition}{Définition}
\newtheorem{propriete}{Propriété}
\newtheorem{corollaire}{Corollaire}
\newtheorem{propdef}{Propriété - Définition}
\newtheorem{theoreme}{Théorème}
\theoremstyle{plain}
\theorembodyfont{\normalfont}
\newtheorem{exerciceT}{Exercice}
\theoremstyle{nonumberplain}
\newtheorem{remarque}{Remarque}
\newtheorem{notation}{Notation}
\newtheorem{probleme}{Problème}
\theoremsymbol{\ensuremath{\blacksquare}}
\newtheorem{preuve}{Preuve}
\theoremsymbol{}
\theoremstyle{nonumberbreak}
\newtheorem{exemple}{Exemple}

\newenvironment{exercice}{\begin{framed}\begin{exerciceT}}{\end{exerciceT}\end{framed}}

\setlength{\parsep}{0pt}
\setlength{\parskip}{5pt}
\setlength{\parindent}{0pt}
\setlength{\itemsep}{7pt}

\setlist{noitemsep}
%\setlist[1]{\labelindent=\parindent} % < Usually a good idea
\setlist[itemize]{leftmargin=*}
\setlist[itemize,1]{label=$\triangleright$}
\setlist[enumerate]{labelsep=*, leftmargin=1.5pc}
\setlist[enumerate,1]{label=\arabic*., ref=\arabic*}
\setlist[enumerate,2]{label=\emph{\alph*}),
ref=\theenumi.\emph{\alph*}}
\setlist[enumerate,3]{label=\roman*), ref=\theenumii.\roman*}
\setlist[description]{font=\sffamily\bfseries}

\usepackage{multicol}
\setlength{\columnseprule}{0pt}

\everymath{\displaystyle\everymath{}}

\title{Étude de la fonction logarithme}
\author{\bsc{Jumel}}
\date{janvier 2018}

\begin{document}

\maketitle

On note $\R_+^* = \interoo{0;+\infty}$ et si $x$ et $y$ sont deux
éléments de $E$, on note $(x,y) \in E^2$.

\section{Activités préparatoires}

\subsection{Petit détour par la physique}

On sait qu'en physique et plus particulièrement en acoustique que
doubler le volume ne fait qu'augmenter de \np[dB]{3} et que si on désire
augmenter de \np[dB]{6}, on aura quadruplé le volume. On dit, à ce titre
que le décibel est une échelle logarithmique.

Comme le vocabulaire utilisé le montre, les fonctions logarithmiques
transforment les produits en somme. Étudions plus précisément la
signification mathématique de cette affirmation.

\subsection{Étude de l'équation fonctionnelle}

\begin{exercice}\label{prop:consequence}
  Soit $f$ une fonction définie sur $\interoo{0;+\infty}$ et vérifiant
  l'équation fonctionnelle \begin{equation} f(xy) = f(x) + f(y).
  \label{eq:fonc_log}\end{equation}

  Démontrer rigoureusement que :
  \begin{itemize}
    \item $f(1) = 0$ ;
    \item $\forall x \in \R_+^*,\ f\left(\frac1x\right) = -f(x)$ ;
    \item $\forall (x,y)\in\R_+^{*2} f\left(\frac{x}y\right) = f(x) -
      f(y)$ ;
    \item $\forall n\in\N,\ \forall x\in\R_+^*,\ f(x^n) = nf(x)$ ;
    \item $\forall n\in\Z,\ \forall x\in\R_+^*,\ f(x^n) = nf(x)$ ;
    \item $\forall x \in \R_+^*,\ f(\sqrt{x}) = \frac12f(x)$.
  \end{itemize}
\end{exercice}

Ces résultats entretiennent une familiarité avec une autre équation
fonctionnelle.

\begin{exercice}
  Donner la relation fonctionnelle de l'exponentielle.
\end{exercice}

\begin{exercice}\label{prop:fondamentale}
  Soient $f$ une fonction qui satisfait à \ref{eq:fonc_log} et $g$ une
  fonction qui satisfait à la relation fonctionnelle de l'exponentielle.
  Montrer que :
  \begin{itemize}
    \item pour tout $(x,y)$ réels strictement positifs, $g(f(xy)) =
      g(f(x))g(f(y))$ ;
    \item pour tout $(x',y')$ réels, $f(g(x'+y')) = f(g(x')) + f(g(y'))$
  \end{itemize}

  En déduire, pour un $y$, un $x'$ et un $y'$ bien choisis que $\forall
  x\in \R^*,\ f(g(x)) = x = g(f(x))$.
\end{exercice}

\begin{definition}
  On dit que des fonctions $f$ et $g$ telles que pour tout $x$ de $I$,
  intervalle de $\R$, $f(g(x)) = g(f(x)) = x$ sont réciproques l'une de
  l'autre.
\end{definition}

\begin{remarque}
  On note également $f\circ g = g\circ f = Id_I$, où $Id_I:I\to I,x
  \mapsto x$ est la fonction identité qui a un élément associe lui même.
\end{remarque}

\subsection{Une conséquence du théorème des valeurs intermédiaires}

\begin{exercice}
  \begin{itemize}
    \item Rappeler le théorème des valeurs intermédiaires.
    \item Justifier que $\forall y > 0,$ il existe un unique $x\in R$
      tel que $y = e^x$.
  \end{itemize}
\end{exercice}

Comme conséquence des considérations précédentes, on peut définir une
nouvelle fonction, réciproque de la fonction exponentielle.

\begin{definition}
  La fonction réciproque de la fonction $\exp$ s'appelle le logarithme
  népérien, notée $\ln$. Elle est définie de $\interoo{0,+\infty} \to
  \R$ et à chaque élément $x$ de $\interoo{0,+\infty}$ associe l'unique
  solution de l'équation $x = e^y$.
\end{definition}

Quelques conséquences immédiates :
\begin{itemize}
  \item pour tout nombre $x >0,\ e^{\ln(x)} = x$ (c'est la définition) ;
  \item pour tout nombre $x,\ \ln(e^x) = x$ (attention aux conditions
    sur $x$) ;
  \item $\ln(1) = 0$ et $\ln(e) = 1$.
\end{itemize}

\begin{propriete}
  Pour tous nombres $x$ et $y$ strictement positifs, $\ln(xy) = \ln(x) +
  \ln(y)$.
\end{propriete}
\begin{preuve}
  Voir l'exercice \ref{prop:fondamentale}
\end{preuve}

L'exercice \ref{prop:consequence} appliqué à la fonction $\ln$ permet de
démontrer un certain nombre de propriétés «calculatoires» de $\ln$.

Ces propriétés étaient à la base de la règle à caclul.

\begin{exercice}
  Rechercher le fonctionnement de la règle à calcul.
\end{exercice}

\section{Étude de la fonction $\ln$}

On a vu que la fonction exponentielle et la fonction logarithme étaient
intimement liées. Explorons plus en détail ce lien en particulier dans
le cadre de l'étude de la fonction logarithme.

\subsection{Quelques conjectures}

\begin{exercice}
  \begin{itemize}
    \item Démontrer que les points $A:(1,2)$ et $B(2,1)$ sont
      symétriques l'un de l'autre par rapport à $\Delta$.
    \item Étendre cette démonstration au cas général des points
      $M:(x,y)$ et $P:(y,x)$.
    \item Construire par symétrie la courbe représentative de la
      fonction logarithme.
  \end{itemize}
\end{exercice}

\begin{center}
  \begin{tikzpicture}
    \tkzInit[xmin=-3,xmax=5,ymin=-3,ymax=5]
    \tkzGrid
    \tkzAxeXY
    \draw [dashed] plot [domain=-3:5] (\x,\x) node [above right] {
    $\Delta$ } ;
    \draw [thick] plot [domain=-3:1.61] (\x,{exp(\x)}) ;
  \end{tikzpicture}
\end{center}

\begin{propriete}
  Pour tout nombre $x$ strictement positif et pour tout nombre $y$, \[
  \ln(x) = y \iff x = e^y .\]
\end{propriete}
\begin{preuve}
  On montre le sens direct puis le sens réciproque.
  \begin{itemize}
    \item[$\Rightarrow$] Soient $x > 0$ et $y$. $\ln(x) = y \implies
      e^{\ln(x)} = e^y \implies x = e^y$.
    \item[$\Leftarrow$] Soient $x > 0$ et $y$. $x = e^y \implies \ln(x)
      = \ln(e^y) \implies \ln(x) = y$.
  \end{itemize}
\end{preuve}

\begin{exercice}
  Quelles conjectures peut-on former à propos de la fonction $\ln$ :
  \begin{itemize}
    \item continue ou non ;
    \item croissante ou décroissante ;
    \item limite en $-\infty$ et $+\infty$.
  \end{itemize}
\end{exercice}

On peut cependant, sans attendre, démontrer que la fonction est
strictement croissante.

\begin{propriete}
  La fonction $\ln$ est strictement croissante sur
  $\interoo{0;+\infty}$.
  \begin{preuve}
    Soient $0< x < y$ deux réels. On peut écrire $e^{\ln(x)} <
    e^{\ln(y)}$. Or $\exp$ est strictement croissante ($a<b \iff e^a <
    e^b$) d'où $\ln(x) < \ln(y)$.
  \end{preuve}
\end{propriete}

\begin{exercice}
  \begin{itemize}
    \item Compléter $0<x<y \iff $ ;
    \item Démontrer que $x = y \iff \ln(x) = \ln(y)$.
  \end{itemize}
\end{exercice}

\subsection{Compléments sur l'étude de fonction}

\begin{remarque}
  On admet que la fonction $\ln$ est continue sur $\R_+^*$. Le lecteur
  intéressé pourra consulter
  \href{https://fr.wikipedia.org/wiki/Th\%C3\%A9or\%C3\%A8me_de_la_bijection}{https://fr.wikipedia.org/wiki/Théorème\_de\_la\_bijection}
  qui justifie ce point là.
\end{remarque}

\begin{propriete}
  La fonction $\ln$ est dérivable sur $\interoo{0;+\infty}$ et sa
  dérivée $\ln'$ est la fonction inverse ($\forall x \in \R_+^*,\
  \ln'(x) = \frac1x$).
\end{propriete}
\begin{preuve}
  On rappelle que $(f\circ u)' = u'\times f'\circ u$. Prenons $f = \exp$
  et $u = \ln$.

  Ainsi, on a $Id' = \ln'\times \exp' \circ \ln$

  Or $Id: x\mapsto x$ donc $Id' = 1$ et $\exp' \circ \ln = \exp \circ
  \ln = Id$

  Donc pour $x \in \R_+^*,\ 1 = x \ln'(x)$ et donc $\ln'(x) = \frac1x.$
\end{preuve}

\begin{exercice}
  Réécrire cette démonstration en détaillant avec la notation $f(u(x))$.

  On suppose que $f$ et $g$ sont deux fonctions continues, dérivables,
  réciproques l'une de l'autre et que de plus, $f'$ ne s'annule pas.
  Exprimer $g'$ en fonction de $f$, $f'$ et $g$.
\end{exercice}

\begin{exercice}
  Une autre démonstration avec le taux d'accroissement.

  Soit $x$ un élément de $\R_+^*$, construisons $\tau_x(h)$ la fonction
  de la variable $h$ qui donne le taux d'accroissement entre $x$ et $x +
  h$. On va démontrer que $\lim_{h\to 0} \tau_x(h) = \frac1x$.

  \begin{enumerate}
    \item Écrire $\tau_x(h)$.
    \item On pose $u = \ln(x+h)$ et $v = \ln(x)$. Exprimer $x+h$ et $x$
      en fonction de $u$ et $v$ puis démontrer que $\tau_x(h) = \frac{u
      -v}{e^u - e^v}$.
    \item On pose $k = u - v$.
      \begin{enumerate}
        \item Exprimer $u$ en fonction de $v$ et $k$.
        \item En déduire $\tau_x(h)$.
        \item Démontrer que $\lim_{h\to0}k = 0$.
        \item En écrivant le nombre dérivé de l'exponentielle en $v$,
          démontrer que $\lim_{k\to0}\frac{e^{v+k} - e^v}{k} = x$ et en
          déduire $\lim_{k\to0}\frac{k}{e^{v+k} - e^v}$.
        \item En utilisant le théorème sur la limite d'une composée de
          fonction, conclure quant à $\lim_{h\to 0}\tau_x(h)$
      \end{enumerate}
    \item Conclure.
  \end{enumerate}
\end{exercice}

\begin{remarque}
  La démonstration proposée et l'exercice présentant une autre
  démonstration présentent tous les deux le même défaut de reposer sur
  la dérivée d'une composée de fonction ou sur la limite d'une composée
  de fonction\footnote{Mais la première proposition dépend \textit{in
  fine} de la deuxième}.
\end{remarque}

On peut aussi démontrer les limites suivantes.

\begin{propriete}
  \begin{multicols}{2}
    \begin{itemize}
      \item $\lim_{x\to+\infty}\ln x = +\infty$
      \item $\lim_{x\to 0}\ln x = -\infty$
    \end{itemize}
  \end{multicols}
\end{propriete}
\begin{preuve}
  \begin{itemize}
    \item Soit $A = e^M$. Pour tout $x>A$, $\ln(x) > M$ par stricte
      croissance de l'exponentielle (ou du logarithme). On en déduit le
      résultat.
    \item On pose $X = \frac1x$, puis on procéde par composition de
      limite.
  \end{itemize}
\end{preuve}

\begin{exercice}
  Réécrire le deuxième point de la preuve en entier.
\end{exercice}

On peut résumer les variations et les limites dans le tableau de
variation suivant.

\begin{center}
  \begin{tikzpicture}
    \tkzTabInit[espcl=9]
    {$x$/1,$\ln'(x)$/1,$\ln(x)$/2}
    {0,$+\infty$}
    \tkzTabLine{,+,}
    \tkzTabVar{D-/$-\infty$,+/$+\infty$}
    \tkzTabVal{1}{2}{0.33}{1}{0}
    \tkzTabVal{1}{2}{0.67}{e}{1}
  \end{tikzpicture}
\end{center}

On peut donc tracer la courbe représentatitive suivante.

\begin{center}
  \begin{tikzpicture}
    \tkzInit[xmin=-2,xmax=12,ymin=-5,ymax=3]
    \tkzGrid
    \tkzAxeXY
    \draw [thick] plot [smooth,domain=0.007:1] (\x,{ln(\x)}) ;
    \draw [thick] plot [smooth,domain=1:12] (\x,{ln(\x)}) ;
  \end{tikzpicture}
\end{center}

\subsection{Dévelopements limités et autres formules}

On a aussi, de façon analogue aux résultats dits de développements
limités ou de comparaisons, les résultats suivants.

\begin{propriete}~\\[-6ex]
  \begin{multicols}{3}
    \begin{itemize}
      \item $\lim_{x\to0}\frac{\ln(x+1)}x = 1$
      \item $\lim_{x\to +\infty}\frac{\ln(x)}x = 0$
      \item $\lim_{x\to 0}x\ln(x) = 0$
    \end{itemize}
  \end{multicols}
\end{propriete}
\begin{preuve}
  Le premier se montrer par un raisonnement sur le nombre dérivé en 1.
  Les deux autres, par un changement de variable.
\end{preuve}

\begin{exercice}
  Démontrer rigoureusement ces trois limites.
\end{exercice}

La première s'interprète comme le comportement des fonctions $x\mapsto
x$ et $x\mapsto \ln(x+1)$ dans un petit intervalle centré autour de 0
est «similaire». Les suivantes s'interprète comme pour les
exponentielles, ici on dira que $\ln$ croit moins vite que $x\mapsto x$.

On a aussi une dérivée importante à noter.
\begin{propriete}
  Soit $u$ une fonction dérivable strictement positive (c'est à dire à
  valeurs dans $\R_+^*$) définie sur un intervalle $I$.

  $x\mapsto \ln(u(x))$ est dérivable et pour tout $x$ de $I$,
  $(\ln(u(x))' = \frac{u'(x)}{u(x)}$.
\end{propriete}

\begin{exercice}
  Réécrire la proposition précédente avec la notation formelle des
  fonction pour l'expression de la dérivée.
\end{exercice}

\section{D'autres avatars du logarithme}

\subsection{Le logarithme «non naturel»}

\begin{definition}
  Soit $a$ un réel strictement positif.

  On appelle logarithme en base $a$ de $x$ et on note $\log_a(x)$ le
  nombre $\log_a(x) = \frac{\ln(x)}{\ln(a)}$.
\end{definition}

Parmi toutes les bases $a$, deux bases retiennent particulièrement
l'attention des mathématiciens et des physiciens : la base 2 et la base
10.

\begin{remarque}
  Lorsque $a = 10$, on omet souvent de préciser la base et on note
  $\log_a = \log$.
\end{remarque}

On peut se poser la question des variations de ces fonctions.

\begin{propriete}
  Si $a >1$, alors $\log_a$ a les même variations que $\ln$. Si $a<1$,
  alors $\log_a$ a des variations opposées.
\end{propriete}

\begin{exercice}
  Démontrer cette proposition.
\end{exercice}

\subsection{Retour à la physique}

Et voilà, on sait désormais pourquoi on parle d'échelle logarithmique
pour le pH ou pour les dB.

\end{document}
