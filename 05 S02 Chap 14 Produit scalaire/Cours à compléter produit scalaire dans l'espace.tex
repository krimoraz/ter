\documentclass[12pt,a4paper,french]{article}
\usepackage[utf8]{inputenc}
\usepackage[T1]{fontenc}
\usepackage{babel}
\usepackage{ntheorem}
\usepackage{amsmath}
\usepackage{amsfonts}
\usepackage{amssymb}

\usepackage{array}

\usepackage{kpfonts}

\usepackage[bookmarks=false,colorlinks,linkcolor=blue]{hyperref}

\pdfminorversion 7
\pdfobjcompresslevel 3

\usepackage{tabularx}
\usepackage[autolanguage,np]{numprint}
\usepackage{enumitem}

\usepackage{tipfr}
\usepackage{pgf}
\usepackage{tikz}
\usepackage{tkz-euclide}
\usetkzobj{all}
%\usetikzlibrary{hobby}

\usepackage[top=1.7cm,bottom=2cm,left=2cm,right=2cm]{geometry}

\usepackage{lastpage}

\usepackage{esvect}
\usepackage{marginnote}

\usepackage{wrapfig}

\usepackage[defaultlines=5,all]{nowidow}


\makeatletter
\renewcommand{\@evenfoot}%
        {\hfil \upshape \small page {\thepage} de \pageref{LastPage}}
\renewcommand{\@oddfoot}{\@evenfoot}

\renewcommand{\maketitle}%
{\framebox{%
    \begin{minipage}{1.0\linewidth}%
      \begin{center}%
        \Large \@title ~-- \@author \\%
        \@date%
      \end{center}%
    \end{minipage}}%
  \normalsize%
  \vspace{1cm}%
}

\pgfdeclarepatternformonly{mes_hachures}
{\pgfpoint{-0.1cm}{-0.1cm}}
{\pgfpoint{0.9cm}{0.5cm}}
{\pgfpoint{0.8cm}{0.4cm}}
{\pgfpathmoveto{\pgfpointorigin}
  \pgfpathlineto{\pgfpoint{0.8cm}{0.4cm}}
\pgfusepath{stroke}}

%Des macros pour les noms d'ensmbles
\newcommand{\R}{\mathbf{R}}
\newcommand{\Q}{\mathbf{Q}}
\newcommand{\Z}{\mathbf{Z}}
\newcommand{\C}{\mathbf{C}}
\newcommand{\N}{\mathbf{N}}

\newcommand{\norme}[1]{\left\lVert #1 \right\rVert}
\newcommand{\abs}[1]{\left\lvert #1 \right\rvert}

%Une macro récursive pour l'intérieru des vecteurs
%http://tex.stackexchange.com/questions/19693/arguments-of-custom-commands-as-comma-separated-list

\newcommand\vecteur[2][\\]{%
    \global\def\my@delim{#1}%
    \left(\negthinspace\begin{matrix}
        \my@vector #2,\relax\noexpand\@eolst%
    \end{matrix}\right)}

%Une macro pour les vecteurs
\def\my@vector #1,#2\@eolst{%
   \ifx\relax#2\relax
      #1
   \else
      #1\my@delim
      \my@vector #2\@eolst%
   \fi}

%Une macro récursive pour mettre formater l'intérieur des intervalles
\def\my@intervalle #1;#2\@eolst{%
  \ifx\relax#2\relax
    #1
  \else
    \my@intervalle #2\@eolst%
  \fi}

%Quatre macros pour les quatres types d'intervalles
\newcommand{\interff}[1]{%
  \left[\my@intervalle #1;\relax\noexpand\@eolst%
  \right]
}
\newcommand{\interfo}[1]{%
  \left[\my@intervalle #1;\relax\noexpand\@eolst%
  \right[}
\newcommand{\interof}[1]{%
  \left]\my@intervalle #1;\relax\noexpand\@eolst%
  \right]}
\newcommand{\interoo}[1]{%
  \left]\my@intervalle #1;\relax\noexpand\@eolst%
  \right[}

\makeatother


\theoremstyle{break}
\newtheorem{definition}{Définition}
\newtheorem{propriete}{Propriété}
\newtheorem{propdef}{Propriété - Définition}
\newtheorem{theoreme}{Théorème}
\theoremstyle{plain}
\theorembodyfont{\normalfont}
\newtheorem{exercice}{Exercice}
\theoremstyle{nonumberplain}
\newtheorem{remarque}{Remarque}
\newtheorem{probleme}{Problème}
\newtheorem{preuve}{Preuve}
\theoremstyle{nonumberbreak}
\newtheorem{exemple}{Exemple}
\newcommand{\qed}{~\hfill$\Box$}

\setlength{\parsep}{0pt}
\setlength{\parskip}{5pt}
%\setlength{\parindent}{0pt}
\setlength{\itemsep}{7pt}

\setlist{noitemsep}
%\setlist[1]{\labelindent=\parindent} % < Usually a good idea
\setlist[itemize]{leftmargin=*}
\setlist[itemize,1]{label=$\triangleright$}
\setlist[enumerate]{labelsep=*, leftmargin=1.5pc}
\setlist[enumerate,1]{label=\arabic*., ref=\arabic*}
\setlist[enumerate,2]{label=\emph{\alph*}),
ref=\theenumi.\emph{\alph*}}
\setlist[enumerate,3]{label=\roman*), ref=\theenumii.\roman*}
\setlist[description]{font=\sffamily\bfseries}

\usepackage{framed}
\newenvironment{acompleter}{%
  \begin{framed}}
  {\end{framed}%
}

\usepackage{multicol}
\setlength{\columnseprule}{0pt}

\everymath{\displaystyle\everymath{}}

\title{Produit scalaire dans l'espace}
\author{\bsc{Jumel}}
\date{janvier 2017}

\begin{document}

\noindent\maketitle

Au niveau de la classe de Terminale, on n'explore que le produit
scalaire «canonique», obtenu par extensin d'une des définitions et
propriétés vues en classe de Première. Cette définition sera
éventuellement revue sous un nouvel angle dans l'enseignement supérieur.

\section{Définition et premières propriétés}

\begin{acompleter}
  \begin{definition}
    Soient deux vecteurs $\vv{u} = \vecteur{x, y, z}$ et $\vv{v} =
    \vecteur{x',y',z'}$ de l'espace, exprimés dans une même base dite
    canonique.

    \noindent Le nombre réel $xx' + yy' + zz'$ est appelé produit
    scalaire de $\vv{u}$ et $\vv{v}$. Il est noté $\vv{u}\cdot\vv{v}$.
  \end{definition}
\end{acompleter}

\begin{remarque}
  \begin{itemize}
    \item L'expression du produit scalaire dépend de la base choisie.
    \item La base canonique, ou orthonormée, est la base telle que pour
      tout vecteur $\vv{e}$ de celle-ci, $\norme{\vv{e}} = 1$ et pour
      tout couple de deux vecteurs distincts de cette base, leur produit
      scalaire est nul.
    \item Les élements qui «changent l'échellent des vecteurs» sont
      appellés les scalaires
  \end{itemize}
\end{remarque}

On peut d'ailleurs rappeller une propriété du produit scalaire qui reste
vraie dans l'espace.

\begin{propriete}
  \[ \forall \vv{u} \in E,\ \vv{u}\cdot\vv{u} = \norme{\vv{u}}^2 \]
\end{propriete}

Si les vecteurs $\vv{u}$ et $\vv{v}$ ne sont pas colinéaires, il est
toujours possible de se placer dans le plan formé par ces deux vecteurs
et on a la propriété suivante.

\begin{propriete}
  Soit $M$ un point de l'espace.

  \noindent Dans le plan $(O,\vv{u},\vv{v})$, le produit scalaire égale
  le cosinus de l'angle orienté $(\vv{u},\vv{v})$.
\end{propriete}

On dispose toujours de la caractérisation suivante, fort utile en
pratique.

\begin{propriete}
  Soient $\vv{u}$ et $\vv{v}$ deux vecteurs de l'espace non nuls. \[
  \vv{u}\cdot\vv{v} = 0 \iff \vv{u} \perp \vv{v}. \]
\end{propriete}

La précision que les vecteurs $\vv{u}$ et $\vv{v}$ ne sont pas nuls est
importante. En effet, on rappelle les deux propriétés suivantes, qui
découlent de la définition.

\begin{propriete}
  \[ \forall (\vv{u},\vv{v})\in E^2,\ \vv{u}\cdot\vv{v} =
  \vv{v}\cdot\vv{u} \]
\end{propriete}

\begin{propriete}
  \[ \forall \vv{u} \in E,\ \vv{u}\cdot\vv{0} = 0 \]
\end{propriete}

Enfin, on rappelle une dernière proposition importante et également
utile en partique, à savoir utiliser.

\begin{acompleter}
  \begin{propriete}
    Soit $\vv{u}$ un vecteur de l'espace.

    \noindent Si, pour tout vecteur $\vv{v}$ du plan,
    $\vv{u}\cdot\vv{v} = 0$, alors $\vv{u} = 0$.
  \end{propriete}
\end{acompleter}

\section{Vecteur normal à un plan}

\subsection{Approche vectorielle}

Si dans le plan, la notion d'orthogonalité permettait de caractériser
les droites, dans l'espace, cette notion permettra de caractériser les
plans. En effet, donnons nous un vecteur $\vv{u} = \vecteur{x,y,z}$. Les
vecteurs $\vv{v} = \vecteur{0,-z,y}$ et $\vv{w}  = \vecteur{-z,0,x}$
sont deux vecteurs orthogonaux à $\vv{u}$, mais ne sont pas colinéaires.

\begin{acompleter}
  \begin{exercice}
    Vérifier l'affirmation précédente.
    \vspace{3cm}
  \end{exercice}
\end{acompleter}

Les vecteurs $\vv{v}$ et $\vv{w}$ définissent un plan.

\begin{acompleter}
  \begin{exercice} Caractériser, sous forme d'une combinaison linéaire,
    un vecteur du plan $(\vv{v},\vv{w})$.\vspace{1cm}
  \end{exercice}
\end{acompleter}

On peut enfin démontrer que le vecteur $\vv{u}$ est orthogonal à tout
vecteur du plan ainsi défini.

\begin{acompleter}
  \begin{exercice}
    Démontrer que si $\vv{n}$ est orthogonal à deux vecteurs non
    colinéaires d'un plan, alors il est orthogonal à tout vecteur de ce
    plan. \vspace{2cm}
  \end{exercice}
\end{acompleter}

On peut se servir de cette dernière assertion pour démontrer la
proposition suivante.

\begin{propriete}
  Une droite est orthogonale à toute droite du plan si et seulement si
  elle est orthogonale à deux droites sécantes de ce plan.
  \begin{preuve}~\\
    Sens direct : On suppose qu'une droite est orthogonale à toutes les
    droites d'un plan. Elle est donc orthogonale à deux droites sécantes
    de ce plan.\\
    Sens réciproque : On suppose qu'une droite est orthogonale à deux
    droites sécantes du plan. Donc elle possède un vecteur directeur
    orthogonal à deux vecteurs non colinaires du plan. Donc le vecteur
    directeur choisi est orthogonal à tout vecteur du plan, qu'on peut
    considérer comme vecteur directeur d'une droite. Donc la droite est
    orthogonale à toute droite du plan. \qed
  \end{preuve}
\end{propriete}
On peut désormais définir le vecteur normal à un plan et utiliser une
nouvelle caractérisation des plans.

\begin{acompleter}
  \begin{definition}
    On dit que $\vv{u}$ est un vecteur normal au plan.
  \end{definition}
  \begin{propriete}
    La donnée d'un point et d'un vecteur normal définissent enièrement
    un plan de de l'espace.
  \end{propriete}
\end{acompleter}

\subsection{Approche «analytique»}

Soit un plan $\mathscr{P}$ défini par un vecteur normal $\vv{n} =
\vecteur{a,b,c}$ et un point $A:\vecteur{x';y';z'}$. On va démontrer que
dans ce cas, on peut trouver une expression analytique, sous la forme
d'une unique équation qui caractérisera l'appartenance d'un point à un
plan, et donc par voie de conséquence, l'appartenance d'un vecteur à ce
plan.

\begin{acompleter}
  \begin{exercice}
    Redonner la condition de colinéarité de deux vecteurs. \vspace{2cm}
  \end{exercice}
\end{acompleter}

Simplifions le problème posé en se ramenant à un cas plus simple : le
cas où $A:\vecteur{0;0;0}$.

\begin{acompleter}
  \begin{exercice}
    Démontrer que si $\vv{n}$ est orthogonal à $\vv{AM}$, alors $\vv{n}$
    est orthogonal à $\vv{OM'}$ où $M'$ est le translaté de $M$ de
    vecteur $\vv{AO}$.\vspace{3cm}
  \end{exercice}
\end{acompleter}

On se donne désormais un point $M$ appartenant à un plan $\mathscr{P}$
passant par l'origine.

\begin{acompleter}
  \begin{exercice}
    Justifier que si $M$ a pour coordonnées $M:\vecteur{x;y;z}$ et
    $\vv{n} = \vecteur{a,b,c}$ est un vecteur normal au plan
    $\mathscr{P}$, alors les coordonnées de $M$ vérifient $ax + by + cz
    = 0$.\vspace{2cm}
  \end{exercice}
\end{acompleter}

Revenons au premier problème, c'est-à-dire, celui où $A$ n'était pas
l'origine du repère.

\begin{acompleter}
  \begin{exercice}
    En écrivant un vecteur du plan $\mathscr{P}$ sous la forme
    $\vecteur{x - d_1, y - d_2, z - d_3}$, annuler le produit scalaire
    de ce vecteur avec le vecteur normal $\vecteur{a,b,c}$.\vspace{2cm}
  \end{exercice}
\end{acompleter}

On peut désormais qualifier cette dernière équation.

\begin{acompleter}
  \begin{definition}
    Une équation de la forme $ax + by + cz + d =0$ est l'équation
    analytique d'un plan.

    \noindent Les coefficients en $x$, $y$ et $z$ sont les composantes
    du vecteur normal à ce plan, et $d$ est obtenu en multipliant terme
    à terme les composantes du vecteur normal et les coordonnées du
    point.
  \end{definition}
\end{acompleter}

\section{Quelques applications pratiques}

Il est en général pertinent de se poser la question de savoir quelle
représentation (analytique ou paramétrique) est la plus pertinente pour
un plan. Certaines questions, typiquement, l'appartenance d'un point à
un plan peuvent se traiter aussi bien avec l'une ou l'autre des
représentations du plan. En revanche, l'intersection de deux plans est
généralement aisée à traiter avec la représentation paramétrique, alors
que l'intersection de trois plans se traite davantage avec la
représentation analytique.

Parmi les autres applications, on peut citer les propriétés suivantes.

\begin{propriete}
  Deux plans parallèles ont des vecteurs normaux colinéaires
\end{propriete}

\begin{theoreme}[du toit]
  Si deux plans $\mathscr{P}$ et $\mathscr{P}'$ sont sécants selon une
  droite $\mathscr{D}$ et que $\mathscr{P}$ et $\mathscr{P}"$ sont
  sécants selon une droite $\mathscr{D}'$ parallèle à $\mathscr{D}$,
  alors $\mathscr{P}'$ et $\mathscr{P}"$ sont sécants selon une droite
  $\mathscr{D}"$ parallèle $\mathscr{D}$ et à $\mathscr{D}'$.
\end{theoreme}

\begin{center}
  \begin{tikzpicture}
    \tkzInit
    \tkzDefPoint(0,0){A}
    \tkzDefPoint(2,0){B}
    \tkzDefPoint(1,1){C}

    %\tkzDrawLines[add=1 and 1](A,B A,C B,C)
    \begin{scope}[shift=(45:3)]
      \tkzDefPoint(0,0){A'}
      \tkzDefPoint(2,0){B'}
      \tkzDefPoint(1,1){C'}
    \end{scope}
    %\tkzDrawLine[dashed, add=1 and 0](A',B')

  \end{tikzpicture}
\end{center}

\end{document}
