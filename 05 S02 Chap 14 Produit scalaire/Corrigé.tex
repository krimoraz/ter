\documentclass[11pt]{article}
\usepackage[T1]{fontenc}
\usepackage[utf8]{inputenc}
\usepackage{fourier}
\usepackage[scaled=0.875]{helvet}
\renewcommand{\ttdefault}{lmtt}
\usepackage{amsmath,amssymb,makeidx}
\usepackage[normalem]{ulem}
\usepackage{fancybox}
\usepackage{enumerate}
\usepackage{tabularx}
\usepackage{ulem}
\usepackage{dcolumn}
\usepackage{textcomp}
\newcommand{\euro}{\eurologo{}}
\usepackage{pstricks,pst-plot,pst-text,pst-tree}
\everymath{\displaystyle}
\setlength\paperheight{297mm}
\setlength\paperwidth{210mm}
\setlength{\textheight}{23,5cm}
\newcommand{\R}{\mathbb{R}}
\newcommand{\N}{\mathbb{N}}
\newcommand{\D}{\mathbb{D}}
\newcommand{\Z}{\mathbb{Z}}
\newcommand{\Q}{\mathbb{Q}}
\newcommand{\C}{\mathbb{C}}
\newcommand{\vect}[1]{\mathchoice%
{\overrightarrow{\displaystyle\mathstrut#1\,\,}}%
{\overrightarrow{\textstyle\mathstrut#1\,\,}}%
{\overrightarrow{\scriptstyle\mathstrut#1\,\,}}%
{\overrightarrow{\scriptscriptstyle\mathstrut#1\,\,}}}
\renewcommand{\theenumi}{\textbf{\arabic{enumi}}}
\renewcommand{\labelenumi}{\textbf{\theenumi.}}
\renewcommand{\theenumii}{\textbf{\alph{enumii}}}
\renewcommand{\labelenumii}{\textbf{\theenumii.}}
\def\Oij{$\left(\text{O},~\vect{\imath},~\vect{\jmath}\right)$}
\def\Oijk{$\left(\text{O},~\vect{\imath},~ \vect{\jmath},~ \vect{k}\right)$}
\def\Ouv{$\left(\text{O},~\vect{u},~\vect{v}\right)$}
\usepackage{fancyhdr} 
\textheight 23,5cm 
\voffset -1.5cm \oddsidemargin 0pt 
\usepackage{esvect}
\usepackage[frenchb]{babel}
\usepackage[np]{numprint}

\begin{document}
\setlength\parindent{0mm}

\vspace{0,25cm}

\textbf{\textsc{Exercice 1}}

\medskip

On considère le cube ABCDEFGH d'arête de longueur 1 représenté ci-dessous.

L'espace est rapporté au repère orthonormal $\left(\text{A}~;~\vect{\text{AB}},\,\vect{\text{AD}},\,\vect{\text{AE}}\right)$.

\medskip

\begin{enumerate}
  \item Démontrer que le vecteur $\vect{n}$ de coordonnées $(1~;~0~;~1)$
    est un vecteur normal au plan (BCE).

    $\vv{BC} = \vv{AD} = \left(\begin{matrix}0\\1\\0\end{matrix}\right)$

    $\vv{BC} \cdot \vv{n} = 0$

    $\vv{BE} = \vv{BA} + \vv{AE} = -\vv{AB} + \vv{AE} =
    \left(\begin{matrix}-1\\0\\1\end{matrix}\right)$

    $\vv{BE} \cdot \vv{n} = -1\times 1 + 0\times 0 + 1\times 1 = 0$.

    Donc $\vv{n}$ est orthogonal à $(BCE)$.

  \item Déterminer une équation du plan (BCE).

    $x + z + d =0$ est une équation du plan. Le point $B$ de coordonnées
    $(1;0;0)$ appartient au plan $(BCE)$ donc $d = -1$.

    L'équation du plan est $\boxed{x + z - 1 = 0}$

  \item On note $(\Delta)$ la droite perpendiculaire en E au plan (BCE).

    Déterminer une représentation paramétrique de la droite $(\Delta)$.

    $\vv{n}$ est un vecteur directeur de cette droite $(\Delta)$. On a
    donc \[(\Delta): \left\lbrace\begin{matrix}x = t\\y = 0\\ z = 1 +
    t\end{matrix}\right.,t\in\R \]
  \item Démontrer que la droite $(\Delta)$ est sécante au plan (ABC) en
    un point R, symétrique de B par rapport à A.

    $(ABC) = (ABD)$. $\vv{AE}$ est orthogonal à $\vv{AB}$ et $\vv{AD}$.
    On en déduit que $\boxed{z=0}$ est une équation cartésienne du plan
    $(ABC)$.

    Le point d'intersection vérifie le système d'équation
    $\left\lbrace\begin{matrix}x = t\\y = 0\\ z = 1 + t \\ z  = 0
    \end{matrix}\right.$ qui est équivalent à
    $\left\lbrace\begin{matrix}x = t\\y = 0\\ z = 1 + t \\ t + 1 =
    0\end{matrix}\right.$ dont l'unique solution est $(-1;0;0)$. Si on
    donne au point $R$ ses coordonnées, on a bien que $\vv{RA} =
    \vv{AB}$, ce qui correspond à la représentation vectorielle d'une
    symétrie centrale.

  \item Démontrer que $\vv{DR} -\vv{DB} + 2\vv{DC} = \vv{0}$

    $\vv{DR} -\vv{DB} + 2\vv{DC} = \vv{DA} + \vv{AR} - \vv{DA} - \vv{AB}
    + 2 \vv{DC} = 2\vv{BA} + 2\vv{DC} = 2\vv{BA} + 2\vv{AB} = \vv{0}$

    car $\vv{AR} = \vv{AB}$ et $\vv{DC} = \vv{AB}$.
\end{enumerate}

\end{document}
