\documentclass[11pt]{article}
\usepackage[T1]{fontenc}
\usepackage[utf8]{inputenc}
\usepackage{fourier}
\usepackage[scaled=0.875]{helvet}
\renewcommand{\ttdefault}{lmtt}
\usepackage{amsmath,amssymb,makeidx}
\usepackage[normalem]{ulem}
\usepackage{fancybox}
\usepackage{enumerate}
\usepackage{tabularx}
\usepackage{ulem}
\usepackage{dcolumn}
\usepackage{textcomp}
\newcommand{\euro}{\eurologo{}}
\usepackage{pstricks,pst-plot,pst-text,pst-tree}
\everymath{\displaystyle}
\setlength\paperheight{297mm}
\setlength\paperwidth{210mm}
\setlength{\textheight}{23,5cm}
\newcommand{\R}{\mathbb{R}}
\newcommand{\N}{\mathbb{N}}
\newcommand{\D}{\mathbb{D}}
\newcommand{\Z}{\mathbb{Z}}
\newcommand{\Q}{\mathbb{Q}}
\newcommand{\C}{\mathbb{C}}
\newcommand{\vect}[1]{\mathchoice%
{\overrightarrow{\displaystyle\mathstrut#1\,\,}}%
{\overrightarrow{\textstyle\mathstrut#1\,\,}}%
{\overrightarrow{\scriptstyle\mathstrut#1\,\,}}%
{\overrightarrow{\scriptscriptstyle\mathstrut#1\,\,}}}
\renewcommand{\theenumi}{\textbf{\arabic{enumi}}}
\renewcommand{\labelenumi}{\textbf{\theenumi.}}
\renewcommand{\theenumii}{\textbf{\alph{enumii}}}
\renewcommand{\labelenumii}{\textbf{\theenumii.}}
\def\Oij{$\left(\text{O},~\vect{\imath},~\vect{\jmath}\right)$}
\def\Oijk{$\left(\text{O},~\vect{\imath},~ \vect{\jmath},~ \vect{k}\right)$}
\def\Ouv{$\left(\text{O},~\vect{u},~\vect{v}\right)$}
\usepackage{fancyhdr} 
\textheight 23,5cm 
\voffset -1.5cm \oddsidemargin 0pt 
\usepackage{esvect}
\usepackage[frenchb]{babel}
\usepackage[np]{numprint}

\begin{document}
\setlength\parindent{0mm}

\vspace{0,25cm}

\textbf{\textsc{Exercice 1}}

\medskip

On considère le cube ABCDEFGH d'arête de longueur 1 représenté ci-dessous.

L'espace est rapporté au repère orthonormal $\left(\text{A}~;~\vect{\text{AB}},\,\vect{\text{AD}},\,\vect{\text{AE}}\right)$.

\medskip

\begin{enumerate}
  \item Démontrer que le vecteur $\vect{n}$ de coordonnées $(1~;~0~;~1)$
    est un vecteur normal au plan (BCE).
  \item Déterminer une équation du plan (BCE).
  \item On note $(\Delta)$ la droite perpendiculaire en E au plan (BCE).

    Déterminer une représentation paramétrique de la droite $(\Delta)$.
  \item Démontrer que la droite $(\Delta)$ est sécante au plan (ABC) en
    un point R, symétrique de B par rapport à A.
  \item Démontrer que $\vv{DR} -\vv{DB} + 2\vv{DC} = \vv{0}$
\end{enumerate}

\medskip
\psset{unit=1cm}
\begin{center}
  \begin{pspicture}(6,6)
    \psline(0.4,4)(0.4,0.6)(4,0)(5.3,1.5)(5.3,5)(4,3.5)(0.4,4)(1.7,5.5)(5.3,5)%EABCGFEHG
    \psline(4,0)(4,3.5)
    \uput[ul](0.4,4){E} \uput[dl](0.4,0.6){A} \uput[dr](4,0){B} \uput[r](5.3,1.5){C}
    \uput[ur](5.3,5){G} \uput[ul](4,3.5){F} \uput[u](1.7,5.5){H} \uput[ul](1.7,2.1){D}
    \psline[linestyle=dashed](0.4,0.6)(1.7,2.1)(1.7,5.5)
    \psline[linestyle=dashed](1.7,2.1)(5.3,1.5)
  \end{pspicture}
\end{center}

\end{document}
