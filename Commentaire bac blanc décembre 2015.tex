\documentclass[12pt,a4paper]{article}
\usepackage[utf8]{inputenc}
\usepackage[T1]{fontenc}
\usepackage[french]{babel}
\usepackage{ntheorem}
\usepackage{amsmath}
\usepackage{amsfonts}

\usepackage{kpfonts}
\usepackage{textcomp}
\newcommand{\euro}{\eurologo{}}

\usepackage[bookmarks=false,colorlinks,linkcolor=blue]{hyperref}
\usepackage{breakurl}

\pdfminorversion 7
\pdfobjcompresslevel 3

\frenchbsetup{ItemLabels=\textbullet,}

\usepackage{tabularx}
\usepackage{enumitem}

\usepackage{pgf}
\usepackage{tikz}
\usepackage{tkz-euclide}
\usetkzobj{all}
\usetikzlibrary{hobby}
\usepackage{tkz-tab}
\usepackage[top=1.4cm,bottom=1.4cm,left=2cm,right=2cm,includehead,
includefoot]{geometry}

\usepackage{lastpage}

\usepackage{fancybox}

\usepackage[autolanguage]{numprint}
\newcommand{\np}{\numprint}

\usepackage{ifthen}
\usepackage{fancyhdr}
\pagestyle{fancy}

\lhead{\ifthenelse{\value{page}=0}{Nom:\dotfill\hfill Prénom: \dotfill
\hfill ~}{}}
\rhead{}
\chead{}

\rfoot{\upshape \small page {\thepage} de \pageref{LastPage}}
\lfoot{}
\cfoot{}
\renewcommand{\headrulewidth}{0pt}
\renewcommand{\footrulewidth}{0pt}

\usepackage{pdfmarginpar}

\makeatletter
\renewcommand{\maketitle}%
{\framebox{%
    \begin{minipage}{0.98\linewidth}%
      \begin{center}%
        \Large \@title ~-- \@author \\%
        \@date%
      \end{center}%
  \end{minipage}}%
  \normalsize%
  %\vspace{1cm}%
}

\makeatother

\theoremstyle{break}
\newtheorem{definition}{Définition}
\theorembodyfont{\normalfont}
\theoremstyle{nobreak}
\newtheorem{exercice}{Exercice}

%\theoremstyle{plain}
\theoremstyle{nonumberplain}
\newtheorem{probleme}{Problème}

% Mise en forme des labels dans les énumérations
\renewcommand{\labelenumi}{\textbf{\theenumi.}}
\renewcommand{\labelenumii}{\textbf{\theenumii)}}
\renewcommand{\theenumi}{\arabic{enumi}}
\renewcommand{\theenumii}{\alph{enumii}}

\renewcommand{\labelitemi}{$\bullet$}

\setlength{\parsep}{0pt}
\setlength{\parskip}{5pt}
\setlength{\parindent}{0pt}
\setlength{\itemsep}{7pt}

\usepackage{multicol}
\setlength{\columnseprule}{0pt}

\everymath{\displaystyle\everymath{}}

\newcommand{\N}{\mathbf{N}}
\newcommand{\Z}{\mathbf{Z}}
\newcommand{\Q}{\mathbf{Q}}
\newcommand{\R}{\mathbf{R}}

\newcommand{\ligne}[1]{%
  \begin{tikzpicture}[]
    \draw[white] (0,#1+0.8) -- (15,#1+0.8) ;
    \foreach \i in {1,...,#1}
    { \draw[dotted] (0,\i) -- (\linewidth,\i) ; }
    \draw[white] (0,0.6) -- (15,0.6) ;
  \end{tikzpicture}%
}

\usepackage{xcolor}
\usepackage{framed}
\usepackage{algorithm}
\usepackage{algpseudocode}
\definecolor{fond}{RGB}{136,136,136}
\definecolor{sicolor}{RGB}{128,0,128}
\definecolor{tantquecolor}{RGB}{221,111,6}
\definecolor{pourcolor}{RGB}{187,136,0}
\definecolor{bloccolor}{RGB}{128,0,0}
\newenvironment{cadrecode}{%
  \def\FrameCommand{{\color{fond}\vrule width
  5pt}\fcolorbox{fond}{white}}%
  \MakeFramed {\hsize \linewidth \advance\hsize-\width
\FrameRestore}\begin{footnotesize}}%
{\end{footnotesize}\endMakeFramed}
\makeatletter
\def\therule{\makebox[\algorithmicindent][l]{\hspace*{.5em}\color{fond}
\vrule width 1pt height .75\baselineskip depth .25\baselineskip}}%
\newtoks\therules
\therules={}
\def\appendto#1#2{\expandafter#1\expandafter{\the#1#2}}
\def\gobblefirst#1{#1\expandafter\expandafter\expandafter{\expandafter\@gobble\the#1}}%
\def\Ligne{\State\unskip\the\therules}% 
\def\pushindent{\appendto\therules\therule}%
\def\popindent{\gobblefirst\therules}%
\def\printindent{\unskip\the\therules}%
\def\printandpush{\printindent\pushindent}%
\def\popandprint{\popindent\printindent}%
\def\Variables{\Ligne \textcolor{bloccolor}{\textbf{VARIABLES}}}
\def\Si#1{\Ligne \textcolor{sicolor}{\textbf{SI}} #1
\textcolor{sicolor}{\textbf{ALORS}}}%
\def\Sinon{\Ligne \textcolor{sicolor}{\textbf{SINON}}}%
\def\Pour#1#2#3{\Ligne \textcolor{pourcolor}{\textbf{POUR}} #1
  \textcolor{pourcolor}{\textbf{ALLANT\_DE}} #2
\textcolor{pourcolor}{\textbf{A}} #3}%
\def\Tantque#1{\Ligne \textcolor{tantquecolor}{\textbf{TANT\_QUE}} #1
\textcolor{tantquecolor}{\textbf{FAIRE}}}%
\algdef{SE}[WHILE]{DebutTantQue}{FinTantQue}
{\pushindent \printindent
\textcolor{tantquecolor}{\textbf{DEBUT\_TANT\_QUE}}}
{\printindent \popindent
\textcolor{tantquecolor}{\textbf{FIN\_TANT\_QUE}}}%
\algdef{SE}[FOR]{DebutPour}{FinPour}
{\pushindent \printindent
\textcolor{pourcolor}{\textbf{DEBUT\_POUR}}}
{\printindent \popindent
\textcolor{pourcolor}{\textbf{FIN\_POUR}}}%
\algdef{SE}[IF]{DebutSi}{FinSi}%
{\pushindent \printindent
\textcolor{sicolor}{\textbf{DEBUT\_SI}}}
{\printindent \popindent
\textcolor{sicolor}{\textbf{FIN\_SI}}}%
\algdef{SE}[IF]{DebutSinon}{FinSinon}
{\pushindent \printindent
\textcolor{sicolor}{\textbf{DEBUT\_SINON}}}
{\printindent \popindent
\textcolor{sicolor}{\textbf{FIN\_SINON}}}%
\algdef{SE}[PROCEDURE]{DebutAlgo}{FinAlgo}
{\printandpush
\textcolor{bloccolor}{\textbf{DEBUT\_ALGORITHME}}}%
{\popandprint
\textcolor{bloccolor}{\textbf{FIN\_ALGORITHME}}}%
\makeatother
\newenvironment{algobox}%
{%
  \begin{ttfamily}
    \begin{algorithmic}[1]
      \begin{cadrecode}
        \labelwidth 1.5em
        \leftmargin\labelwidth
        \addtolength{\leftmargin}{\labelsep}
      }
      {%
      \end{cadrecode}
    \end{algorithmic}
  \end{ttfamily}
}

\newcommand{\maximaout}[3]{%
\begin{framed}%
  {\ttfamily \textcolor{red}{(\%i#1)} #2\\%
      \textcolor{brown}{(\%o#1)} #3}%
\end{framed}%
}
\usepackage{wasysym}

\newcommand{\vf}{%
  \Square{}~ Vrai \-\- \Square{}~ Faux%
}

\renewcommand{\Vec}[1]{\overrightarrow{#1}}
\newcommand{\ProSca}[2]{#1 \cdot #2}

\title{Commentaires sur 29 copies de bac blanc}
\author{\bsc{Jumel}}
\date{décembre 2015}

\begin{document}

\maketitle

\emph{Compétences :}
\begin{itemize}
  \item Étudier la limite d'une somme, d'un produit ou d'un quotient de
    deux suites ;
  \item Utilisation des théorèmes de comparaison ;
  \item Savoir mener un raisonnement par récurrence ;
  \item Construire un arbre pondéré de probabilités ;
  \item Théorème des valeurs intermédiaires ;
  \item Limites de fonctions (et interprétations graphiques) ;
  \item Dérivation et variation d'une fonction :
  \item Géométrie dans l'espace.
\end{itemize}

\section*{Introduction}
La rédaction est assez pauvre, lorsque présente. On trouve des
expressions au sens bien étrange : «on fait …» qui ne correspondent pas
au niveau de langue attendue en classe de Terminale générale. On trouve
également, dans les copies ayant fait un effort de rédaction de
nombreuses fautes de syntaxe élémentaire, qui nuisent à l'impression
globale de la copie.

De nombreuses copies donnent l'impression que les connaissances des
années antérieures ne sont pas maîtrisées, c'est dommage, les
compétences de Terminale sont donc faiblement acquises. En particulier,
la validité des égalités n'est jamais vérifiée ou annoncée.

Beaucoup d'élèves n'ont pas du prendre le temps de se relire : ils ont
laissé passé de graves incohérences soit entre leurs réponses
successives soit entre l'énoncé et leurs réponses. Cela indique une
déconnexion entre les questions qui n'est pas réelle. On notera
d'ailleurs que bien peu d'élèves ont traité les questions
d'interprétation du premier exercice.

\section*{Commentaires sur les exercices}

\begin{exercice}[Une étude de suites, dans un cas concret]~

  L'objet de cet exercice était d'étudier une modélisation de
  l'évolution de population particulière. Sans être nécessairement
  introduit en tant que tel, le coefficient $k$ correspond à un taux de
  renouvellement, $u_n$ à une population exprimée en millions.

  Dans le premier cas, $k<1$ (pas de renouvellement des populations) on
  observe une décroissance de la population et même une extinction.

  Dans le deuxième cas, et c'est le but du tableau, on constate une
  stabilisation de la population vers une valeur qui ne dépend pas de la
  population initiale, mais surtout du facteur de renouvellement. Ainsi,
  lorsque le taux de natalité est inférieur à deux, on obtient une
  convergence vers le demi-million d'individus.

  Une étude plus détaillée se trouve sur
  \href{https://interstices.info/jcms/i_56750/modeliser-la-dynamique-des-populations-animales-la-predation}{interstices.info
  : modeliser-la-dynamique-des-popu\-lations-animales-la-predation}

  Le but de l'exercice était de vérifier que les élèves savent rédiger
  un raisonnement par récurrence, déterminer la limite d'une suite
  définie par une relation de récurrence et exploiter ce résultat. Une
  des question sert précisément à montrer l'existence d'une racine.
  Parmi les points à maîtriser, la connaissance des fonctions
  polynomiales de degré 2 était un atout certain qui évitait un
  développement fastidieux.

  Dans l'ensemble l'exercice n'a pas été bien réussi par les élèves, qui
n'ont pas su interpréter les résultats en terme d'évolution de la
  population.

  \underline{Cas \no{} 1}
  \begin{enumerate}
    \item Il s'agit d'une récurrence qu'il faut écrire précisément dans
      laquelle il n'y a pas de difficulté technique autre que sur les
      inégalités dans l'hérédité.
    \item L'objet de cette question est le calcul de la limite de la
      suite $u_n$. Ici, celle-ci est encadrée par 0 et une suite
      géométrique de raison $k\in\left]0;1\right[$. Le théorème
      d'encadrement nous fournit la limite de la suite.
    \item Puisque la limite est nulle, il y a extinction de la
      population.
  \end{enumerate}
  \underline{Cas \no{} 2}
  \begin{enumerate}
    \item \begin{enumerate}
           \item Un algorithme assez basique de calcul qui permet
             simplement de remplir  de façon itérative un tableau de
             valeur d'une suite récurrente.

             Il suffisait d'effectuer un calcul ou de mettre en œuvre
             rapidement l'algorithme sur la calculatrice.
           \item On cherchait juste à établir une conjecture sensée à
             partir du tableau de valeurs précédent. Quelques élèves ont
             fournit des résultats étranges.
         \end{enumerate}
       \item On rentrait désormais dans l'étude approfondie de la
         fonction $f$ proposé pour en déduire des informations sur la
         convergence de la suite, éventuellement une version explicite.
         \begin{enumerate}
           \item L'étude des variations d'un polynôme de degré 2 est du
             programme de Seconde, consolidée par une étude exhaustive
             en classe de Première S.
           \item La question précédente permettait de mettre en lumière
             la continuité de la fonction précédente et d'utiliser les
             connaissances de Terminale S : le théorème des valeurs
             intermédiaires qui permet d'affirmer de façon générale que
             l'image d'un intervalle est un intervalle, sous réserve de
             continuité de la fonction.
           \item Une récurrence évidente permettait de démontrer cette
             inégalité en utilisant le résultat de la question
             précédente qui pouvait être admis ici.
           \item Suite croissante majorée.
           \item La population se stabilise vers \np{473000} individus.
         \end{enumerate}
  \end{enumerate}
\end{exercice}

\begin{exercice}[Une étude de fonction]~

  Cet exercice consistait en une étude de fonction dédiée à la recherche
  d'une solution $\alpha$ pour une certaine fonction donnée (partie A)
  et au calcul de l'image de cette solution par la fonction (partie B)
  et enfin, l'étude de la direction asymptotique de la fonction (partie
  C)

  \noindent\textbf{\underline{Partie A}}

  Sans l'écrire explicitement, on ne s'intéresse qu'au numérateur de la
  fonction de la partie B. En effet, c'est lui qui peut s'annuler,
  annulant ainsi la fonction $f$.

  \begin{enumerate}
    \item Il s'agissait de déterminer des limites en $+\infty$ et
      $-\infty$. Ici le calcul se menait de la même façon dans les deux
      cas et pouvait être expédié rapidement pour la deuxième limite.

      D'autre part, certains élèves écrivent des égalités avec $x$ au
      dénominateur sans que ce soit une limite en $+\infty$, c'est à
      dire que l'égalité n'est pas valide (ex : $3x^2 -1 = x(3x -
      \frac1x)$ n'est vraie que si $x>A>0$.)
    \item Le calcul de la dérivée n'a pas posé de problème ici aux
      élèves. Cependant, une meilleure connaissance des fonctions
      polynomiales de degré 2 aurait évité de se lancer dans le calcul
      du discriminant pour trouver le tableau de signe.

      Le sens de variation en fonction du signe de la dérivée est une
      connaissance qui semble assez solide chez les élèves. On trouve
      cependant des incohérences ici. Il ne faut pas hésiter à tracer la
      fonction sur la calculatrice afin de valider le résultat écrit.

      Certains élèves écrivent des tableaux de variations laissant
      entendre qu'une fonction peut-être croissante de -7 à -9
      sous-entendant que $-7<-9$. Il est impératif de se relire pour
      éviter de telles erreurs.
    \item La question a été souvent abordée de façon très incomplète, il
      fallait justifier que la fonction était continue, strictement
      monotone sur 3 intervalles différents, et qu'un seul contenait 0.
      La conclusion se faisait en citant le théorème (ou son corollaire)
      des valeurs intermédiaires.

      Quelques élèves ont cité le théorème de la bijection, qui n'est
      plus au programme.
    \item Question bien réussie dans l'ensemble, validant une démarche
      par balayage à la calculatrice. Peu d'élève semblent avoir écrit
      un algorithme sur leur calculatrice ou utilisé les fonctions
      idoines d'icelle.
    \item Une partie des élèves n'a pas fait le lien avec la question
      précédente. $\alpha$ a désormais un statut unique de nombre et
      peut être utilisé. La question de l'encadrement a d'ailleurs
      permis de le placer sur la droite des nombres réels.
  \end{enumerate}

  \noindent\textbf{\underline{Partie B}}

  On considère une fonction définie un intervalle donné. De nombreux
  candidats ont étudié la fonction sur l'ensemble de son intervalle, ce
  qui n'était pas demandé ici, d'autant plus que cette fonction
  présente, de part sa parité, une symétrie centrale.

  \begin{enumerate}
    \item Il s'agissait d'une question sur les limites séparées en deux
      questions intermédiaires.
      \begin{enumerate}
        \item La plus part des candidats a su exploiter correctement la
          limite ici. On regrette que beaucoup d'entre-eux se laissent
          aller à démontrer d'abord qu'il s'agit d'une forme
          indéterminée puis à «lever l'indétermination». Les
          abréviations sont à proscrire ici.
        \item En revanche, beaucoup de candidats n'ont pas bien compris
          que l'essence de cette question était de mettre en évidence
          une limite infinie aux bornes de l'intervalle de définition et
          donc la présence d'une asymptote verticale.
      \end{enumerate}
    \item La question a été bien réussie dans l'ensemble pour les
      candidats qui s'y sont frottés, c'est à dire une grande majorité.
      Cependant, beaucoup n'ont pas fait le lien avec la partie
      précédente : $f'$ s'annule et change de signe en $\alpha$, ce qui
      d'ailleurs motive la question 2.
    \item Quelques élèves ont essayé d'encadrer $\alpha$ ou ses valeurs
      approchées pour essayer de démontrer la formule proposée. Bien peu
      y sont arrivés.

      Encore moins et c'est dommage ont donné un encadrement de
      $f(\alpha)$.
  \end{enumerate}

  \noindent\textbf{\underline{Partie C}}

  On s'intéresse dans cette partie au comportement asymptotique de la
  fonction, c'est à dire son comportement au voisinage de $+\infty$.

  \begin{enumerate}
    \item Quelques élèves n'ont pas tracé la droite correctement. Les
      résultats s'en ressentent. Il s'agit d'élèves tombant souvent dans
      la première catégorie (cf. infra.)
    \item La conjecture a été faite souvent moyennement, sa
      démonstration très rarement.
    \item Le calcul fait précédemment devait permettre d'affirmer que
      $MN\approx 0$ quand $x\to \infty$. Peu l'ont traité, et même peu
      ont répondu à la question.
  \end{enumerate}

\end{exercice}

\begin{exercice}[Suite et probabilités]~

  L'exercice consistait en un tirage de dé un peu particulier avec une
  répétition d'une des expériences puis la construction de la suite des
  résultats, ainsi que la question de sa convergence. La suite ainsi
  construite était une suite de probabilité, donc à valeurs dans
  $\left[0;1\right]$, ce que quelques élèves semblent avoir oublié au
  moment de conclure.

  \begin{enumerate}
    \item Le rôle de cette question était d'observer ce qui se passe
      avant la répétions afin de bien fixer les idées. Cette question a
      été plutôt bien traitée.
      \begin{enumerate}
        \item L'arbre de probabilité a été assez bien complété dans
          l'ensemble.
        \item Sans forcément justifier (l'arbre était au dessus), la
          plus part des candidats ont utilisé la formule des
          probabilités totales.
      \end{enumerate}
    \item On répétait désormais la deuxième partie de l'expérience.
      \begin{enumerate}
        \item Une partie des élèves donne les probabilités certainement
          par extraction de la réponse suivante. Il est rare d'avoir
          la justification de la répétition d'une expérience, avec la
          considération d'indépendance des événements.
        \item On trouve un certain nombre de rédaction étranges ici pour
          cette question et bien peu sont valables. Il s'agissait
          formellement pourtant de la même réponse qu'à la question
          1.b).
        \item On introduisait une suite ici, celle de la probabilité de
          tirer le dé rouge. On voulait, en utilisant la formule de
          Bayes retrouver la probabilité d'avoir tirer un dé
          particulier.

          Un élève a pensé à vérifier que cette égalité était vraie pour
          $n=1$, mais dans le cadre d'un raisonnement par récurrence qui
          était inutile ici.
        \item La limite de cette suite a donné lieu à des résultats
          étranges : certains élèves ont oublié qu'il s'agissait de
          probabilité et que le résultat était entre 0 et 1. Bien peu se
          sont donnés la peine de vérifier la cohérence du résultat avec
          l'énoncé de départ. En effet, il y'a «plus de chance que le 6
          vienne du dé rouge que du dé vert» donc si on connaît la
          probabilité d'avoir obtenu 6, la probabilité de venir du dé
          rouge est grande.
      \end{enumerate}
    \item On admettait ici que la suite était croissante, ce qui
      simplifiait l'étude.
      \begin{enumerate}
        \item L'existence de cet entier $n_0$ est garantie par la
          définition de la convergence d'une suite. D'ailleurs, celle-ci
          étant croissante, elle est majorée par sa limite.
        \item L'algorithme proposé ici était celui qui calcule les
          termes de la suite : on définit $E$ la précision, puis tant
          que $ 1 - p_n \geqslant E$, on incrémente $n$.
      \end{enumerate}
  \end{enumerate}
\end{exercice}

\begin{exercice}[Géométrie dans l'espace]~

  L'exercice faisait appel à un certain nombre de connaissance des
  années antérieures, en particulier du collège, ou de la classe de
  Seconde. Peu l'ont abordé.

  Beaucoup ont perdu temps à rédiger longuement des réponses à des
  questions qui ne leur était pas demandées ou qui possédaient des
  réponses très simples.

  \begin{enumerate}
    \item
      \begin{enumerate}
        \item Deux méthodes permettait de répondre à la question :
          \begin{itemize}
            \item en utilisant le théorème des milieux ;
            \item en utilisant la proportionnalité des vecteurs.
          \end{itemize}
          Dans l'ensemble, la question a été assez bien traitée.
        \item On pouvait dans cette question raisonner par l'absurde :
          montrer, en utilisant le théorème de Thalès ou la
          proportionnalité des vecteurs, que les droites ne pouvaient
          pas être parallèles et comme elles étaient dans le même plan,
          alors, elles étaient sécantes.

          Quelques élèves n'ont pas bien lu l'énoncé et ont effectué la
          démonstration pour l'affirmation qui était admise.
        \item L'intersection de plan est une droite, certains élèves
          semblent l'avoir oublié. C'est dommage. Les points
          d'intersections trouvés à la question précédentes
          fournissaient la réponse.
      \end{enumerate}
    \item Cette question devait se traiter uniquement à l'aide des
      relations vectorielles. Beaucoup ont cherché à développer
      longuement le calcul espérant perdre le lecteur et affirmer
      triomphalement le résultat. Cette technique ne fonctionne guère.
      \begin{enumerate}
        \item Placer le point $G$ dans le plan aurait grandement aidé
          les candidats à trouver l'intuition de décomposition
          convenable. Cependant, quelques uns, certainement après une
          recherche au brouillon, ont été en mesure de répondre avec
          satisfaction à la question.

          L'interprétation en tant que points coplanaires n'a que peu
          été traitée.
        \item Le point $G$ était défini partout comme un isobarycentre,
          notion qui n'est désormais plus au programme. On trouvait
          néanmoins que $G$ était le milieu du sagement $[KE]$ puis que
          $G$ était le centre de gravité du triangle $IJK$.
      \end{enumerate}
    \item Cette question permettait de tester des connaissances de cours
      sur les notions de représentation de droites ou de plan. La plus
      part des élèves ne prennent d'ailleurs pas les «meilleurs
      \footnote{au sens de la simplicité}» vecteurs directeurs.
      \begin{enumerate}
        \item Il s'agissait d'une question de cours. Elle a été assez
          bien traité dans l'ensemble, bien que de nombreux élèves
          n'aient pas choisie une représentation plus simple qui
          simplifierait les calculs suivants.

          Il est à noter que trop d'élèves ne précisent pas que le
          paramètre «vit» dans l'ensemble $\R$ des nombres réels (et
          qu'à ce sujet tout droite de l'espace est une bijection de la
          droite des réels.)
        \item On peut refaire les mêmes remarques à cette question qu'à
          la question précédente.
        \item Comme les coordonnées du point $H$ étaient données, la
          démonstration se limitait à une vérification d'appartenance.
        \item La réponse est positive et pouvait se faire par des
          considérations de géométrie (penser à la construction du
          centre de gravité d'un triangle, et à la projection
          homothétique d'un centre de gravité d'un triangle sur un
          autre.)
      \end{enumerate}
  \end{enumerate}
\end{exercice}

\section*{Conclusion}

Deux types d'élèves semblent se dégager :
\begin{itemize}
  \item des élèves pour qui un effort conséquent est attendu ;
  \item des élèves qui réussissent moyennement et pour lesquels un
      travail régulier permettra d'appréhender le bac tranquillement.
\end{itemize}

Pour les seconds, les bases semblent être présentes et un travail
régulier pendant l'année leur permettra de réussir l'épreuve de
mathématiques du bac.

Pour les premiers, des efforts conséquents de remise à niveau vont être
nécessaire. Une stratégie est donc à mettre en place.

\end{document}
