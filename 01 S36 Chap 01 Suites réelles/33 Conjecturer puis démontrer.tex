\documentclass[a4paper,12pt,french]{article}

\usepackage[utf8]{inputenc}
\usepackage[T1]{fontenc}
\usepackage[upright]{kpfonts}

\usepackage{amsmath,amsfonts,amssymb}
\usepackage[margin=1.4cm]{geometry}
\usepackage{enumitem}
\usepackage[lastexercise]{exercise}
\usepackage{multicol}
\usepackage{tikz}
\usepackage{tkz-tab}
\usetikzlibrary{babel}
\usepackage{babel}

\setlength{\columnseprule}{0pt}

%\setlength{\parindent}{0pt}
\setlist{noitemsep}
%\setlist[1]{\labelindent=\parindent} % < Usually a good idea
\setlist[itemize]{leftmargin=*}
\setlist[itemize,1]{label=$\blacktriangleright$}
\setlist[enumerate]{labelsep=*, leftmargin=1.5pc}
\setlist[enumerate,1]{label=\arabic*., ref=\arabic*}
\setlist[enumerate,2]{label=\emph{\alph*}),
ref=\theenumi.\emph{\alph*}}
\setlist[enumerate,3]{label=\roman*), ref=\theenumii.\roman*}
\setlist[description]{font=\sffamily\bfseries}

\renewcommand{\ExerciseName}{Exercice}
\renewcommand{\AnswerName}{Réponse de l'exercice}

\newcommand{\abs}[1]{\left\lvert #1\right\rvert}

\newcommand{\N}{\mathbf{N}}
\newcommand{\R}{\mathbf{R}}

\everymath{\displaystyle\everymath{}}

\title{Activité de recherche}
\date{}

\begin{document}

\maketitle

\begin{Exercise}[number=33]
  La suite $(u_n)$ est définie par $u_0 = 1$ et pour tout entier naturel
  $n$, $u_n = \dfrac{u_n}{u_n + 2}$.

  \emph{Objectif :} Exprimer de façon explicite les termes de la suite
  $(u_n)$ définie par récurrence.

  \begin{enumerate}
    \item Calculez sous forme fractionnaire les termes $u_1$, $u_2$,
      $u_3$, $u_4$, $u_5$ et $u_6$.
    \item Calculez la différence entre les dénominateurs de deux termes
      conséscutifs allant de $u_1$ à $u_6$.
    \item
      \begin{enumerate}
        \item Énoncez une conjecture et testez-là sur les termes connus.
        \item Prouvez, par récurrence, que votre conjecture est vraie.
      \end{enumerate}
  \end{enumerate}
\end{Exercise}
\pagebreak
\begin{Answer}[number=104]
\end{Answer}
\end{document}
