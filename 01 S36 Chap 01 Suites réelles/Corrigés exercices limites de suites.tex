\documentclass[a4paper,12pt,french]{article}

\usepackage[utf8]{inputenc}
\usepackage[T1]{fontenc}
\usepackage[margin=1.4cm]{geometry}
\usepackage{amsmath,amsfonts,amssymb}
\usepackage{enumitem}

\usepackage{babel}
\usepackage[lastexercise]{exercise}

\renewcommand{\ExerciseName}{Exercice}
\renewcommand{\AnswerName}{Réponse de l'exercice}

\newcommand{\N}{\mathbf{N}}

\everymath{\displaystyle\everymath{}}

\title{Correction exercices}
\date{6 octobre 2015}

\begin{document}

\maketitle

\begin{Exercise}[number=19]
  \begin{enumerate}[label=\alph*)]
    \item $u_n = \dfrac{5n^2 - 5}{2n(n+1)} $ de la forme quotient de
      limites infinies, mais $\forall n \in \N,\ u_n = \dfrac{5 (n-1)
      (n+1) }{2n (n+1)} = \dfrac52 \times \left(\dfrac{n}{n} +
      \dfrac{1}{n}\right)$

      On a donc $\lim_{n\to\infty} u_n = \dfrac52 $
    \item $u_n = \frac{7n + 3}{n^2}$ est a priori de la forme quotient
      de limites infinies, mais $\forall n \in \N^*\ u_n = \frac7n +
      \frac3{n^2}$. Sous cette forme de somme, on a
      $\lim_{n\to\infty}u_n = 0$
  \end{enumerate}
\end{Exercise}
\begin{Exercise}[number=20]
  Étude du comportement a l'infini des suites $u$, $v$ et $w$ définies
  pour tout $n$ entier supérieur ou égal à 1 par : \[ u_n = \frac{3n^2
    -4}{n+1},\ v_n = \frac{u_n}{n}\ \text{et}\ w_n = u_n - 3.\]

\end{Exercise}
\begin{Answer}[number=20]
  \begin{itemize}[label=\textbullet]
    \item $u_n = \frac{3n^2}{n+1} - \frac{4}{n+1}$, cette seconde limite
      étant nulle, intéressons nous à $\frac{3n^2}{n+1} = \frac{n\times
      3n}{n\left(1+\frac1n\right)}$ qui est de limite infinie.

        On a donc $\boxed{\lim_{n\to\infty}u_n = +\infty}$.
      \item $v_n = \frac{3n^2-4}{n(n+1)}$. Après écriture en différence
        de deux fractions, la première fraction peut se factoriser en
        $\frac{3n}{n\left(1+\frac1n\right)}$ d'où on obtient que la
        limite est $\boxed{\lim_{n\to\infty}v_n = 3}$.
      \item On a clairement que $\boxed{\lim_{n\to\infty}w_n = +
        \infty}$.
    \end{itemize}
\end{Answer}

\begin{Exercise}
  Déterminez, si elle existe, la limite de la suite $(u_n)_{n\in\N}$
  proposée.
  \begin{enumerate}[label=\alph*)]
    \item $u_n = \sqrt{2n^2 - 5}$ avec $n \geqslant 2$.
    \item $u_n = \sqrt{n^2 + 3n}$.
  \end{enumerate}
\end{Exercise}

\begin{Answer}
  \begin{enumerate}[label=\alph*)]
    \item $\forall n\in \N, n\geqslant 2 \implies 2n^2 - 5 \geqslant
      2n^2 \implies u_n \geqslant \sqrt2 n$.

      Posons $v_n = \sqrt2 n,\ \forall n\in \N$. Ainsi, à partir du rang
      2, on a $u_n \geqslant v_n$ et il est clair que
      $\lim_{n\to\infty}v_n = +\infty$. En utilisant le théorème de
      comparaison, on a que $\lim_{n\to\infty}u_n = +\infty$.
    \item En factorisant par exemple, on a $\lim_{n\to\infty}u_n =
      +\infty$
  \end{enumerate}
\end{Answer}

\begin{Exercise}
  Déterminez, si elle existe, la limite de la suite $(u_n)_{n\in\N}$
  proposée.
  \begin{enumerate}[label=\alph*)]
    \item $u_n = \sqrt{2n^2 - 5} - 2n$ avec $n \geqslant 2$.
    \item $u_n = n\left(\sqrt{2 + \frac1n} - \sqrt2 \right)$ avec $n>0$.
  \end{enumerate}
\end{Exercise}

\begin{Answer}
  \begin{enumerate}[label=\alph*)]
    \item Factorisons : $\forall n\geqslant 2 ,\ u_n = \sqrt{2n^2}
      \sqrt{1 - \frac5{n^2}} -2n = n\left(\sqrt2 \sqrt{1-\frac5{n^2}} -
      2 \right)$. On peut donc conclure $\lim_{n\to+\infty}u_n = -\infty$
    \item En majorant, on obtient que $u_n \geqslant n$ et donc que
    $\lim_{n\to+\infty}u_n = +\infty$
  \end{enumerate}
\end{Answer}

\begin{Exercise}
  Déterminez, si elle existe, la limite de la suite $(u_n)_{n\in\N}$
  proposée.
  \begin{enumerate}[label=\alph*)]
    \item $u_n = \sqrt{2n+1}-\sqrt{2n-1}$.
    \item $u_n = \frac{n}{\sqrt{n+1}+\sqrt{n+2}}$.
  \end{enumerate}
\end{Exercise}

\begin{Answer}
  \begin{enumerate}[label=\alph*)]
      \item $\forall n>0 ,\ \sqrt{2n+1}-\sqrt{2n-1} = \frac{2}{
      \sqrt{2n+1}+\sqrt{2n-1}}$ dont la limite est 0.
    \item En factorisant, on trouve $\frac{n}{\sqrt{n}\times v_n}$ avec
      $\lim_{n\to+\infty}v_n=1$ d'où $\lim_{n\to+\infty}u_n=+\infty$
  \end{enumerate}
\end{Answer}

\begin{Exercise}
  Déterminez, si elle existe, la limite de la suite $(u_n)_{n\in\N}$
  proposée.
  \begin{enumerate}[label=\alph*)]
    \item $u_n = \frac{n}{\sqrt{n+1}} -\frac{n}{\sqrt{n+2}}$.
    \item $u_n = \frac{3n - \sqrt{9n^2 +1}}{\sqrt{n^2 + 5}}$
  \end{enumerate}
\end{Exercise}

\begin{Answer}
  \begin{enumerate}[label=\alph*)]
    \item $\lim_{n\to+\infty}u_n = 0$
    \item $\lim_{n\to+\infty}u_n = 0$
  \end{enumerate}
\end{Answer}

\begin{Exercise}[number=82]
  Sont représentées sur le dessin ci-dessous, la fonction définie pour
  $x$ positif par $f(x) = \sqrt{3x}$ et la droite d'équation $y=x$.

  On considère la suite $u$ définie par $u_0\in[1;3]$ et pour tout
  entier naturel $n$, $u_{n+1} = \sqrt{3u_n}$.

  \begin{enumerate}[label=\textbf{\arabic*.}]
    \item L'intervalle $I = [1;3]$ semble contenir tous les termes de la
      suite. Démontrez par récurrence que tous les termes de $u$
      appartiennent effectivement à $I$.
    \item Conjecturez le sens de variation de la suite $u$ puis démontrer
      votre conjecture.
    \item Démontrez que la suite $u$ est convergent et conjecturez sa
      limite.
  \end{enumerate}
\end{Exercise}

\begin{Answer}
  \begin{enumerate}[label=\textbf{\arabic*.}]
    \item \begin{itemize}[label=\textbullet]
        \item Initialisation : $u_0\in I$ par définition.
        \item Hérédité : Soit $n\in\N$, on suppose que $u_n \in I$.

          $1\geqslant u_n \geqslant 3 \implies 3 \geqslant 3u_n
          \geqslant 9 \implies 1 \geqslant \sqrt3 \geqslant u_n
          \geqslant 3$.

          Donc $u_{n+1} \in I$.
        \item Conclusion : $\forall n\in\N u_n \in I$.
      \end{itemize}
    \item La suite semble être croissante.

      Soit $n\in \N$, calculons $\frac{u_{n+1}}{u_n} = \sqrt3
      \frac{\sqrt{u_n}}{u_n} = \frac{\sqrt{3}}{\sqrt{u_n}} \leqslant 1$,
      car $u_n\geqslant 3$.

      La suite est donc croissante pour tout $n\in\N$.
    \item La suite est croissante majorée, elle est donc convergente.

      De plus, sa limite semble être 3.
  \end{enumerate}
\end{Answer}

\end{document}
