\documentclass[12pt,a4paper]{article}
\usepackage[utf8]{inputenc}
\usepackage[T1]{fontenc}
\usepackage[frenchb]{babel}
\usepackage{ntheorem}
\usepackage{amsmath}
\usepackage{amsfonts}

\usepackage{array}

\usepackage{kpfonts}

\usepackage[bookmarks=false,colorlinks,linkcolor=blue]{hyperref}

\pdfminorversion 7
\pdfobjcompresslevel 3

\usepackage{tabularx}

\usepackage{pgf}
\usepackage{tikz}
\usepackage{tkz-euclide}
\usetkzobj{all}
\usepackage[top=1.7cm,bottom=2cm,left=2cm,right=2cm]{geometry}

\usepackage{lastpage}

\usepackage{marginnote}

\usepackage{wrapfig}

\usepackage[autolanguage]{numprint}
\newcommand{\np}{\numprint}

\makeatletter
\renewcommand{\@evenfoot}%
        {\hfil \upshape \small page {\thepage} de \pageref{LastPage}}
\renewcommand{\@oddfoot}{\@evenfoot}

\renewcommand{\maketitle}%
{\framebox{%
    \begin{minipage}{1.0\linewidth}%
      \begin{center}%
        \Large \@title ~-- \@author \\%
        \@date%
      \end{center}%
    \end{minipage}}%
  \normalsize%
  %\vspace{1cm}%
}

\newcommand{\R}{\mathbf{R}}
\newcommand{\N}{\mathbf{N}}
\newcommand{\Vecteur}{\overrightarrow}
\newcommand{\norme}[1]{\left\lVert #1 \right\rVert}
\newcommand{\vabs}[1]{\left\lvert #1 \right\rvert}
\newcommand{\ioo}[2]{\left]#1~;~#2\right[}

\makeatother


\theoremstyle{break}
\newtheorem{definition}{Définition}
\newtheorem{propriete}{Propriété}
\newtheorem{propdef}{Propriété - Définition}
\newtheorem{theoreme}{Théorème}
\theoremstyle{plain}
\theorembodyfont{\normalfont}
\newtheorem{exercice}{Exercice}
\theoremstyle{nonumberplain}
\newtheorem{remarque}{Remarque}
\newtheorem{probleme}{Problème}
\newtheorem{preuve}{Preuve}
\theoremstyle{nonumberbreak}
\newtheorem{exemple}{Exemple}

\usepackage{enumitem}

\setlength{\parsep}{0pt}
\setlength{\parskip}{5pt}
\setlength{\parindent}{0pt}
\setlength{\itemsep}{7pt}

\usepackage{multicol}
\setlength{\columnseprule}{0pt}

\everymath{\displaystyle\everymath{}}

\title{Raisonnement par récurrence, applucation aux suites}
\author{Terminale S}
\date{septembre 2017}

\begin{document}

\maketitle

\section{Généralités sur les entiers et les suites}

\subsection{Raisonnement par récurrence}

Le raisonnement par \emph{récurrence} est un type de raisonnement
permettant de montrer une propriété sur les entiers.

Il repose sur le principe (ou axiome) énoncé par Peano pour la
construction des entiers naturels, ensemble dont on suppose l'existence
\emph{a priori} :
\begin{enumerate}[label=(\roman*)]
  \item L'élément appelé \emph{zéro} et noté 0 existe et est un entier
    naturel.
  \item Tout entier naturel $n$ possède un \emph{unique} successeur,
    noté $s(n)$.
  \item Aucun entier naturel n'a pour successeur 0.
  \item Deux entiers naturels ayant le même successeur sont égaux.
  \item Si un ensemble contient 0 et contient le successeur de chacun de
    ses éléments, alors il est égal à l'ensemble des entiers naturels,
    noté $\N$.
\end{enumerate}

Quelques remarques sur ces différents points.

Le premier point est un point d'existence, aussi fondamental que
l'existence des nombres naturels. C'est d'ailleurs quasiment lui qui
confirme, avec l'axiome 5 l'existence des entiers naturels.

Le successeur d'un entier $n$ est souvent noté $n+1$. On peut utiliser
intensivement cette proposition pour définir addition et multiplication
et démontrer leurs propriétés -- associativité, commutativité et
distributivité.

On peut voir $s$ comme une fonction (axiome 2) injective (axiome 4) de
$\N$ dans $\N$.

\framebox{\begin{minipage}{0.99\linewidth}
    \emph{Donner la définition de fonction injective}
    \vspace{3cm}
  \end{minipage}
}

La dernière proposition est aussi appellée axiome d'induction et
correspond au cœur même du raisonnement par récurrence et peut se
reformuler :

\framebox{\begin{minipage}{0.99\linewidth}
    Si une proprosition dépendant d'un entier naturel $n$ est vraie pour
    $n = 0$, puis que si elle est vraie pour un élément arbitraire $k$,
    on en déduit qu'elle est vraie pour $s(k)$ le successeur de $k$,
    alors elle est vraie pour tous les entiers naturels.
  \end{minipage}
}

Exemple de raisonnement par récurrence : démontrer que pour tout $n$
entier naturel, $0 + 1 + 2 + 3 + \dots + n = \frac{n(n+1)}2$.

\framebox{\begin{minipage}{0.99\linewidth}
    Soit $P_n$ la proposition «$0 + 1 + 2 + 3 + \dots + n = \frac{n(n+1)}2$».
    \\
    \emph{Initialisation}\\
    Vérifions $P_0$ : En effet, comme $0 = \frac{0 \times (0 + 1)}2 = 0$,
    $P_0$ est vraie et la récurrence est initialisée.
    \emph{Hérédité}\\
    On suppose que $P_n$ est vraie pour $n$ un entier arbitrairement choisi,
    montrons alors que $P_{n+1}$ est vraie.
    \begin{align*}
      0 + 1 + 2 + 3 + \cdots + n + (n +1) & = \frac{n(n+1)}2 + (n + 1)           \\
                                          & = \frac{n}2(n+1) + (n+1)             \\
                                          & = \left( \frac{n}2 + 1 \right) (n+1) \\
                                          & = \frac{n+2}{2} (n+1)                \\
                                          & = \frac{(n+1)(n+2)}2
    \end{align*}
    \emph{Conclusion}\\
    Pour tout $n$ entier naturel, la proposition $P_n$ : «$0 + 1 + 2 + 3 +
    \dots + n = \frac{n(n+1)}2$» est vraie.
  \end{minipage}
}

La structure de ce raisonnement doit bien apparaître, éventuellement en
mettant en exergue la proposition, ici «$P(n) : 1 + 2 + 3 + \dots + n =
\frac{n(n+1)}2$».

Entraînez vous à écrire la proposition $P_n$ pour différentes lettres, ou
pour $n + 1$.

Les trois étapes du raisonnement doivent clairement apparaître.

\subsection{La notion de suite}

Les suites sont un objet mathématique à la fois simple à définir mais
extremement riche et permettant d'amorçer nombre de raisonnement. En
voici une définition :
\begin{definition}
  On appelle \emph{suite}, en précisant la nature des éléments, une
  collection infinie et ordonnée d'éléments de même nature, indexés par
  les entiers naturels, c'est à dire numérotés.
\end{definition}

On peut donner quelques exemples :
\begin{itemize}
  \item les entiers naturels ;
  \item les entiers pairs ;
  \item les nombres premiers ;
\end{itemize}

On note l'ensemble de la suite $(u_n)_{n\in\N}$ et $u_0$, $u_1$, \dots
$u_n$ en sont les éléments.

Lorsque tous les éléments sont égaux entre eux, on parle de suite
constante.

Pour les suites plus élaborées que celles que nous venons de présenter,
il existe deux façon de les définir :
\begin{itemize}
  \item pour tout $n$ entier naturel, $u_n = f(n)$ où $f$ est une
    fonction ;
  \item pour tout $n$ entier naturel $u_{n+1} = f(u_n)$ où $f$ est une
    fonction, $u_0$ donné.
\end{itemize}
La deuxième définition s'appelle définition par récurrence d'une suite
et on tâchera de se ramener à la première formulation.

\framebox{
  \begin{minipage}{0.99\linewidth}
    \emph{Montrer qu'une suite arithmétique définie pour tout entier
    naturel $n$ par $u_{n+1} = u_n + r$ peut se mettre sous la forme
    $u_n = u_0 + rn$, où $u_0$, le premier terme, est connu.\\[4.8cm]
  Donner la somme des $N$ premiers termes.\\[2.7cm]}
  \end{minipage}
}

\subsection{Quelques compléments sur les suites}

\begin{definition}
  On dit qu'une suite est
  \begin{itemize}
    \item arithmétique si la différence entre un terme et le suivant est
      une constante $r$ qu'on appelle raison. On a alors $u_n = u_0 +
      rn$ ;
    \item géométrique si le quotient entre un terme et le suivant est
      une constante $q$ qu'on appelle la raison.
  \end{itemize}
\end{definition}

\framebox{
  \begin{minipage}{0.99\linewidth}
    \emph{Montrer qu'une suite géométrique définie pour tout entier
    naturel $n$ par $u_{n+1} = q\times u_n $ peut se mettre sous la
  forme $u_n = u_0 \times q^n $.\\[4.8cm]}
\end{minipage}
}

\framebox{
  \begin{minipage}{0.99\linewidth}
    \emph{
    Montrer par récurrence que la somme des $N$ premiers termes vaut
    $u_0 \times \frac{1 - q^{N+1}}{1 - q}$ si $q\neq 1$ \\[4.8cm]}
  \end{minipage}
}

Enfin, pour conclure cette partie, on peut parler de la monotonie d'une
suite.

\begin{definition}
  On dit qu'une suite est monotone croissante (respectivement
  décroissante) lorsque pour tout entier naturel $n$, $u_{n+1} \geq u_n$
  (respectivement $u_{n+1} \leq u_n$.)
\end{definition}

\section{Limite d'une suite}

\subsection{Limite finie d'une suite}

On dit qu'une suite admet une limite finie $\ell$ lorsqu'à partir d'un
certain rang, tous les termes de la suite sont au voisinage de cette
limite. Ce voisinage s'entend à une précision $\varepsilon$ près. On a
donc la définition suivante :
\begin{definition}\label{def:suite:superieur}
  La suite $(u_n)_{n\in\N}$ admet $\ell$ pour limite si \[ \forall
    \varepsilon > 0, \exists N_0\in\N | \forall n \in\N , n\geq N_0
  \implies \vabs{u_n - \ell} \leq \varepsilon \]
  On note alors $\lim_{n\to+\infty} u_n = \ell$
\end{definition}

On ne perd pas en généralité si on remplace l'inégalité précédente par
une inégalité stricte.

Interprétons les différents éléments de cette définition :
\begin{itemize}
  \item $\exists N_0\in\N, | \forall n \in\N , n\geq N_0$ signifie qu'on se
    place à partir d'un certain rang ;
  \item $\forall \varepsilon > 0$ on s'est fixé une précision, c'est le
    diamètre d'un certain intervalle, et ça doit donc être valable pour tous
    les intervalles ;
  \item $\vabs{u_n - \ell} \leq \varepsilon$ peut se lire $u_n \in [ \ell -
    \varepsilon ; \ell + \varepsilon ]$.
\end{itemize}

On peut désormais prendre la proposition suivante, équivalente à la
définition \ref{def:suite:superieur} :

\framebox{
  \begin{minipage}{0.99\linewidth}
    $\lim_{n\to+\infty} u_n = \ell$ si et seulement si pour tout
    intervalle contenant $\ell$, celui-ci contient aussi tous les termes de
    la suite à partir d'un certain rang.
  \end{minipage}
}

\begin{remarque}
  Si la limite d'une suite existe, alors elle est unique.
\end{remarque}

\begin{minipage}{0.99\linewidth}
  \emph{Preuve} (par l'absurde, en considérant que $\ell_1$ et $\ell_2$ sont
  deux limites distinctes)
  \vspace{3cm}
\end{minipage}

Donnons quelques exemples de suites qui présentent une limite finie et
donnons la valeur de celle-ci.

\begin{itemize}
  \item Pour $u_n = \frac1n$, $\lim_{n\to +\infty}u_n = 0$ ;
  \item Pour $u_n = \frac1{n^2}$, $\lim_{n\to +\infty}u_n = 0$ ;
  \item Pour $u_n = \frac1{\sqrt{n}}$, $\lim_{n\to +\infty}u_n = 0$ ;
\end{itemize}

On remarque que cette définition ne donne aucun moyen pratique de
calculer la limite d'une suite ni même de dire si celle-ci existe.

\begin{remarque}[Critère de Cauchy (admis et hors programme)]~\\
  $(u_n)_{n\in\N}$ converge si, et seulement si, \[\forall \varepsilon >
  0,\ \exists N_0\in\N,\ \forall (n,p)\in\N^2,\ n\geq p\geq N_0 \implies
\vabs{u_n - u_p} < \varepsilon. \]
\end{remarque}

\subsection{Limite infinie d'une suite}

On dit qu'une suite admet une limite infinie ($+\infty$ ou $-\infty$)
lorsqu'à partir d'un certain rang, tous les termes de la suite
appartiennent à tous les intervalles de la forme $\ioo{A}{+\infty}$, où
$A$ est un réel arbitraitement grand.

\begin{definition}\label{def:suiteinf:superieur}
  La suite $(u_n)_{n\in\N}$ admet $+\infty$ pour limite si \[ \forall
    A\in\R,\ \exists N_0\in\N\ |\ \forall n\in\N,\ n\geq N_0 \implies
  u_n > A \]
\end{definition}

On peut aussi écrire la proposition suivante, équivalente à la
définition \ref{def:suiteinf:superieur} :

\framebox{
  \begin{minipage}{0.99\linewidth}
    $\lim_{n\to+\infty} u_n = +\infty$ si et seulement si tout
    intervalle de la forme $[A; +\infty [$ contient aussi tous les termes de
    la suite à partir d'un certain rang.
  \end{minipage}
}

Donnons quelques exemples de suites qui présentent une limite infinie.

\begin{itemize}
  \item Pour $u_n = n$, $\lim_{n\to +\infty}u_n = +\infty$ ;
  \item Pour $u_n = {n^2}$, $\lim_{n\to +\infty}u_n = +\infty$ ;
  \item Pour $u_n = \sqrt{n}$, $\lim_{n\to +\infty}u_n = +\infty$ ;
\end{itemize}

\begin{remarque}~\\
  Une suite de limite infinie est dite divergente vers $\pm\infty$.
\end{remarque}

\begin{remarque}[absence de limite]~\\
  Certaines suites ne présentent pas de limite. C'est le cas de la suite
  définie pour tout $n$ naturel par $u_n = (-1)^n$.
\end{remarque}

\section{Calcul effectif de la limite d'une suite}

Dans cette partie, on va voir comment calculer de façon efficace la
limite d'une suite.

\subsection{Théorèmes de comparaion}

\begin{theoreme}
  Soient $(u_n)_{n\in\N}$, $(v_n)_{n\in\N}$ et $(w_n)_{n\in\N}$, trois
  suites. Si à partir d'un certain rang ($p$) on a : \[ u_n \geq v_n
    \text{ et } \lim_{n\to\infty} v_n = +\infty \text{ alors }
  \lim_{n\to\infty} u_n = +\infty \]
\end{theoreme}

\framebox{
  \begin{minipage}{0.99\linewidth}
    \emph{Preuve :}
      \vspace{4cm}
  \end{minipage}
}

\begin{exercice}
  Adapter le théorème au cas où $\lim_{n\to\infty} v_n = -\infty$.

  \framebox{
    \begin{minipage}{0.99\linewidth}
      \hspace{\linewidth}
      \vspace{2.5cm}
    \end{minipage}
  }
\end{exercice}

\begin{theoreme}[d'encadrement (admis)]
  Soient $(u_n)_{n\in\N}$, $(v_n)_{n\in\N}$ et $(w_n)_{n\in\N}$, trois
  suites. Si à partir d'un certain rang ($p$) on a : \[ w_n \leq u_n
    \leq w_n \text{ et } \lim_{n\to\infty} v_n = \lim_{n\to\infty}w_n =
  \ell \text{ alors } \lim_{n\to\infty} u_n = \ell \]
\end{theoreme}

\begin{exemple}
  \begin{itemize}
    \item Démontrer que la suite $(v_n)_{n\in\N}$ définie par $\forall
      n\in\N,\ v_n = n + \sin n $ diverge vers $+\infty$.

      \framebox{
        \begin{minipage}{0.99\linewidth}
          \hspace{\linewidth}
          \vspace{3cm}
        \end{minipage}
      }
    \item Démontrer que la suite $(u_n)_{n\in\N}$ définie par $\forall
      n\in\N,\ u_n = \frac{\sin n}{n + 1} $ converge vers 0.

      \framebox{
        \begin{minipage}{0.99\linewidth}
          \hspace{\linewidth}
          \vspace{3cm}
        \end{minipage}
      }
  \end{itemize}
\end{exemple}

\subsection{Opérations sur les limites}

Il paraît assez naturel de se poser la question de savoir que sont les
limites potentielles de la somme (algébrique), du produit et du quotient
de deux suites. Raisonnons par «condition nécessaire» : si deux suites
$u$ et $v$ possèdent toutes deux une limite finie, la limite de la somme
est également finie et est la somme des limites. Admettons que cette
condition est aussi «suffisante». On donne ainsi, sans justification le
tableau suivant pour la limite de la somme :

\begin{center}
  \renewcommand{\arraystretch}{1.2}
  \begin{tabular}{|l|*{6}{>{\hfill$}p{1cm}<{$\hfill~}|}}
    \hline
    $u$ a pour limite    & \ell   & \ell      & \ell      & +\infty   & -\infty    & +\infty \\ \hline
    $v$ a pour limite    & \ell'  & +\infty   & -\infty   & +\infty   & -\infty    & -\infty \\ \hline
    %$u+v$ a pour limite & $l+l'$ & $+\infty$ & $-\infty$ & $+\infty$ & %$-\infty$ & $+\infty$ \\ \hline
    $u+v$ a pour limite  &        &           &           &           &            & \\ \hline
  \end{tabular}
\end{center}

Pour le produit, on donne aussi le tableau suivant, pour $\ell \neq 0$
et $\ell' \neq 0$ :

\begin{center}
  \renewcommand{\arraystretch}{1.2}
  \begin{tabular}{|l|*{6}{>{\hfill$}p{1cm}<{$\hfill~}|}}
    \hline
    $u$ a pour limite         & \ell  & \ell    & \ell    & +\infty & -\infty & +\infty \\ \hline
    $v$ a pour limite         & \ell' & +\infty & -\infty & +\infty & -\infty & -\infty \\ \hline
    $u\times v$ a pour limite &       &         &         &         &         & \\ \hline
    \end{tabular}
\end{center}

Attention, dans le cas où $\ell = 0$, on a des formes indéterminées :
\begin{center}
  \renewcommand{\arraystretch}{1.2}
  \begin{tabular}{|l|*{3}{>{\hfill$}p{1cm}<{$\hfill~}|}}
    \hline
    $u$ a pour limite         & 0     & 0       & 0       \\ \hline
    $v$ a pour limite         & \ell' & +\infty & -\infty \\ \hline
    $u\times v$ a pour limite &       &         &         \\ \hline
    \end{tabular}
\end{center}

Pour le quotient, on a toujours dans le cas où $\ell \neq 0$ et $\ell'
\neq 0$ :
\begin{center}
  \renewcommand{\arraystretch}{1.2}
  \begin{tabular}{|l|*{5}{>{\hfill$}p{1cm}<{$\hfill~}|}}
    \hline
    $u$ a pour limite         & \ell  & \ell    & \ell    & +\infty & -\infty \\ \hline
    $v$ a pour limite         & \ell' & +\infty & -\infty & \ell' & \ell' \\ \hline
    $\frac{u}{v}$ a pour limite &       &         &         &         &       \\ \hline
    \end{tabular}
\end{center}

Pour lever les cas d'indétermination, on essaie de revenir à un autre
cas connu, en factorisant l'expression de la suite.

Faire les exercices 11 à 19 p 31 et 32.

\pagebreak

\section{Suites majorées ou minorées et monotonie}

\subsection{Un premier exemple pour comprendre}

On s'intéresse au comportement à l'infini de la suite définie pour tout
entier naturel $n$ par $q^n$ où $q$ est un réel strictement supérieur à
1.

\framebox{
  \begin{minipage}{0.99\linewidth}
    \emph{Démontrer par récurrence l'inégalité de Bernoulli : $(1 + a)^n
    \geqslant 1 + na, a > 0$}
      \vspace{4cm}
  \end{minipage}
}

Attention, ici l'hypothèse $a>0$ prend toute son importance !

\framebox{
  \begin{minipage}{0.99\linewidth}
    \emph{Utiliser le théorème de majoration des suites pour conclure :}
      \vspace{4cm}
  \end{minipage}
}

On démontre également que les suites géométriques de raison $\vabs{q} <
1$ convergent vers 0.

\framebox{
  \begin{minipage}{0.99\linewidth}
    \emph{Preuve :}
      En effet, à partir du rang $n_0$, tous les intervalles ouverts
      contenus dans l'intervalle
      $\ioo{-\frac{u_0}{\vabs{q}^{n_0+1}}}{\frac{u_0}{\vabs{q}^{n_0+1}}}$
      contiennent tous les termes de la suite.
  \end{minipage}
}

\subsection{Croissance, monotonie et limite}

\begin{theoreme}[admis]
  Toute suite strictement croissante majorée admet une limite.
\end{theoreme}

\begin{theoreme}
  Toute suite strictement croissante non-majorée diverge vers $+\infty$.
\end{theoreme}

\framebox{
  \begin{minipage}{0.99\linewidth}
    \emph{Preuve :}
      \vspace{4cm}
  \end{minipage}
}



\end{document}
