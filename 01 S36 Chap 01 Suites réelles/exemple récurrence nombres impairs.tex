% rubber: setlist arguments --shell-escape
\documentclass[varwidth,convert]{standalone}

\usepackage[utf8]{inputenc}
\usepackage[T1]{fontenc}
\usepackage{amsmath}

\begin{document}
Démontrons que $\forall n \in \mathbf{N}, 1 + 3 + \cdots + (2n+1) =
(n+1)^2$.

Soit $P(n)$ la proposition « $1 + 3 + \cdots + (2n+1) = (n+1)^2$».

\emph{Initialisation}

Vérifions $P(0)$.

Pour $P(0)$, la somme vaut 1 et $(n+1)^2 = 1$.

Donc $P(0)$ est vraie.

\emph{Hérédité}

Soit $n$ un entier naturel supérieur ou égal à 1.

Supposons que $P(n)$ est vraie, démontrons $P(n+1)$ : « $1 + 3 + \cdots
+ (2n+1) + (2(n+1)+1) = (n+1 +1)^2$.

$1 + 3 + \cdots + (2n+1) + (2(n+1)+1) = (n+1)^2 + (2(n+1)+1)$ en
utilisant l'hypothèse de réuccurence.

En factorisant l'identité remarquable, on obtient

\centering $\boxed{1 + 3 + \cdots + (2n+1) + (2(n+1)+1) = (n+1 +1)^2}$.

\emph{Conclusion}

Le principe de récurrence nous permet désormais d'affirmer que $\forall
n \in \mathbf{N}, 1 + 3 + \cdots + (2n+1) = (n+1)^2$.
\end{document}
