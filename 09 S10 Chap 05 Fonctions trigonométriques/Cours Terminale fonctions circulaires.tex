\documentclass[11pt,a4paper,french]{article}
\usepackage[utf8]{inputenc}
\usepackage[T1]{fontenc}
\usepackage{babel}
\usepackage[thmmarks]{ntheorem}
\usepackage{amsmath}
\usepackage{amsfonts}
\usepackage{amssymb}

\usepackage{array}

\usepackage{kpfonts}

\usepackage[bookmarks=false,colorlinks,linkcolor=blue,pdfusetitle]{hyperref}

\pdfminorversion 7
\pdfobjcompresslevel 3

\usepackage{tabularx}
\usepackage[autolanguage,np]{numprint}
\usepackage{enumitem}

\usepackage{tipfr}
\usepackage{pgf}
\usepackage{tikz}
\usepackage{tkz-base}
\usepackage{tkz-euclide}
\usetkzobj{all}
\usetikzlibrary{hobby}
\usepackage{tkz-tab}

\usepackage[top=1.7cm,bottom=2cm,left=2cm,right=2cm]{geometry}

\usepackage{lastpage}

\usepackage{esvect}
\usepackage{marginnote}

\usepackage{wrapfig}

\usepackage[defaultlines=5,all]{nowidow}


\makeatletter
\renewcommand{\@evenfoot}%
        {\hfil \upshape \small page {\thepage} de \pageref{LastPage}}
\renewcommand{\@oddfoot}{\@evenfoot}

\renewcommand{\maketitle}%
{\framebox{%
    \begin{minipage}{1.0\linewidth}%
      \begin{center}%
        \Large \@title ~-- \@author \\%
        \@date%
      \end{center}%
    \end{minipage}}%
  \normalsize%
  %\vspace{1cm}%
}

%Des macros pour les noms d'ensmbles
\newcommand{\R}{\mathbf{R}}
\newcommand{\Q}{\mathbf{Q}}
\newcommand{\Z}{\mathbf{Z}}
\newcommand{\C}{\mathbf{C}}
\newcommand{\N}{\mathbf{N}}

\newcommand{\Cf}{\mathscr{C}}

\newcommand{\norme}[1]{\left\lVert #1 \right\rVert}
\newcommand{\abs}[1]{\left\lvert #1 \right\rvert}
\newcommand{\diff}[1]{\mathrm{d}\,#1}

%Une macro récursive pour l'intérieru des vecteurs
%http://tex.stackexchange.com/questions/19693/arguments-of-custom-commands-as-comma-separated-list

\newcommand\vecteur[2][\\]{%
    \global\def\my@delim{#1}%
    \left(\negthinspace\begin{matrix}
        \my@vector #2,\relax\noexpand\@eolst%
    \end{matrix}\right)}

%Une macro pour les vecteurs
\def\my@vector #1,#2\@eolst{%
   \ifx\relax#2\relax
      #1
   \else
      #1\my@delim
      \my@vector #2\@eolst%
   \fi}

%Une macro récursive pour mettre formater l'intérieur des intervalles
\def\my@intervalle #1;#2\@eolst{%
  \ifx\relax#2\relax
    #1
  \else
    \my@intervalle #2\@eolst%
  \fi}

%Quatre macros pour les quatres types d'intervalles
\newcommand{\interff}[1]{%
  \left[\my@intervalle #1;\relax\noexpand\@eolst%
  \right]
}
\newcommand{\interfo}[1]{%
  \left[\my@intervalle #1;\relax\noexpand\@eolst%
  \right[}
\newcommand{\interof}[1]{%
  \left]\my@intervalle #1;\relax\noexpand\@eolst%
  \right]}
\newcommand{\interoo}[1]{%
  \left]\my@intervalle #1;\relax\noexpand\@eolst%
  \right[}

\makeatother


\usepackage{framed}

\theoremstyle{break}
\newtheorem{definition}{Définition}
\newtheorem{propriete}{Propriété}
\newtheorem{corollaire}{Corollaire}
\newtheorem{propdef}{Propriété - Définition}
\newtheorem{theoreme}{Théorème}
\newtheorem{lemme}{Lemme}
\theoremstyle{plain}
\theorembodyfont{\normalfont}
\newtheorem{exerciceT}{Exercice}
\theoremstyle{nonumberplain}
\newtheorem{remarque}{Remarque}
\newtheorem{notation}{Notation}
\newtheorem{probleme}{Problème}
\theoremsymbol{\ensuremath{\blacksquare}}
\newtheorem{preuve}{Preuve}
\theoremsymbol{}
\theoremstyle{nonumberbreak}
\newtheorem{exemple}{Exemple}

\newenvironment{exercice}{\begin{framed}\begin{exerciceT}}{\end{exerciceT}\end{framed}}

\setlength{\parsep}{0pt}
\setlength{\parskip}{5pt}
%\setlength{\parindent}{0pt}
\setlength{\itemsep}{7pt}

\setlist{noitemsep}
%\setlist[1]{\labelindent=\parindent} % < Usually a good idea
\setlist[itemize]{leftmargin=*}
\setlist[itemize,1]{label=$\triangleright$}
\setlist[enumerate]{labelsep=*, leftmargin=1.5pc}
\setlist[enumerate,1]{label=\arabic*., ref=\arabic*}
\setlist[enumerate,2]{label=\emph{\alph*}),
ref=\theenumi.\emph{\alph*}}
\setlist[enumerate,3]{label=\roman*), ref=\theenumii.\roman*}
\setlist[description]{font=\sffamily\bfseries}

\usepackage{multicol}
\setlength{\columnseprule}{0pt}

\everymath{\displaystyle\everymath{}}

\title{Fonctions circulaires}
\author{\bsc{Jumel}}
\date{mars 2017}

\begin{document}

\noindent\maketitle

\bigskip

\section*{Activités préparatoires}

\begin{exercice}
  Placer sur le cercle ci-dessous le cosinus et le sinus de l'angle
  $\alpha$.

  \begin{center}
    \begin{tikzpicture}[scale=2]
      \tkzInit[xmin=-1,xmax=1,ymin=-1,ymax=1]
      \tkzDrawXY
      \draw plot [smooth,domain=0:360] ({cos(\x)}, {sin(\x)} ) ;
      \draw (0,0) -- (36:1) ;
      \draw [red,fill=red!30] plot [smooth,domain=0:36] ({0.3*cos(\x)},
      {0.3*sin(\x)} ) -- (0,0) -- cycle ;
      \draw (18:0.3) node [right] {$\alpha$} ;
    \end{tikzpicture}
  \end{center}
\end{exercice}

\begin{definition}
  \begin{itemize}
    \item On appelle fonction cosinus, notée $\cos$ la fonction de $\R
      \to \R,\ x\mapsto \cos x$ ;
    \item On appelle fonction sinus, notée $\sin$ la fonction de $\R \to
      \R,\ x\mapsto \sin x$.
  \end{itemize}
\end{definition}

On admet ici que ces deux fonctions sont continues sur $\R$.

\begin{exercice}
  Compléter les relations ci-dessous :
  \begin{multicols}{2}
    \begin{itemize}
      \item $\cos (a+\pi) = \dotfill$
      \item $\cos (a-\pi) = \dotfill$
      \item $\cos (-a) = \dotfill$
      \item $\cos (a+\frac{\pi}2) = \dotfill$
      \item $\cos (a-\frac{\pi}2) = \dotfill$
      \item $\cos (a+b) = \dotfill$
      \item $\cos (a-b) = \dotfill$
      \item $\cos (2a) = \dotfill$
      \item $\sin (a+\pi) = \dotfill$
      \item $\sin (a-\pi) = \dotfill$
      \item $\sin (-a) = \dotfill$
      \item $\sin (a+\frac{\pi}2) = \dotfill$
      \item $\sin (a-\frac{\pi}2) = \dotfill$
      \item $\sin (a+b) = \dotfill$
      \item $\sin (a-b) = \dotfill$
      \item $\sin (2a) = \dotfill$
    \end{itemize}
  \end{multicols}
\end{exercice}

\begin{exercice}
  Donner l'argument qui permet d'écrire $\cos^2 x + \sin^2 x = 1$.
\end{exercice}

Il est particulièrement utile de connaître le tableau suivant :

\begin{center}
  \begin{tabular}{%
      |c|*{5}{c|}%
    } \hline
    $\alpha$ & 0 & $\frac{\pi}6$ &  $\frac{\pi}4$ &  $\frac{\pi}3$ &
    $\frac{\pi}2$ \\ \hline
    $\sin \alpha$ & 0 & $\frac12$ & $\frac{\sqrt{2}}2$ &
    $\frac{\sqrt{3}}2$ & $\frac{\sqrt{4}}{2} = 1$ \\ \hline
    $\cos \alpha$ & $\frac{\sqrt{4}}2 = 1$ & $\frac{\sqrt{3}}2$ &
    $\frac{\sqrt{2}}2$ & $\frac{1}{2}$ & 0 \\ \hline
  \end{tabular}
\end{center}

\section{Étude des fonctions sinus et cosinus}

Il s'agit ici d'une étude au sens de l'étude de fonction. Pour cela,
nous donnerons d'abord la dérivée des ces fonctions, étudierons les
limites éventuelles (ou les majorants et minorants s'ils existent) et
dresseront le tableau de variation.

Une autre partie sera ensuite consacrée à la représentation graphique de
telles fonctions.

\subsection{Dérivées et études des fonctions}

\begin{lemme}
  La fonction $x \mapsto \frac{\sin x}{x}$ est continue et vaut 1 en 0.
\end{lemme}

Donnons une preuve géométrique de ce lemme.

\begin{exercice}
  Démontrons que $\sin h \leqslant h \leqslant \tan h$.

  \begin{enumerate}
    \item Préciser dans quel intervalle cette inégalité est non triviale
      (i.e. non évidente)
    \item $\sin h \leqslant h$.
      \begin{enumerate}
        \item Compléter le schéma en plaçant un point $S$ représentant
          le sinus de l'angle représenté.
          \begin{center}
            \begin{tikzpicture}[scale=4]
              \draw plot [smooth, domain=0:18] ({cos(\x)}, {sin(\x)} ) ;
              \draw (0,0) node [left] {$O$} -- (1,0) node [right] {$U$} ;
              \draw (0,0) -- (18:1) node [above] {$X$} ;
              \draw (1,0) -- (1,{tan(18)}) node [right] {$T$} -- (18:1) ;
            \end{tikzpicture}
          \end{center}
          On donne $OU = \np{1}$.
        \item Donner un argument géométrique permettant de conclure.
      \end{enumerate}
    \item $h \leqslant \tan h$
      \begin{enumerate}
        \item Quelle longueur correspond à la tangente ?
        \item Donner les aires des triangles $OSX$ et $OUT$ ainsi que du
          secteur angulaire $OUX$.
        \item En déduire $\sin h \cos h < h < \tan h$
      \end{enumerate}
    \item Quels théorèmes ou propriétés utiliser pour en déduire
      $\lim_{h\to 0} \dfrac{h}{\sin h}$ puis $\lim_{h\to 0} \dfrac{\sin
      h}h$ ?
  \end{enumerate}
\end{exercice}

On obtient ainsi une démonstration de ce résultat qui ne fait pas appel
à la dérivée du sinus.

\begin{exercice}\label{ex:nb_deriv_cos}
  Une démonstration du nombre dérivé de la fonction $\cos$.

  \begin{enumerate}
    \item \begin{enumerate}
        \item Rappeler la formule de linéarisation du cosinus.
        \item Donner $\lim_{h\to 0}\cos h$ et $\lim_{h\to 0}\sin h$.
      \end{enumerate}
    \item Soient $x$ et $h$ deux réels.
      \begin{enumerate}
        \item Linéariser $\cos(x+h)$
        \item Montrer que $\cos(x + h) - \cos x = \cos x\cos h - \sin x
          \sin h - \cos x$
        \item En déduire une simplification de $\lim_{h\to
          0}\dfrac{\cos(x + h) - \cos x}{h}$
      \end{enumerate}
  \end{enumerate}
\end{exercice}

\begin{remarque}
  Ici, on ne pouvait pas utiliser le nombre dérivé du sinus en 0, puisque
  c'est justement ce qu'on cherche à montrer.
\end{remarque}

\begin{propriete}
  \begin{itemize}
    \item La fonction cosinus est dérivable sur $\R$ et sa dérivée est
      $\cos'x = -\sin x$ ;
    \item la fonction sinus est dérivable sur $\R$ et sa dérivée est
      $\sin'x = \cos x$.
  \end{itemize}
\end{propriete}
\begin{preuve}
  \begin{itemize}
    \item Voir les deux exercices ci-dessus ;
    \item Reprendre l'exercice \ref{ex:nb_deriv_cos} en linéarisant le
      sinus.
  \end{itemize}
\end{preuve}

\begin{exercice}
  Écrire la dérivée des fonctions $\cos$ et $\sin$ en terme de
  déphasage.
\end{exercice}

De par sa définition géométrique, on a pour tout $x$ réel, $-1\leqslant
\cos x \leqslant 1$ et $-1 \leqslant \sin x \leqslant 1$. On peut
démontrer également que $\cos$ et $\sin$ n'ont pas de limite en
$+\infty$. Il suffit de le montrer sur une suite.

\begin{exercice}
  \begin{enumerate}
    \item Exprimer la suite $\cos(n\pi)$ en fonction de -1 et de $n$.
    \item Même question pour la suite $\sin\left(\frac{n\pi}2\right)$.
    \item Que vaut la limite de $n\pi$.
    \item Conclure.
  \end{enumerate}
\end{exercice}

Pour des raisons de périodicité (nous reviendrons sur ce terme plus
tard), nous pouvons limiter les tableaux de variations.

Pour le cosinus :
\begin{center}
  \begin{tikzpicture}
    \tkzTabInit[]
    {$x$/1, $-\sin x$/1, $\cos$/3}
    {0, $\frac{\pi}2$, $\pi$, $\frac{3\pi}2$, $2\pi$}
    \tkzTabLine{,-, ,-,0,+, ,+}
    \tkzTabVar{+/1,R,-/-1,R,+/1}
    \tkzTabVal{1}{3}{0.5}{}{0}
    \tkzTabVal{3}{5}{0.5}{}{0}
  \end{tikzpicture}
\end{center}

Pour le sinus :
\begin{center}
  \begin{tikzpicture}
    \tkzTabInit[]
    {$x$/1, $\cos x$/1, $\sin$/3}
    {$-\pi$, $-\frac{\pi}2$, $0$, $\frac{\pi}2$, $\pi$}
    \tkzTabLine{,-,0,+, ,+,0,-}
    \tkzTabVar{+/0,-/-1,R,+/1,-/0}
    \tkzTabVal{2}{4}{0.5}{}{0}
  \end{tikzpicture}
\end{center}

\subsection{Représentations graphiques et compléments directs}

À l'aide des tableaux de variations, on peut en déduire les
représentation graphiques des ces fonctions.

\begin{center}
  \begin{tikzpicture}
    \tkzInit[ymin=-1.2,ymax=1.2,xmin=-8,xmax=8]
    \tkzDrawXY

    \draw [thick,red] plot [smooth,samples=500,
    domain={-2.5*3.14}:{2.5*3.14}] (\x, {cos(\x r)}) node [below] {$\cos
    x$} ;
    \draw [thick,blue] plot [smooth,samples=500,
    domain={-2.5*3.14}:{2.5*3.14}] (\x, {sin(\x r)}) node [right] {$\sin
    x$} ;
  \end{tikzpicture}
\end{center}

On peut aussi donner de nouvelles relations qui complèterons les
tableaux des dérivées et des primitives usuelles.

\begin{propriete}
  \begin{multicols}{2}
    \begin{itemize}
      \item La primitive de $\cos$ est $\sin$.
      \item La primitive de $\sin$ est $-\cos$.
      \item Une façon de les mémoriser est de retenir que primitive et
        dérivée sont «réciproques» l'une de l'autre, ainsi, pour $f\in\{
        \cos,\sin\}$, la primitive est $f(x - \frac{\pi}2)$.
    \end{itemize}
    \columnbreak
    \begin{itemize}
      \item La dérivée de $x\mapsto \cos(\omega x + \varphi)$ est
        $x\mapsto -\omega\sin(\omega x + \varphi)$.
      \item La dérivée de $x\mapsto \sin(\omega x + \varphi)$ est
        $x\mapsto \omega\cos(\omega x + \varphi)$.
      \item La primitive de $x\mapsto \cos(\omega x + \varphi)$ est
        $x\mapsto \frac1{\omega}\sin(\omega x + \varphi)$.
      \item La primitive de $x\mapsto \sin(\omega x + \varphi)$ est
        $x\mapsto \frac{-1}{\omega}\cos(\omega x + \varphi)$.
    \end{itemize}
  \end{multicols}
\end{propriete}

\section{Notions complémentaires sur les fonctions}

\subsection{Quelques propriétés en plus}

\begin{definition}
  On dit qu'une fonction $f$ définie sur $I$ est :
  \begin{itemize}
    \item paire si $\forall x\in I,\ f(-x) = f(x)$ ;
    \item impaire si $\forall x\in I,\ f(-x) = -f(x)$ ;
  \end{itemize}
\end{definition}

\begin{exercice}
  \begin{enumerate}
    \item Préciser qui de sinus et de cosinus est paire ou impaire.
    \item Interpréter la parité en terme de symétrie de la
      représentation graphique.
    \item Donner des exemples de fonctions paires et impaires.
  \end{enumerate}
\end{exercice}

\begin{definition}
  On dit qu'une fonction $f$ définie sur $I$ est périodique s'il existe
  un réel $T$ tel que $\forall x \in I,\ f(x+T) = f(x)$.

  Le réel $T_0 = \min \left\{ T | f(x+T) = f(x) \right \}$ est appelé
  \emph{période} et on dit que $f$ est $T_0$ périodique.
\end{definition}

\begin{exercice}
  Préciser la période des fonctions cosinus et sinus.
\end{exercice}

\pagebreak

\subsection{Le rapport du sinus et du cosinus}

\begin{definition}[tangente]
  On définit la fonction tangente comme la fonction de $\R \setminus
  \left\{\frac{(2n+1)\pi}2,\ n\in\Z \right\}\to \R, x\mapsto \frac{\sin
  x}{\cos x}$
\end{definition}

\begin{propriete}
  La fonction tangente est impaire, périodique, de période $\pi$.
\end{propriete}
\begin{preuve}
  Soit $x \in \R \setminus \left\{\frac{(2n+1)\pi}2,\ n\in\Z \right\}$.
  $\tan(-x) = \frac{\sin(-x)}{\cos(-x)} = \frac{-\sin x}{\cos x} = -\tan
  x$.

  $\tan(x+\pi) = \frac{\sin(x + \pi)}{\cos(x + \pi)} = \frac{- \sin x}{-
  \cos x} = \tan x$.
\end{preuve}

\begin{propriete}
  $\tan$ est continue dérivable sur son ensemble de définition et $\tan'
  = 1 + \tan^2 = \frac1{\cos^2}$.
\end{propriete}
\begin{preuve}
  $\tan$ est le quotient de deux fonctions continues dérivables.

  $\tan' = \frac{\sin' \cos - \sin \cos'}{\cos^2} = \frac{\cos \cos +
  \sin \sin}{\cos^2}$.
\end{preuve}

\begin{propriete}
  $\lim_{\substack{x\to\frac{(2n+1)\pi}2^{\varepsilon}\\n\in\Z}} \tan x =
  \varepsilon\infty$ avec $\varepsilon = \pm 1$.
\end{propriete}
\begin{preuve}
  Il suffit d'étudier $\cos$ au voisinage de $(2n+1)\frac{\pi}2$.
\end{preuve}

\begin{center}
  \begin{tikzpicture}
    \tkzTabInit{$x$/1,$1+\tan^2 x$/1,$\tan x$/3}{$-\frac{\pi}2$,
    $\frac{\pi}2$}
    \tkzTabLine{,+,}
    \tkzTabVar{-/$-\infty$,+/$+\infty$}
    \tkzTabVal{1}{2}{0.5}{0}{0}
  \end{tikzpicture}
\end{center}

Tracé de la courbe représentative de la tangente :

\begin{center}
  \begin{tikzpicture}[yscale=0.34]
    \tkzInit[ymin=-10,ymax=10,xmin=-8,xmax=8]
    \tkzDrawXY

    \foreach \i in {-2,...,2} {
      \draw [thick,red] plot [smooth,samples=500,
      domain={-3.14/2+0.1 + 3.14*\i}:{3.14/2-0.1 + 3.14*\i}] (\x,
      {tan(\x r)}) ;
    }
    \draw [red] (9,8) node { $\tan x$ } ;
  \end{tikzpicture}
\end{center}

\subsection{Des fonctions qui découlent du sinus et du cosinus : leurs
réciproques}

\begin{definition}[injection, surjection, bijection]
  \begin{itemize}
    \item Une fonction telle que deux éléments $x_1 \neq x_2$ entraine
      $f(x_1) \neq f(x_2)$ est dite injective ;
    \item une fonction telle que pour tout élément $y$, il existe $x$
      telle que $f(x) = y$ est dite surjective ;
    \item une fonction à la fois injective et surjective est dite
      bijective, on dit aussi que tout élément possède un et un seul
      antécédent.
  \end{itemize}
\end{definition}

Par exemple, toute fonction continue monotone (donc qui respecte le
théorème des valeurs intermédiaires) est bijective.

\begin{propriete}
  \begin{itemize}
    \item La restriction de $\cos$ à l'intervalle $\interff{0;\pi}$ est
      bijective.
    \item La restriction de $\sin$ à l'intervalle
      $\interff{-\frac{\pi}2;\frac{\pi}2}$ est bijective.
  \end{itemize}
\end{propriete}
\begin{preuve}
  Il suffit pour cela de considèrer les variations de ces fonctions.
\end{preuve}

\begin{propriete}
  Une fonction continue bijective, à dérivée continue bijective admet
  une réciproque continue bijective.
\end{propriete}

Cette dernière affirmation n'est pas démontrée, mais on peut en donner
un exemple : les fonctions $\exp$ et $\ln$.

\begin{definition}
  \begin{itemize}
    \item La réciproque de $\cos$ s'appelle $\arccos$ et est définie sur
      $\interff{-1;1}$.
    \item La réciproque de $\sin$ s'appelle $\arcsin$ et est définie sur
      $\interff{-1;1}$.
  \end{itemize}
\end{definition}

\begin{propriete}
  \begin{itemize}
    \item $\arccos' x = \frac{-1}{\sqrt{1-x^2}}$
    \item $\arcsin' x = \frac{1}{\sqrt{1-x^2}}$
  \end{itemize}
\end{propriete}

Ces deux dernières dérivées sont particulièrement utiles dans le cadre
de l'intégration. On retrouve ainsi des résultats assez intéressants.

\section*{Réferences :}

\begin{itemize}
  \item \url{http://xavier.hubaut.info/coursmath/ana/sinus.htm}
\end{itemize}

\end{document}
