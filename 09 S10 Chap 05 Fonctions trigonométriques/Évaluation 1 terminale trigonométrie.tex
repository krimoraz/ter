\documentclass[12pt,a4paper,french]{article}
\usepackage[utf8]{inputenc}
\usepackage[T1]{fontenc}
\usepackage{babel}
\usepackage[thmmarks]{ntheorem}
\usepackage{amsmath}
\usepackage{amsfonts}
\usepackage{amssymb}

\usepackage{array}

\usepackage{lmodern}
\usepackage{kpfonts}

\usepackage[bookmarks=false,colorlinks,linkcolor=blue,pdfusetitle]{hyperref}

\pdfminorversion 7
\pdfobjcompresslevel 3

\usepackage{tabularx}
\usepackage[autolanguage,np]{numprint}
\usepackage{enumitem}

\usepackage{tipfr}
\usepackage{pgf}
\usepackage{tikz}
\usepackage{tkz-euclide}
\usetkzobj{all}
\usetikzlibrary{hobby}
\usepackage{tkz-tab}

\usepackage[top=1.9cm,bottom=2cm,left=2cm,right=2cm]{geometry}

\usepackage{lastpage}

\usepackage{esvect}
\usepackage{marginnote}

\usepackage{wrapfig}

\usepackage[defaultlines=5,all]{nowidow}


\usepackage[]{algorithm2e}

\usepackage{ifthen}
\usepackage{fancyhdr}
\pagestyle{fancy}


\makeatletter

\count1=\year \count2=\year
\ifnum\month<8\advance\count1by-1\else\advance\count2by1\fi

\setlength{\headheight}{14.5pt}
\renewcommand{\headrulewidth}{0pt}
\renewcommand{\footrulewidth}{0pt}
\cfoot{\textsl{\footnotesize{Année \number\count1/\number\count2}}}

\rfoot{%
  \ifthenelse{\value{page}=0}{%
  }
  {%
    \footnotesize{Page \thepage/ \pageref{LastPage}}
  }
}

\rhead{}

\lhead{%
  \ifthenelse{\value{page}=1}{%
    Nom:\dotfill\hfill Prénom: \dotfill \hfill
    Classe: \@author \dots%
  }
  { }
}

\renewcommand{\maketitle}%
{\framebox{%
    \begin{minipage}{1.0\linewidth}%
      \begin{center}%
        \Large \@title ~-- \@author \\%
        \@date%
      \end{center}%
    \end{minipage}}%
  \normalsize%
  %\vspace{1cm}%
}

%Des macros pour les noms d'ensmbles
\newcommand{\R}{\mathbf{R}}
\newcommand{\Q}{\mathbf{Q}}
\newcommand{\Z}{\mathbf{Z}}
\newcommand{\C}{\mathbf{C}}
\newcommand{\N}{\mathbf{N}}

\newcommand{\norme}[1]{\left\lVert #1 \right\rVert}
\newcommand{\abs}[1]{\left\lvert #1 \right\rvert}
\newcommand{\diff}{\mathop{}\mathopen{}\mathrm{d}}
%Une macro récursive pour l'intérieru des vecteurs
%http://tex.stackexchange.com/questions/19693/arguments-of-custom-commands-as-comma-separated-list

\newcommand\vecteur[2][\\]{%
    \global\def\my@delim{#1}%
    \left(\negthinspace\begin{matrix}
        \my@vector #2,\relax\noexpand\@eolst%
    \end{matrix}\right)}

%Une macro pour les vecteurs
\def\my@vector #1,#2\@eolst{%
   \ifx\relax#2\relax
      #1
   \else
      #1\my@delim
      \my@vector #2\@eolst%
   \fi}

%Une macro récursive pour mettre formater l'intérieur des intervalles
\def\my@intervalle #1;#2\@eolst{%
  \ifx\relax#2\relax
    #1
  \else
    \my@intervalle #2\@eolst%
  \fi}

%Quatre macros pour les quatres types d'intervalles
\newcommand{\interff}[1]{%
  \left[\my@intervalle #1;\relax\noexpand\@eolst%
  \right]
}
\newcommand{\interfo}[1]{%
  \left[\my@intervalle #1;\relax\noexpand\@eolst%
  \right[}
\newcommand{\interof}[1]{%
  \left]\my@intervalle #1;\relax\noexpand\@eolst%
  \right]}
\newcommand{\interoo}[1]{%
  \left]\my@intervalle #1;\relax\noexpand\@eolst%
  \right[}

\makeatother


\usepackage{framed}

\theoremstyle{break}
\newtheorem{definition}{Définition}
\newtheorem{propriete}{Propriété}
\newtheorem{corollaire}{Corollaire}
\newtheorem{propdef}{Propriété - Définition}
\newtheorem{theoreme}{Théorème}
\theoremstyle{plain}
\theorembodyfont{\normalfont}
\newtheorem{exerciceT}{Exercice}
\theoremstyle{nonumberplain}
\newtheorem{remarque}{Remarque}
\newtheorem{notation}{Notation}
\newtheorem{probleme}{Problème}
\theoremsymbol{\ensuremath{\blacksquare}}
\newtheorem{preuve}{Preuve}
\theoremsymbol{}
\theoremstyle{nonumberbreak}
\newtheorem{exemple}{Exemple}

\newenvironment{exercice}{\begin{framed}\begin{exerciceT}}{\end{exerciceT}\end{framed}}

\setlength{\parsep}{0pt}
\setlength{\parskip}{5pt}
\setlength{\parindent}{0pt}
\setlength{\itemsep}{7pt}

\setlist{noitemsep}
%\setlist[1]{\labelindent=\parindent} % < Usually a good idea
\setlist[itemize]{leftmargin=*}
\setlist[itemize,1]{label=$\triangleright$}
\setlist[enumerate]{labelsep=*, leftmargin=1.5pc}
\setlist[enumerate,1]{label=\arabic*., ref=\arabic*}
\setlist[enumerate,2]{label=\emph{\alph*}),
ref=\theenumi.\emph{\alph*}}
\setlist[enumerate,3]{label=\roman*), ref=\theenumii.\roman*}
\setlist[description]{font=\sffamily\bfseries}

\usepackage{multicol}
\setlength{\columnseprule}{0pt}

\usepackage[]{exsheets}
\SetupExSheets{headings=block}

\everymath{\displaystyle\everymath{}}

\title{Évaluation \no{11} : trigonométrie}
\author{\bsc{Ts}}
\date{janvier 2018}

\begin{document}

\maketitle

\begin{tabular}{|p{6em}|p{26em}|p{6em}|}\hline
   & & \\
   & & \\
   \hfill\Huge /\totalpoints* & & \\
   & & \\
   & & \\ \hline
\end{tabular}


\begin{question}[ID=trigonometrie;bac;Nouvelle-Caledonie;2013]
  ~\\[-6ex]
  \phantom{a}\hfill\textbf{(\GetQuestionProperty{points}{\CurrentQuestionID} points)}\\

  \textit{Pour l'énoncé suivant, indiquer si la proposition
    correspondante est vraie ou fausse et proposer une justification de
  la réponse choisie.}

  On considère une fonction $f$, sa dérivée $f'$ et son unique primitive
  $F$ s'annulant en $x=0$. Les représentations graphiques de ses trois
  fonctions sont données (dans le désordre) par les courbes ci-dessous.

  \textbf{Proposition :} «La courbe 3 est la représentation graphique de
  $f$.» \addpoints*{3}

  \textbf{Courbe 1}

  \begin{tikzpicture}[xscale=2,yscale=0.7]
    \tkzInit[xmax=3.5,xmin=-2,ymin=-4.1,ymax=4,ystep=2]
    \tkzAxeX[trig=2, label options = {below right = 7 pt} ]
    \tkzAxeY[orig = false]

    \draw [thick, green] plot [smooth,domain=-1.57:3.14] (\x,
    {-2*sin(2*\x r)}) ;
  \end{tikzpicture}

  \textbf{Courbe 2}

  \begin{tikzpicture}[xscale=2,yscale=0.7]
    \tkzInit[xmax=3.5,xmin=-2,ymin=-2.1,ymax=2]
    \tkzAxeX[trig=2, label options = {below right = 7 pt} ]
    \tkzAxeY[orig = false,step = 2]

    \draw [thick, red] plot [smooth,domain=-1.57:3.14] (\x,
    {2*cos(2*\x r)}) ;
  \end{tikzpicture}

  \textbf{Courbe 3}

  \begin{tikzpicture}[xscale=2,yscale=0.7]
    \tkzInit[xmax=3.5,xmin=-2,ymin=-1.5,ymax=1,ystep=0.5]
    \tkzAxeX[trig=2, label options = {below right = 7 pt} ]
    \tkzAxeY[orig = false, step=0.5]

    \draw [thick, blue] plot [smooth,domain=-1.57:3.14] (\x,
    {2*sin(2*\x r)}) ;
  \end{tikzpicture}

  \blank[style=dotted,width=6\linewidth,linespread=1.7]{}

  \blank[style=dotted,width=6\linewidth,linespread=1.7]{}

  \blank[style=dotted,width=6\linewidth,linespread=1.7]{}

  \blank[style=dotted,width=8\linewidth,linespread=1.7]{}


\end{question}
\begin{solution}
  Il s'agit d'une partie du sujet de Nouvelle-Calédonie 2013.

  La proposition est fausse : plusieurs arguments permettent de
  l'affirmer. Par exemple, les courbes 1 et 2 son manifestement des
  sinus de signe opposés et donc l'une est primitive et l'autre est
  dérivée. On peut utiliser le facteur multiplicatif pour les ranger
  dans l'ordre :
  \begin{itemize}
    \item la courbe 2 est la fonction $f$ ;
    \item la courbe 1 est la dérivée de $f$, son amplitude est deux fois
      plus élevée ;
    \item la courbe 3 est la primitive $F$, son amplitude est deux fois
      moins élevée.
  \end{itemize}
\end{solution}


\newpage
\section*{Correction}
\printsolutions
\vfill
\hrule
\vfill
\section*{Correction}
\printsolutions
\vfill
\hrule
\vfill
\section*{Correction}
\printsolutions
\end{document}
