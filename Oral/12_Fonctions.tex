\documentclass[12pt,a4paper,french]{article}

\usepackage[utf8]{inputenc}
\usepackage[T1]{fontenc}
\usepackage{lmodern}
\usepackage{fourier}

\usepackage{amsmath,amsfonts,amssymb}

\usepackage{babel}

\author{}
\title{Sujet \no{12}}
\date{}

\setlength{\parindent}{0pt}

\begin{document}

\maketitle
\begin{center}Ne rien inscrire sur le sujet.

  Le sujet est à rendre en fin d'exercice.
\end{center}

\bigskip

\textbf{Exercice 1}

Question de cours : Donner puis démontrer
$\lim_{x\to+\infty}\frac{e^x}x$.

\medskip

\textbf{Exercice 2}

Frej a remarqué qu'il mettait en moyenne 20 minutes pour aller à son
lycée. On considère que sa durée de trajet, modélisée par une variable
aléatoire $D$ suit une loi normale de paramètes $\mu = 20$ et $\sigma$
inconnu.

\begin{enumerate}
  \item En partant à 7h40 pour 8h, quelle est la probabilité qu'il
    arrive à l'heure ?
  \item Voyant qu'il est trop souvent en retard, il décide de partir à
    7h30. La probabilité de retard est désormais de 0,0228.

    Déterminer $\sigma$.
  \item À quelle heure doit-il partir pour arriver à l'heure avec une
    probabilité de 99\% ?
\end{enumerate}

\end{document}
