\documentclass[12pt,a4paper,french]{article}

\usepackage[utf8]{inputenc}
\usepackage[T1]{fontenc}
\usepackage{lmodern}
\usepackage{fourier}

\usepackage{amsmath,amsfonts,amssymb}

\usepackage{babel}

\author{}
\title{Sujet \no{10}}
\date{}

\setlength{\parindent}{0pt}

\begin{document}

\maketitle
\begin{center}Ne rien inscrire sur le sujet.

  Le sujet est à rendre en fin d'exercice.
\end{center}

\bigskip

\textbf{Exercice 1}

Question de cours : démontrer qu'une droite $\mathcal{D}$ est
parallèle à un plan $\mathcal{P}$ si et seulement si elle est parallèle
à deux droites sécantes de ce plan.

\medskip

\textbf{Exercice 2}

On considère la fonction de la variable réelle $x$ définie sur
l'ensemble des nombres réels positifs par $f(x) = \frac{2x - \sqrt{x}}{2
+ \sqrt{x}}$.

\begin{enumerate}
  \item Montrer que $f$ est dérivable sur $\left]0 ; +\infty\right[$ et
    que $f'(x) = \frac{ x + 4\sqrt{x} - 1}{\sqrt{x}(2 + \sqrt{x})^2}$
  \item Résoudre l'équation $X^2 + 4X - 1 =0$ en déduire le signe de
    $f'(x)$.
  \item Dresser le tableau de variation complet de $f$.
\end{enumerate}

\end{document}
