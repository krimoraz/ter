\documentclass[12pt,a4paper,french]{article}

\usepackage[utf8]{inputenc}
\usepackage[T1]{fontenc}
\usepackage{lmodern}
\usepackage{fourier}

\usepackage{amsmath}

\usepackage{babel}

\author{}
\title{Sujet \no{6}}
\date{}

\setlength{\parindent}{0pt}

\begin{document}

\maketitle
\begin{center}Ne rien inscrire sur le sujet.

  Le sujet est à rendre en fin d'exercice.
\end{center}

\bigskip

\textbf{Exercice 1}

On donne la suite $(u_n)$ définie par $\forall n\in \mathbf{N},
u_{n+1} = \frac{1}{\sqrt{2}}u_{n} + 1$ et $u_0 = 0$.
\begin{enumerate}
  \item La suite $(u_n)$ est ici définie par $u_{n+1} = f(u_n)$. Quelle
    condition respecte la fonction $f$ ici pour pouvoir dire si $u_n$
    converge vers $\ell$, alors $f(\ell) = \ell$ ?
  \item Calculer la valeur de $\ell$.
  \item Soit la suite $(v_n)$ définie par $\forall n\in \mathbf{N}, v_n
    = u_n - \frac{2}{2 - \sqrt{2}}$.
    \begin{enumerate}
      \item Montrer que $(v_n)$ est décroissante.
      \item En admettant que la suite $(v_n)$ est minorée par 0, que
        peut-on dire de la convergence de cette suite ?
      \item En déduire la limite de la suite $(u_n)$.
    \end{enumerate}
\end{enumerate}

\medskip

\textbf{Exercice 2}

En rentrant de soirée, Léo sait qu'il va arriver dans la station de
métro la plus proche entre 1h et 1h30. On fait l'hypothèse que son heure
d'arrivée suit une loi uniforme. Depuis minuit, il passe un métro toutes
les 20 minutes dans cette station. Quelle est la probabilité que Léo
attente :
\begin{enumerate}
  \item moins de 5 minutes ;
  \item plus d'un quart d'heure.
\end{enumerate}

On rappelle que :
\begin{itemize}
  \item pour une loi à densité $f$ continue, $P(X \leq x) =
    \int_{-\infty}^x f(t) \mathrm{d}\,t$ ;
  \item la densité de probabilité de la loi uniforme entre $a$ et $b$
    est $f : \left\lbrace \begin{array}{lr} \frac1{b-a} & x \in ]a;b[
    \\ 0 & \text{sinon}\end{array}\right.$
\end{itemize}

\end{document}
