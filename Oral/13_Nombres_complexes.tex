\documentclass[12pt,a4paper,french]{article}

\usepackage[utf8]{inputenc}
\usepackage[T1]{fontenc}
\usepackage{lmodern}
\usepackage{fourier}

\usepackage{amsmath,amsfonts,amssymb}

\usepackage{babel}

\author{}
\title{Sujet \no{13}}
\date{}

\setlength{\parindent}{0pt}

\begin{document}

\maketitle
\begin{center}Ne rien inscrire sur le sujet.

  Le sujet est à rendre en fin d'exercice.
\end{center}

\bigskip

\textbf{Exercice 1}

Calcul de $\cos\frac{2\pi}{5}$

\begin{enumerate}
  \item Montrer que tout nombre complexe $z\neq 1$, $1 + z + z^2 + Z^3 +
    Z^4 = \frac{1 - z^5}{1 - z}$.
  \item Démontrer alors que pour $z_0 = e^{i\frac{2\pi}5}$, $\left(z_0^2
    + \frac1{z_0^2} \right) + \left(z_0 + \frac1{z_0}\right) + 1 = 0$
  \item Montrer que $\left(z_0^2 + \frac1{z_0^2} \right) = \left(z_0 +
    \frac1{z_0}\right)^2 - 2$.
  \item On admet que
    \begin{itemize}
      \item $z_0 + \frac1{z_0} = 2\cos\frac{2\pi}5$ ;
      \item si on pose $X = 2\cos\frac{2\pi}5$, alors $X$ est solution
        de $X^2 + X - 3 = 0$.
    \end{itemize}
    Calculer la valeur de $\cos\frac{2\pi}5$.
\end{enumerate}

\medskip

\textbf{Exercice 2}

Question de cours : Démontrer que si $F$ est une primitive de $f$ sur un
intervalle $[a;b]$ alors $f$ admet une infinité de primitive de la forme
$x\mapsto F(x) + k,\ \in\mathbf{R}$.

\end{document}
