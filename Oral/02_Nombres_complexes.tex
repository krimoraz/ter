\documentclass[12pt,a4paper,french]{article}

\usepackage[utf8]{inputenc}
\usepackage[T1]{fontenc}
\usepackage{lmodern}
\usepackage{fourier}

\usepackage{babel}

\author{}
\title{Sujet \no{2}}
\date{}

\setlength{\parindent}{0pt}

\begin{document}

\maketitle
\begin{center}Ne rien inscrire sur le sujet.

  Le sujet est à rendre en fin d'exercice.
\end{center}

\bigskip

\textbf{Exercice 1}

On désigne par $A$ et $B$ les points du plan complexe d'affixe $z_A = 1$
et $z_B = i$.
\begin{enumerate}
  \item Déterminer l'ensemble des points $M$ d'affixe $z$ du plan
    complexe tels que $\left\lvert z - 1 \right\rvert = \left\lvert z -
    i \right\rvert$.
  \item Montrer que l'ensemble des points $M$ d'affixe $z$ du plan
    complexe tel que $\frac{z - i}{z -1}$ soit un imaginaire pur est un
    cercle de diamètre $[AB]$ privé du point $A$.
  \item Écrire une équation permettant de trouver les points
    d'intersection de la droite et du cercle. On ne demande pas de la
    résoudre ici.
\end{enumerate}

\medskip

\textbf{Exercice 2}

On considère un sac opaque contenant trois billes rouges et deux billes
bleues indiscernables au toucher.

On effectue un tirage sans remise.

\begin{enumerate}
  \item Quelle est la probabilité de tirer une bille bleue, puis une
    bille rouge ?
  \item Quelle est la probabilité de tirer une bille rouge puis une
    bille bleue.
  \item Les événements «tirer une bille bleue puis une bille rouge» et
    «tirer une bille rouge puis une bille bleue» sont ils indépendants.
\end{enumerate}

\end{document}
