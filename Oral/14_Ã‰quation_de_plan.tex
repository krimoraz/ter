\documentclass[12pt,a4paper,french]{article}

\usepackage[utf8]{inputenc}
\usepackage[T1]{fontenc}
\usepackage{lmodern}
\usepackage{fourier}

\usepackage{amsmath,amsfonts,amssymb}

\usepackage{babel}

\author{}
\title{Sujet \no{14}}
\date{}

\setlength{\parindent}{0pt}

\begin{document}

\maketitle
\begin{center}Ne rien inscrire sur le sujet.

  Le sujet est à rendre en fin d'exercice.
\end{center}

\bigskip

\textbf{Exercice 1}

On donne, dans un repère orthnormé l'équation paramétrique suivante :
\[ \left\lbrace \begin{array}{lc}
    x = 1 - s + 4t &                                 \\
    y = 2 + 2s -t  & s\in\mathbf{R}, t\in \mathbf{R} \\
    z = -1 +s + 2t &                                 \\
\end{array}\right.\]

\begin{enumerate}
  \item Cette représentation paramétrique définit-elle un plan ?
  \item Déterminer un vecteur normal à ce plan.
  \item En déduire une équation cartésienne.
\end{enumerate}

\medskip

\textbf{Exercice 2}

Question de cours : Démontrer que si $f$ est une fonction continue
positive et croissante sur un intervalle $[a;b]$ alors la fonction
$F:x\mapsto \int_a^x f(t) \mathrm{d}\,t$ est définie et dérivable et que
$F' = f$.

\end{document}
