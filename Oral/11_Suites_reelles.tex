\documentclass[12pt,a4paper,french]{article}

\usepackage[utf8]{inputenc}
\usepackage[T1]{fontenc}
\usepackage{lmodern}
\usepackage{fourier}

\usepackage{amsmath,amsfonts,amssymb}

\usepackage{babel}

\author{}
\title{Sujet \no{11}}
\date{}

\setlength{\parindent}{0pt}

\begin{document}

\maketitle
\begin{center}Ne rien inscrire sur le sujet.

  Le sujet est à rendre en fin d'exercice.
\end{center}

\bigskip

\textbf{Exercice 1}

Question de cours : démontrer qu'une droite $\mathcal{D}$ est
orthogonale à un plan $\mathcal{P}$ si et seulement si elle est
orthogonale à deux droites sécantes de ce plan.

\medskip

\textbf{Exercice 2}

On considère la suite numérique $(u)$ défine par $u_0 = \frac12$ et pour
tout $n$ entier naturel, $u_{n+1} = \frac{3u_n}{1 + 2u_n}$.

\begin{enumerate}
  \item Démontrer, par récurrence, que la suite est minorée par 0.
  \item On admet que la suite est majorée par 1. Montrer qu'elle est
    croissante. Justifier alors qu'elle converge.
  \item Soit $v_n = \frac{u_n}{1 - u_n}$. Montrer que $v_n$ est une
    suite géométrique de raison 3.
  \item On admet que $u_n = \frac{3_n}{3^n + 1}$. Donner sa limite.
\end{enumerate}

\end{document}
