\documentclass[12pt,a4paper,french]{article}

\usepackage[utf8]{inputenc}
\usepackage[T1]{fontenc}
\usepackage{lmodern}
\usepackage{fourier}

\usepackage{amsmath,amsfonts,amssymb}

\usepackage{babel}

\author{}
\title{Sujet \no{15}}
\date{}

\setlength{\parindent}{0pt}

\begin{document}

\maketitle
\begin{center}Ne rien inscrire sur le sujet.

  Le sujet est à rendre en fin d'exercice.
\end{center}

\bigskip

\textbf{Exercice 1}

Question de cours : Démontrer que si $A$ et $B$ sont indépendants, alors
$A$ et $\overline{B}$ le sont aussi.

\medskip

\textbf{Exercice 2}

On considère la fonction $f$ définie sur $]0;+\infty[$ par $f(x) = \ln x
+ x^2$.

\begin{enumerate}
  \item Déterminer les limites de $f$ aux bornes de son intervalle de
    définition.
  \item Calculer $f'(x)$ et étudier son signe.
  \item Dresser le tableau de variation de $f$.
  \item Montrer que l'équation $f(x) = 0$ admet une unique solution
    $\alpha$ dans l'intervalle $]0;+\infty[$.
  \item Donner une valeur approchée de $\alpha$ à $10^{-2}$ près.
\end{enumerate}

\end{document}
