\documentclass[12pt,a4paper,french]{article}

\usepackage[utf8]{inputenc}
\usepackage[T1]{fontenc}
\usepackage{lmodern}
\usepackage{fourier}

\usepackage{babel}

\author{}
\title{Sujet \no{8}}
\date{}

\setlength{\parindent}{0pt}

\begin{document}

\maketitle
\begin{center}Ne rien inscrire sur le sujet.

  Le sujet est à rendre en fin d'exercice.
\end{center}

\bigskip

\textbf{Exercice 1}

On donne la suite $(u_n)$ définie par $\forall n\in \mathbf{N},
u_{n+1} = \frac{1}{\sqrt{2}}u_{n} + 1$ et $u_0 = 0$.
\begin{enumerate}
  \item La suite $(u_n)$ est ici définie par $u_{n+1} = f(u_n)$. Quelle
    condition respecte la fonction $f$ ici pour pouvoir dire si $u_n$
    converge vers $\ell$, alors $f(\ell) = \ell$ ?
  \item Calculer la valeur de $\ell$.
  \item Soit la suite $(v_n)$ définie par $\forall n\in \mathbf{N}, v_n
    = u_n - \frac{2}{2 - \sqrt{2}}$.
    \begin{enumerate}
      \item Montrer que $(v_n)$ est décroissante.
      \item En admettant que la suite $(v_n)$ est minorée par 0, que
        peut-on dire de la convergence de cette suite ?
      \item En déduire la limite de la suite $(u_n)$.
    \end{enumerate}
\end{enumerate}

\medskip

\textbf{Exercice 2}

Question de cours : Démontrer que la loi exponentielle (définie par la
densité de probabilité $f(t) = \lambda e^{-\lambda t}$) définie bien une
loi sans mémoire, c'est-à-dire que pour $t$ et $s$ deux réels positifs,
$P_{T>s}(T > s+t) = P(T > t)$.

\end{document}
