\begin{question}[topic=suites]
On donne la suite $(u_n)$ définie par $\forall n\in \mathbf{N},
u_{n+1} = \frac{1}{\sqrt{2}}u_{n} + 1$ et $u_0 = 0$.
\begin{enumerate}
  \item La suite $(u_n)$ est ici définie par $u_{n+1} = f(u_n)$. Quelle
    condition respecte la fonction $f$ ici pour pouvoir dire si $u_n$
    converge vers $\ell$, alors $f(\ell) = \ell$ ?
  \item Calculer la valeur de $\ell$.
  \item Soit la suite $(v_n)$ définie par $\forall n\in \mathbf{N}, v_n
    = u_n - \frac{2}{2 - \sqrt{2}}$.
    \begin{enumerate}
      \item Montrer que $(v_n)$ est décroissante.
      \item En admettant que la suite $(v_n)$ est minorée par 0, que
        peut-on dire de la convergence de cette suite ?
      \item En déduire la limite de la suite $(u_n)$.
    \end{enumerate}
\end{enumerate}
\end{question}

\begin{question}[topic=probabilités]
On considère un sac opaque contenant trois billes rouges et deux billes
bleues indiscernables au toucher.

On effectue un tirage sans remise.

\begin{enumerate}
  \item Quelle est la probabilité de tirer une bille bleue, puis une
    bille rouge ?
  \item Quelle est la probabilité de tirer une bille rouge puis une
    bille bleue.
  \item Les événements «tirer une bille bleue puis une bille rouge» et
    «tirer une bille rouge puis une bille bleue» sont ils indépendants.
\end{enumerate}
\end{question}

\begin{question}[topic=complexes]
On désigne par $A$ et $B$ les points du plan complexe d'affixe $z_A = 1$
et $z_B = i$.
\begin{enumerate}
  \item Déterminer l'ensemble des points $M$ d'affixe $z$ du plan
    complexe tels que $\left\lvert z - 1 \right\rvert = \left\lvert z -
    i \right\rvert$.
  \item Montrer que l'ensemble des points $M$ d'affixe $z$ du plan
    complexe tel que $\frac{z - i}{z -1}$ soit un imaginaire pur est un
    cercle de diamètre $[AB]$ privé du point $A$.
  \item Écrire une équation permettant de trouver les points
    d'intersection de la droite et du cercle. On ne demande pas de la
    résoudre ici.
\end{enumerate}
\end{question}

\begin{question}[topic=fonction]
On considère la fonction de la variable réelle $x$ définie sur
l'ensemble des nombres réels positifs par $f(x) = \frac{2x - \sqrt{x}}{2
+ \sqrt{x}}$.

\begin{enumerate}
  \item Montrer que $f$ est dérivable sur $\left]0 ; +\infty\right[$ et
    que $f'(x) = \frac{ x + 4\sqrt{x} - 1}{\sqrt{x}(2 + \sqrt{x})^2}$
  \item Résoudre l'équation $X^2 + 4X - 1 =0$ en déduire le signe de
    $f'(x)$.
  \item Dresser le tableau de variation complet de $f$.
\end{enumerate}
\end{question}

\begin{question}[topic=géométrie]
On considère la droite $\Delta$ de représentation paramétrique \[ \Delta
  : \left\lbrace \begin{array}{lr} x = 1 - 4t & \\ y = 3 + t & t\in
\mathbf{R} \\ z = 1 - t & \end{array}\right. \]

\begin{enumerate}
  \item Donner un point et un vecteur directeur de $\Delta$.
  \item Le point $M (-3;4;1)$ appartient-il à la droite $\Delta$.
  \item Proposer une autre représentation paramétrique de cette droite.
\end{enumerate}
\end{question}

\begin{question}[topic=loi_continue]
En rentrant de soirée, Léo sait qu'il va arriver dans la station de
métro la plus proche entre 1h et 1h30. On fait l'hypothèse que son heure
d'arrivée suit une loi uniforme. Depuis minuit, il passe un métro toutes
les 20 minutes dans cette station. Quelle est la probabilité que Léo
attente :
\begin{enumerate}
  \item moins de 5 minutes ;
  \item plus d'un quart d'heure.
\end{enumerate}

On rappelle que :
\begin{itemize}
  \item pour une loi à densité $f$ continue, $P(X \leq x) =
    \int_{-\infty}^x f(t) \mathrm{d}\,t$ ;
  \item la densité de probabilité de la loi uniforme entre $a$ et $b$
    est $f : \left\lbrace \begin{array}{lr} \frac1{b-a} & x \in ]a;b[
    \\ 0 & \text{sinon}\end{array}\right.$
\end{itemize}
\end{question}

\begin{question}[topic=exponentielle]
Question de cours : Démontrer que s'il existe une fonction $f$ telle que
pour tout $x$ réel, $f'(x) = f(x)$ et $f(0) = 1$, alors cette fonction
est unique.

Démontrer de plus que $f(a+b) = f(a)f(b)$.
\end{question}

\begin{question}[topic=loi_exponentielle]
Question de cours : Démontrer que la loi exponentielle (définie par la
densité de probabilité $f(t) = \lambda e^{-\lambda t}$) définie bien une
loi sans mémoire, c'est-à-dire que pour $t$ et $s$ deux réels positifs,
$P_{T>s}(T > s+t) = P(T > t)$.
\end{question}

\begin{question}[topic=recurrence]
Discuter (et démontrer évenutellement) pour des valeurs de $n$ entières
l'inégalité $5^n \geqslant 4^n + 3^n$.
\end{question}

\begin{question}[topic=géométrie]
Question de cours : démontrer qu'une droite $\mathcal{D}$ est
parallèle à un plan $\mathcal{P}$ si et seulement si elle est parallèle
à deux droites sécantes de ce plan.
\end{question}

\begin{question}[topic=suite]
On considère la suite numérique $(u)$ défine par $u_0 = \frac12$ et pour
tout $n$ entier naturel, $u_{n+1} = \frac{3u_n}{1 + 2u_n}$.

\begin{enumerate}
  \item Démontrer, par récurrence, que la suite est minorée par 0.
  \item On admet que la suite est majorée par 1. Montrer qu'elle est
    croissante. Justifier alors qu'elle converge.
  \item Soit $v_n = \frac{u_n}{1 - u_n}$. Montrer que $v_n$ est une
    suite géométrique de raison 3.
  \item On admet que $u_n = \frac{3_n}{3^n + 1}$. Donner sa limite.
\end{enumerate}
\end{question}

\begin{question}[topic=loi_continue]
Fred a remarqué qu'il mettait en moyenne 20 minutes pour aller à son
lycée. On considère que sa durée de trajet, modélisée par une variable
aléatoire $D$ suit une loi normale de paramètes $\mu = 20$ et $\sigma$
inconnu.

\begin{enumerate}
  \item En partant à 7h40 pour 8h, quelle est la probabilité qu'il
    arrive à l'heure ?
  \item Voyant qu'il est trop souvent en retard, il décide de partir à
    7h30. La probabilité de retard est désormais de 0,0228.

    Déterminer $\sigma$.
  \item À quelle heure doit-il partir pour arriver à l'heure avec une
    probabilité de 99\% ?
\end{enumerate}
\end{question}

\begin{question}[topic=exponentielle]
Question de cours : Donner puis démontrer
$\lim_{x\to+\infty}\frac{e^x}x$.
\end{question}

\begin{question}[topic=cos:complexes]
Calcul de $\cos\frac{2\pi}{5}$

\begin{enumerate}
  \item Montrer que tout nombre complexe $z\neq 1$, $1 + z + z^2 + z^3 +
    z^4 = \frac{1 - z^5}{1 - z}$.
  \item Démontrer alors que pour $z_0 = e^{i\frac{2\pi}5}$, $\left(z_0^2
    + \frac1{z_0^2} \right) + \left(z_0 + \frac1{z_0}\right) + 1 = 0$
  \item Montrer que $\left(z_0^2 + \frac1{z_0^2} \right) = \left(z_0 +
    \frac1{z_0}\right)^2 - 2$.
  \item On admet que
    \begin{itemize}
      \item $z_0 + \frac1{z_0} = 2\cos\frac{2\pi}5$ ;
      \item si on pose $X = 2\cos\frac{2\pi}5$, alors $X$ est solution
        de $X^2 + X - 3 = 0$.
    \end{itemize}
    Calculer la valeur de $\cos\frac{2\pi}5$.
\end{enumerate}
\end{question}

\begin{question}[topic=intégrale]
Question de cours : Démontrer que si $F$ est une primitive de $f$ sur un
intervalle $[a;b]$ alors $f$ admet une infinité de primitive de la forme
$x\mapsto F(x) + k,\ \in\mathbf{R}$.
\end{question}

\begin{question}[topic=géométrie]
On donne, dans un repère orthnormé l'équation paramétrique suivante :
\[ \left\lbrace \begin{array}{lc}
    x = 1 - s + 4t &                                 \\
    y = 2 + 2s -t  & s\in\mathbf{R}, t\in \mathbf{R} \\
    z = -1 +s + 2t &                                 \\
\end{array}\right.\]

\begin{enumerate}
  \item Cette représentation paramétrique définit-elle un plan ?
  \item Déterminer un vecteur normal à ce plan.
  \item En déduire une équation cartésienne.
\end{enumerate}
\end{question}

\begin{question}[topic=probabilités]
Question de cours : Démontrer que si $A$ et $B$ sont indépendants, alors
$A$ et $\overline{B}$ le sont aussi.
\end{question}

\begin{question}[topic=fonction]
On considère la fonction $f$ définie sur $]0;+\infty[$ par $f(x) = \ln x
+ x^2$.

\begin{enumerate}
  \item Déterminer les limites de $f$ aux bornes de son intervalle de
    définition.
  \item Calculer $f'(x)$ et étudier son signe.
  \item Dresser le tableau de variation de $f$.
  \item Montrer que l'équation $f(x) = 0$ admet une unique solution
    $\alpha$ dans l'intervalle $]0;+\infty[$.
  \item Donner une valeur approchée de $\alpha$ à $10^{-2}$ près.
\end{enumerate}
\end{question}

\begin{question}[topic=nombres_complexes]
Question de cours : montrer les trois affirmations suivantes :
\begin{itemize}
  \item Affirmation 1 : $z + \overline{z} = 2\Re(z)$
  \item Affirmation 2 : $z - \overline{z} = 2i\Im(z)$
  \item Affirmation 3 : $z \times \overline{z} = \lvert z \rvert^2$
\end{itemize}
\end{question}

\begin{question}[topic=fonction]
Soit la fonction $f$ définie sur $[-1;1]$ par $f(x) = (1-x^2)e^x$.

\begin{enumerate}
  \item Calculer la dérivée de $f$ et en déduire les variations de la
    fonctions $f$. Préciser les valeurs $f(-1)$ et $f(1)$.
  \item Donner la valeur de $x_{\max}$ pour laquelle le maximum est
    atteint et donner $f(x_{\max})$.
  \item On donne $f''(x) = \left(-x^2-4x-1\right)e^{x}$. Montrer que
    $f(x) - 2f'(x) + f''(x) = - 2e^x$.

    En déduire $\int_{-1}^1 f(x) \mathrm\,x$.
  \item Interpréter géométriquement ce résultat.
\end{enumerate}
\end{question}

