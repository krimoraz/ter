\begin{question}[topic=suites]
  On donne la suite $(u_n)$ définie par $\forall n\in \mathbf{N},
  u_{n+1} = \frac{1}{\sqrt{2}}u_{n} + 1$ et $u_0 = 0$.
  \begin{enumerate}
    \item La suite $(u_n)$ est ici définie par $u_{n+1} = f(u_n)$. Quelle
      condition respecte la fonction $f$ ici pour pouvoir dire si $u_n$
      converge vers $\ell$, alors $f(\ell) = \ell$ ?
    \item Calculer la valeur de $\ell$.
    \item Soit la suite $(v_n)$ définie par $\forall n\in \mathbf{N}, v_n
      = u_n - \frac{2}{2 - \sqrt{2}}$.
      \begin{enumerate}
        \item Montrer que $(v_n)$ est décroissante.
        \item En admettant que la suite $(v_n)$ est minorée par 0, que
          peut-on dire de la convergence de cette suite ?
        \item En déduire la limite de la suite $(u_n)$.
      \end{enumerate}
  \end{enumerate}
\end{question}

\begin{question}[topic=probabilités]
  On considère un sac opaque contenant trois billes rouges et deux billes
  bleues indiscernables au toucher.

  On effectue un tirage sans remise.

  \begin{enumerate}
    \item Quelle est la probabilité de tirer une bille bleue, puis une
      bille rouge ?
    \item Quelle est la probabilité de tirer une bille rouge puis une
      bille bleue.
    \item Les événements «tirer une bille bleue puis une bille rouge» et
      «tirer une bille rouge puis une bille bleue» sont ils indépendants.
  \end{enumerate}
\end{question}

\begin{question}[topic=complexes]
  On désigne par $A$ et $B$ les points du plan complexe d'affixe $z_A = 1$
  et $z_B = i$.
  \begin{enumerate}
    \item Déterminer l'ensemble des points $M$ d'affixe $z$ du plan
      complexe tels que $\left\lvert z - 1 \right\rvert = \left\lvert z -
      i \right\rvert$.
    \item Montrer que l'ensemble des points $M$ d'affixe $z$ du plan
      complexe tel que $\frac{z - i}{z -1}$ soit un imaginaire pur est un
      cercle de diamètre $[AB]$ privé du point $A$.
    \item Écrire une équation permettant de trouver les points
      d'intersection de la droite et du cercle. On ne demande pas de la
      résoudre ici.
  \end{enumerate}
\end{question}

\begin{question}[topic=fonction]
  On considère la fonction de la variable réelle $x$ définie sur
  l'ensemble des nombres réels positifs par $f(x) = \frac{2x - \sqrt{x}}{2
  + \sqrt{x}}$.

  \begin{enumerate}
    \item Montrer que $f$ est dérivable sur $\left]0 ; +\infty\right[$ et
      que $f'(x) = \frac{ x + 4\sqrt{x} - 1}{\sqrt{x}(2 + \sqrt{x})^2}$
    \item Résoudre l'équation $X^2 + 4X - 1 =0$ en déduire le signe de
      $f'(x)$.
    \item Dresser le tableau de variation complet de $f$.
  \end{enumerate}
\end{question}

\begin{question}[topic=géométrie]
  On considère la droite $\Delta$ de représentation paramétrique \[ \Delta
    : \left\lbrace \begin{array}{lr} x = 1 - 4t & \\ y = 3 + t & t\in
  \mathbf{R} \\ z = 1 - t & \end{array}\right. \]

  \begin{enumerate}
    \item Donner un point et un vecteur directeur de $\Delta$.
    \item Le point $M (-3;4;1)$ appartient-il à la droite $\Delta$.
    \item Proposer une autre représentation paramétrique de cette droite.
  \end{enumerate}
\end{question}

\begin{question}[topic=loi_continue]
  En rentrant de soirée, Léo sait qu'il va arriver dans la station de
  métro la plus proche entre 1h et 1h30. On fait l'hypothèse que son heure
  d'arrivée suit une loi uniforme. Depuis minuit, il passe un métro toutes
  les 20 minutes dans cette station. Quelle est la probabilité que Léo
  attente :
  \begin{enumerate}
    \item moins de 5 minutes ;
    \item plus d'un quart d'heure.
  \end{enumerate}

  On rappelle que :
  \begin{itemize}
    \item pour une loi à densité $f$ continue, $P(X \leq x) =
      \int_{-\infty}^x f(t) \mathrm{d}\,t$ ;
    \item la densité de probabilité de la loi uniforme entre $a$ et $b$
      est $f : \left\lbrace \begin{array}{lr} \frac1{b-a} & x \in ]a;b[
    \\ 0 & \text{sinon}\end{array}\right.$
  \end{itemize}
\end{question}

\begin{question}[topic=exponentielle]
  Question de cours : Démontrer que s'il existe une fonction $f$ telle que
  pour tout $x$ réel, $f'(x) = f(x)$ et $f(0) = 1$, alors cette fonction
  est unique.

  Démontrer de plus que $f(a+b) = f(a)f(b)$.
\end{question}

\begin{question}[topic=loi_exponentielle]
  Question de cours : Démontrer que la loi exponentielle (définie par la
  densité de probabilité $f(t) = \lambda e^{-\lambda t}$) définie bien une
  loi sans mémoire, c'est-à-dire que pour $t$ et $s$ deux réels positifs,
  $P_{T>s}(T > s+t) = P(T > t)$.
\end{question}

\begin{question}[topic=recurrence]
  Discuter (et démontrer évenutellement) pour des valeurs de $n$ entières
  l'inégalité $5^n \geqslant 4^n + 3^n$.
\end{question}

\begin{question}[topic=géométrie]
  Question de cours : démontrer qu'une droite $\mathcal{D}$ est
  parallèle à un plan $\mathcal{P}$ si et seulement si elle est parallèle
  à deux droites sécantes de ce plan.
\end{question}

\begin{question}[topic=suite]
  On considère la suite numérique $(u)$ défine par $u_0 = \frac12$ et pour
  tout $n$ entier naturel, $u_{n+1} = \frac{3u_n}{1 + 2u_n}$.

  \begin{enumerate}
    \item Démontrer, par récurrence, que la suite est minorée par 0.
    \item On admet que la suite est majorée par 1. Montrer qu'elle est
      croissante. Justifier alors qu'elle converge.
    \item Soit $v_n = \frac{u_n}{1 - u_n}$. Montrer que $v_n$ est une
      suite géométrique de raison 3.
    \item On admet que $u_n = \frac{3_n}{3^n + 1}$. Donner sa limite.
  \end{enumerate}
\end{question}

\begin{question}[topic=loi_continue]
  Fred a remarqué qu'il mettait en moyenne 20 minutes pour aller à son
  lycée. On considère que sa durée de trajet, modélisée par une variable
  aléatoire $D$ suit une loi normale de paramètes $\mu = 20$ et $\sigma$
  inconnu.

  \begin{enumerate}
    \item En partant à 7h40 pour 8h, quelle est la probabilité qu'il
      arrive à l'heure ?
    \item Voyant qu'il est trop souvent en retard, il décide de partir à
      7h30. La probabilité de retard est désormais de 0,0228.

      Déterminer $\sigma$.
    \item À quelle heure doit-il partir pour arriver à l'heure avec une
      probabilité de 99\% ?
  \end{enumerate}
\end{question}

\begin{question}[topic=exponentielle]
  Question de cours : Donner puis démontrer
  $\lim_{x\to+\infty}\frac{e^x}x$.
\end{question}

\begin{question}[topic=cos:complexes]
  Calcul de $\cos\frac{2\pi}{5}$

  \begin{enumerate}
    \item Montrer que tout nombre complexe $z\neq 1$, $1 + z + z^2 + z^3 +
      z^4 = \frac{1 - z^5}{1 - z}$.
    \item Démontrer alors que pour $z_0 = e^{i\frac{2\pi}5}$, $\left(z_0^2
      + \frac1{z_0^2} \right) + \left(z_0 + \frac1{z_0}\right) + 1 = 0$
    \item Montrer que $\left(z_0^2 + \frac1{z_0^2} \right) = \left(z_0 +
      \frac1{z_0}\right)^2 - 2$.
    \item On admet que
      \begin{itemize}
        \item $z_0 + \frac1{z_0} = 2\cos\frac{2\pi}5$ ;
        \item si on pose $X = 2\cos\frac{2\pi}5$, alors $X$ est solution
          de $X^2 + X - 3 = 0$.
      \end{itemize}
      Calculer la valeur de $\cos\frac{2\pi}5$.
  \end{enumerate}
\end{question}

\begin{question}[topic=intégrale]
  Question de cours : Démontrer que si $F$ est une primitive de $f$ sur un
  intervalle $[a;b]$ alors $f$ admet une infinité de primitive de la forme
  $x\mapsto F(x) + k,\ \in\mathbf{R}$.
\end{question}

\begin{question}[topic=géométrie]
  On donne, dans un repère orthnormé l'équation paramétrique suivante :
  \[ \left\lbrace \begin{array}{lc}
      x = 1 - s + 4t &                                 \\
      y = 2 + 2s -t  & s\in\mathbf{R}, t\in \mathbf{R} \\
      z = -1 +s + 2t &                                 \\
  \end{array}\right.\]

  \begin{enumerate}
    \item Cette représentation paramétrique définit-elle un plan ?
    \item Déterminer un vecteur normal à ce plan.
    \item En déduire une équation cartésienne.
  \end{enumerate}
\end{question}

\begin{question}[topic=probabilités]
  Question de cours : Démontrer que si $A$ et $B$ sont indépendants, alors
  $A$ et $\overline{B}$ le sont aussi.
\end{question}

\begin{question}[topic=fonction]
  On considère la fonction $f$ définie sur $]0;+\infty[$ par $f(x) = \ln x
  + x^2$.

  \begin{enumerate}
    \item Déterminer les limites de $f$ aux bornes de son intervalle de
      définition.
    \item Calculer $f'(x)$ et étudier son signe.
    \item Dresser le tableau de variation de $f$.
    \item Montrer que l'équation $f(x) = 0$ admet une unique solution
      $\alpha$ dans l'intervalle $]0;+\infty[$.
    \item Donner une valeur approchée de $\alpha$ à $10^{-2}$ près.
  \end{enumerate}
\end{question}

\begin{question}[topic=nombres_complexes]
  Question de cours : montrer les trois affirmations suivantes :
  \begin{itemize}
    \item Affirmation 1 : $z + \overline{z} = 2\Re(z)$
    \item Affirmation 2 : $z - \overline{z} = 2i\Im(z)$
    \item Affirmation 3 : $z \times \overline{z} = \lvert z \rvert^2$
  \end{itemize}
\end{question}

\begin{question}[topic=fonction]
  Soit la fonction $f$ définie sur $[-1;1]$ par $f(x) = (1-x^2)e^x$.

  \begin{enumerate}
    \item Calculer la dérivée de $f$ et en déduire les variations de la
      fonctions $f$. Préciser les valeurs $f(-1)$ et $f(1)$.
    \item Donner la valeur de $x_{\max}$ pour laquelle le maximum est
      atteint et donner $f(x_{\max})$.
    \item On donne $f''(x) = \left(-x^2-4x-1\right)e^{x}$. Montrer que
      $f(x) - 2f'(x) + f''(x) = - 2e^x$.

      En déduire $\int_{-1}^1 f(x) \mathrm\,x$.
    \item Interpréter géométriquement ce résultat.
  \end{enumerate}
\end{question}

\begin{question}[topic=suite]
  On considère la suite $(w_n)$ définie par $w_0=2$ et
  $w_n=\dfrac15w_{n-1}+\dfrac12$ pour tout entier $n\geqslant1$.

  Montrer que $w_n=\dfrac{11}{8}\left(\dfrac{1}{5}\right)^n+\dfrac{5}{8}$
  pour tout $n\geqslant0$.
\end{question}

\begin{question}[topic=suite]
  On considère la suite définie par $u_1=8$ et
  ${u_{n+1}=\dfrac{1}{2}u_n+2n-3}$ pour tout $n\in\N^*$.
  \begin{enumerate}
    \item Montrer que $u_n\geqslant n$ pour tout $n\geqslant 4$.
    \item En déduire $\displaystyle \lim_{n\rightarrow +\infty} u_n$.
  \end{enumerate}
\end{question}

\begin{question}[topic=fonction]
  La fonction $f$ définie sur $\mathbb{R}$ par :
  \[f(x)=\left\{\begin{array}{@{}ll}
        2+\sqrt{5} & \text{si~} x\leqslant0 \\
        \sqrt{9+4\sqrt{5}} & \text{si~} x>0 \\
  \end{array}\right..\]
  est-elle continue en 0 ?
\end{question}

\begin{question}[topic=fonction]
  \begin{enumerate}
    \item Soit $n\in\mathbb{N}^*$  et $f$ la fonction définie sur
      $\mathbb{R}$ par : \[f(x)=(1+x)^n-1-nx.\]
      Établir le sens de variation  de $f$ sur $[-1~;~+\infty[$.
    \item
      \begin{enumerate}
        \item Établir \textbf{l'inégalité de Bernoulli} :
          \[(1+x)^n\geqslant 1+nx\]
          pour tout $n\in\mathbb{N}^*$ et tout $x\in[-1~;~+\infty[$.
        \item Pour quelle(s) valeur(s) de $x$ a-t-on l'égalité ?
      \end{enumerate}
  \end{enumerate}
\end{question}

\begin{question}[topic=fonction]
  Soit la fonction $f$ définie sur $[0~;~+\infty[$ par :
  \[f(x)=\dfrac{2x-\sqrt{x}}{2+\sqrt{x}}.\]
  \begin{enumerate}
    \item Montrer que $f$ est dérivable sur $]0~;~+\infty[$ et que :
      \[f'(x)=\dfrac{x+4\sqrt{x}-1}{\sqrt{x}\left(2+\sqrt{x}\right)^2}.\]
    \item Résoudre l'équation $X^2+4X-1=0$.\par
      En déduire le signe de $f'(x)$.
    \item Dresser le tableau de variation complet de $f$.
  \end{enumerate}
\end{question}

\begin{question}[topic=exponetielle]
  Soit $f$ la fonction définie pour tout $x\in\left[0~;~1\right]$ par :
  \[f(x)=2x-2\e^{-x}+\e^{-1}.\]
  \begin{enumerate}
    \item Dresser le tableau de variation de $f$ sur $\left[0~;~1\right]$.
    \item Démontrer que la fonction $f$ s'annule une fois et une seule sur
      l'intervalle $\left[0~;~1\right]$ en un réel $\alpha$.\par
      Donner la valeur de $\alpha$ arrondie au centième.
  \end{enumerate}
\end{question}

\begin{question}[topic=logarithme]
  On considère la fonction $f$ définie sur $ \left] 0 \, ;+\infty \right[ $
  par $f(x)=x \ln x$.
  \begin{enumerate}
    \item Étudier les limites de $f$ en 0 et en $+\infty$.
    \item Pour tout réel $x>0$, calculer $f^{\prime}(x)$.
    \item Étudier le signe de  $f^{\prime}(x)$ et en déduire les variations
      de $f$.
    \item En déduire que $f$ admet un minimum sur $\left] 0 \,
      ;+\infty\right[ $ que l'on précisera.
\end{enumerate}
\end{question}

\begin{question}[topic=logarithme]
  En 2015, la population d'une ville compte \\250 000 habitants. Chaque
  année, cette population diminue de $2 \, \%$. À partir de quelle année la
  population passera-t-elle au-dessous de 100 000 habitants?
\end{question}

\begin{question}[topic=logarithme]
  Déterminer le plus petit entier naturel $n$ tel que :
  \[ 1+5+5^2+...+5^n \geqslant 10^9. \]
\end{question}

\begin{question}[topic=integration]
  On souhaite calculer l'intégrale suivante :
  \[ I = \int_0^1 \dfrac{x}{x+1}\,\mathrm{d}x. \]
  \begin{enumerate}
    \item Expliquer pourquoi $f : x \mapsto \dfrac{x}{x+1}$ ne correspond à
      aucune forme de dérivée connue.
    \item En remarquant que $x = x + 1 - 1$ , démontrer que pour tout $x
      \neq -1$, $f(x)$ peut s'écrire sous la forme \[f(x) = \alpha+
      \dfrac{\beta}{x+1}\] où $\alpha$ et $\beta$ sont deux réels à
      déterminer.
    \item En déduire que $I = 1 - \ln(2)$.
  \end{enumerate}
\end{question}

\begin{question}[topic=complexes]
  On pourra utiliser comme prérequis que pour tous nombres complexes $z_1$
  et $z_2$ on a : $$\overline{z_1\times z_2}=\overline{z_1}\times
  \overline{z_2}.$$
  \begin{enumerate}
    \item Démontrer par récurrence que pour tout entier naturel $n\geqslant
      1$ on a $\overline{z^n}=\overline{z}^n$
    \item En déduire que pour tout entier naturel $n\geqslant 1$ et pour
      tout nombre complexe $z$ : $z^n+\overline{z}^n$ est un nombre réel.
    \item Démontrer que pout tout entier naturel $n$ le complexe
      $(2-\ci)^{2n}+(3-4\ci)^n$ est un nombre réel.
  \end{enumerate}
\end{question}

\begin{question}[topic=geometrie]
  On considère le cube suivant, d'arête 1, où $I$ et $J$ sont les milieux
  respectifs de $[EH]$ et $[FG]$. Soit $M$ un point appartenant à $[IJ]$.

  \begin{center}
    \def \dx {-0.4}
    \def \r {0.3}
    \begin{tikzpicture}[general, scale=2]
      %points
      \coordinate (A) at (0,0); \draw (A) node[above left] {$A$};
      \coordinate (B) at (\dx,-0.5); \draw (B) node[below left] {$B$};
      \coordinate (C) at (1+\dx,-0.5); \draw (C) node[below] {$C$};
      \coordinate (D) at (1,0); \draw (D) node[right] {$D$};
      \coordinate (E) at (0,1); \draw (E) node[above left] {$E$};
      \coordinate (F) at (\dx,0.5); \draw (F) node[left] {$F$};
      \coordinate (G) at (1+\dx,0.5); \draw (G) node[right] {$G$};
      \coordinate (H) at (1,1); \draw (H) node[above right] {$H$};
      \coordinate (I) at (0.5,1); \draw (I) node[above right] {$I$}; \draw (I) node {$\times$};
      \coordinate (J) at (0.5+\dx,0.5); \draw (J) node[above] {$J$}; \draw (J) node {$\times$};
      \coordinate (M) at (0.5+\dx/3,5/6); \draw (M) node[right] {$M$};
      %arêtes
      \draw (F)--(B)--(C)--(D)--(H)--(G)--(C); \draw (G)--(F)--(E)--(H);
      \draw[dashed] (A)--(B); \draw[dashed] (A)--(D); \draw[dashed] (A)--(E);
      %IJ, MB et MC
      \draw[thin] (I)--(J);
      \draw[color=B1] (M)--(B); \draw[color=B1] (M)--(C);
    \end{tikzpicture}
  \end{center}

  \begin{enumerate}
    \item \begin{enumerate}
      \item Démontrer que pour $M \neq J$, les triangles $MJB$ et $MJC$ sont rectangles en $J$.
      \item En déduire que $MB = MC$.
      \item En déduire que $\vv{MB} \cdot \vv{MC} = MB^2 - \dfrac{1}{2}$.
    \end{enumerate}

  \item Déterminer la ou les positions du point $M$ pour que $\vv{MB} \cdot \vv{MC}=1$.
\end{enumerate}
\end{question}

\begin{question}[topic=géométrie]
  Soit $SABC$ un tétraèdre de tel que les triangles $ABC$ et $SBC$ soient
  isocèles respectivement en $A$ et $S$ :
  \begin{center}
    %\includegraphics[scale=0.3]{./images/tetra_triangles_isoceles.eps}
    \begin{tikzpicture}[general, scale=3]
      %points
      \coordinate (A) at (0,0); \draw (A) node[below  left] {$A$};
      \coordinate (C) at (1,0); \draw (C) node[below right] {$C$};
      \coordinate (B) at (0.4,0.4); \draw (B) node[right] {$B$};
      \coordinate (S) at (0.6,1.17); \draw (S) node[above] {$S$};
      %segments
      \draw[color=D1] (C)--(A)--(B);
      \draw[color=J1] (B)--(S)--(C);
      \draw[color=B1, ->, thick] (B)--(C);
      \draw[color=B1, ->, thick] (A)--(S);
      %codage
      \draw[color=D1] (0.5,0) node {{\boldmath $\bullet$}};
      \draw[color=D1] (0.2,0.2) node {{\boldmath $\bullet$}};
      \draw[color=J1] (0.5,0.74) node {{\boldmath $\approx$}};
      \draw[color=J1] (0.8,0.59) node {{\boldmath $\approx$}};
    \end{tikzpicture}
  \end{center}

  Démontrer que $\vv{AS} \cdot \vv{BC} = 0$.
\end{question}

\begin{question}[topic=géométrie]
  On considère un cube $ABCDEFGH$ de côté 1 et de centre $O$.
  \begin{center}
    \def \dx {-0.4}
    \def \r {0.3}
    \begin{tikzpicture}[general, scale=2]
      %points
      \coordinate (A) at (0,0); \draw (A) node[above left] {$A$};
      \coordinate (B) at (\dx,-0.5); \draw (B) node[below left] {$B$};
      \coordinate (C) at (1+\dx,-0.5); \draw (C) node[below] {$C$};
      \coordinate (D) at (1,0); \draw (D) node[right] {$D$};
      \coordinate (E) at (0,1); \draw (E) node[above left] {$E$};
      \coordinate (F) at (\dx,0.5); \draw (F) node[left] {$F$};
      \coordinate (G) at (1+\dx,0.5); \draw (G) node[right] {$G$};
      \coordinate (H) at (1,1); \draw (H) node[above right] {$H$};
      \coordinate (O) at ({(1+\dx)/2},0.25); \draw (O) node[above] {$O$};
      %arêtes
      \draw (F)--(B)--(C)--(D)--(H)--(G)--(C); \draw (G)--(F)--(E)--(H);
      \draw[dashed] (A)--(B); \draw[dashed] (A)--(D); \draw[dashed] (A)--(E);
      %(BH) et (EC)
      \draw[thin, color=C1, dashed] (B)--(H); \draw[thin, color=C1, dashed] (E)--(C);
      %angle
      \fill[color=A1,opacity=0.3,domain=228:291] (O)
        --plot ({(1+\dx)/2+\r*cos(\x)}, {0.25+\r*sin(\x)})
        --cycle;
      \draw[color=A1]({(1+\dx)/2},-0.15) node {{\boldmath $\alpha$}};
    \end{tikzpicture}
  \end{center}
  Calculer une valeur approchée de la mesure de l'angle $\alpha =
  \widehat{BOC}$ au degré près.
\end{question}

\begin{question}[topic=geometrie]
  Dans l'espace muni d'un repère orthonormé, soient $\vv{n} \begin{pmatrix}
  \alpha \\ \beta \\ \gamma \end{pmatrix}$ un vecteur non nul et $A(x_A\,;
  y_A\,; z_A)$ un point.\\
  Démontrer qu'une équation cartésienne du plan $(\mathcal{P})$, admettant
  $\vv{n}$ pour vecteur normal et passant par $A$ est de la forme $ax + by +
  cz + d = 0$. On donnera $a$, $b$, $c$ et $d$ en fonction des coordonnées
  de $\vv{n}$ et de $A$.
\end{question}

\begin{question}[topic=geometrie]
Dans l'espace muni d'un repère orthonormé $(O\,; \vv{i}, \vv{j}, \vv{k})$,
déterminer une équation cartésienne du plan $(\mathcal{P})$ passant par
$A(-1\,; 2\,; -1)$ et de vecteur normal $\vv{n}\begin{pmatrix} 1 \\ 2 \\ -3
\end{pmatrix}$.
\end{question}

\begin{question}[topic=probabilités]
  Dans un immeuble, on donne la répartition des appartements suivant :
   \begin{itemize}
     \item que ce soit un studio ou non ;
     \item qu'il soit occupé par une seule personne ou bien par plusieurs personnes.
   \end{itemize}
     \begin{center}\small
  \begin{tabularx}{0.95\linewidth}{4}\hline
  & Studio & Pas studio & Total \\\hline
   Seule & {8} & & {15} \\\hline
   Plusieurs  & {2} & & {7}   \\\hline
   Total & {10} & {12} & {22} \\\hline
  \end{tabularx}
  \end{center}
  \vspace{-0.75\baselineskip}\begin{enumerate}
    \item Déterminer les valeurs manquantes dans le tableau.
    \item Quand on choisit un appartement au hasard dans l'immeuble, on appelle $S$ l'évènement \og l'appartement est un studio \fg{} et $PL$ l'évènement \og l'appartement est occupé par plusieurs personnes \fg{}.
        \begin{enumerate}
          \item Calculer $P(S)$, $P_{\overline{S}}(PL)$ et $P_{PL}(S)$.
          \item Les évènements $S$ et $PL$ sont-ils indépendants ?
        \end{enumerate}
  \end{enumerate}
\end{question}

\begin{question}[topic=probabilités]
  Après les contrôles de mathématiques, 60\% du temps, Issa dit \og Je
  suis sûr que j'ai loupé \fg.

  Ses amis sont pourtant formels : \og Quand il dit ça, il a quand même 15
  ou plus les 3/4 du temps.
  Et quand il ne dit rien, on peut être sûr à 95\% qu'il va avoir 15
  ou plus. \fg

  Après un devoir de mathématiques, on considère les évènements :
  \begin{itemize}
    \item $L$ : \og Issa dit qu'il a manqué le devoir \fg ;
    \item $B$ : \og Issa a 15 ou plus au devoir \fg.
  \end{itemize}
  \vspace{-1\baselineskip}
  \begin{enumerate}
    \item Représenter la situation par un arbre de probabilité :
    \item Calculer $P(L\cap B)$ et interpréter cette probabilité dans les
      termes de l'énoncé.
    \item Calculer la probabilité qu'il ne dise rien et qu'il ait moins de 15.
  \end{enumerate}
\end{question}

\begin{question}[topic=loi_continue]
  La durée de vie, en années, d'un atome radioactif peut être modélisée par
  une variable aléatoire $D$ suivant une loi exponentielle de paramètre
  $\lambda$.

  On appelle demi-vie de cet élément le nombre réel $T$ tel que la
  probabilité que cet atome se désintègre avant $T$ années soit égale à
  \nombre{0,5}.

  Ainsi, la demi-vie du carbone 14 est \nombre{5730} ans.
  \begin{enumerate}
    \item Calculer le paramètre $\lambda$ dans le cas du carbone 14.
    \item Calculer la probabilité qu'un atome de carbone 14 se désintègre:
      \begin{colenumerate}{2}
      \item avant \nombre{1000} ans ;
      \item après \nombre{10000} ans.
      \end{colenumerate}
    \item Déterminer la valeur de $a$ telle que $P(D<a)=\nombre{0,95}$ pour
      le carbone 14.

      Interpréter ce résultat dans le contexte de l'exercice.
  \end{enumerate}
\end{question}
