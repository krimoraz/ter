\documentclass[12pt,a4paper,french]{article}

\usepackage[utf8]{inputenc}
\usepackage[T1]{fontenc}
\usepackage{lmodern}
\usepackage{fourier}

\usepackage{amsmath}

\usepackage{babel}

\author{}
\title{Sujet \no{5}}
\date{}

\setlength{\parindent}{0pt}

\begin{document}

\maketitle
\begin{center}Ne rien inscrire sur le sujet.

  Le sujet est à rendre en fin d'exercice.
\end{center}

\bigskip

\textbf{Exercice 1}

On considère la droite $\Delta$ de représentation paramétrique \[ \Delta
  : \left\lbrace \begin{array}{lr} x = 1 - 4t & \\ y = 3 + t & t\in
\mathbf{R} \\ z = 1 - t & \end{array}\right. \]

\begin{enumerate}
  \item Donner un point et un vecteur directeur de $\Delta$.
  \item Le point $M (-3;4;1)$ appartient-il à la droite $\Delta$.
  \item Proposer une autre représentation paramétrique de cette droite.
\end{enumerate}

\medskip

\textbf{Exercice 2}

En rentrant de soirée, Léo sait qu'il va arriver dans la station de
métro la plus proche entre 1h et 1h30. On fait l'hypothèse que son heure
d'arrivée suit une loi uniforme. Depuis minuit, il passe un métro toutes
les 20 minutes dans cette station. Quelle est la probabilité que Léo
attente :
\begin{enumerate}
  \item moins de 5 minutes ;
  \item plus d'un quart d'heure.
\end{enumerate}

On rappelle que :
\begin{itemize}
  \item pour une loi à densité $f$ continue, $P(X \leq x) =
    \int_{-\infty}^x f(t) \mathrm{d}\,t$ ;
  \item la densité de probabilité de la loi uniforme entre $a$ et $b$
    est $f : \left\lbrace \begin{array}{lr} \frac1{b-a} & x \in ]a;b[
    \\ 0 & \text{sinon}\end{array}\right.$
\end{itemize}

\end{document}
