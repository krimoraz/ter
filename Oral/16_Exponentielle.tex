\documentclass[12pt,a4paper,french]{article}

\usepackage[utf8]{inputenc}
\usepackage[T1]{fontenc}
\usepackage{lmodern}
\usepackage{fourier}

\usepackage{amsmath,amsfonts,amssymb}

\usepackage{babel}

\author{}
\title{Sujet \no{16}}
\date{}

\setlength{\parindent}{0pt}

\begin{document}

\maketitle
\begin{center}Ne rien inscrire sur le sujet.

  Le sujet est à rendre en fin d'exercice.
\end{center}

\bigskip

\textbf{Exercice 1}

Soit la fonction $f$ définie sur $[-1;1]$ par $f(x) = (1-x^2)e^x$.

\begin{enumerate}
  \item Calculer la dérivée de $f$ et en déduire les variations de la
    fonctions $f$. Préciser les valeurs $f(-1)$ et $f(1)$.
  \item Donner la valeur de $x_{\max}$ pour laquelle le maximum est
    atteint et donner $f(x_{\max})$.
  \item On donne $f''(x) = \left(-x^2-4x-1\right)e^{x}$. Montrer que
    $f(x) - 2f'(x) + f''(x) = - 2e^x$.

    En déduire $\int_{-1}^1 f(x) \mathrm\,x$.
  \item Interpréter géométriquement ce résultat.
\end{enumerate}

\medskip

\textbf{Exercice 2}

Question de cours : montrer les trois affirmations suivantes :
\begin{itemize}
  \item Affirmation 1 : $z + \overline{z} = 2\Re(z)$
  \item Affirmation 2 : $z - \overline{z} = 2i\Im(z)$
  \item Affirmation 3 : $z \times \overline{z} = \lvert z \rvert^2$
\end{itemize}

\end{document}
