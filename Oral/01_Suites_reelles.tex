\documentclass[12pt,a4paper,french]{article}

\usepackage[utf8]{inputenc}
\usepackage[T1]{fontenc}
\usepackage{lmodern}
\usepackage{fourier}

\usepackage{babel}

\author{}
\title{Sujet \no{1}}
\date{}

\setlength{\parindent}{0pt}

\begin{document}

\maketitle
\begin{center}Ne rien inscrire sur le sujet.

  Le sujet est à rendre en fin d'exercice.
\end{center}

\bigskip

\textbf{Exercice 1}

On donne la suite $(u_n)$ définie par $\forall n\in \mathbf{N},
u_{n+1} = \frac{1}{\sqrt{2}}u_{n} + 1$ et $u_0 = 0$.
\begin{enumerate}
  \item La suite $(u_n)$ est ici définie par $u_{n+1} = f(u_n)$. Quelle
    condition respecte la fonction $f$ ici pour pouvoir dire si $u_n$
    converge vers $\ell$, alors $f(\ell) = \ell$ ?
  \item Calculer la valeur de $\ell$.
  \item Soit la suite $(v_n)$ définie par $\forall n\in \mathbf{N}, v_n
    = u_n - \frac{2}{2 - \sqrt{2}}$.
    \begin{enumerate}
      \item Montrer que $(v_n)$ est décroissante.
      \item En admettant que la suite $(v_n)$ est minorée par 0, que
        peut-on dire de la convergence de cette suite ?
      \item En déduire la limite de la suite $(u_n)$.
    \end{enumerate}
\end{enumerate}

\medskip

\textbf{Exercice 2}

On considère un sac opaque contenant trois billes rouges et deux billes
bleues indiscernables au toucher.

On effectue un tirage sans remise.

\begin{enumerate}
  \item Quelle est la probabilité de tirer une bille bleue, puis une
    bille rouge ?
  \item Quelle est la probabilité de tirer une bille rouge puis une
    bille bleue.
  \item Les événements «tirer une bille bleue puis une bille rouge» et
    «tirer une bille rouge puis une bille bleue» sont ils indépendants.
\end{enumerate}

\end{document}
