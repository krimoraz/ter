\documentclass[12pt,a4paper,french]{article}

\usepackage[utf8]{inputenc}
\usepackage[T1]{fontenc}
\usepackage{lmodern}
\usepackage{fourier}

\usepackage{babel}

\author{}
\title{Sujet \no{9}}
\date{}

\setlength{\parindent}{0pt}

\begin{document}

\maketitle
\begin{center}Ne rien inscrire sur le sujet.

  Le sujet est à rendre en fin d'exercice.
\end{center}

\bigskip

\textbf{Exercice 1}

Discuter (et démontrer évenutellement) pour des valeurs de $n$ entières
l'inégalité $5^n \geqslant 4^n + 3^n$.

\medskip

\textbf{Exercice 2}

Question de cours : Démontrer que la loi exponentielle (définie par la
densité de probabilité $f(t) = \lambda e^{-\lambda t}$) définie bien une
loi sans mémoire, c'est-à-dire que pour $t$ et $s$ deux réels positifs,
$P_{T>s}(T > s+t) = P(T > t)$.

\end{document}
