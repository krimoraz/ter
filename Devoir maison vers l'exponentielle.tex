\documentclass[11pt,a4paper,french]{article}
\usepackage[utf8]{inputenc}
\usepackage[T1]{fontenc}
\usepackage{babel}
\usepackage[thmmarks]{ntheorem}
\usepackage{amsmath}
\usepackage{amsfonts}
\usepackage{amssymb}

\usepackage{array}

\usepackage{kpfonts}

\usepackage[bookmarks=false,colorlinks,linkcolor=blue]{hyperref}

\pdfminorversion 7
\pdfobjcompresslevel 3

\usepackage{tabularx}
\usepackage[autolanguage,np]{numprint}
\usepackage{enumitem}

\usepackage{tipfr}
\usepackage{pgf}
\usepackage{tikz}
\usepackage{tkz-base}
\usepackage{tkz-euclide}
\usetkzobj{all}
\usetikzlibrary{hobby}
\usepackage{tkz-tab}

\usepackage[top=1.7cm,bottom=2cm,left=2cm,right=2cm]{geometry}

\usepackage{lastpage}

\usepackage{esvect}
\usepackage{marginnote}

\usepackage{wrapfig}

\usepackage[defaultlines=5,all]{nowidow}


\makeatletter
\renewcommand{\@evenfoot}%
        {\hfil \upshape \small page {\thepage} de \pageref{LastPage}}
\renewcommand{\@oddfoot}{\@evenfoot}

\renewcommand{\maketitle}%
{\framebox{%
    \begin{minipage}{1.0\linewidth}%
      \begin{center}%
        \Large \@title ~-- \@author \\%
        \@date%
      \end{center}%
    \end{minipage}}%
  \normalsize%
  %\vspace{1cm}%
}

%Des macros pour les noms d'ensmbles
\newcommand{\R}{\mathbf{R}}
\newcommand{\Q}{\mathbf{Q}}
\newcommand{\Z}{\mathbf{Z}}
\newcommand{\C}{\mathbf{C}}
\newcommand{\N}{\mathbf{N}}

\newcommand{\Cf}{\mathscr{C}}

\newcommand{\norme}[1]{\left\lVert #1 \right\rVert}
\newcommand{\abs}[1]{\left\lvert #1 \right\rvert}
\newcommand{\diff}{\mathop{}\mathopen{}\mathrm{d}}

%Une macro récursive pour l'intérieru des vecteurs
%http://tex.stackexchange.com/questions/19693/arguments-of-custom-commands-as-comma-separated-list

\newcommand\vecteur[2][\\]{%
    \global\def\my@delim{#1}%
    \left(\negthinspace\begin{matrix}
        \my@vector #2,\relax\noexpand\@eolst%
    \end{matrix}\right)}

%Une macro pour les vecteurs
\def\my@vector #1,#2\@eolst{%
   \ifx\relax#2\relax
      #1
   \else
      #1\my@delim
      \my@vector #2\@eolst%
   \fi}

%Une macro récursive pour mettre formater l'intérieur des intervalles
\def\my@intervalle #1;#2\@eolst{%
  \ifx\relax#2\relax
    #1
  \else
    \my@intervalle #2\@eolst%
  \fi}

%Quatre macros pour les quatres types d'intervalles
\newcommand{\interff}[1]{%
  \left[\my@intervalle #1;\relax\noexpand\@eolst%
  \right]
}
\newcommand{\interfo}[1]{%
  \left[\my@intervalle #1;\relax\noexpand\@eolst%
  \right[}
\newcommand{\interof}[1]{%
  \left]\my@intervalle #1;\relax\noexpand\@eolst%
  \right]}
\newcommand{\interoo}[1]{%
  \left]\my@intervalle #1;\relax\noexpand\@eolst%
  \right[}

\makeatother


\usepackage{mdframed}

\theoremstyle{break}
\newtheorem{definition}{Définition}
\newtheorem{propriete}{Propriété}
\newtheorem{proposition}{Proposition}
\newtheorem{corollaire}{Corollaire}
\newtheorem{propdef}{Propriété - Définition}
\newtheorem{theoreme}{Théorème}
\theoremstyle{plain}
\theorembodyfont{\normalfont}
\newtheorem{exerciceT}{Exercice}
\theoremstyle{nonumberplain}
\newtheorem{remarque}{Remarque}
\newtheorem{notation}{Notation}
\newtheorem{probleme}{Problème}
\theoremsymbol{\ensuremath{\blacksquare}}
\newtheorem{preuve}{Preuve}
\theoremsymbol{}
\theoremstyle{nonumberbreak}
\newtheorem{exemple}{Exemple}


\setlength{\parsep}{0pt}
\setlength{\parskip}{5pt}
%\setlength{\parindent}{0pt}
\setlength{\itemsep}{7pt}

\setlist{noitemsep}
%\setlist[1]{\labelindent=\parindent} % < Usually a good idea
\setlist[itemize]{leftmargin=*}
\setlist[itemize,1]{label=$\triangleright$}
\setlist[enumerate]{labelsep=*, leftmargin=1.5pc}
\setlist[enumerate,1]{label=\arabic*., ref=\arabic*}
\setlist[enumerate,2]{label=\emph{\alph*}),
ref=\theenumi.\emph{\alph*}}
\setlist[enumerate,3]{label=\roman*), ref=\theenumii.\roman*}
\setlist[description]{font=\sffamily\bfseries}

\usepackage{multicol}
\setlength{\columnseprule}{0pt}

\usepackage[]{exsheets}
\SetupExSheets{
  headings = block,
  question/pre-hook = \mdframed,
  question/post-hook = \endmdframed,
}


\everymath{\displaystyle\everymath{}}

\title{Devoir maison \no 5 : Exponentielle}
\author{\bsc{Jumel}}
\date{pour le 20 avril 2017}

\begin{document}

\noindent\maketitle

\bigskip

Le but de ce devoir est de montrer que l'exponentielle est une fonction
«naturelle« qui émerge dans des cas assez simples. En particulier, on
justifiera ainsi l'existence d'une fonction $f$ définie sur $\R$ telle
que $f' = f$.

On pose $f_n \colon z \mapsto \left( 1 + \frac{z}n \right)^n$. La
dérivée de cette fonction est $f_n'(z) = \frac{1}n n \left( 1 + \frac{z}n
\right)^{n-1} = f_{n-1}(z)$.

Par passage à la limite, on a $\lim_{n\to+\infty} f'_n (z) =
\lim_{n\to+\infty} f_{n-1} (z)$

On a donc, une convergence ponctuelle.

La suite de fonction $(f'_n)$ possède la même limite que la suite
$(f_n)$. Il faut encore démontrer que la convergence est normale.

Cette notion est hors de portée d'élèves de TS pour l'instant.


Cependant, on peut évaluer la suite de fonction en 0. Il est facile de
vérifier que pour tout $n \geqslant 0, f_n(0) = 1$.

On a ainsi une suite de fonctions dont la limite est telle que $f' = f$
et $f(0) = 1$.

% http://paramanands.blogspot.com/2014/05/theories-of-exponential-and-logarithmic-functions-part-2_10.html

% https://math.stackexchange.com/questions/879742/is-this-a-valid-proof-of-lim-n-rightarrow-infty-1-fracznn-ez?noredirect=1&lq=1

%https://math.stackexchange.com/questions/365029/intuitive-proofs-that-lim-limits-n-to-infty-left1-frac-xn-rightn-ex?noredirect=1&lq=1

%http://www.les-mathematiques.net/phorum/read.php?4,831190,831232

%https://math.stackexchange.com/questions/882741/limit-of-1-x-nn-when-n-tends-to-infinity

%https://math.stackexchange.com/questions/358830/about-lim-left1-frac-xn-rightn

%http://mathsfg.net.free.fr/terminale/TS2011/exp/exponentiellescoursv2TS.pdf
%https://www.math.u-psud.fr/~perrin/CAPES/analyse/fonctions/definition-exponentielle.pdf
%http://www.les-mathematiques.net/phorum/file.php?4,file=19894,filename=m04c31ea.pdf
%(capes externe 2004)

\end{document}
