\documentclass[12pt,french,a4paper]{scrartcl}

\usepackage[utf8]{inputenc}
\usepackage[T1]{fontenc}
\usepackage{lmodern}
\usepackage{kpfonts}

\begin{document}

On veut montrer que le couple résultat d'une division euclidienne est
unique.

On suppose donc qu'il existe $(q,r)$ et $(q',r')$ tels que \[ a = bq +r
\] et \[a = bq' + r'\] avec $0 \leqslant r \leq b$ et $0 \leqslant r'
\leq b$

En égalant les deux expressions, on obtient $bq + r = bq' + r'$ soit
encore en isolant les termes $b(q -q') = r' - r$. \hfill $(1)$

Supposons $q > q'$. Ainsi $q - q' > 0$ donc $q - q' \geqslant 1$.

Or $r' -r < b$ donc la condition $q \neq q'$ n'est clairement pas
possible.

$(1) \ \text{et}\ q = q' \implies r = r'$ permet de conclure.
\end{document}
