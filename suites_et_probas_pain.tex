% vim: set ft=tex:
\documentclass[12pt,french,a4paper]{article}

\input{../commons.tex.inc}

\title{Suites et probabilités : évaluation \no{2} }
\date{10 octobre 2017}
\author{}

\begin{document}

\maketitle
\thispagestyle{fancy}

\begin{center}
  \tikz{
    \draw (0,0) rectangle (18,3) ;
    \draw (3,0) rectangle (15,3) ;
    \draw (0,3) node [anchor=north west] {\large /5} ;
    \draw (3,3) node [anchor=north west] { \large Appréciation } ;
    \draw (15,3) node [anchor=north west] { \large Signature } ;
  }

%  Pour certaines questions, plusieurs réponses peuvent être possible.
\end{center}

\begin{question}
  Chaque jour Bill doit décider s'il achète ou non du pain.
  \begin{itemize}
    \item S'il a acheté du pain un jour, la probabilité qu'il en achète le
      lendemain est \np{0.3}.
    \item S'il n'en a pas acheté un jour, la probabilité qu'il en achète le
      lendemain est \np{0.8}.
  \end{itemize}
  Pour tout entier $n \in \N^*$, on appelle $A_n$ l'événement «Bill achète
  du pain le jour $n$ » et on note $p_n = p(A_n)$.

  Aujourd'hui, premier jour, Bill a acheté du pain et $p_1 = 1$.
  \begin{enumerate}
    \item Calculer $p_2$ et $p_3$.
    \item Représenter la situation par un arbre de probabilité sur lequel
      figurent $A_n$, $\overline{A_n}$, $A_{n+1}$ et $\overline{A_{n+1}}$.
    \item Montrer que $p_{n+1} = 0,8 - 0,5 p_n$.
    \item Montrer que $p_n = \frac7{15} \brk*{\frac12}^{n-1} + \frac8{15}$
      pour tout $n \in \N^*$.
    \item \begin{enumerate}
      \item Déterminer $\lim_{n\to+\infty} p_n$.
      \item Interpréter cette limite en termes concrets.
    \end{enumerate}
  \end{enumerate}
\end{question}


\end{document}
