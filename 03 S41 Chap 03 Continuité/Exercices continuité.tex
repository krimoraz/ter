\documentclass[12pt,a4paper,french]{article}
\usepackage[utf8]{inputenc}
\usepackage[T1]{fontenc}
\usepackage{babel}
\usepackage[thmmarks]{ntheorem}
\usepackage{amsmath}
\usepackage{amsfonts}
\usepackage{amssymb}

\usepackage{array}

\usepackage{lmodern}
\usepackage{kpfonts}
\usepackage{cmbright}

\usepackage[top=1.7cm,bottom=2cm,left=2cm,right=2cm]{geometry}

\usepackage{ifthen}
\usepackage{fancyhdr}
\pagestyle{fancy}

\count1=\year \count2=\year
\ifnum\month<8\advance\count1by-1\else\advance\count2by1\fi
\pagestyle{fancy}
\cfoot{\textsl{\footnotesize{Année \number\count1/\number\count2}}}
\lfoot{\textsl{\footnotesize{Lycée \textsc{LaSalle Saint-Denis}}}}
\rfoot{\footnotesize{Page \thepage/ \pageref{LastPage}}}
\rhead{}
%\lhead{\ifthenelse{\value{page}=1}{Nom:\dotfill\hfill Prénom: \dotfill
%\hfill ~}{}}
\renewcommand{\headrulewidth}{0pt}
\renewcommand{\footrulewidth}{0pt}

\pdfminorversion 7
\pdfobjcompresslevel 3

%\frenchbsetup{ItemLabels=\textbullet,}

\usepackage[bookmarks=false,colorlinks,linkcolor=blue,pdfusetitle]{hyperref}

\hypersetup{%
  pdftitle = {Exercices sur la continuité},
  pdfauthor = {Vincent-Xavier Jumel},
  pdfkeywords = {TS, exercices, continuité, TVI},
}

\usepackage{tabularx}
\usepackage[autolanguage,np]{numprint}
\usepackage{enumitem}

\usepackage{tipfr}
\usepackage{pgf}
\usepackage{tikz}
\usepackage{tkz-euclide}
\usetkzobj{all}
%\usetikzlibrary{hobby}
\usepackage{tkz-tab}

\usepackage{lastpage}
\usepackage{esvect}
\usepackage{marginnote}

\usepackage{wrapfig}

\usepackage[defaultlines=5,all]{nowidow}

\usepackage[tikz]{bclogo}


\makeatletter

\renewcommand{\maketitle}%
{\framebox{%
    \begin{minipage}{1.0\linewidth}%
      \begin{center}%
        \Large \@title ~-- \@author \\%
        \@date%
      \end{center}%
    \end{minipage}}%
  \normalsize%
  %\vspace{1cm}%
}

\pgfdeclarepatternformonly{mes_hachures}
{\pgfpoint{-0.1cm}{-0.1cm}}
{\pgfpoint{0.9cm}{0.5cm}}
{\pgfpoint{0.8cm}{0.4cm}}
{\pgfpathmoveto{\pgfpointorigin}
  \pgfpathlineto{\pgfpoint{0.8cm}{0.4cm}}
\pgfusepath{stroke}}

%Des macros pour les noms d'ensembles
\newcommand{\R}{\mathbf{R}}
\newcommand{\Q}{\mathbf{Q}}
\newcommand{\Z}{\mathbf{Z}}
\newcommand{\C}{\mathbf{C}}
\newcommand{\N}{\mathbf{N}}

\newcommand{\norme}[1]{\left\lVert #1 \right\rVert}
\newcommand{\abs}[1]{\left\lvert #1 \right\rvert}

%Une macro récursive pour l'intérieur des vecteurs
%http://tex.stackexchange.com/questions/19693/arguments-of-custom-commands-as-comma-separated-list

\newcommand\vecteur[2][\\]{%
    \global\def\my@delim{#1}%
    \left(\negthinspace\begin{matrix}
        \my@vector #2,\relax\noexpand\@eolst%
    \end{matrix}\right)}

%Une macro pour les vecteurs
\def\my@vector #1,#2\@eolst{%
   \ifx\relax#2\relax
      #1
   \else
      #1\my@delim
      \my@vector #2\@eolst%
   \fi}

%Une macro récursive pour mettre formater l'intérieur des intervalles
\def\my@intervalle #1;#2\@eolst{%
  \ifx\relax#2\relax
    #1
  \else
    \my@intervalle #2\@eolst%
  \fi}

%Quatre macros pour les quatre types d'intervalles
\newcommand{\interff}[1]{%
  \left[\my@intervalle #1;\relax\noexpand\@eolst%
  \right]
}
\newcommand{\interfo}[1]{%
  \left[\my@intervalle #1;\relax\noexpand\@eolst%
  \right[}
\newcommand{\interof}[1]{%
  \left]\my@intervalle #1;\relax\noexpand\@eolst%
  \right]}
\newcommand{\interoo}[1]{%
  \left]\my@intervalle #1;\relax\noexpand\@eolst%
  \right[}

\makeatother


\usepackage{mdframed}

\theoremstyle{break}
\newtheorem{definition}{Définition}
\newtheorem{propriete}{Propriété}
\newtheorem{corollaire}{Corollaire}
\newtheorem{propdef}{Propriété - Définition}
\newtheorem{theoreme}{Théorème}
\theoremstyle{plain}
\theorembodyfont{\normalfont}
\newtheorem{exerciceT}{Exercice}
\theoremstyle{nonumberplain}
\newtheorem{remarque}{Remarque}
\newtheorem{notation}{Notation}
\newtheorem{probleme}{Problème}
\theoremsymbol{\ensuremath{\blacksquare}}
\newtheorem{preuve}{Preuve}
\theoremsymbol{}
\theoremstyle{nonumberbreak}
\newtheorem{exemple}{Exemple}

\newenvironment{exercice}{\begin{framed}\begin{exerciceT}}{\end{exerciceT}\end{framed}}

\setlength{\parsep}{0pt}
\setlength{\parskip}{5pt}
\setlength{\parindent}{0pt}
\setlength{\itemsep}{7pt}

\setlist{noitemsep}
%\setlist[1]{\labelindent=\parindent} % < Usually a good idea
\setlist[itemize]{leftmargin=*}
\setlist[itemize,1]{label=$\triangleright$}
\setlist[enumerate]{labelsep=*, leftmargin=1.5pc}
\setlist[enumerate,1]{label=\arabic*., ref=\arabic*}
\setlist[enumerate,2]{label=\emph{\alph*}),
ref=\theenumi.\emph{\alph*}}
\setlist[enumerate,3]{label=\roman*), ref=\theenumii.\roman*}
\setlist[description]{font=\sffamily\bfseries}

\usepackage{multicol}
\setlength{\columnseprule}{0pt}

\usepackage[]{exsheets}
\SetupExSheets{
  headings = block-subtitle,
%  question/pre-hook = \mdframed,
%  question/post-hook = \endmdframed,
  counter-format = qu[1],
}
\usepackage{exsheets-listings}

\everymath{\displaystyle\everymath{}}

\title{Exercices sur la continuité}
\author{\bsc{Ts 3}}
\date{novembre 2016}

\tikzstyle{inclue}=[draw, circle,fill, inner sep = 1pt]
\tikzstyle{exclue}=[draw, circle, fill = white,inner sep=1pt]

\begin{document}

\maketitle

\bigskip

\begin{question}[subtitle={Pour démarrer},class=3,topic={continuité}]
  Indiquer si les fonctions présentées sont continues. Préciser
  éventuellement leur points de discontinuité et les valeurs des
  fonctions en ces points.

  \begin{multicols}{3}
    \begin{center}
      \begin{tikzpicture}[scale=0.5,>=latex]
        \clip (-5,-5) rectangle (5,5) ;
        \draw [->] (-5,0) -- (5,0) ;
        \draw [->] (0,-5) -- (0,5) ;
        \draw [thin,dotted] (-5,-5) grid (5,5) ;
        \draw [thick] plot [domain=-5:-2] (\x,{(\x+2)^2}) ;
        \draw [thick] plot [domain=-2:2] (\x,\x+2) ;
        \draw [thick] plot [domain=2:5] (\x,0.5*\x+3) ;
      \end{tikzpicture}
    \end{center}

    \begin{center}
      \begin{tikzpicture}[scale=0.5,>=latex]
        \clip (-5,-5) rectangle (5,5) ;
        \draw [->] (-5,0) -- (5,0) ;
        \draw [->] (0,-5) -- (0,5) ;
        \draw [thin,dotted] (-5,-5) grid (5,5) ;
        \foreach \i in {-5,-3,...,3} {
          \begin{scope}[xshift=\i cm]
            \draw [thick] plot [domain=0:2] (\x,\x-1) ;
            \draw (0,-1) node [inclue] {} ;
            \draw (2,1) node [exclue] {} ;
          \end{scope}
        }
      \end{tikzpicture}
    \end{center}

    \begin{center}
      \begin{tikzpicture}[scale=0.5,>=latex]
        \clip (-5,-5) rectangle (5,5) ;
        \draw [->] (-5,0) -- (5,0) ;
        \draw [->] (0,-5) -- (0,5) ;
        \draw [thin,dotted] (-5,-5) grid (5,5) ;
        \draw [thick] plot [domain=-5:-2] (\x,{(\x+2)^2}) node [exclue]
        {} ;
        \draw [thick] plot (-2,-1) node [thick,inclue] {} ;
        \draw [thick] plot [domain=-2:2] (\x,\x+2) node [inclue] {} ;
        \draw [thick] (2,2) node [exclue] {}  -- plot [domain=2:5] (\x,0.5*\x+1) ;
      \end{tikzpicture}
    \end{center}
  \end{multicols}
\end{question}

\begin{question}[subtitle={Avec des expressions},
  topic={continuité}, class=3]
  Pour les fonctions suivantes indiquer le domaine de définition de la
  fonction et si elle présente des discontinuités.
  \begin{multicols}{4}
    \begin{enumerate}
      \item $x\mapsto \sqrt{1 - x^2}$
      \item $x\mapsto \frac{\sin x}x$
      \item $x\mapsto \frac{3x - 1}{2x + 5}$
      \item $x\mapsto \left\lfloor x \right\rfloor$
    \end{enumerate}
  \end{multicols}
\end{question}

\begin{question}[subtitle={Une fonction définie par morceaux},
  topic={continuité},class=2]
  On considère la fonction définie sur $\R$ par \[ f(x) = \left\lbrace
  \begin{matrix}  x^2 - 4x + 5\ \text{si}\ x<2 \\
  -x^2 + 4x -3 \ \text{sinon} \end{matrix} \right. \]
  \begin{enumerate}
    \item La fonction $f$ est-elle continue ?
    \item Calculer $f(1)$.
    \item Tracer la courbe représentative de la fonction $f$ sur la
      calculatrice.
  \end{enumerate}
\end{question}

\begin{lstquestion}[pre={On considère le code Python suivant},
  post={%
    \begin{enumerate}%
      \item Que renvoie ce code pour 1 ; 2 ; 1,99 ; 2,01 ?
      \item Transformer ce code en une fonction Python.
      \item Écrire une fonction définie de la même façon que dans
        l'exercice précédent.
      \item La fonction ainsi définie est-elle continue ?
    \end{enumerate}%
  },
  options={
    subtitle={Algorihtme},
    class=3,topic={continuité},
  },
  listings={language=Python}]
  x = float(input())
  if x >= 2:
      print(x+2)
  else:
      print(x**2)
\end{lstquestion}

\begin{question}[subtitle={Image d'un intervalle}, topic={continuité},
  class=3]
  Donner l'image de l'intervalle $\interff{0,1}$ par les fonctions
  suivantes :
  \begin{enumerate}
    \item $x\mapsto x^2$
    \item $x\mapsto 5\sqrt{\frac{1}{25} - \frac{x^2}{25}}$
    \item $x\mapsto \sin{x}$
  \end{enumerate}
\end{question}

\begin{bclogo}[noborder,logo=\bcoutil]{Précision}%
  Dans un tableau de variation, la flèche continue entre deux valeurs
  traduit à la fois la monotonie et la continuité de la fonction.  Une
  double barre permet de préciser une discontinuité.

  Une étude de fonction comprend :
  \begin{itemize}
    \item l'ensemble de définition ; de continuité; de dérivabilité ;
    \item le calcul des limites aux bornes des intervalles précédents ;
    \item le calcul de la dérivée ;
    \item un tableau contenant l'ensemble de définition, le signe de la
      dérivée et les variations de la fonction.
  \end{itemize}
\end{bclogo}

\begin{question}[subtitle={Étude de fonction},class=3,
  topic={continuité}]
  Étudier la fonction définie pour tout $x$ appartenant à $\R \setminus
  \{1\}$ par $f(x) = \frac{-4x^2 + 1}{x^3 - 1}$
\end{question}

\begin{question}[subtitle={Une autre étude de fonction},
  class=2,topic={continuité}]
  On considère la fonction $f$ définie sur $I = \interff{-4,1}$ par
  $f(x) = x^3 + 6x^2 + 9x + 3$.

  Elle admet le tableau de variation suivant :
  \begin{center}
    \begin{tikzpicture}
      \tkzTabInit[espcl=2]
      {$x$ / 1, $f'(x)$ / 1, $f$ / 2}
      { -4, -3, -1, 1}
      \tkzTabLine{,+,,-,,+,}
      \tkzTabVar{-/-1,+/3,-/-1,+/19}
    \end{tikzpicture}
  \end{center}
  \begin{enumerate}
    \item Justifier que $f$ est continue sur $I$.
    \item Que vaut $f(I)$ ? Quelle est sa nature ? Justifier.
    \item L'équation $f(x) = 2$ possède-t-elle des solutions ? Si oui,
      les dénombrer.
    \item Justifier que l'équation $f(x) = 4$ admet une unique solution
      sur $I$.
    \item Écrire un algorithme qui permet d'encadrer cette solution à
      $10^{-1}$ près.
    \item Que vaut cette solution d'après votre algorithme. Détaillez
      les étapes intermédiaires dans un tableau.
  \end{enumerate}
\end{question}

\begin{question}[subtitle={Résolution d'une équation}, class=3,
  topic={continuité}]
  On considère l'équation de la variable $x$ réelle \[x^3 - 4x^2 + 4x -
  1 = 0 \]
  \begin{enumerate}
    \item Dénombrer les solutions d'une telle équation.
    \item Justifier que 1 est une solution.
    \item Justifier cette équation admet une seule solution sur
      l'intervalle $\interff{2,3}$.
    \item Donner un encadrement au centième de cette solution.
      %les deux autres solutions sont \frac32 ± \frac{\sqrt{5}}2
    \item En admettant que $\frac{3 - \sqrt{5}}2$ est une solution de
      l'équation, déterminer de façon exacte la troisième solution.
  \end{enumerate}
\end{question}

\begin{bclogo}[noborder,logo=\bcoutil]{}%
  On rappelle que si $\alpha$ est une racine d'un polynôme $P = a_0 +
  a_1X + \cdots + a_nX^n$ de degré $n$ ($a_n \neq 0$), alors $P$ s'écrit
  $P = (X-\alpha)Q$ où $Q$ est un polynôme de degré $n -1$ qui
  s'écrit $Q = b_0 + b_1X + \cdots b_{n-1}X^{n-1}$. On a peut exprimer
  les coefficients $a_i$ en fonction de $\alpha$ et des $b_i$ avec des
  relations comme : \[ \left\lbrace \begin{array}{l} a_0 = -b_0\alpha \\
    a_n = b_{n-1} \\ a_i = b_{i-1} - \alpha b_i \ \forall i \in
  \{1,\dots,n-1\} \end{array} \right. \]
\end{bclogo}

\begin{question}[subtitle={Vers le bac : Nouvelle calédonie 2000},
  topic={continuité},class=3]
  \begin{multicols}{2}
    Dans tout ce problème le plan est rapporté à un repère orthonormal
    $(0;\vv{\imath};\vv{\jmath})$ d'unité graphique \np[cm]{2}.

    Soit la fonction numérique $u$ définie sur $\R$ par \[ u(x) =
    \sqrt{x^2 + 1} + x \] et $\mathscr{C}$ sa courbe représentative.
    \begin{enumerate}
      \item \begin{enumerate}
          \item Déterminer la limite en $-\infty$.
          \item Montrer que pour tout $x$ réel : \[ u(x) = \frac1{\sqrt{x^2
            + 1} + x} .\] En déduire la limite de $u$ en $+\infty$.
        \end{enumerate}
      \item \begin{enumerate}
          \item Montrer que $x\mapsto u(x) + 2x$ tend vers 0 quand $x$ tend
            vers $-\infty$.
          \item Montrer que pour tout $x$ réel, on a $u(x) > 0$.

            En déduire le signe de $u(x) + 2x$.
          \item Interpréter graphiquement ces résultats.
        \end{enumerate}
      \item \begin{enumerate}
          \item Montrer que la dérive de la fonction $u$ est la fonction
            $u'$ définie par \[ u'(x) = \frac{-u(x)}{\sqrt{x^2 + 1}} .\]
          \item Étudier les variations de la fonction $u$.
        \end{enumerate}
      \item Tracer la courbe $\mathscr{C}$ et son asymptote oblique.
    \end{enumerate}
  \end{multicols}
\end{question}

\begin{bclogo}[noborder,logo=\bcoutil]{}%
  Après avoir étudié une fonction, on cherche généralement à la
  tracer, ou au moins à donner son allure. Voici quelques conseils.
  \begin{itemize}
    \item On place les bornes (limites) lorsqu'elles sont finies et les
      asymptotes éventuelles.
    \item On place les tangentes horizontales (maxima et minima) aux
      valeurs indiquées dans la tableau de variation.
    \item On peut se donner quelques points remarquables comme les zéros
      de la fonction (les points où elle s'annule.)
    \item On peut aussi compléter avec quelques tangentes en
      particuliers aux bornes lorsqu'elles sont finies et aux zéros.
  \end{itemize}
  Si tout ces éléments sont en place, l'allure de la courbe devrait se
  dégager. Il suffit désormais de la tracer sans lever le crayon si elle
  est continue.
\end{bclogo}

\begin{question}[subtitle={Prolongement par continuité},
  class=3,topic={continuité}]
  \begin{enumerate}
    \item Étudier la continuité de la fonction $f : \R\setminus \{ 0 \}
      \to \R,
      x\mapsto \frac{\sin x}x$.
    \item Tracer la fonction sur la calculatrice et zoomer autour de 0.
      Que se passe-t-il ?

      Formuler une conjecture sur la valeur à donner à $f(0)$ pour
      compléter la fonction de façon continue.
    \item Quelle est la nature de la forme $\lim_{\substack{x\to 0\\x >
      0}} \frac{\sin x}x$
    \item On va chercher à faire un encadrement de la fonction sinus en
      0.
      \begin{enumerate}
        \item Étudier sur $\interff{0,1}$ la fonction $x\mapsto \sin x -
          x$. En particulier, on regardera son sens de variation et son
          maximum sur cet intervalle.
        \item En déduire que $\sin x \leqslant x$.
        \item Proposer une méthode pour montrer que $x - \frac{x^3}6
          \leqslant \sin x$
      \end{enumerate}
    \item Donner un encadrement de $\sin x$. En déduire la limite de $f$
      en 0. La fonction $f$ ainsi complétée est-elle continue ?
    \item Quelle est l'image de $\R$ par $f$ ?
  \end{enumerate}
\end{question}

\begin{bclogo}[noborder,logo=\bcinfo]{Remarque}%
  \begin{itemize}
    \item Pour ce dernier exercice, on peut trouver une démonstration
      géométrique de cette inégalité.
    \item Ce résultat géométrique sera particulièrement utile pour
      démontrer la dérivée de la fonction cosinus.
  \end{itemize}
\end{bclogo}

\end{document}

\pagebreak

\section*{Solutions des exercices}

\vspace{-7mm}

\printsolutions

\end{document}
