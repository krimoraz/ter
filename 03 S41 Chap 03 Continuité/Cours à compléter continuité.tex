\documentclass[12pt,a4paper]{article}
\usepackage[utf8]{inputenc}
\usepackage[T1]{fontenc}
\usepackage[frenchb]{babel}
\usepackage{ntheorem}
\usepackage{amsmath}
\usepackage{amsfonts}

\usepackage{array}

\usepackage{kpfonts}

\usepackage[bookmarks=false,colorlinks,linkcolor=blue,pdfusetitle]{hyperref}

\pdfminorversion 7
\pdfobjcompresslevel 3

\usepackage{tabularx}

\usepackage{pgf}
\usepackage{tikz}
\usepackage{tkz-euclide}
\usetkzobj{all}
\usetikzlibrary{hobby}

\usepackage[top=1.7cm,bottom=2cm,left=2cm,right=2cm]{geometry}

\usepackage{lastpage}

\usepackage{marginnote}

\usepackage{wrapfig}

\usepackage[defaultlines=5,all]{nowidow}

\usepackage[autolanguage]{numprint}
\newcommand{\np}{\numprint}

\makeatletter
\renewcommand{\@evenfoot}%
        {\hfil \upshape \small page {\thepage} de \pageref{LastPage}}
\renewcommand{\@oddfoot}{\@evenfoot}

\renewcommand{\maketitle}%
{\framebox{%
    \begin{minipage}{1.0\linewidth}%
      \begin{center}%
        \Large \@title ~-- \@author \\%
        \@date%
      \end{center}%
    \end{minipage}}%
  \normalsize%
  %\vspace{1cm}%
}

\pgfdeclarepatternformonly{mes_hachures}
{\pgfpoint{-0.1cm}{-0.1cm}}
{\pgfpoint{0.9cm}{0.5cm}}
{\pgfpoint{0.8cm}{0.4cm}}
{\pgfpathmoveto{\pgfpointorigin}
  \pgfpathlineto{\pgfpoint{0.8cm}{0.4cm}}
\pgfusepath{stroke}}

\newcommand{\R}{\mathbf{R}}
\newcommand{\N}{\mathbf{N}}
\newcommand{\Vecteur}{\overrightarrow}
\newcommand{\norme}[1]{\left\lVert #1 \right\rVert}
\newcommand{\abs}[1]{\left\lvert #1 \right\rvert}
\newcommand{\vabs}[1]{\left\lvert #1 \right\rvert}
\newcommand{\inff}[2]{\left[#1~;~#2\right]}
\newcommand{\info}[2]{\left[#1~;~#2\right[}
\newcommand{\inof}[2]{\left]#1~;~#2\right]}
\newcommand{\inoo}[2]{\left]#1~;~#2\right[}

\makeatother


\theoremstyle{break}
\newtheorem{definition}{Définition}
\newtheorem{propriete}{Propriété}
\newtheorem{propdef}{Propriété - Définition}
\newtheorem{theoreme}{Théorème}
\theoremstyle{plain}
\theorembodyfont{\normalfont}
\newtheorem{exercice}{Exercice}
\theoremstyle{nonumberplain}
\newtheorem{remarque}{Remarque}
\newtheorem{probleme}{Problème}
\newtheorem{preuve}{Preuve}
\theoremstyle{nonumberbreak}
\newtheorem{exemple}{Exemple}

% Mise en forme des labels dans les énumérations
\renewcommand{\labelenumi}{\textbf{\theenumi.}}
\renewcommand{\labelenumii}{\textbf{\theenumii)}}
\renewcommand{\theenumi}{\arabic{enumi}}
\renewcommand{\theenumii}{\alph{enumii}}


\renewcommand{\labelitemi}{$\bullet$}

\setlength{\parsep}{0pt}
\setlength{\parskip}{5pt}
\setlength{\parindent}{0pt}
\setlength{\itemsep}{7pt}

\usepackage{multicol}
\setlength{\columnseprule}{0pt}

\everymath{\displaystyle\everymath{}}

\title{Continuité des fonctions}
\author{Terminale S}
\date{octobre 2016}

\begin{document}

\maketitle

Remarque liminaire : dans un certain nombre d'ouvrages de l'enseignement
supérieur, la notion de continuité vient avant la notion de limite. Ces
deux notions sont fondamentales et sont à la base de l'analyse telle
qu'on la pratique de façon courante et sont des notions ancrées dans
notre inconscient.

\section{Continuité sur un intervalle}

\subsection{Continuité en un point}

\framebox{
  \begin{minipage}{0.99\linewidth}
    \begin{definition}
      On dit que $f$, fonction définie sur $I$ est continue en $a\in I$
      si
    \end{definition}
    \vspace{2cm}
  \end{minipage}
}

Si on se souvient de la définition de la limite en un point, on peut
également dire que \[ \forall \varepsilon > 0\, \exists \alpha > 0,
\abs{x - a} \leq \alpha \implies \abs{f(x) - f(a)} \leq \varepsilon .\]
Autrement dit encore, si pour tout voisinage de $f(a)$, il existe un
voisinage de $x$ tel que tous les points de ce voisinage ont une image
dans le voisinage de $f(a)$, alors la fonction est continue en $a$.

\subsection{Continuité en tous les points d'un intervalle}

\framebox{
  \begin{minipage}{0.99\linewidth}
    \begin{definition}
      On dit que $f$, fonction définie sur $I$ est continue sur $I$ si
    \end{definition}
    \vspace{2cm}
  \end{minipage}
}

Attention, toutes les fonctions ne sont pas continues.

\begin{exemple}[Fonction partie entière]~\\

\begin{center}
  \begin{tikzpicture}[>=latex]
    \draw [thin, lightgray] (-1,-1) grid [step=1] (5,5) ;
    \draw [thick,->] (-1,0) -- (5,0) ;
    \draw [thick,->] (0,-1) -- (0,5) ;
    \foreach \x in {-1,...,4} {
      \draw[very thick,red] (\x,\x) node [circle,draw,fill,red,inner
      sep=1.2pt] {} -- (\x + 1,\x) node [circle,draw,fill,red,inner
      sep=1.6pt] {} (\x +1, \x) node [circle,draw,fill,white,inner
      sep=0.8pt] {} ;
      \draw (\x+1,\x) node [rectangle,draw,fill,white,inner sep = 1.6pt,
      anchor=west]
      {} ;
    }
  \end{tikzpicture}
\end{center}
Cette fonction, souvent noté $E$ ou $x\mapsto \lfloor x\rfloor$, associe
à tout nombre $x$ réel l'entier naturel $n$ tel que $n\leq x < n+1$,
comme par exemple $\lfloor\pi\rfloor = 3$.

\end{exemple}

\section{Théorème des valeurs intermédiaires}

\subsection{Énoncé du théorème}

\framebox{
  \begin{minipage}{0.99\linewidth}
    \begin{theoreme}[des valeurs intermédiaires]
      Soit $f$ une fonction définie sur $I = [a;b],\ a < b$.
      Si $f$ est continue sur $I$, alors 
    \end{theoreme}
    \vspace{2cm}
  \end{minipage}
}

\begin{center}
  \begin{tikzpicture}[scale=0.8,
      use Hobby shortcut,
      tangent/.style={%
        in angle={(180+#1)},
        Hobby finish,
        designated Hobby path=next,
        out angle=#1,
      },
    ]
    \draw [dotted, help lines, line width=0.6pt] (-6,-6) grid (6,6) ;
    \draw [dotted, help lines, step=0.5] (-6,-6) grid (6,6) ;
    \draw [very thick, ->] (0,-6) -- (0,6) ;
    \draw [very thick, ->] (-6,0) -- (6,0) ;
    \draw [thick] ([tangent=-45]-4,5) .. (-2.5,1.5) ..
    ([tangent=0]-1,1) .. (0,1.5) .. ([tangent=0]1,2) .. (2,1.5) ..
    ([tangent=0]4,0) ;
    \foreach \x/\y in {-4/5,-2.5/1.5,0/1.5,2/1.5,4/0}
    { \draw (\x,\y) node [circle,fill,inner sep=1pt] {} ; }
    \draw (-4,5) node [above right] {$\mathscr{C}_f$} ;
    \draw[thick] (-4,1.5) node[circle,fill,inner sep=1pt] {} -- (4,1.5)
    node[circle,fill,inner sep=1pt] {} ;
    \draw (4,1.5) node[right] {$k$} ;
  \end{tikzpicture}
\end{center}

\pagebreak
\begin{remarque}~\\[-3mm]
  \begin{itemize}
    \item $f$ est juste continue, pas forcément monotone ;
    \item on dit aussi que l'image d'un intervalle par $f$, notée $f(I)$
      est un intervalle ;
    \item l'existence de solutions d'une équation se ramène à un
      problème de continuité.
  \end{itemize}
\end{remarque}

\subsection{Corollaires du théorème}

Si on suppose de plus que la fonction est strictement monotone, alors on
a l'unicité du réel dans l'intervalle image. On va donc démontrer la
proposition suivante :

\framebox{
  \begin{minipage}{0.99\linewidth}
    \begin{theoreme}
      Soit $f$ une fonction définie sur $I = \inff{a}{b}\, a < b$.

      Si $f$ est continue et strictement croissante alors
      \begin{itemize}
        \item l'image de l'intervalle $\inff{a}{b}$ par $f$ est
          $\inff{f(a)}{f(b)}$ ;
        \item pour tout $k\in \inff{f(a)}{f(b)}$ l'équation $f(x) = k$
          possède une unique solution dans $\inff{a}{b}$ ;
        \item on dit que la fonction $f$ est bijective.
      \end{itemize}
    \end{theoreme}
  \end{minipage}
}

\end{document}
