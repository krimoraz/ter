\documentclass[a4paper,french,12pt]{article}

\input{../commons.tex.inc}

\usepackage[pyfuture=all, autoprint=true, gobble=auto]{pythontex}

\title{Séance 3 : simulation d'une loi de probabilité}
\author{Vincent-Xavier \bsc{Jumel}}
\begin{document}

\maketitle
\thispagestyle{fancy}

\begin{definition}
  On dit qu'une variable aléatoire $X$ suit une loi binomiale lorsqu'elle
  modélise la situation suivante :
  \begin{itemize}
    \item on répéte $n$ fois une même expérience aléatoire ;
    \item la répétition est supposée indépendante ;
    \item cette expérience aléatoire n'a que deux issues possibles (schéma
      de Bernoulli) avec une probabilité $p$.
  \end{itemize}
  On dit alors que $n$ et $p$ sont les paramètres de la loi et on note $X
  \leadsto \mathcal{B}(n,p)$.
\end{definition}

\noindent On considère le code suivant :
\begin{pyblock}
  from random import random
  X = random()
  print(X)
\end{pyblock}

\begin{question}
  \begin{enumerate}
    \item Préciser le résultat de ces trois lignes.
    \item Compléter ces trois lignes pour simuler une loi de Bernoulli de
      paramètre $p = 0,2$.
    \item Simuler, avec une boucle, 10 répétitions de cette loi de
      Bernoulli.
  \end{enumerate}
\end{question}

On désire désormais construire la loi de probabilités. Pour cela, on va
utiliser la structure de liste. Voici quelques éléments sur cette structure.

\begin{pyblock}
  l = [] # définit une liste vide
  l = [1,2,3] # définit une liste à trois éléments
  l[1] # renvoie le 2eme élément de la liste
  len(l) # est la longueur de la liste
  l[1] = 4 # remplace le 2eme élément par 4
  l.append(4) # ajoute 4 à la fin de la liste
  l.pop() # renvoie 4 et retire le dernier élément de la liste
  print(l) # affiche la liste
  l = 10*[0] # initialise une liste à 10 éléments tous égaux à zéro
\end{pyblock}

\begin{question}
  On considère une expérience aléatoire dont la loi de probabilité est la
  loi binomiale de paramètres $n = 25$ et $p = 0,1$. Écrire un programme
  Python qui explicite la loi de probabilité sous forme d'une liste à 26
  éléments.

  Vous rendrez votre réponse sous la forme d'un fichier déposé dans l'espace
  Restitution de la classe initulé «Nom Prénom loi binomiale.py»
\end{question}


\end{document}
