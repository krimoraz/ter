\documentclass[11pt,a4paper,french]{article}
\usepackage[utf8]{inputenc}
\usepackage[T1]{fontenc}
\usepackage{babel}
\usepackage[thmmarks]{ntheorem}
\usepackage{amsmath}
\usepackage{amsfonts}
\usepackage{amssymb}

\usepackage{array}

\usepackage{kpfonts}

\usepackage[bookmarks=false,colorlinks,linkcolor=blue]{hyperref}

\pdfminorversion 7
\pdfobjcompresslevel 3

\usepackage{tabularx}
\usepackage[autolanguage,np]{numprint}
\usepackage{enumitem}

\usepackage{tipfr}
\usepackage{pgf}
\usepackage{tikz}
\usepackage{tkz-base}
\usepackage{tkz-euclide}
\usetkzobj{all}
\usetikzlibrary{hobby}
\usepackage{tkz-tab}

\usepackage[top=1.7cm,bottom=2cm,left=2cm,right=2cm]{geometry}

\usepackage{lastpage}

\usepackage{esvect}
\usepackage{marginnote}

\usepackage{wrapfig}

\usepackage[defaultlines=5,all]{nowidow}


\makeatletter
\renewcommand{\@evenfoot}%
        {\hfil \upshape \small page {\thepage} de \pageref{LastPage}}
\renewcommand{\@oddfoot}{\@evenfoot}

\renewcommand{\maketitle}%
{\framebox{%
    \begin{minipage}{1.0\linewidth}%
      \begin{center}%
        \Large \@title ~-- \@author \\%
        \@date%
      \end{center}%
    \end{minipage}}%
  \normalsize%
  %\vspace{1cm}%
}

%Des macros pour les noms d'ensmbles
\newcommand{\R}{\mathbf{R}}
\newcommand{\Q}{\mathbf{Q}}
\newcommand{\Z}{\mathbf{Z}}
\newcommand{\C}{\mathbf{C}}
\newcommand{\N}{\mathbf{N}}

\newcommand{\Cf}{\mathscr{C}}

\newcommand{\norme}[1]{\left\lVert #1 \right\rVert}
\newcommand{\abs}[1]{\left\lvert #1 \right\rvert}
\newcommand{\diff}[1]{\mathrm{d}\,#1}

%Une macro récursive pour l'intérieru des vecteurs
%http://tex.stackexchange.com/questions/19693/arguments-of-custom-commands-as-comma-separated-list

\newcommand\vecteur[2][\\]{%
    \global\def\my@delim{#1}%
    \left(\negthinspace\begin{matrix}
        \my@vector #2,\relax\noexpand\@eolst%
    \end{matrix}\right)}

%Une macro pour les vecteurs
\def\my@vector #1,#2\@eolst{%
   \ifx\relax#2\relax
      #1
   \else
      #1\my@delim
      \my@vector #2\@eolst%
   \fi}

%Une macro récursive pour mettre formater l'intérieur des intervalles
\def\my@intervalle #1;#2\@eolst{%
  \ifx\relax#2\relax
    #1
  \else
    \my@intervalle #2\@eolst%
  \fi}

%Quatre macros pour les quatres types d'intervalles
\newcommand{\interff}[1]{%
  \left[\my@intervalle #1;\relax\noexpand\@eolst%
  \right]
}
\newcommand{\interfo}[1]{%
  \left[\my@intervalle #1;\relax\noexpand\@eolst%
  \right[}
\newcommand{\interof}[1]{%
  \left]\my@intervalle #1;\relax\noexpand\@eolst%
  \right]}
\newcommand{\interoo}[1]{%
  \left]\my@intervalle #1;\relax\noexpand\@eolst%
  \right[}

\makeatother


\usepackage{framed}

\theoremstyle{break}
\newtheorem{definition}{Définition}
\newtheorem{propriete}{Propriété}
\newtheorem{corollaire}{Corollaire}
\newtheorem{propdef}{Propriété - Définition}
\newtheorem{theoreme}{Théorème}
\newtheorem{lemme}{Lemme}
\theoremstyle{plain}
\theorembodyfont{\normalfont}
\newtheorem{exerciceT}{Exercice}
\theoremstyle{nonumberplain}
\newtheorem{remarque}{Remarque}
\newtheorem{notation}{Notation}
\newtheorem{probleme}{Problème}
\theoremsymbol{\ensuremath{\blacksquare}}
\newtheorem{preuve}{Preuve}
\theoremsymbol{}
\theoremstyle{nonumberbreak}
\newtheorem{exemple}{Exemple}

\newenvironment{exercice}{\begin{framed}\begin{exerciceT}}{\end{exerciceT}\end{framed}}

\setlength{\parsep}{0pt}
\setlength{\parskip}{5pt}
%\setlength{\parindent}{0pt}
\setlength{\itemsep}{7pt}

\setlist{noitemsep}
%\setlist[1]{\labelindent=\parindent} % < Usually a good idea
\setlist[itemize]{leftmargin=*}
\setlist[itemize,1]{label=$\triangleright$}
\setlist[enumerate]{labelsep=*, leftmargin=1.5pc}
\setlist[enumerate,1]{label=\arabic*., ref=\arabic*}
\setlist[enumerate,2]{label=\emph{\alph*}),
ref=\theenumi.\emph{\alph*}}
\setlist[enumerate,3]{label=\roman*), ref=\theenumii.\roman*}
\setlist[description]{font=\sffamily\bfseries}

\usepackage{multicol}
\setlength{\columnseprule}{0pt}

\everymath{\displaystyle\everymath{}}

\title{Nombres complexes}
\author{\bsc{Jumel}}
\date{février 2017}

\begin{document}

\noindent\maketitle

\bigskip

Les titres des sous parties de ce cours peuvent sembler ésotérique à
première vue, mais seront justifiée par la suite.

\section*{Activités préparatoires}

Afin de bien aborder les notions, commençons par travailler sur des
choses connues : l'équation du second degré.

\begin{exercice}~\\
  \textbf{Partie A}

  Résoudre les équations suivantes :
  \begin{itemize}
    \item $x^2 + 2x + 1 = 0$
    \item $x^2 - 3x + 2 = 0$
    \item $x^2 - 3x - 4 = 0$
  \end{itemize}

  \noindent\textbf{Partie B}

  Soient $a$ et $b$ deux réels.
  \begin{enumerate}
    \item Développer $(X - a)(X - b)$
    \item Que représentent $a$ et $b$ pour l'équation $X^2 - (a+b)X +
      ab$ ?
    \item En déduire une méthode permettant de trouver deux nombres
      connaissant leur somme et leur produit.
  \end{enumerate}

   \noindent\textbf{Partie C}

   Développer $(a+b)^3$.
\end{exercice}

On peut en profiter pour revenir sur une méthode de résolution générale
pour des équations de degré 3. Avant de s'attaquer à l'exercice,
étudions une propriété essentielle des polynomes et son application
immédiate.

\begin{propriete}
  Deux polynomes sont égaux si et seulement s'ils ont les mêmes
  coefficients devant les termes de même degré.
\end{propriete}

L'exercice qui vient permet de préciser cette notion.

\begin{exercice}~\\
  On considère le polynome $2 x^{3} - 11 x^{2} + 18 x - 9$ dont on
  cherche les racines, c'est à dire les solutions de l'équation $2 x^{3}
  - 11 x^{2} + 18 x - 9= 0$.
  \begin{enumerate}
    \item Vérifier que 1 est une racine de ce polynome.
    \item \begin{enumerate}
        \item Justifier que 1 est aussi une racine du polynome $(x - 1)
          (ax^2 + bx + c)$
        \item En développant ce polynome et en l'identifiant à $2 x^{3}
          - 11 x^{2} + 18 x - 9$ donner la valeur de $a$, $b$ et $c$.
        \item Résoudre l'équation $2 x^{2} - 9 x + 9 = 0$
        \item Donner l'ensembe des racines.

          Que constate-t-on ?
      \end{enumerate}
    \item Écrire le polynome $2 x^{3} - 11 x^{2} + 18 x - 9$ sous forme
      factorisée.
  \end{enumerate}
\end{exercice}

\begin{exercice}~\\
  \noindent\textbf{Partie A}

  De la même façon qu'on dispose d'une méthode générale pour factoriser
  les polynomes de degré 1 et 2 (Calcul du discriminant dans le deuxième
  cas), on va chercher une méthode générale pour trouver les racines des
  polynomes de degré 3.

  On va, pour simplifier, supposer que notre polynome de degré 3 s'écrit
  $P = X^3 + bX^2 + cX + d$ et on notera $P: X \mapsto P$ la fonction
  polynomiale associée.

  \begin{enumerate}
    \item Justifier que tout polynome de degré 3 possède au moins une
      solution sur $\R$.
    \item On pose $X = X' - \frac{b}3$. Montrer que $P$ s'écrit $P =
      X'^3 + pX' + q$.

      Préciser les valeurs de $p$ et $q$.
    \item Les cas «triviaux»
      \begin{enumerate}
        \item Résoudre l'équation $P(x) = 0$ dans le cas $p = 0$
          Préciser dans ce cas la condition sur $b,\ c$ et $d$.
          Que se passe-t-il si de plus $q=0$ ?
        \item Résoudre l'équation $P(x) = 0$ dans le cas $q = 0$, $p
          \neq 0$.
          Quelle solution «évidente» retrouve-t-on ? Ce résultat
          était-il prévisible ?
          Quel type d'équation doit-on ensuite résoudre ?
      \end{enumerate}
    \item Le cas général

      On va chercher $X'$ sous la forme $X' = u + v$.
      \begin{enumerate}
        \item Substituer $u + v$ à $X'$ et développer.
        \item On impose $uv = -\frac{p}3$. Justifier que c'est possible
          ?
        \item Réécrire l'équation avec cette nouvelle condition.
        \item Quelle condition doit ont avoir sur $q$, $u^3$ et $v^3$.
        \item Justifier que résoudre l'équation se ramène désormais à
          trouver deux cubes ($u^3$ et $v^3$) connaissant leur somme et
          leur produit.
          Préciser le système d'équation qu'ils vérifient.
          Donner une méthode permettant de résoudre un tel système.
        \item Justifier qu'ensuite, si $x_1$ est la solution trouvée,
          alors :
          \begin{itemize}
            \item si $u$ et $v$ sont égaux, $x_1$ est une racine double
              et en déduire la dernière solution ;
            \item si $u$ et $v$ sont différents, en déduire $x_2$ et
              $x_3$.
          \end{itemize}
      \end{enumerate}
  \end{enumerate}

  \noindent\textbf{Partie B}

  Sur la résolution de l'équation du troisième degré, Cardan démontra
  que la racine certaine pouvait se calculer explicitement à l'aide de
  la formule $x = \sqrt[3]{\frac{-q}2 + \sqrt{\frac{q^2}4 +
  \frac{p^3}{27}}} + \sqrt[3]{\frac{-q}2 - \sqrt{\frac{q^2}4 +
  \frac{p^3}{27}}}$.

  \begin{enumerate}
    \item Résoudre l'exemple historique du à Cardan, $x^3 -x -1 =0$, en
      appliquant la méthode décrite ci-dessus.
    \item Résoudre $x^3 -3x + 2 =0$.
    \item Résoudre l'exemple historique du à Bombelli : $x^3 - 15x -4 =
      0$.
      \begin{enumerate}
        \item Vérifier que $4$ est solution de l'équation considérée.
        \item Quel problème semble être posé par la méthode ?
        \item On va supposer que $\sqrt{-121}$ est une écriture
          légitime. Justifier qu'une autre écriture serait $\sqrt{-121}
          = 11\sqrt{-1}$. Quelle propriété des radicaux a-t-on utilisé ?
        \item On va chercher un cube qui vaut $2 \pm 11\sqrt{-1}$ sous
          la forme $2 + a\sqrt{-1}$. Développer $(2+a\sqrt{-1})^3$ en
          utilisant $\sqrt{-1}\times \sqrt{-1} = -1$. Effectuer le même
          calcul avec $(2-a\sqrt{-1})^3$ et identifier l'un à $2 +
          11\sqrt{-1}$ et l'autre à $2 - 11\sqrt{-1}$. Que vaut $a$ ?
        \item Retrouver alors par le calcul $x = 4$ en additionnant les
          nombres trouvés.
      \end{enumerate}
  \end{enumerate}
\end{exercice}

\pagebreak

\noindent\maketitle

\section{Une extension du corps des réels}

\subsection{Une construction qui ajoute des propriétés}

L'idée générale des nombres complexes est d'étendre les capacités de
calculs, éventuellement en recourant à des «nombres astucieux»
permettant de mener des calculs. On rencontre cette idée plusieurs fois
dans le développement des mathématiques. Ainsi, l'ensemble des nombres
$\N$ muni de l'opération d'addition a été étendu en $\Z$ afin de donner
un sens à $a+b =0$ et à permis d'exprimer de nouvelles opérations. De la
même façon, $\Z$ a été étendu à $\Q$ pour donner un sens à $ab = 1$.
Enfin, $\Q$, par une méthode bien plus complexe peut s'étendre à $\R$
qui est construit de façon à donner un sens à des nombres tels que
$\sqrt{2}$. Cependant, même dans $\R$ des nombres tels que $\sqrt{-1}$
n'ont pas de sens.

On considère $x$ et $y$, tout deux éléments de $\R$, qu'on note
$(x,y) \in \R \times \R$ et $(x',y') \in \R \times \R$. On construit
deux «nouvelles» opérations :
\begin{itemize}
  \item une addition définie par $(x,y) + (x',y') = (x + x', y + y')$ ;
  \item une multiplication définie par $(x,y) \times (x',y') = (xx' -
    yy', xy' + x'y)$.
\end{itemize}

\begin{exercice}[Quelques démonstrations]~\\
  En utilisant les notations et les règles de calculs ci-dessus :
  \begin{enumerate}
    \item Montrer que $\varphi (x,y) \mapsto (x,0)$ permet de retrouver
      l'ensemble des nombres réels. (Appliquer les règles de calcul)
    \item Que vaut $(0,1)\times (0,1)$ ?
    \item Vérifier que
      \begin{enumerate}
        \item $(0,0)$ est l'élément neutre de + et $(1,0)$ celui de
          $\times$ ;
        \item ces deux opérations sont associatives et commutatives ($(a
          * b) * c = a * (b * c)$ et $a * b = b * a$, $* \in
          \{+,\times\}$) ;
        \item que $a \times ( b + c ) = a \times b + a \times c$.
      \end{enumerate}
  \end{enumerate}
\end{exercice}

On peut désormais définir le corps des nombres complexes ainsi que
l'unité imaginaire.

\begin{definition}[Unité imaginaire]
  On pose $i$ défini par $i^2 = -1$.
\end{definition}

\begin{remarque}
  On note parfois $j$, en particulier en physique, pour ne pas confondre
  avec $i(t)$ l'intensité d'un courant. On trouve également, en
  particulier dans la littérature anglo-saxonne $\sqrt{-1}$ ou
  $(-1)^{\frac12}$.

  L'école française (bourbakiste) reste attachée à $i$.
\end{remarque}

\begin{definition}[Corps des complexes]
  On note $\R\times\R$ muni des opérations définies ci-dessus le coprs
  $\C$ des nombres complexes. On note souvent les éléments de ce corps
  $z = a + ib$, $z\in \C,\ (a,b)\in \R^2$.
\end{definition}

\begin{remarque}
  On parle d'écriture cartésienne.
\end{remarque}

\begin{propriete}[Règles de calculs]
  Soient $z$ et $z'$ deux nombres complexes s'écrivant $z = a + i b$ et
  $z' = a' + i b'$.
  \begin{itemize}
    \item $z + z' = a + a' + i(b + b')$
    \item $zz' = aa' - bb' + i(ab' + a'b)$
  \end{itemize}
\end{propriete}

On a une identification «canonique».

\begin{propriete}
  Si $b = 0$, on identifie le nombre complexe $z = a +i0$ au nombre réel
  $a$.\\ On dit alors que $\R$ est plongé dans $\C$ et on note $\R
  \subset \C$.
\end{propriete}

\begin{definition}
  Soit $z$ un nombre complexe s'écrivant $z = a +ib$. On définit et note
  $a = \Re(z)\in\R $ la partie réelle de $z$ et $b = \Im(z) \in \R$ la
  partie imaginaire de $z$.
\end{definition}

\subsection{Exploration de ce nouvel espace de nombres}

Les nombres réels vont continuer de jouer un rôle prépondérant dans ce
nouvel espace de nombres et on peut noter d'une certaine façon que $\C
= \R + i\R$. Cherchons une application bijective, qui préserve les
opérations et les nombres réels. Pour préciser, on va chercher $\varphi
\colon \C \to \C$ telle que pour tous nombres complexes $z$ et $z'$ :
\begin{itemize}
  \item $\varphi(z+z') = \varphi(z) + \varphi(z')$
  \item $\varphi(zz') = \varphi(z) \times \varphi(z')$
  \item si $x\in \R$, $\varphi(x) = x$
\end{itemize}

\begin{exercice}
  Soit $\varphi$ une application satisfaisant les conditions ci-dessus.
  \begin{enumerate}
    \item Justifier que $\varphi(0) = 0$ en utilisant la première
      condition.
    \item Justifier que $\varphi(1) = 1$ en utilisant la deuxième
      condition.
    \item Exprimer $\varphi(a + ib)$
    \item Justifier que $\varphi(i) = \pm i$
    \item Quelle application célèbre obtient-on en prenant $\varphi(i) =
      i$ ?
  \end{enumerate}
\end{exercice}

On a donc une application qui préserve les opérations (morphisme) de
$\C$ dans $\C$ (endomorphisme) et bijective (donc on peut définir
l'application réciproque, automorphisme.)

\begin{definition}[conjugaison]
  L'application $\varphi \colon \C \to \C, z = a + ib \mapsto a - ib$
  s'appelle la conjugaison.\\ On note $\varphi(z) = \overline{z}$ ou
  encore $z^*$.
\end{definition}

\begin{propriete}
  \begin{itemize}
    \item $\overline{z + z'} = \overline{z} + \overline{z'}$
    \item $\overline{zz'} = \overline{z} \times \overline{z'}$
    \item $\overline{\overline{z}} = z$
  \end{itemize}
\end{propriete}

La conjugaison est une application très importante : on peut s'en servir
pour calculer certaines quantités.

\begin{exercice} Soit $z = a + ib$ un nombre complexe. Calculer
  \begin{enumerate}
    \item $z + \overline{z}$
    \item $z - \overline{z}$
    \item $z \times \overline{z}$
  \end{enumerate}
\end{exercice}

Enfin, pour finir cette partie, on peut noter que désormais toutes les
équations du second degré à coefficient réels ont une solution,
éventuellement complexe. En effet, soit $ax^2 + bx +c =0$ une équation
du second degré à coefficients réels, si $\Delta = b^2 - 4ac < 0$ on ne
dira plus que «l'équation ne possède pas de solutions réelles» mais on
dira que l'équation possède \emph{deux solutions complexes conjuguées}.
Ces solutions seront $x_1 = \frac{-b + i\sqrt{\abs{\Delta}}}{2a}$ et son
conjugué $x_2 = \frac{-b - i\sqrt{\abs{\Delta}}}{2a}$.

\section{Une approche géométrique des nombres complexes}

\subsection{Quelques éléments de géométrie}

Le corps des nombres complexes a été construit comme $\R\times\R$ et les
premiers éléments qui en ont été donnés l'ont été sous la forme de
couples. Il est assez naturel de penser à une représentation
géométrique.

\begin{definition}[affixe d'un point]
  Soit $M: \vecteur{x;y}$ un point du plan. On appelle affixe du point
  $M$ et on note le nombre complexe $z_M = x + iy$.
\end{definition}

On peut faire de même avec un vecteur en procédant de la façon suivante.

\begin{definition}[affixe d'un vecteur]
  Soit $\vv{MP}: \vecteur{x,y}$ un vecteur du plan. On appelle affixe du
  vecteur $\vv{MP}$ et on note le nombre complexe $z_{\vv{MP}} = z_P -
  z_M$.
\end{definition}

Ainsi, à tout nombre complexe $z$ on peut associer un point du plan (dit
d'Argand-Cauchy) et réciproquement. Puisqu'on algébrise la géométrie,
intéressont nous aux conséquences et implications.

Commençons par l'opération de conjugaison.

\begin{exercice}~\\[-5mm]
  \begin{enumerate}
    \item Citer une application géométrique telle que si elle est
      appliquée deux fois consécutives, elle renvoie sur le point
      d'origine.
    \item Parmi les applications géométriques trouvées précédement,
      étudier leurs éléments invariants et comparer au fait que la
      conjugaison laisse les réels invariant.
    \item Conclure.
  \end{enumerate}
\end{exercice}

\begin{propriete}
  La conjugaison correspond à une symétrie axiale d'axe l'axe des
  abscisses.
\end{propriete}

Puisqu'on a vu une première symétrie, axiale, intéressons nous à la
symétrie centrale.

\begin{exercice}~\\[-5mm]
  \begin{enumerate}
    \item La symétrie centrale possède-t-elle également cette propriété
      d'involution, c'est-à-dire que si elle est appliquée deux fois, on
      revient sur le point d'origine ?
    \item Combien la symétrie centrale possède-t-elle de point fixe ?
      Quel(s) (est|sont) il(s) ?
    \item Donner une application algébrique de la symétrie centrale de
      centre $O: \vecteur{0;0}$.
  \end{enumerate}
\end{exercice}

On peut noter une proposition permettant de réaliser une symétrie axiale
d'axe l'axe des ordonnées.

\begin{propriete}
  L'application $s:\C\to\C,\ z\mapsto -\overline{z}$ est la symétrie
  axiale d'axe l'axe des ordonnées.
\end{propriete}

Parmi les applications géométriques, on peut aussi noter les
applications suivantes ainsi que leurs variantes algébriques dans
l'ensemble des nombres complexes.

\begin{propriete}
  \begin{itemize}
    \item $z \mapsto z+a$ est la translation de vecteur $\vv{OA}$ où $a$
      est l'affixe du point $A$.
    \item $z \mapsto z + \overline{z} = \Re(z)$ est la projection sur
      $(Ox)$ parallèlement à $(Oy)$.
    \item $z \mapsto z - \overline{z} = \Im(z)$ est la projection sur
      $(Oy)$ parallèlement à $(Ox)$.
  \end{itemize}
\end{propriete}


\subsection{Une analyse fonctionnelle pour une égalité}

Avant de se lancer dans une analyse fonctionnelle et en dégager un
nouvelle notion, apportons quelques précisions.

\begin{exercice}Soit $z$ un nombre complexe. On considère l'application
  $\psi:\C \to \R, z \mapsto \sqrt{z\times \overline{z}}$
  \begin{enumerate}
    \item Vérifier que $\psi(z) = \psi(-z) = \psi(\overline{z})$.
    \item Exprimer $z^{-1}$ en fonction de $\overline{z}$ et $\psi(z)$.
    \item Montrer que $\Re(z) \leqslant \psi(z)$ et $\Im(z) \leqslant
      \psi(z)$. Préciser quelles sont les conditions d'égalité de
      $\Re(z) = \psi(z)$.
    \item \begin{enumerate}
        \item $\forall z\in \C,\ \psi(z) = 0 \iff z = 0$
        \item $\forall (z,z') \in \C^2,\ \psi(zz') = \psi(z) \psi(z')$
        \item $\forall (z,z') \in \C^2,\ \psi(z + z') \leqslant \psi(z)
          + \psi(z')$
      \end{enumerate}
  \end{enumerate}
\end{exercice}

L'application $\psi$ définit en fait une norme, qu'on note
$\abs{\cdot}$.

\begin{definition}[Module]
  L'application $\psi$ définie ci-dessus s'appelle le \emph{module} du
  nombre complexe $z$. On le note $\abs{z}$.
\end{definition}

\begin{propriete}
  Le module définit une norme.
\end{propriete}

La notion de module prendra son sens avec celle d'argument que nous
allons développer.

\begin{exercice}Soit $t$ un nombre réel. On pose $e^{it} = \cos t +
  i\sin t$.
  \begin{enumerate}
    \item \begin{enumerate}
        \item Exprimer $e^{-it}$.
        \item En déduire que $e^{-it} = \overline{e^{it}}$.
        \item Calculer $\abs{e^{it}}$.
      \end{enumerate}
    \item Calculer $\frac{\diff e^{it}}{\diff t}$
    \item Vérifier que $e^{it}\times e^{it'} = e^{i(t+t')}$.
  \end{enumerate}
\end{exercice}

\begin{definition}[Argument]
  Tout nombre complexe peut s'écrire sous la forme $z = re^{i\theta} =
  r\cos \theta + ri\sin \theta$, où $r = \abs{z}$ . On dit que
  $re^{i\theta}$ est la forme polaire du nombre complexe et $\theta$ est
  un \emph{argument} du nombre complexe $z$, noté $\arg(z)$.
\end{definition}

\begin{remarque} L'angle $\theta$ n'est défini qu'à $2\pi$ près, on note
  que si $\theta$ est un argument, $\theta + 2k\pi,\ k\in\Z$ est aussi
  un argument de $z$.
\end{remarque}

\begin{propriete}
  $\forall (z,z') \in \C^2,\ \arg(zz') = \arg(z) + \arg(z') \ [2\pi]$.
\end{propriete}

\begin{propriete}
  \begin{itemize}
    \item La distance $AB = \abs{z_B - z_A}$ où $A$ et $B$ ont pour
      affixes respectives $z_A$ et $z_B$.
    \item L'angle $\widehat{\left[Ox\right),\vv{AB}} = \arg(z_B - z_A)
      \ [2\pi]$
  \end{itemize}
\end{propriete}

\section*{Références}

\begin{itemize}
  \item \url{http://serge.mehl.free.fr/chrono/Bombelli.html}
  \item \url{http://serge.mehl.free.fr/anx/equ_deg3.html}
  \item Ramis, Deschamps, Odoux : Tome 1. Algèbre
\end{itemize}

\end{document}
