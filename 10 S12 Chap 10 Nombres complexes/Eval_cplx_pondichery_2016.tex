\documentclass[12pt,a4paper,french]{article}
\usepackage[utf8]{inputenc}
\usepackage[T1]{fontenc}
\usepackage{babel}
\usepackage[thmmarks]{ntheorem}
\usepackage{amsmath}
\usepackage{amsfonts}
\usepackage{amssymb}

\usepackage{array}

\usepackage{lmodern}
\usepackage{kpfonts}

\usepackage[bookmarks=false,colorlinks,linkcolor=blue,pdfusetitle]{hyperref}

\pdfminorversion 7
\pdfobjcompresslevel 3

\usepackage{tabularx}
\usepackage[autolanguage,np]{numprint}
\usepackage{enumitem}

\usepackage{tipfr}
\usepackage{pgf}
\usepackage{tikz}
\usepackage{tkz-euclide}
\usetkzobj{all}
\usetikzlibrary{hobby}
\usepackage{tkz-tab}

\usepackage[top=1.9cm,bottom=2cm,left=2cm,right=2cm]{geometry}

\usepackage{lastpage}

\usepackage{esvect}
\usepackage{marginnote}

\usepackage{wrapfig}

\usepackage[defaultlines=5,all]{nowidow}


\usepackage[]{algorithm2e}

\usepackage{ifthen}
\usepackage{fancyhdr}
\pagestyle{fancy}


\makeatletter

\count1=\year \count2=\year
\ifnum\month<8\advance\count1by-1\else\advance\count2by1\fi

\setlength{\headheight}{14.5pt}
\renewcommand{\headrulewidth}{0pt}
\renewcommand{\footrulewidth}{0pt}
\cfoot{\textsl{\footnotesize{Année \number\count1/\number\count2}}}

\rfoot{%
  \ifthenelse{\value{page}=1}{%
  }
  {%
    \footnotesize{Page \thepage/ \pageref{LastPage}}
  }
}

\rhead{}

\lhead{%
  \ifthenelse{\value{page}=1}{%
    Nom:\dotfill\hfill Prénom: \dotfill \hfill
    Classe: \@author \dots%
  }
  { }
}

\renewcommand{\maketitle}%
{\framebox{%
    \begin{minipage}{1.0\linewidth}%
      \begin{center}%
        \Large \@title ~-- \@author \\%
        \@date%
      \end{center}%
    \end{minipage}}%
  \normalsize%
  %\vspace{1cm}%
}

%Des macros pour les noms d'ensmbles
\newcommand{\R}{\mathbf{R}}
\newcommand{\Q}{\mathbf{Q}}
\newcommand{\Z}{\mathbf{Z}}
\newcommand{\C}{\mathbf{C}}
\newcommand{\N}{\mathbf{N}}

\newcommand{\norme}[1]{\left\lVert #1 \right\rVert}
\newcommand{\abs}[1]{\left\lvert #1 \right\rvert}
\newcommand{\diff}{\mathop{}\mathopen{}\mathrm{d}}
%Une macro récursive pour l'intérieru des vecteurs
%http://tex.stackexchange.com/questions/19693/arguments-of-custom-commands-as-comma-separated-list

\newcommand\vecteur[2][\\]{%
    \global\def\my@delim{#1}%
    \left(\negthinspace\begin{matrix}
        \my@vector #2,\relax\noexpand\@eolst%
    \end{matrix}\right)}

%Une macro pour les vecteurs
\def\my@vector #1,#2\@eolst{%
   \ifx\relax#2\relax
      #1
   \else
      #1\my@delim
      \my@vector #2\@eolst%
   \fi}

%Une macro récursive pour mettre formater l'intérieur des intervalles
\def\my@intervalle #1;#2\@eolst{%
  \ifx\relax#2\relax
    #1
  \else
    \my@intervalle #2\@eolst%
  \fi}

%Quatre macros pour les quatres types d'intervalles
\newcommand{\interff}[1]{%
  \left[\my@intervalle #1;\relax\noexpand\@eolst%
  \right]
}
\newcommand{\interfo}[1]{%
  \left[\my@intervalle #1;\relax\noexpand\@eolst%
  \right[}
\newcommand{\interof}[1]{%
  \left]\my@intervalle #1;\relax\noexpand\@eolst%
  \right]}
\newcommand{\interoo}[1]{%
  \left]\my@intervalle #1;\relax\noexpand\@eolst%
  \right[}

\makeatother


\usepackage{framed}

\theoremstyle{break}
\newtheorem{definition}{Définition}
\newtheorem{propriete}{Propriété}
\newtheorem{corollaire}{Corollaire}
\newtheorem{propdef}{Propriété - Définition}
\newtheorem{theoreme}{Théorème}
\theoremstyle{plain}
\theorembodyfont{\normalfont}
\newtheorem{exerciceT}{Exercice}
\theoremstyle{nonumberplain}
\newtheorem{remarque}{Remarque}
\newtheorem{notation}{Notation}
\newtheorem{probleme}{Problème}
\theoremsymbol{\ensuremath{\blacksquare}}
\newtheorem{preuve}{Preuve}
\theoremsymbol{}
\theoremstyle{nonumberbreak}
\newtheorem{exemple}{Exemple}

\newenvironment{exercice}{\begin{framed}\begin{exerciceT}}{\end{exerciceT}\end{framed}}

\setlength{\parsep}{0pt}
\setlength{\parskip}{5pt}
\setlength{\parindent}{0pt}
\setlength{\itemsep}{7pt}

\setlist{noitemsep}
%\setlist[1]{\labelindent=\parindent} % < Usually a good idea
\setlist[itemize]{leftmargin=*}
\setlist[itemize,1]{label=$\triangleright$}
\setlist[enumerate]{labelsep=*, leftmargin=1.5pc}
\setlist[enumerate,1]{label=\arabic*., ref=\arabic*}
\setlist[enumerate,2]{label=\emph{\alph*}),
ref=\theenumi.\emph{\alph*}}
\setlist[enumerate,3]{label=\roman*), ref=\theenumii.\roman*}
\setlist[description]{font=\sffamily\bfseries}

\usepackage{multicol}
\setlength{\columnseprule}{0pt}

\usepackage[]{exsheets}
\SetupExSheets{headings=block}

\everymath{\displaystyle\everymath{}}

\title{Évaluation \no{11} : Nombres complexes}
\author{\bsc{Ts}}
\date{mai 2017}


\usepackage{pst-plot,pst-tree,pstricks,pst-node,pst-text}
\usepackage{pst-eucl}
\usepackage{pstricks-add}
\newcommand{\vect}[1]{\mathchoice%
{\overrightarrow{\displaystyle\mathstrut#1\,\,}}%
{\overrightarrow{\textstyle\mathstrut#1\,\,}}%
{\overrightarrow{\scriptstyle\mathstrut#1\,\,}}%
{\overrightarrow{\scriptscriptstyle\mathstrut#1\,\,}}}
\def\Oij{$\left(\text{O},~\vect{\imath},~\vect{\jmath}\right)$}
\def\Oijk{$\left(\text{O},~\vect{\imath},~\vect{\jmath},~\vect{k}\right)$}
\def\Ouv{$\left(\text{O},~\vect{u},~\vect{v}\right)$}

\begin{document}

\maketitle

\begin{tabular}{|p{6em}|p{26em}|p{6em}|}\hline
   & & \\
   & & \\
   \hfill\Huge /\totalpoints* & & \\
   & & \\
   & & \\ \hline
\end{tabular}


\begin{question}[ID=complexes;géométrie;bac;pondichery;2016]

\medskip

L'objectif de cet exercice est de trouver une méthode pour construire à la règle et au compas
un pentagone régulier.

\medskip

\parbox{0.5\linewidth}{Dans le plan complexe muni d'un repère
orthonormé direct \Ouv, on considère le pentagone régulier $A_0A_1A_2A_3A_4$, de centre $O$ tel
que $\vect{OA_0} = \vect{u}$.

On rappelle que dans le pentagone régulier $A_0A_1A_2A_3A_4$, ci-contre :

\setlength\parindent{1cm}
\begin{itemize}
\item[$\bullet~~$] les cinq côtés sont de même longueur;
\item[$\bullet~~$] les points $A_0,\:A_1,\:A_2,\:A_3$ et $A_4$
appartiennent au cercle trigonométrique ;
\item[$\bullet~~$] pour tout entier $k$ appartenant à $\{0~;~1~;~2~;~3\}$ on a 

$\left(\vect{OA_k}~;~\vect{OA_{k+1}}\right) = \dfrac{2\pi}{5}$.
\end{itemize}
\setlength\parindent{0cm}}\hfill
\parbox{0.49\linewidth}{\psset{unit=2.5cm}
\begin{pspicture}(-1.2,-1.1)(1.2,1.1)
\psaxes[linewidth=1.25pt]{->}(0,0)(-1.2,-1.1)(1,1)
\pspolygon(1;0)(1;72)(1;144)(1;216)(1;288)
\uput[d](0.5,0){$\vect{u}$}\uput[l](0,0.5){$\vect{v}$}
\uput[dl](0,0){O}
\uput[ur](1;0){$A_0$} \uput[ur](1;72){$A_1$} \uput[ul](1;144){$A_2$} \uput[dl](1;216){$A_3$} \uput[dr](1;288){$A_4$} 
\end{pspicture}
}

\medskip

\begin{enumerate}
\item On considère les points $B$ d'affixe $- 1$ et $J$ d'affixe $\dfrac{\text{i}}{2}$.

Le cercle $(\mathcal{C})$ de centre $J$ et de rayon $\dfrac{1}{2}$ coupe le segment $[BJ]$ en un point $K$.

Calculer $BJ$, puis en déduire $BK$.
\item 
	\begin{enumerate}
		\item Donner sous forme exponentielle l'affixe du point $A_2$. Justifier brièvement.
		\item Démontrer que $BA_2\,^2 = 2 +  2\cos \left(\dfrac{4\pi}{5}\right)$.
		\item Un logiciel de calcul formel affiche les résultats ci-dessous, que l'on pourra utiliser
sans justification :
\begin{center}
\begin{tabularx}{0.5\linewidth}{|c|X|}\hline
\multicolumn{2}{|l|}{$\blacktriangleright$ Calcul formel}\\\hline
1&cos (4*pi/5)\\
&\rule[-3mm]{0mm}{9mm}$\to \dfrac{1}{4}\left(- \sqrt{5} - 1\right)$\\ \hline
2&sqrt((3 - sqrt(5))/2)\\ \hline
&\rule[-3mm]{0mm}{9mm}$\to \dfrac{1}{2}\left(\sqrt{5} - 1\right)$\\ \hline
\end{tabularx}

\og sqrt \fg{} signifie \og racine carrée\fg
\end{center}

En   déduire, grâce à ces résultats, que $BA_2 = BK$.
 	\end{enumerate}
\item  Dans le repère \Ouv{} donné en annexe, construire à la règle et au compas un
pentagone régulier. N'utiliser ni le rapporteur ni les graduations de la règle et laisser
apparents les traits de construction.
  \addpoints*{6}
\end{enumerate}

\end{question}

\begin{question}
  \emph{Les deux parties \text{A} et \text{B} peuvent être traitées de façon indépendante}

\bigskip

\textbf{Partie A}

\medskip

Des études statistiques ont permis de modéliser le temps hebdomadaire, en heures, de
connexion à internet des jeunes en France âgés de 16 à 24 ans par une variable aléatoire $T$
suivant une loi normale de moyenne $\mu = 13,9$ et d'écart type $\sigma$.

La fonction densité de probabilité de $T$ est représentée ci-dessous :

\begin{center}
\psset{xunit=0.3cm,yunit=1.8cm,algebraic=true}
\begin{pspicture}(-1.5,-0.1)(30,3)
%\psgrid
\psline(-1.5,0)(30,0)
\multido{\n=-1+1}{31}{\psline(\n,0)(\n,-0.05)}
\def\m{13.9}% moyenne
\def\s{4.1}% écart type
\def\f{30/(\s*sqrt(2*PI))*EXP((-((x-\m)/\s)^2)/2)}
\psplot[plotpoints=1000,linecolor=blue]{-1.5}{30}{\f}
\uput[d](0,0){0}\uput[d](1,0){1}\uput[d](10,0){10}\uput[d](13.9,0){13,9}
\psline[linestyle=dashed](13.9,0)(13.9,2.9)
\end{pspicture}
\end{center}

\begin{enumerate}
\item On sait que $p(T \geqslant 22) =  0,023$.

En exploitant cette information :
	\begin{enumerate}
		\item hachurer sur le graphique donné un annexe, deux domaines distincts
dont l'aire est égale à $0,023$ ;
		\item déterminer $P(5,8 \leqslant T \leqslant 22)$. Justifier le résultat.
Montrer qu'une valeur approchée de $\sigma$ au dixième est $4,1$.
	\end{enumerate}
\item On choisit un jeune en France au hasard.

Déterminer la probabilité qu'il soit connecté à internet plus de 18 heures par semaine.

Arrondir au centième.
\end{enumerate}

\bigskip

\textbf{Partie B}

\medskip

Dans cette partie, les valeurs seront arrondies au millième.

La Hadopi (Haute Autorité pour la diffusion des Œuvres et la Protection des droits sur
Internet) souhaite connaître la proportion en France de jeunes âgés de 16 à 24 ans pratiquant
au moins une fois par semaine le téléchargement illégal sur internet. Pour cela, elle envisage
de réaliser un sondage.

Mais la Hadopi craint que les jeunes interrogés ne répondent pas tous de façon sincère. Aussi,
elle propose le protocole $(\mathcal{P})$ suivant :

\begin{tabularx}{\linewidth}{|X}
On choisit aléatoirement un échantillon de jeunes âgés de 16 à 24 ans.\\
Pour chaque jeune de cet échantillon :
\setlength\parindent{1cm}
\begin{itemize}
\item[$\bullet~~$] le jeune lance un dé équilibré à 6 faces ;
l'enquêteur ne connaît pas le résultat du lancer ;
\item[$\bullet~~$] l'enquêteur pose la question : \og Effectuez-vous un téléchargement illégal au
moins une fois par semaine ? \fg{} ;
\end{itemize}
\setlength\parindent{0cm}\\
\begin{tabular}{|m{11cm}|}\hline
$\blacksquare$~~si le résultat du lancer est pair alors le jeune doit répondre à la question par
\og Oui \fg{} ou \og Non\fg{} de façon sincère ;\\
$\blacksquare$~~ si le résultat du lancer est \og 1 \fg{} alors le jeune doit répondre \og Oui \fg{} ;\\
$\blacksquare$~~ si le résultat du lancer est \og 3 ou 5 \fg{} alors le jeune doit répondre \og Non \fg.\\ \hline
\end{tabular}\\
\end{tabularx}

\medskip

Grâce à ce protocole, l'enquêteur ne sait jamais si la réponse donnée porte sur la question
posée ou résulte du lancer de dé, ce qui encourage les réponses sincères.

\medskip

On note $p$ la proportion inconnue de jeunes âgés de 16 à 24 ans qui pratiquent au moins une
fois par semaine le téléchargement illégal sur internet.

\medskip

\begin{enumerate}
\item \emph{Calculs de probabilités}

On choisit aléatoirement un jeune faisant partie du protocole $(\mathcal{P})$.

On note : $R$ l'évènement \og le résultat du lancer est pair\fg,

$O$ l'évènement \og le jeune a répondu Oui \fg.

Reproduire et compléter l'arbre pondéré ci-dessous :

\begin{center}
\pstree[treemode=R,nodesep=2.5pt]{\TR{}}
{\pstree{\TR{$R$}}
	{\TR{$O$}
	\TR{$\overline{O}$}
	}
\pstree{\TR{$\overline{R}$}}
	{\TR{$O$}
	\TR{$\overline{O}$}
	}
}
\end{center}
\smallskip

En déduire que la probabilité $q$ de l'évènement \og le jeune a répondu Oui\fg{} est :

\[q = \dfrac{1}{2}p + \dfrac{1}{6}.\]

\item \emph{Intervalle de confiance}

\medskip

	\begin{enumerate}
		\item À la demande de l'Hadopi, un institut de sondage réalise une enquête selon le
protocole $(\mathcal{P})$. Sur un échantillon de taille \np{1500}, il dénombre $625$ réponses \og Oui \fg.

Donner un intervalle de confiance, au niveau de confiance de 95\,\%, de la proportion $q$
de jeunes qui répondent \og Oui \fg{} à un tel sondage, parmi la population des jeunes
français âgés de 16 à 24 ans.
		\item Que peut-on en conclure sur la proportion $p$ de jeunes qui pratiquent au moins une fois
par semaine le téléchargement illégal sur internet ?
	\end{enumerate}
\end{enumerate}

\addpoints*{8}

\end{question}


%\newpage
%\section*{Correction}
%\printsolutions
\end{document}
