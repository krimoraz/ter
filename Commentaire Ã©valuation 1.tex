\documentclass[12pt,a4paper]{article}
\usepackage[utf8]{inputenc}
\usepackage[T1]{fontenc}
\usepackage[french]{babel}
\usepackage{ntheorem}
\usepackage{amsmath}
\usepackage{amsfonts}

\usepackage{kpfonts}
\usepackage{textcomp}
\newcommand{\euro}{\eurologo{}}

\usepackage[bookmarks=false,colorlinks,linkcolor=blue,pdfusetitle]{hyperref}

\pdfminorversion 7
\pdfobjcompresslevel 3

\frenchbsetup{ItemLabels=\textbullet,}

\usepackage{tabularx}
\usepackage{enumitem}

\usepackage{pgf}
\usepackage{tikz}
\usepackage{tkz-euclide}
\usetkzobj{all}
\usetikzlibrary{hobby}
\usepackage{tkz-tab}
\usepackage[top=1.4cm,bottom=1.4cm,left=2cm,right=2cm,includehead,
includefoot]{geometry}

\usepackage{lastpage}

\usepackage{fancybox}

\usepackage[autolanguage]{numprint}
\newcommand{\np}{\numprint}

\usepackage{ifthen}
\usepackage{fancyhdr}
\pagestyle{fancy}

\lhead{\ifthenelse{\value{page}=1}{Nom:\dotfill\hfill Prénom: \dotfill
\hfill ~}{}}
\rhead{}
\chead{}

\rfoot{\upshape \small page {\thepage} de \pageref{LastPage}}
\lfoot{}
\cfoot{}
\renewcommand{\headrulewidth}{0pt}
\renewcommand{\footrulewidth}{0pt}

\usepackage{pdfmarginpar}

\makeatletter
\renewcommand{\maketitle}%
{\framebox{%
    \begin{minipage}{0.98\linewidth}%
      \begin{center}%
        \Large \@title ~-- \@author \\%
        \@date%
      \end{center}%
  \end{minipage}}%
  \normalsize%
  %\vspace{1cm}%
}

\makeatother

\theoremstyle{break}
\newtheorem{definition}{Définition}
\theorembodyfont{\normalfont}
\theoremstyle{nobreak}
\newtheorem{exercice}{Exercice}

%\theoremstyle{plain}
\theoremstyle{nonumberplain}
\newtheorem{probleme}{Problème}

% Mise en forme des labels dans les énumérations
\renewcommand{\labelenumi}{\textbf{\theenumi.}}
\renewcommand{\labelenumii}{\textbf{\theenumii)}}
\renewcommand{\theenumi}{\arabic{enumi}}
\renewcommand{\theenumii}{\alph{enumii}}

\renewcommand{\labelitemi}{$\bullet$}

\setlength{\parsep}{0pt}
\setlength{\parskip}{5pt}
\setlength{\parindent}{0pt}
\setlength{\itemsep}{7pt}

\usepackage{multicol}
\setlength{\columnseprule}{0pt}

\everymath{\displaystyle\everymath{}}

\newcommand{\N}{\mathbf{N}}
\newcommand{\Z}{\mathbf{Z}}
\newcommand{\Q}{\mathbf{Q}}
\newcommand{\R}{\mathbf{R}}

\newcommand{\ligne}[1]{%
  \begin{tikzpicture}[]
    \draw[white] (0,#1+0.8) -- (15,#1+0.8) ;
    \foreach \i in {1,...,#1}
    { \draw[dotted] (0,\i) -- (\linewidth,\i) ; }
    \draw[white] (0,0.6) -- (15,0.6) ;
  \end{tikzpicture}%
}

\usepackage{xcolor}
\usepackage{framed}
\usepackage{algorithm}
\usepackage{algpseudocode}
\definecolor{fond}{RGB}{136,136,136}
\definecolor{sicolor}{RGB}{128,0,128}
\definecolor{tantquecolor}{RGB}{221,111,6}
\definecolor{pourcolor}{RGB}{187,136,0}
\definecolor{bloccolor}{RGB}{128,0,0}
\newenvironment{cadrecode}{%
  \def\FrameCommand{{\color{fond}\vrule width
  5pt}\fcolorbox{fond}{white}}%
  \MakeFramed {\hsize \linewidth \advance\hsize-\width
\FrameRestore}\begin{footnotesize}}%
{\end{footnotesize}\endMakeFramed}
\makeatletter
\def\therule{\makebox[\algorithmicindent][l]{\hspace*{.5em}\color{fond}
\vrule width 1pt height .75\baselineskip depth .25\baselineskip}}%
\newtoks\therules
\therules={}
\def\appendto#1#2{\expandafter#1\expandafter{\the#1#2}}
\def\gobblefirst#1{#1\expandafter\expandafter\expandafter{\expandafter\@gobble\the#1}}%
\def\Ligne{\State\unskip\the\therules}% 
\def\pushindent{\appendto\therules\therule}%
\def\popindent{\gobblefirst\therules}%
\def\printindent{\unskip\the\therules}%
\def\printandpush{\printindent\pushindent}%
\def\popandprint{\popindent\printindent}%
\def\Variables{\Ligne \textcolor{bloccolor}{\textbf{VARIABLES}}}
\def\Si#1{\Ligne \textcolor{sicolor}{\textbf{SI}} #1
\textcolor{sicolor}{\textbf{ALORS}}}%
\def\Sinon{\Ligne \textcolor{sicolor}{\textbf{SINON}}}%
\def\Pour#1#2#3{\Ligne \textcolor{pourcolor}{\textbf{POUR}} #1
  \textcolor{pourcolor}{\textbf{ALLANT\_DE}} #2
\textcolor{pourcolor}{\textbf{A}} #3}%
\def\Tantque#1{\Ligne \textcolor{tantquecolor}{\textbf{TANT\_QUE}} #1
\textcolor{tantquecolor}{\textbf{FAIRE}}}%
\algdef{SE}[WHILE]{DebutTantQue}{FinTantQue}
{\pushindent \printindent
\textcolor{tantquecolor}{\textbf{DEBUT\_TANT\_QUE}}}
{\printindent \popindent
\textcolor{tantquecolor}{\textbf{FIN\_TANT\_QUE}}}%
\algdef{SE}[FOR]{DebutPour}{FinPour}
{\pushindent \printindent
\textcolor{pourcolor}{\textbf{DEBUT\_POUR}}}
{\printindent \popindent
\textcolor{pourcolor}{\textbf{FIN\_POUR}}}%
\algdef{SE}[IF]{DebutSi}{FinSi}%
{\pushindent \printindent
\textcolor{sicolor}{\textbf{DEBUT\_SI}}}
{\printindent \popindent
\textcolor{sicolor}{\textbf{FIN\_SI}}}%
\algdef{SE}[IF]{DebutSinon}{FinSinon}
{\pushindent \printindent
\textcolor{sicolor}{\textbf{DEBUT\_SINON}}}
{\printindent \popindent
\textcolor{sicolor}{\textbf{FIN\_SINON}}}%
\algdef{SE}[PROCEDURE]{DebutAlgo}{FinAlgo}
{\printandpush
\textcolor{bloccolor}{\textbf{DEBUT\_ALGORITHME}}}%
{\popandprint
\textcolor{bloccolor}{\textbf{FIN\_ALGORITHME}}}%
\makeatother
\newenvironment{algobox}%
{%
  \begin{ttfamily}
    \begin{algorithmic}[1]
      \begin{cadrecode}
        \labelwidth 1.5em
        \leftmargin\labelwidth
        \addtolength{\leftmargin}{\labelsep}
      }
      {%
      \end{cadrecode}
    \end{algorithmic}
  \end{ttfamily}
}

\newcommand{\maximaout}[3]{%
\begin{framed}%
  {\ttfamily \textcolor{red}{(\%i#1)} #2\\%
      \textcolor{brown}{(\%o#1)} #3}%
\end{framed}%
}
\usepackage{wasysym}

\newcommand{\vf}{%
  \Square{}~ Vrai \-\- \Square{}~ Faux%
}

\renewcommand{\Vec}[1]{\overrightarrow{#1}}
\newcommand{\ProSca}[2]{#1 \cdot #2}

\title{Commentaires sur l'évaluation \no 1}
\author{TS3}
\date{2 octobre 2015}

\begin{document}

\maketitle

\emph{Compétences :}
\begin{itemize}
  \item Étudier la limite d'une somme, d'un produit ou d'un quotient de
    deux suites ;
  \item Savoir mener un raisonnement par récurrence ;
  \item Construire un arbre pondéré de probabilités.
\end{itemize}

\begin{exercice}[Limite de suite]~

  Dans l'ensemble, l'exercice a été assez bien réussi. Attention, parmi
  les erreurs (non sanctionnées pour l'instant) :
  \begin{itemize}
    \item il faut préciser que la factorisation est vraie $\forall
      n\in\N$, en écrivant par exemple «$\forall n\in\N,\ u_n = n^2 -3n
      = n(n-3)$ ;
    \item il faut écrire si la forme est indéterminée en toutes lettres
      ;
    \item il faut citer la règle utilisée («produit de deux suites de
      limite infinie»).
  \end{itemize}
  Pour les théorèmes de comparaison ou d'encadrement, il faut préciser
  le «domaine de validité» des encadrements ou majoration, c'est-à-dire
  le rang à partir du quel cette majoration est vraie. Celui-ci
  peut-être nul.

  Attention à bien définir les suites utilisées si vous le faites :
  «Soit la suite $(v_n)_{n\in\N}$ définie par $\forall n\in\N,\ v_n =
  f(n)$, où $f(n)$ est une expression.
\end{exercice}

\begin{exercice}[Récurrence]~

  Plusieurs points de vigilance :
  \begin{itemize}
    \item $P(n)$ est une proposition, c'est à dire une phrase que je
      peut mettre entre guillemets et qui contient la lettre
      \emph{variable} $n$ ;
    \item $P(n)$ ne peut contenir le quantificateur universel
      ($\forall$) appliqué à la variable $n$ ;
    \item si la propriété est à démontrer pour $n\in\N$, il faut
      initialiser à $n=0$, si la propriété est à démontrer pour
      $n\in\N^*$, il faut initialiser à $n=1$ ;
    \item il faut écrire tous les calculs algébrique permettant
      d'obtenir la proposition $P(n+1)$, un bon début est de partir de
      l'hypothèse de récurrence et de «dérouler les calculs» jusqu'à
      obtenir l'expression présente dans $P(n+1)$ ;
    \item les lignes de calculs entretiennent souvent un rapport les
      unes entres elles, il faut donc le matérialiser : « les égalités
      suivantes sont successcivement équivalentes» ou placer le symbole
      d'équivalence logique «$\iff$» devant chacune des lignes ;
    \item ne pas oublier de conclure dans la récurrence : c'est le cumul
      de l'initialisation et de la vérification de l'hérédité, qui
      valide l'affirmation finale.
  \end{itemize}
\end{exercice}

\begin{exercice}[Probabilités conditionnelles]~

  Exercice bien réussi dans l'ensemble, à l'exception de deux erreurs
  fondamentales :
  \begin{itemize}
    \item il faut préciser la signification des événements considérés ;
    \item une probabilité est un nombre compris entre 0 et 1.
  \end{itemize}
\end{exercice}

\end{document}
