\documentclass[12pt,a4paper,french]{article}
\usepackage[utf8]{inputenc}
\usepackage[T1]{fontenc}
\usepackage{babel}
\usepackage[thmmarks]{ntheorem}
\usepackage{amsmath}
\usepackage{amsfonts}
\usepackage{amssymb}

\usepackage{array}

\usepackage{lmodern}
\usepackage{kpfonts}

\usepackage[bookmarks=false,colorlinks,linkcolor=blue,pdfusetitle]{hyperref}

\pdfminorversion 7
\pdfobjcompresslevel 3

\usepackage{tabularx}
\usepackage[autolanguage,np]{numprint}
\usepackage{enumitem}

\usepackage{tipfr}
\usepackage{pgf}
\usepackage{tikz}
\usepackage{tkz-euclide}
\usetkzobj{all}
\usetikzlibrary{hobby}
\usepackage{tkz-tab}

\usepackage[top=1.7cm,bottom=2cm,left=2cm,right=2cm]{geometry}

\usepackage{lastpage}

\usepackage{esvect}
\usepackage{marginnote}

\usepackage{wrapfig}

\usepackage[defaultlines=5,all]{nowidow}


\makeatletter
\renewcommand{\@evenfoot}%
        {\hfil \upshape \small page {\thepage} de \pageref{LastPage}}
\renewcommand{\@oddfoot}{\@evenfoot}

\renewcommand{\maketitle}%
{\framebox{%
    \begin{minipage}{1.0\linewidth}%
      \begin{center}%
        \Large \@title ~-- \@author \\%
        \@date%
      \end{center}%
    \end{minipage}}%
  \normalsize%
  %\vspace{1cm}%
}

\pgfdeclarepatternformonly{mes_hachures}
{\pgfpoint{-0.1cm}{-0.1cm}}
{\pgfpoint{0.9cm}{0.5cm}}
{\pgfpoint{0.8cm}{0.4cm}}
{\pgfpathmoveto{\pgfpointorigin}
  \pgfpathlineto{\pgfpoint{0.8cm}{0.4cm}}
\pgfusepath{stroke}}

%Des macros pour les noms d'ensmbles
\newcommand{\R}{\mathbf{R}}
\newcommand{\Q}{\mathbf{Q}}
\newcommand{\Z}{\mathbf{Z}}
\newcommand{\C}{\mathbf{C}}
\newcommand{\N}{\mathbf{N}}

\newcommand{\norme}[1]{\left\lVert #1 \right\rVert}
\newcommand{\abs}[1]{\left\lvert #1 \right\rvert}

%Une macro récursive pour l'intérieru des vecteurs
%http://tex.stackexchange.com/questions/19693/arguments-of-custom-commands-as-comma-separated-list

\newcommand\vecteur[2][\\]{%
    \global\def\my@delim{#1}%
    \left(\negthinspace\begin{matrix}
        \my@vector #2,\relax\noexpand\@eolst%
    \end{matrix}\right)}

%Une macro pour les vecteurs
\def\my@vector #1,#2\@eolst{%
   \ifx\relax#2\relax
      #1
   \else
      #1\my@delim
      \my@vector #2\@eolst%
   \fi}

%Une macro récursive pour mettre formater l'intérieur des intervalles
\def\my@intervalle #1;#2\@eolst{%
  \ifx\relax#2\relax
    #1
  \else
    \my@intervalle #2\@eolst%
  \fi}

%Quatre macros pour les quatres types d'intervalles
\newcommand{\interff}[1]{%
  \left[\my@intervalle #1;\relax\noexpand\@eolst%
  \right]
}
\newcommand{\interfo}[1]{%
  \left[\my@intervalle #1;\relax\noexpand\@eolst%
  \right[}
\newcommand{\interof}[1]{%
  \left]\my@intervalle #1;\relax\noexpand\@eolst%
  \right]}
\newcommand{\interoo}[1]{%
  \left]\my@intervalle #1;\relax\noexpand\@eolst%
  \right[}

\makeatother


\usepackage{framed}

\theoremstyle{break}
\newtheorem{definition}{Définition}
\newtheorem{propriete}{Propriété}
\newtheorem{corollaire}{Corollaire}
\newtheorem{propdef}{Propriété - Définition}
\newtheorem{theoreme}{Théorème}
\theoremstyle{plain}
\theorembodyfont{\normalfont}
\newtheorem{exerciceT}{Exercice}
\theoremstyle{nonumberplain}
\newtheorem{remarque}{Remarque}
\newtheorem{notation}{Notation}
\newtheorem{probleme}{Problème}
\theoremsymbol{\ensuremath{\blacksquare}}
\newtheorem{preuve}{Preuve}
\theoremsymbol{}
\theoremstyle{nonumberbreak}
\newtheorem{exemple}{Exemple}

\newenvironment{exercice}{\begin{framed}\begin{exerciceT}}{\end{exerciceT}\end{framed}}

\setlength{\parsep}{0pt}
\setlength{\parskip}{5pt}
\setlength{\parindent}{0pt}
\setlength{\itemsep}{7pt}

\setlist{noitemsep}
%\setlist[1]{\labelindent=\parindent} % < Usually a good idea
\setlist[itemize]{leftmargin=*}
\setlist[itemize,1]{label=$\triangleright$}
\setlist[enumerate]{labelsep=*, leftmargin=1.5pc}
\setlist[enumerate,1]{label=\arabic*., ref=\arabic*}
\setlist[enumerate,2]{label=\emph{\alph*}),
ref=\theenumi.\emph{\alph*}}
\setlist[enumerate,3]{label=\roman*), ref=\theenumii.\roman*}
\setlist[description]{font=\sffamily\bfseries}

\usepackage{multicol}
\setlength{\columnseprule}{0pt}

\usepackage[]{exsheets}
\SetupExSheets{headings=block}

\everymath{\displaystyle\everymath{}}

\title{Évaluation \no{7} : exponentielle}
\author{\bsc{Ts 3}}
\date{4 février 2016}

\begin{document}

\maketitle

\begin{tabular}{|p{6em}|p{26em}|p{6em}|}\hline
   & & \\
   & & \\
   \hfill\Huge /\totalpoints* & & \\
   & & \\
   & & \\ \hline
\end{tabular}


\begin{question}[ID=71p100]
  ~\\[-6ex]
  \phantom{a}\hfill\textbf{(\GetQuestionProperty{points}{71p100} points)}\\
  Selon le Suisse Hans Jenny (1899-1992), spécialiste de l'étude des
  sols, la quantité $Q$ d'azote dans un sol cultivé (en kg/ha) évolue en
  fonction du temps $t$ (en année) selon la loi : \[ Q(t) = Q_e - (Q_e -
  Q_0)e^{-0,06t} .\]
  \begin{enumerate}
    \item En supposant $Q_0 < Q_e$, étudiez les variations sur
      $\interfo{0;+\infty}$ de la fonction $Q$. Vérifiez que que
      $\lim_{t\to+\infty}Q(t) = Q_e$. \addpoints*{3}

      Cette quantité $Q_e$ indépendante de $Q_0$ apparaît donc comme la
      valeur à laquelle se stabilise la quantité d'azote $Q$.
    \item
      \begin{enumerate}
        \item Vérifiez que la vitesse d'évolution à l'instant $t$,
          $\frac{\mathrm{d}Q}{\mathrm{d}t} = Q'(t)$, peut s'écrire sous
          la forme $0,06Q_e - 0,06Q(t)$. \addpoints*{1}

          Cette vitesse résulte de deux facteurs : le sol absorbe
          l'azote atmosphérique à une vitesse constante $v$ et élimine
          son azote à une vitesse proportionnelle à $Q$.
        \item Jenny évalue $v$ à 38 kg par ha et par an (en l'absence
          d'engrais et de résidus végétaux.). Déterminez $Q_e$.
          \addpoints*{1}
      \end{enumerate}
  \end{enumerate}

  \blank[style=dotted,width=15\linewidth,linespread=1.7]{}
\end{question}
\begin{solution}
  \begin{enumerate}
    \item $Q_0 < Q_e \implies Qe - Q_0 > 0$. $Q'(t) = -0,06(Qe -
      Q_0)e^{-0,06t} < 0$ et donc $Q$ est strictement décroissante.

      De plus, $\lim_{t\to+\infty} e^{-0,06t} = 0$ donc
      $\lim_{t\to+\infty} Q(t) = Q_e$.

    \item
      \begin{enumerate}
        \item D'après la question précédente, $v = -0,06(Qe -
          Q_0)e^{-0,06t}$

          Or $0,06Q(t) = 0,06Q_e -0,06(Q_e-Q_0)e^{-0,06t}$

          D'où $v = 0,06(Q_e-Q_0)e^{-0,06t} = 0,06Q_e - 0,06Q(t)$

    \item $Q'(t) = v - \lambda Q(t)$. On trouve donc $v = 0,06 Q_e$ et
      donc $Q_e \approx 633,33$ kg/an.
  \end{enumerate}
  \end{enumerate}
\end{solution}

\newpage
\section*{Correction}
\printsolutions
\vfill
\hrule
\vfill
\section*{Correction}
\printsolutions
\vfill
\hrule
\vfill
\section*{Correction}
\printsolutions
\vfill
\hrule
\vfill
\section*{Correction}
\printsolutions



\end{document}
