\documentclass[a4paper,12pt,frenchb]{article}

\input{../../commons.tex.inc}

\title{Évaluation : théoèrme des valeurs intermédiaires}
\author{semaine \no{48} -- 11\up{ième} semaine de cours}
\date{27 novembre 2017}

\SetWatermarkText{}
\parindent0pt

\begin{document}


\maketitle

\thispagestyle{fancy}

\begin{question}[subtitle={TVI}]

  \bsc{Partie A}

  Soit la fonction $u$ définie sur $\mathbf{R}$ par :
\[u(x)=2x^3-3x^2-1.\]
\begin{enumerate}
\item Calculer $u'(x)$ puis dresser le tableau de variation de la fonction $u$.
\item Démontrer que l'équation $u(x)=0$ a une unique solution $\alpha$ dans $\mathbf{R}$ et que $1<\alpha<2$.
\item En déduire le signe de $u(x)$ selon les valeurs de $x$.
\item Déterminer une valeur approchée de $\alpha$ à $10^{-3}$.
\end{enumerate}

\bsc{Partie B}

Soit $f$ la fonction définie sur $]-1~;~+\infty[$ par :
\[f(x)=\dfrac{1-x}{1+x^3}.\]
\begin{enumerate}
  \item Déterminer les limites de $f$ en $-1$ et en $+\infty$.
  \item Montrer que $f'(x)=\dfrac{u(x)}{\left(1+x^3\right)^2}$.
  \item Dresser le tableau de variation de $f$ sur $]-1~;~+\infty[$
  \item En remarquant que $2\alpha^3-3\alpha^2-1=0$, montrer que $f(\alpha)=\dfrac{2(1-\alpha)}{3(1+\alpha^2)}$.
\end{enumerate}

\bsc{Partie C}

Soit les fonctions $g$ et $h$ définies sur  $\mathbf{R}^*$ par :
\[g(x)=x^2\left(1-\dfrac{1}{x}\right) \text{~~et~~} h(x)=\dfrac{1}{2}\left(x+\dfrac{1}{x}\right).\]
\begin{enumerate}
  \item Conjecturer avec une calculatrice les positions des courbes  représentatives des fonctions $g$ et $h$.
  \item Montrer que $g(x)-h(x)=\dfrac{u(x)}{2x}$ puis, dresser un tableau de signes de $(g-h)(x)$ sur $\mathbf{R}^*$.\par
Le résultat est-il  conforme à la conjecture ?
\end{enumerate}
\end{question}

\end{document}
