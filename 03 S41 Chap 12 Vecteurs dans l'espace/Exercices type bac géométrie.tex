\documentclass[12pt,a4paper,french]{article}
\usepackage[utf8]{inputenc}
\usepackage[T1]{fontenc}
\usepackage{babel}
\usepackage[thmmarks]{ntheorem}
\usepackage{amsmath}
\usepackage{amsfonts}
\usepackage{amssymb}

\usepackage{array}

\usepackage{lmodern}
\usepackage{kpfonts}

\usepackage[bookmarks=false,colorlinks,linkcolor=blue,pdfusetitle]{hyperref}

\pdfminorversion 7
\pdfobjcompresslevel 3

\usepackage{tabularx}
\usepackage[autolanguage,np]{numprint}
\usepackage{enumitem}

\usepackage{tipfr}
\usepackage{pgf}
\usepackage{tikz}
\usepackage{tkz-euclide}
\usetkzobj{all}
%\usetikzlibrary{hobby}
\usepackage{tkz-tab}

\usepackage[top=1.7cm,bottom=2cm,left=2cm,right=2cm]{geometry}

\usepackage{lastpage}

\usepackage{esvect}
\usepackage{marginnote}

\usepackage{wrapfig}

\usepackage[defaultlines=5,all]{nowidow}


\makeatletter
\renewcommand{\@evenfoot}%
        {\hfil \upshape \small page {\thepage} de \pageref{LastPage}}
\renewcommand{\@oddfoot}{\@evenfoot}

\renewcommand{\maketitle}%
{\framebox{%
    \begin{minipage}{1.0\linewidth}%
      \begin{center}%
        \Large \@title ~-- \@author \\%
        \@date%
      \end{center}%
    \end{minipage}}%
  \normalsize%
  %\vspace{1cm}%
}

\pgfdeclarepatternformonly{mes_hachures}
{\pgfpoint{-0.1cm}{-0.1cm}}
{\pgfpoint{0.9cm}{0.5cm}}
{\pgfpoint{0.8cm}{0.4cm}}
{\pgfpathmoveto{\pgfpointorigin}
  \pgfpathlineto{\pgfpoint{0.8cm}{0.4cm}}
\pgfusepath{stroke}}

%Des macros pour les noms d'ensmbles
\newcommand{\R}{\mathbf{R}}
\newcommand{\Q}{\mathbf{Q}}
\newcommand{\Z}{\mathbf{Z}}
\newcommand{\C}{\mathbf{C}}
\newcommand{\N}{\mathbf{N}}

\newcommand{\norme}[1]{\left\lVert #1 \right\rVert}
\newcommand{\abs}[1]{\left\lvert #1 \right\rvert}

%Une macro récursive pour l'intérieru des vecteurs
%http://tex.stackexchange.com/questions/19693/arguments-of-custom-commands-as-comma-separated-list

\newcommand\vecteur[2][\\]{%
    \global\def\my@delim{#1}%
    \left(\negthinspace\begin{matrix}
        \my@vector #2,\relax\noexpand\@eolst%
    \end{matrix}\right)}

%Une macro pour les vecteurs
\def\my@vector #1,#2\@eolst{%
   \ifx\relax#2\relax
      #1
   \else
      #1\my@delim
      \my@vector #2\@eolst%
   \fi}

%Une macro récursive pour mettre formater l'intérieur des intervalles
\def\my@intervalle #1;#2\@eolst{%
  \ifx\relax#2\relax
    #1
  \else
    \my@intervalle #2\@eolst%
  \fi}

%Quatre macros pour les quatres types d'intervalles
\newcommand{\interff}[1]{%
  \left[\my@intervalle #1;\relax\noexpand\@eolst%
  \right]
}
\newcommand{\interfo}[1]{%
  \left[\my@intervalle #1;\relax\noexpand\@eolst%
  \right[}
\newcommand{\interof}[1]{%
  \left]\my@intervalle #1;\relax\noexpand\@eolst%
  \right]}
\newcommand{\interoo}[1]{%
  \left]\my@intervalle #1;\relax\noexpand\@eolst%
  \right[}

\makeatother


\usepackage{mdframed}

\theoremstyle{break}
\newtheorem{definition}{Définition}
\newtheorem{propriete}{Propriété}
\newtheorem{corollaire}{Corollaire}
\newtheorem{propdef}{Propriété - Définition}
\newtheorem{theoreme}{Théorème}
\theoremstyle{plain}
\theorembodyfont{\normalfont}
\newtheorem{exerciceT}{Exercice}
\theoremstyle{nonumberplain}
\newtheorem{remarque}{Remarque}
\newtheorem{notation}{Notation}
\newtheorem{probleme}{Problème}
\theoremsymbol{\ensuremath{\blacksquare}}
\newtheorem{preuve}{Preuve}
\theoremsymbol{}
\theoremstyle{nonumberbreak}
\newtheorem{exemple}{Exemple}

\newenvironment{exercice}{\begin{framed}\begin{exerciceT}}{\end{exerciceT}\end{framed}}

\setlength{\parsep}{0pt}
\setlength{\parskip}{5pt}
\setlength{\parindent}{0pt}
\setlength{\itemsep}{7pt}

\setlist{noitemsep}
%\setlist[1]{\labelindent=\parindent} % < Usually a good idea
\setlist[itemize]{leftmargin=*}
\setlist[itemize,1]{label=$\triangleright$}
\setlist[enumerate]{labelsep=*, leftmargin=1.5pc}
\setlist[enumerate,1]{label=\textbf{\arabic*.}, ref=\textbf{\arabic*}}
\setlist[enumerate,2]{label=\textbf{\alph*}),
ref=\theenumi.\textbf{\alph*}}
\setlist[enumerate,3]{label=\roman*), ref=\theenumii.\roman*}
\setlist[description]{font=\sffamily\bfseries}

\usepackage{multicol}
\setlength{\columnseprule}{0pt}

\usepackage[]{exsheets}
\SetupExSheets{
  headings = block,
  %question/pre-hook = \mdframed,
  %question/post-hook = \endmdframed,
}

\everymath{\displaystyle\everymath{}}

\title{Exercices : géométrie dans l'espace}
\author{\bsc{Ts 3}}
\date{2016}

\begin{document}

\maketitle

\begin{question}[ID=Metropole_juin_2015]
  Dans un repère orthonormé $(O,I,J,K)$ d'unité \np[cm]{1}, on consièdre
  les points $A \vecteur{0;-1;5}$, $B \vecteur{2;-1;5}$, $C
  \vecteur{11;0;1}$, $D \vecteur{11;4;4}$.

  Un point $M$ se déplace sur la droite $(AB)$ de $A$ vers $B$ à la
  vitesse de \np[cm]{1} par seconde.

  Un point $N$ se déplace sur la droite $(CD)$ de $C$ vers $D$ à la
  vitesse de \np[cm]{1} par seconde.

  À l'instant $t = 0$, le point $M$ est en $A$ et le point $N$ est en
  $C$.

  On note $M_t$ et $N_t$ les positions des points $M$ et $N$ au bout de
  $t$ secondes, $t$ désignant un nombre réel positif.

  On admet que $M_t$ et $N_t$ on pour coordonnées :

  $M_t\vecteur{t;-1;5}$ et $N_t \vecteur{11;\np{0.8}t;1+\np{0.6}t}$.

  \emph{Les questions 1 et 2 sont indépendantes.}

  \begin{enumerate}
    \item \begin{enumerate}
        \item La droite $(AB)$ est parallèle à l'un des axes $(OI)$,
          $(OJ)$ ou $(OK)$. Lequel ?
        \item La droite $(CD)$ se trouve dans un plan $\mathscr{P}$
          parallèle à $(OIJ)$, $(OIK)$ ou $(OJK)$.
        \item Vérifier que la droite $(AB)$, orthogonale au plan
          $\mathscr{P}$ coupe ce plan au point $E \vecteur{11;-1;5}$
        \item Les droites $(AB)$ et $(CD)$ sont elles sécantes ?
      \end{enumerate}
    \item \begin{enumerate}
        \item Montrer que $M_tN_t^2 = 2t^2 - \np{25.2}t + 138$.
        \item À quel instant $t$ la longueur $M_tN_t$ est elle minimale.
      \end{enumerate}
  \end{enumerate}
\end{question}
\begin{solution}
\end{solution}

\begin{question}[ID=Métropole_juin_2014]
  Dans l’espace, on considère un tétraèdre $ABCD$ dont les faces $ABC$,
  $ACD$ et $ABD$ sont des triangles rectangles et isocèles en $A$. On
  désigne par $E$, $F$ et $G$ les milieux respectifs des côtés $[AB]$,
  $[BC]$ et $[CA]$.

  On choisit $AB$ pour unité de longueur et on se place dans le repère
  orthonormé $\vecteur{A;\vv{AB};\vv{AC};\vv{AD}}$ de l'espace.

  \begin{enumerate}
    \item On désigne par $\mathscr{P}$ le plan  qui passe par $A$ et qui
      est orthogonal à la droite $(DF)$.

      On note $H$ le point d'intersection du plan $\mathscr{P}$ et de la
      droite $(DF)$.
      \begin{enumerate}
        \item Donner les coordonnées des points $D$ et $F$.
        \item Donner une représentation paramétrique de la droite
          $(DF)$.
        \item Déterminer une équation cartésienne du plan $\mathscr{P}$.
        \item Calculer les coordonnées du point $H$.
        \item Démontrer que l'angle $\widehat{EHG}$ est un angle droit.
      \end{enumerate}
    \item On désigne par $M$ un point de la droite $(DF)$ et par le $t$
      réel tel que $\vv{DM} = t\vv{DF}$.

      On note $\alpha$ la mesure en radians de l'angle géométrique
      $\widehat{EMG}$.

      Le but de cette question est de déterminer la position du point
      $M$ pour que $\alpha$ soit maximal.

      \begin{enumerate}
        \item Démontrer que $ME^2 = \frac32t^2 - \frac52t + \frac54$.
        \item Démontrer que le triangle $MEG$ est isocèle en $M$.

          En déduire que $ME\sin\left(\frac{\alpha}2\right) =
          \frac1{2\sqrt{2}}$.
        \item Justifier que $\alpha$ est maximale si et seulement si
          $\sin\left(\frac{\alpha}2\right)$ est maximal.

          En déduire que $\alpha$ est maximale si et seulement si $ME^2$ est
          minimal.
        \item Conclure.
      \end{enumerate}
  \end{enumerate}
\end{question}

\begin{question}[ID=Métropole_septembre_2010]
  Le plan est rapporté à un repère orthornormal
  $\vecteur{O;\vv{i};\vv{j};\vv{k}}$.

  Soit $(\mathscr{P})$ le plan d'équation : $3x + y - z - 1 = 0$ et
  $(\mathscr{D})$ la droite dont une représentation paramétrique est
  \[ \left\lbrace \begin{array}{ccc}
      x & = & -t + 1 \\
      y & = & 2t \\
      z & = & -t +2
  \end{array}\right. \text{où $t$ désigne un nombre réel.}\]

  \begin{enumerate}
    \item \begin{enumerate}
        \item Le point $C \vecteur{1;3;2}$ appartient-il au plan
          $(\mathscr{P})$ ? Justifier.
        \item Démontrer que la droite $(\mathscr{D})$ est incluse dans
          le plan $(\mathscr{P})$.
      \end{enumerate}
    \item Soit $(\mathscr{Q})$ le plan passant par le point $C$ et
      orthogonal à la droite $(\mathscr{D})$.
      \begin{enumerate}
        \item Déterminer une équation cartésienne du plan
          $(\mathscr{Q})$.
        \item Calculer les coordonnées du point $I$, point
          d'intersection du plan $(\mathscr{Q})$ et de la droite
          $(\mathscr{D})$.
        \item Montrer que $CI = \sqrt{3}$.
      \end{enumerate}
    \item Soit $t$ un nombre réel et $M_t$ le point de la droite
      $(\mathscr{D})$ de coordonnées $\vecteur{-t +1;2t;-t +2}$.
      \begin{enumerate}
        \item Vérifier que pour tout nombre réel $t$, $CM_t^2 = 6t^2
          -12t + 9$.
        \item Montrer que $CI$ est la valeur minimale de $CM_t$ lorsque
          $t$ décrit l'ensemble des nombres réels.
      \end{enumerate}
  \end{enumerate}
\end{question}

\pagebreak
\printsolutions

\end{document}
