\documentclass[12pt,a4paper,french]{article}
\usepackage[utf8]{inputenc}
\usepackage[T1]{fontenc}
\usepackage{babel}
\usepackage[thmmarks]{ntheorem}
\usepackage{amsmath}
\usepackage{amsfonts}
\usepackage{amssymb}

\usepackage{array}

\usepackage{lmodern}
\usepackage{kpfonts}

\usepackage[bookmarks=false,colorlinks,linkcolor=blue,pdfusetitle]{hyperref}

\pdfminorversion 7
\pdfobjcompresslevel 3

\usepackage{tabularx}
\usepackage[autolanguage,np]{numprint}
\usepackage[inline]{enumitem}

\usepackage{tipfr}
\usepackage{pgf}
\usepackage{tikz}
\usepackage{tkz-euclide}
\usetkzobj{all}
%\usetikzlibrary{hobby}
\usepackage{tkz-tab}

\usepackage[top=1.7cm,bottom=2cm,left=2cm,right=2cm]{geometry}

\usepackage{lastpage}

\usepackage{esvect}
\usepackage{marginnote}

\usepackage{wrapfig}

\usepackage[defaultlines=5,all]{nowidow}

\usepackage[extdef]{delimset}

\makeatletter
\renewcommand{\@evenfoot}%
        {\hfil \upshape \small page {\thepage} de \pageref{LastPage}}
\renewcommand{\@oddfoot}{\@evenfoot}

\renewcommand{\maketitle}%
{\framebox{%
    \begin{minipage}{1.0\linewidth}%
      \begin{center}%
        \Large \@title ~-- \@author \\%
        \@date%
      \end{center}%
    \end{minipage}}%
  \normalsize%
  %\vspace{1cm}%
}

\pgfdeclarepatternformonly{mes_hachures}
{\pgfpoint{-0.1cm}{-0.1cm}}
{\pgfpoint{0.9cm}{0.5cm}}
{\pgfpoint{0.8cm}{0.4cm}}
{\pgfpathmoveto{\pgfpointorigin}
  \pgfpathlineto{\pgfpoint{0.8cm}{0.4cm}}
\pgfusepath{stroke}}

%Des macros pour les noms d'ensmbles
\newcommand{\R}{\mathbf{R}}
\newcommand{\Q}{\mathbf{Q}}
\newcommand{\Z}{\mathbf{Z}}
\newcommand{\C}{\mathbf{C}}
\newcommand{\N}{\mathbf{N}}

\makeatother


\usepackage{mdframed}

\theoremstyle{break}
\newtheorem{definition}{Définition}
\newtheorem{propriete}{Propriété}
\newtheorem{corollaire}{Corollaire}
\newtheorem{propdef}{Propriété - Définition}
\newtheorem{theoreme}{Théorème}
\theoremstyle{plain}
\theorembodyfont{\normalfont}
\newtheorem{exerciceT}{Exercice}
\theoremstyle{nonumberplain}
\newtheorem{remarque}{Remarque}
\newtheorem{notation}{Notation}
\newtheorem{probleme}{Problème}
\theoremsymbol{\ensuremath{\blacksquare}}
\newtheorem{preuve}{Preuve}
\theoremsymbol{}
\theoremstyle{nonumberbreak}
\newtheorem{exemple}{Exemple}

\newenvironment{exercice}{\begin{framed}\begin{exerciceT}}{\end{exerciceT}\end{framed}}

\setlength{\parsep}{0pt}
\setlength{\parskip}{5pt}
\setlength{\parindent}{0pt}
\setlength{\itemsep}{7pt}

\setlist{noitemsep}
%\setlist[1]{\labelindent=\parindent} % < Usually a good idea
\setlist[itemize]{leftmargin=*}
\setlist[itemize,1]{label=$\triangleright$}
\setlist[enumerate]{labelsep=*, leftmargin=1.5pc}
\setlist[enumerate,1]{label=\textbf{\arabic*.}, ref=\textbf{\arabic*}}
\setlist[enumerate,2]{label=\textbf{\alph*}),
ref=\theenumi.\textbf{\alph*}}
\setlist[enumerate,3]{label=\roman*), ref=\theenumii.\roman*}
\setlist[description]{font=\sffamily\bfseries}

\usepackage{multicol}
\setlength{\columnseprule}{0pt}

\usepackage[]{exsheets}
\SetupExSheets{
  headings = block,
  %question/pre-hook = \mdframed,
  %question/post-hook = \endmdframed,
}

\everymath{\displaystyle\everymath{}}

\title{Exercices : géométrie dans l'espace}
\author{\bsc{Ts 3}}
\date{2016}

\begin{document}

\maketitle

\begin{question}

  \textbf{Partie A}

  L'espace est rapporté à un repère orthonormé $\brk*{O, \vv{\imath},
  \vv{\jmath}, \vv{k}}$. On donne les points :

  $A \brk*{-1;0;2}$, $B \brk*{3;2;-4}$, $C \brk*{1;-4;2}$ et $D
  \brk*{5;-2;4}$.

  $I$ et $K$ sont les milieux respectifs des segments $\brk[s]*{AB}$ et
  $\brk[s]*{CD}$. Le point $J$ est tel que $\vv{BJ} = \dfrac14\vv{BC}$

  \begin{enumerate}
    \item
      \begin{enumerate}
        \item Déterminer les coordonnées des points $I$, $J$ et $K$.
        \item Démontrer que ces trois points ne sont pas alignés.
      \end{enumerate}
    \item On donne les vecteurs $\vv{u} \brk*{\begin{matrix} 1 \\ -2
      \\2 \end{matrix}}$ et $\vv{v} \brk*{\begin{matrix} 3 \\ -1 \\ -3
      \end{matrix}}$

      Démontrer que les plans $\brk*{IJK}$ et $\mathscr{P}\brk*{I,
      \vv{u}, \vv{v}}$ sont confondus.
    \item
      \begin{enumerate}
        \item Déterminer une représentation paramétrique de la droite
          $\brk*{AD}$.
        \item Démontrer que le plan $\mathscr{P}$ et la droite
          $\brk*{ArDB}$ sont sécants au point $L$ de coordonnées
          $\brk*{\frac12 ; -\frac12 ; \frac52}$
        \item Vérifier que $\vv{AL} = \frac14 \vv{AD}$
      \end{enumerate}
  \end{enumerate}

  \textbf{Partie B. Généralisation}

  $ABCD$ est un tétraèdre. $I$ et $K$ sont les milieux respectifs des
  arêtes $\brk[s]*{AB}$ et $\brk[s]*{CD}$. Les points $L$ et $J$ sont
  tels que : \[ \vv{AL} = \frac14 \vv{AD} \text{ et } \vv{BJ} = \frac14
  \vv{BC} .\] En choisissant un repère adapté, démontrer que les droites
  $\brk{IK}$ et $\brk{LJ}$ sont concourantes et que les points $I,J,K,L$
  sont coplanaires.

\end{question}
\begin{solution}
  \textbf{Partie A}

  \begin{enumerate}
    \item
      \begin{enumerate}
        \item $I\brk*{1;1;-1}$ et $K\brk*{3;-3;3}$ et utilisant le fait
          que les coordonnées d'un milieu sont la demi-somme des
          coordonnées des extrémités. Pour le point $J$, on peut écrire
          \[
            \begin{array}{rc}
                   & \vv{BJ} = \frac14 \vv{BC} \\
              \iff &                           \\
                   & \vv{BO} + \vv{OJ} = \frac14 \vv{BO} + \frac14
              \vv{OC} \\
              \iff & \\
                   & \vv{OJ} = \frac34 \vv{OB} + \frac14 \vv{OC}
            \end{array}
          \]
          On a donc, en travaillant sur les coordonnées $J\brk*{-\frac12
          ; -\frac32 ; \frac32}$.
        \item On a $\vv{IJ} : \brk*{\begin{matrix}-\frac32 \\ -\frac52 \\
          \frac52 \end{matrix}}$ et $\vv{IK} : \brk*{\begin{matrix} 2
          \\ -4 \\ 4\end{matrix}}$.
          Les deuxième et troisième composantes n'ayant pas le même
          signe, on peut donc conclure que les vecteurs $\vv{IJ}$ et
          $\vv{IK}$ ne sont pas colinéaires et donc les points $I,J,K$
          ne sont pas alignés.
      \end{enumerate}
    \item La question précédente a permis de démontrer que $(IJK)$ est
      un plan.

      On a, par les coordonnées, $\vv{IK} = 2\vv{u}$.

      Les vecteurs $\vv{KJ}$ et $\vv{IJ}$ forment une combinaison
      linéaire de $\vv{v}$, on a donc que les deux plans sont parallèles
      ou confondus. Comme ils contiennent tous deux le point $I$, ils
      sont confondus.

    \item
      \begin{enumerate}
        \item Le vecteur $\vv{AD}$ a pour coordonnées
        $\brk*{\begin{matrix}6 \\ -2 \\ 2\end{matrix}}$. On en déduit
          l'équation paramétriqued de la droite \[ \left\lbrace
            \begin{matrix} x= -1 + 6t \\ y = -2t \\ z=
          2+2t\end{matrix}\right., t\in \R. \]
        \item Il suffit de vérifier que le point $L$ appartient bien aux
          deux ensembles. Pour la droite, on a la première composante
          qui donne $t = \frac14$. Vérifions que cette valeur convient
          pour les autres composantes : on retrouve bien $y =
          \frac{-1}2$ et $z = \frac52$. Donc $L$ appartient à la droite
          $\brk{AD}$.

          Pour le plan, on obtient

        \item On a $\vv{AL} = \brk*{\begin{matrix} \frac32 \\ -\frac12
          \\ \frac12\end{matrix}}$. On a donc l'égalité $\vv{AL} =
          \frac14 \vv{AD}$.
      \end{enumerate}
  \end{enumerate}

  \textbf{Partie B. Généralisation}

  On se place dans le repère $\brk{D,\vv{DA},\vv{DB},\vv{DC}}$. Dans ce
  repère, les coordonnées sont :
  \begin{itemize*}
    \item $I : \brk*{\frac12 ; \frac12 ; 0}$
    \item $K : \brk*{0 ; 0 ; \frac12}$
    \item $L : \brk*{\frac14 ; 0 ; 0}$
    \item $J : \brk*{0 ; \frac34 ; \frac14}$
  \end{itemize*}

  Les vecteurs $\vv{IK}$ et $\vv{LJ}$ sont non colinéaires donc les
  droites qu'ils portent sont sécantes en un point $S$.

  On peut désormais considérer le plan
  $\mathscr{Q}\brk*{S,\vv{IK},\vv{LJ}}$ qui contient les points
  $I,J,K,L$. \hfill$\square$
\end{solution}


\pagebreak
\printsolutions

\end{document}
