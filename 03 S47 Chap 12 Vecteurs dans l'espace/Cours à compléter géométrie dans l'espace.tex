\documentclass[12pt,a4paper,french]{article}
\usepackage[utf8]{inputenc}
\usepackage[T1]{fontenc}
\usepackage{babel}
\usepackage{ntheorem}
\usepackage{amsmath}
\usepackage{amsfonts}
\usepackage{amssymb}

\usepackage{array}

\usepackage{kpfonts}

\usepackage[bookmarks=false,colorlinks,linkcolor=blue,pdfusetitle]{hyperref}

\pdfminorversion 7
\pdfobjcompresslevel 3

\usepackage{tabularx}
\usepackage[autolanguage,np]{numprint}
\usepackage{enumitem}

\usepackage{tipfr}
\usepackage{pgf}
\usepackage{tikz}
\usepackage{tkz-euclide}
\usetkzobj{all}
\usetikzlibrary{hobby}

\usepackage[top=1.7cm,bottom=2cm,left=2cm,right=2cm]{geometry}

\usepackage{lastpage}

\usepackage{esvect}
\usepackage{marginnote}

\usepackage{wrapfig}

\usepackage[defaultlines=5,all]{nowidow}


\makeatletter
\renewcommand{\@evenfoot}%
        {\hfil \upshape \small page {\thepage} de \pageref{LastPage}}
\renewcommand{\@oddfoot}{\@evenfoot}

\renewcommand{\maketitle}%
{\framebox{%
    \begin{minipage}{1.0\linewidth}%
      \begin{center}%
        \Large \@title ~-- \@author \\%
        \@date%
      \end{center}%
    \end{minipage}}%
  \normalsize%
  %\vspace{1cm}%
}

\pgfdeclarepatternformonly{mes_hachures}
{\pgfpoint{-0.1cm}{-0.1cm}}
{\pgfpoint{0.9cm}{0.5cm}}
{\pgfpoint{0.8cm}{0.4cm}}
{\pgfpathmoveto{\pgfpointorigin}
  \pgfpathlineto{\pgfpoint{0.8cm}{0.4cm}}
\pgfusepath{stroke}}

%Des macros pour les noms d'ensmbles
\newcommand{\R}{\mathbf{R}}
\newcommand{\Q}{\mathbf{Q}}
\newcommand{\Z}{\mathbf{Z}}
\newcommand{\C}{\mathbf{C}}
\newcommand{\N}{\mathbf{N}}

\usepackage{framed}

\newcommand{\norme}[1]{\left\lVert #1 \right\rVert}
\newcommand{\abs}[1]{\left\lvert #1 \right\rvert}

%Une macro récursive pour l'intérieru des vecteurs
%http://tex.stackexchange.com/questions/19693/arguments-of-custom-commands-as-comma-separated-list

\newcommand\vecteur[2][\\]{%
    \global\def\my@delim{#1}%
    \left(\negthinspace\begin{smallmatrix}
        \my@vector #2,\relax\noexpand\@eolst%
    \end{smallmatrix}\right)}

%Une macro pour les vecteurs
\def\my@vector #1,#2\@eolst{%
   \ifx\relax#2\relax
      #1
   \else
      #1\my@delim
      \my@vector #2\@eolst%
   \fi}

%Une macro récursive pour mettre formater l'intérieur des intervalles
\def\my@intervalle #1;#2\@eolst{%
  \ifx\relax#2\relax
    #1
  \else
    \my@intervalle #2\@eolst%
  \fi}

%Quatre macros pour les quatres types d'intervalles
\newcommand{\interff}[1]{%
  \left[\my@intervalle #1;\relax\noexpand\@eolst%
  \right]
}
\newcommand{\interfo}[1]{%
  \left[\my@intervalle #1;\relax\noexpand\@eolst%
  \right[}
\newcommand{\interof}[1]{%
  \left]\my@intervalle #1;\relax\noexpand\@eolst%
  \right]}
\newcommand{\interoo}[1]{%
  \left]\my@intervalle #1;\relax\noexpand\@eolst%
  \right[}

%\newcommand*{\versioncomplet}{}
\newcommand{\complet}[2]{%
  \ifdefined\versioncomplet
    \textcolor{blue}{#2}
  \else
   ~\\\hspace{0.98\linewidth}\vspace{#1}
  \fi
}

\makeatother


\theoremstyle{break}
\newtheorem{definition}{Définition}
\newtheorem{propriete}{Propriété}
\newtheorem{propdef}{Propriété - Définition}
\newtheorem{theoreme}{Théorème}
\theoremstyle{plain}
\theorembodyfont{\normalfont}
\newtheorem{exercice}{Exercice}
\theoremstyle{nonumberplain}
\newtheorem{remarque}{Remarque}
\newtheorem{probleme}{Problème}
\newtheorem{preuve}{Preuve}
\theoremstyle{nonumberbreak}
\newtheorem{exemple}{Exemple}

\newcommand{\qed}{\hfill$\Box$}

\setlength{\parsep}{0pt}
\setlength{\parskip}{5pt}
\setlength{\parindent}{0pt}
\setlength{\itemsep}{7pt}

\setlist{noitemsep}
%\setlist[1]{\labelindent=\parindent} % < Usually a good idea
\setlist[itemize]{leftmargin=*}
\setlist[itemize,1]{label=$\triangleright$}
\setlist[enumerate]{labelsep=*, leftmargin=1.5pc}
\setlist[enumerate,1]{label=\arabic*., ref=\arabic*}
\setlist[enumerate,2]{label=\emph{\alph*}),
ref=\theenumi.\emph{\alph*}}
\setlist[enumerate,3]{label=\roman*), ref=\theenumii.\roman*}
\setlist[description]{font=\sffamily\bfseries}

\usepackage{multicol}
\setlength{\columnseprule}{0pt}

\everymath{\displaystyle\everymath{}}

\title{Géométrie dans l'espace}
\author{Terminale S}
\date{octobre 2017}

\begin{document}

\maketitle

\section{Généralités sur les vecteurs}

La géométrie dans l'espace ne constitue pas une révolution par rapport à
la géométrie planaire et on se ramène bien souvent, par choix ou par
projection, à un plan particulier dans lequel le calcul ou le
raisonnement sera facilité.

Un vecteur $\vv{AB}$ de l'espace est toujours l'expression du
déplacement qui amène le point $A$ sur $B$.

\framebox{
  \begin{minipage}{0.99\linewidth}
    On dit souvent qu'un vecteur $\vv{AB}$ est défini par :
    \complet{25mm}{
      \begin{itemize}
        \item une direction ;
        \item un sens ;
        \item une norme.
      \end{itemize}
    }
  \end{minipage}
}

En particulier, on admet l'existence du déplacement nul et on note :

\framebox{
  \begin{minipage}{0.99\linewidth}
    \complet{5mm}{$\vv{AA} = \vv{0}$ pour tout point $A$ du plan (ou de
    l'espace)}
  \end{minipage}
}

L'égalité de deux vecteurs permet de définir un parallélogramme et
réciproquement, si quatre points sont les sommets d'un parallélogramme,
ils définissent l'égalité de deux vecteurs.

On rappelle qu'«on peut toujours faire un détour» : c'est l'égalité de
Chasles :

\framebox{
  \begin{minipage}{0.99\linewidth}
    \complet{5mm}{pour tous points $A,\ B$ et $C$ du plan (ou de
    l'espace), $\vv{AB} = \vv{AC} + \vv{CB}$.}
  \end{minipage}
}

Ce qu'il faut retenir : les propositions, théorèmes et méthodes qui
étaient valables dans le plan restent vrais dans l'espace.

\section{Vecteurs du plan et de l'espace}

\subsection{Vecteurs dans le plan}

Soient $A$, $B$ et $C$ trois points de l'espace (ou d'un plan qu'ils
définissent) non alignés.

\framebox{
  \begin{minipage}{0.99\linewidth}
    Le plan $(ABC)$ est
    \complet{15mm}{l'ensemble des points $M$ tels qu'il existe $x$ et
    $y$ réels tel que $\vv{AM} = x\vv{AB} + y\vv{AC}$.}
  \end{minipage}
}

Cette caractérisation du plan nous permet de définir la notion de
vecteurs coplanaires :

\framebox{
  \begin{minipage}{0.99\linewidth}
    \begin{definition}[Vecteurs coplanaires]
      \complet{15mm}{Trois vecteurs sont dit coplanaires s'il l'un
        d'entre eux est une combinaison linéaire des deux autres, c'est
        dire s'il existe $\alpha$ et $\beta$ réels tels que $\vv{w} =
      \alpha\vv{u} + \beta\vv{v}.$}
    \end{definition}
  \end{minipage}
}

\begin{remarque}~\\[-9mm]

  \begin{itemize}
    \item On utilise le fait que tout point (ou tout vecteur) d'un plan
      peut s'exprimer comme combinaison linéaire de deux vecteurs
      non-colinéaire.
    \item On peut aussi caractériser une droite $(AB)$ comme l'ensemble
      des points $M$ tels que $\vv{AM} = t\vv{AB}$.
    \item On peut désormais définir un plan par deux vecteurs non
      colinéaires et un point.
  \end{itemize}
\end{remarque}

\begin{framed}
  \begin{propriete}
    On dit que trois vecteurs sont coplanaires lorsque la seule
    combinaison linéaire $a\vv{u} + b\vv{v} + c\vv{w} = \vv{0}$ est la
    combinaison $(a,b,c) = (0,0,0)$.
  \end{propriete}
\end{framed}

\subsection{Vecteurs dans l'espace}

Soient $(\vv{u};\vv{v};\vv{w})$ trois vecteurs non-coplanaires et $O$ un
point de l'espace.

\framebox{
  \begin{minipage}{0.99\linewidth}
    \begin{theoreme}[Décomposition]
      Tout point $M$ de l'espace,
      \complet{15mm}{se décompose de façon unique de la façon suivante :
      $\vv{OM} = x\vv{u} + y\vv{v} + z\vv{w}$}

      Tout vecteur $\vv{AB}$ de l'espace,
      \complet{15mm}{se décompose de façon unique de la façon suivante :
      $\vv{s} = x\vv{u} + y\vv{v} + z\vv{w}$}
    \end{theoreme}
  \end{minipage}
}

On admet que cette écriture est unique, c'est à dire que pour deux
points $M$ et $M'$ distints, les triplets $(x;y;z)$ et $(x';y';z')$ sont
différents et réciproquement.

\begin{definition}
  Avec la décomposition précédente :
  \begin{itemize}
    \item Les coordonnées du point $M$ dans le repère
      $(O;\vv{u};\vv{v};\vv{w})$ est le triplet $(x;y;z)$;
    \item les coordonnées du vecteur $\vv{AB}$ dans la base
      $(\vv{u};\vv{v};\vv{w})$.
  \end{itemize}
\end{definition}

Pour le calcul en pratique, on reprendra avec les coordonnées dans
l'espace les mêmes techniques et propriétés que dans l'esapce en
ajoutant une troisième coordonnées.

Ainsi, si $\vv{u} = \vecteur{x,y,z}$ et $\vv{v} = \vecteur{x',y',z'}$,
on a pour tout $\alpha$ et $\beta$ réels, \[ \alpha\vv{u} + \beta\vv{v}
= \complet{-4mm}{\vecteur{\alpha x + \beta x' ,\alpha y + \beta y' ,
\alpha z + \beta z' }}\] %\vecteur{. ,. ,. } \]

Rappellons également que dans une même base, deux vecteurs sont égaux si
et seulement s'ils ont les mêmes coordonnées.

Si $A$ a pour coordonnées $\vecteur[,&]{x_a,y_a,z_a}$ et $B$ a pour
coordonnées $\vecteur[,&]{x_b,y_b,z_b}$, dans le même repère alors :
\begin{itemize}
  \item $\vv{AB} = \complet{-4mm}{(x_b - x_a;\ y_b - y_a;\ z_b - z_a)}$
  \item $I$ milieu de $AB$ a pour coordonnées :
    $\complet{-4mm}{\left(\frac{x_b + x_a}2;\ \frac{y_b + y_a}2;\
      \frac{z_b + z_a}2 \right)}$
\end{itemize}

On démontre également que la norme d'un vecteur $\vv{u} =
\vecteur{x,y,z}$ se calcule avec $\norme{\vv{u}} = \sqrt{x^2 + y^2 +
z^2}$.

Cette dernière proposition n'est vraie que pour un plan dans lequel on
a défini le produit scalaire canonique (voir cours sur le produit
scalaire) et si on se place dans la base canonique pour ce produit
scalaire, c'est à dire la base $(\vv{\imath},\vv{\jmath},\vv{k})$ où
$\vv{\imath} = \vecteur{1,0,0}$, $\vv{\jmath} = \vecteur{0,1,0}$ et
$\vv{k} = \vecteur{0,0,1}$

\begin{propriete}
  On dit qu'une droite est parallèle à un plan si elle est dirigée par
  un vecteur colinéaire combinaison linéaire de deux vecteurs non nuls
  du plan.
\end{propriete}

\begin{exercice}
  Traduire cette proposition en terme d'égalité de vecteurs
  \complet{15mm}{Si $\vv{w}$ est un vecteur dircteur de $\mathcal{D}$ et
    que $\mahtcal{P}$ est dirigé par $\vv{u}$ et $\vv{v}$, la condition
  s'écrit $\vv{w} a \vv{u} + b\vv{v}$.}
\end{exercice}

\pagebreak

\section{Équations de plans et de droites}

\subsection{Équations de droites dans le plan, de plan dans l'espace}

On rappelle que dans le plan, on a :

\framebox{%
  \begin{minipage}{0.9\linewidth}
    \complet{10mm}{%
      Dans le plan, $ax + by + c =0$ est une équation de droite.
  }\end{minipage}
}

Cependant, on ne peut pas généraliser directement à l'espace :

\framebox{%
  \begin{minipage}{0.9\linewidth}
      Dans l'espace, $ax + by + cz + d =0$ est une équation de plan.
  \end{minipage}
}

Cette dernière notion sera développée dans le chapitre sur l'orthogonalité
dans l'espace.

\subsection{Équation paramétrique d'une droite}

Soit $M: \vecteur[,&]{x_0;y_0;z_0}$ un point et $\vv{u} =
\vecteur{a,b,c}$ un vecteur directeur de la droite $\mathscr{D}$.

\framebox{%
  \begin{minipage}{0.9\linewidth}
    \begin{definition}
    \complet{20mm}{%
      On appelle équation paramétrique de la droite l'ensemble des
      points  de coordonnées \[ \left\lbrace\begin{array}{r} x = x_0 +
      at \\ y = y_0 + bt \\ z = z_0 + ct \end{array}\right.,\ t\in\R \]
  }\end{definition}
\end{minipage}
}

\framebox{%
  \begin{minipage}{0.9\linewidth}
    \begin{preuve}
    \complet{20mm}{%
      \begin{itemize}
        \item Soit $P$ un point de $\mathscr{D}$. D'après la définition
          précédente, $\vv{MP} = t\vv{u}$, donc $\vv{u} $ est un vecteur
          directeur.
        \item Soit $P$ un point de la droite $\mathscr{D}$. Il s'exprime
          comme $\vv{MP} = \lambda\vv{u},\ \lambda\in\R$. On a donc que
          les coordonnées de $\vv{MP}$ sont $\vv{MP} = \vecteur{\lambda
          a,\lambda b,\lambda c} = \vecteur{x - x_0, y - y_0, z -z_0}$
      \end{itemize}
      \qed
  }\end{preuve}
\end{minipage}
}



\end{document}
