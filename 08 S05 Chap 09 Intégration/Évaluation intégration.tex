\documentclass[12pt,a4paper,french]{article}
\usepackage[utf8]{inputenc}
\usepackage[T1]{fontenc}
\usepackage{babel}
\usepackage[thmmarks]{ntheorem}
\usepackage{amsmath}
\usepackage{amsfonts}
\usepackage{amssymb}

\usepackage{array}

\usepackage{lmodern}
\usepackage{kpfonts}

\usepackage[bookmarks=false,colorlinks,linkcolor=blue,pdfusetitle]{hyperref}

\pdfminorversion 7
\pdfobjcompresslevel 3

\usepackage{tabularx}
\usepackage[autolanguage,np]{numprint}
\usepackage{enumitem}

\usepackage{tipfr}
\usepackage{pgf}
\usepackage{tikz}
\usepackage{tkz-euclide}
\usetkzobj{all}
\usetikzlibrary{hobby}
\usepackage{tkz-tab}

\usepackage[top=1.9cm,bottom=2cm,left=2cm,right=2cm]{geometry}

\usepackage{lastpage}

\usepackage{esvect}
\usepackage{marginnote}

\usepackage{wrapfig}

\usepackage[defaultlines=5,all]{nowidow}


\usepackage[]{algorithm2e}

\usepackage{ifthen}
\usepackage{fancyhdr}
\pagestyle{fancy}


\makeatletter

\count1=\year \count2=\year
\ifnum\month<8\advance\count1by-1\else\advance\count2by1\fi

\setlength{\headheight}{14.5pt}
\renewcommand{\headrulewidth}{0pt}
\renewcommand{\footrulewidth}{0pt}
\cfoot{\textsl{\footnotesize{Année \number\count1/\number\count2}}}

\rfoot{%
  \ifthenelse{\value{page}=1}{%
  }
  {%
    \footnotesize{Page \thepage/ \pageref{LastPage}}
  }
}

\rhead{}

\lhead{%
  \ifthenelse{\value{page}=1}{%
    Nom:\dotfill\hfill Prénom: \dotfill \hfill
    Classe: \@author \dots%
  }
  { }
}

\renewcommand{\maketitle}%
{\framebox{%
    \begin{minipage}{1.0\linewidth}%
      \begin{center}%
        \Large \@title ~-- \@author \\%
        \@date%
      \end{center}%
    \end{minipage}}%
  \normalsize%
  %\vspace{1cm}%
}

%Des macros pour les noms d'ensmbles
\newcommand{\R}{\mathbf{R}}
\newcommand{\Q}{\mathbf{Q}}
\newcommand{\Z}{\mathbf{Z}}
\newcommand{\C}{\mathbf{C}}
\newcommand{\N}{\mathbf{N}}

\newcommand{\norme}[1]{\left\lVert #1 \right\rVert}
\newcommand{\abs}[1]{\left\lvert #1 \right\rvert}
\newcommand{\diff}{\mathop{}\mathopen{}\mathrm{d}}
%Une macro récursive pour l'intérieru des vecteurs
%http://tex.stackexchange.com/questions/19693/arguments-of-custom-commands-as-comma-separated-list

\newcommand\vecteur[2][\\]{%
    \global\def\my@delim{#1}%
    \left(\negthinspace\begin{matrix}
        \my@vector #2,\relax\noexpand\@eolst%
    \end{matrix}\right)}

%Une macro pour les vecteurs
\def\my@vector #1,#2\@eolst{%
   \ifx\relax#2\relax
      #1
   \else
      #1\my@delim
      \my@vector #2\@eolst%
   \fi}

%Une macro récursive pour mettre formater l'intérieur des intervalles
\def\my@intervalle #1;#2\@eolst{%
  \ifx\relax#2\relax
    #1
  \else
    \my@intervalle #2\@eolst%
  \fi}

%Quatre macros pour les quatres types d'intervalles
\newcommand{\interff}[1]{%
  \left[\my@intervalle #1;\relax\noexpand\@eolst%
  \right]
}
\newcommand{\interfo}[1]{%
  \left[\my@intervalle #1;\relax\noexpand\@eolst%
  \right[}
\newcommand{\interof}[1]{%
  \left]\my@intervalle #1;\relax\noexpand\@eolst%
  \right]}
\newcommand{\interoo}[1]{%
  \left]\my@intervalle #1;\relax\noexpand\@eolst%
  \right[}

\makeatother


\usepackage{framed}

\theoremstyle{break}
\newtheorem{definition}{Définition}
\newtheorem{propriete}{Propriété}
\newtheorem{corollaire}{Corollaire}
\newtheorem{propdef}{Propriété - Définition}
\newtheorem{theoreme}{Théorème}
\theoremstyle{plain}
\theorembodyfont{\normalfont}
\newtheorem{exerciceT}{Exercice}
\theoremstyle{nonumberplain}
\newtheorem{remarque}{Remarque}
\newtheorem{notation}{Notation}
\newtheorem{probleme}{Problème}
\theoremsymbol{\ensuremath{\blacksquare}}
\newtheorem{preuve}{Preuve}
\theoremsymbol{}
\theoremstyle{nonumberbreak}
\newtheorem{exemple}{Exemple}

\newenvironment{exercice}{\begin{framed}\begin{exerciceT}}{\end{exerciceT}\end{framed}}

\setlength{\parsep}{0pt}
\setlength{\parskip}{5pt}
\setlength{\parindent}{0pt}
\setlength{\itemsep}{7pt}

\setlist{noitemsep}
%\setlist[1]{\labelindent=\parindent} % < Usually a good idea
\setlist[itemize]{leftmargin=*}
\setlist[itemize,1]{label=$\triangleright$}
\setlist[enumerate]{labelsep=*, leftmargin=1.5pc}
\setlist[enumerate,1]{label=\arabic*., ref=\arabic*}
\setlist[enumerate,2]{label=\emph{\alph*}),
ref=\theenumi.\emph{\alph*}}
\setlist[enumerate,3]{label=\roman*), ref=\theenumii.\roman*}
\setlist[description]{font=\sffamily\bfseries}

\usepackage{multicol}
\setlength{\columnseprule}{0pt}

\usepackage[]{exsheets}
\SetupExSheets{headings=block}

\everymath{\displaystyle\everymath{}}

\title{Évaluation \no{10} : intégration}
\author{\bsc{Ts}}
\date{10 mars 2016}

\begin{document}

\maketitle

\begin{tabular}{|p{6em}|p{26em}|p{6em}|}\hline
   & & \\
   & & \\
   \hfill\Huge /\totalpoints* & & \\
   & & \\
   & & \\ \hline
\end{tabular}


\begin{question}[ID=primitives;suites;bac;liban;2010]
  ~\\[-6ex]
  \phantom{a}\hfill\textbf{(\GetQuestionProperty{points}{\CurrentQuestionID} points)}\\
  On considère la suite $(u_n)$ définie pour tout entier naturel $n$ par
  \[u_n = \int_0^1 \frac{e^{-nx}}{1 + e^{-x}}\diff x. \]

  \begin{enumerate}
    \item \begin{enumerate}
        \item Montrer que $u_0 + u_1 = 1$. \addpoints*{1}
        \item Calculer $u_1$. En déduire $u_0$. \addpoints*{1}
      \end{enumerate}
    \item Montrer que pour tout entier naturel $n$, $u_n \geqslant 0$.
      \addpoints*{1}
    \item \begin{enumerate}
        \item Montrer que pour tout entier naturel $n$, on a \[ u_n +
          u_{n+1} = \frac{1 - e^{-n}}n . \]
          \addpoints*{1}
        \item En déduire que pour tout entier naturel $n$ non nul, $u_n
          \leqslant \frac{1 - e^{-n}}{n}$.\addpoints*{1}
      \end{enumerate}
    \item Déterminer la limite de la suite $(u_n)$.\addpoints*{1}
  \end{enumerate}

  \blank[style=dotted,width=6\linewidth,linespread=1.7]{}

  \blank[style=dotted,width=6\linewidth,linespread=1.7]{}

  \blank[style=dotted,width=6\linewidth,linespread=1.7]{}

  \blank[style=dotted,width=6\linewidth,linespread=1.7]{}

  \blank[style=dotted,width=6\linewidth,linespread=1.7]{}

  \blank[style=dotted,width=6\linewidth,linespread=1.7]{}

  \blank[style=dotted,width=5\linewidth,linespread=1.7]{}
\end{question}
\begin{solution}
  Il s'agit du sujet du Liban 2010, mais qui est compatible avec le
  programme actuel.

  \begin{enumerate}
    \item \begin{enumerate}
        \item $u_0 + u_1 = \int_0^1 \frac{e^{-0x}}{1 + e^{-x}}\diff x +
          \int_0^1 \frac{e^{-1x}}{1 + e^{-x}}\diff x = \int_0^1 \frac{1
          + e^{-x}}{1 + e^{-x}}\diff x = \int_0^1 \diff x = 1$
        \item $u_1 = \int_0^1 \frac{e^{-1x}}{1 + e^{-x}}\diff x$. Posons
          $u(x) = 1 + e^{-x}$.

          On a $u'(x) = -e^{-x}$ et donc $u_1 = \int_0^1
          \frac{-u'(x)}{u(x)}\diff x = - \int_0^1
          \frac{u'(x)}{u(x)}\diff x$ Une primitive est $\ln(u(x)) =
          \ln(1+e^{-x})$ et donc $u_1 = -(\ln(1+e^{-1}) - \ln(1+e^{0}))
          = \ln(2) - \ln(1+e^{-1})$. On en déduit $u_0 = 1 - \ln(2) +
          \ln(1+e^{-1})$.
      \end{enumerate}
    \item Soit $n$ un entier naturel. $e^{-nx} > 0$, $1+e^{-x} > 0$ donc
      le quotient est positif. Comme l'intégrale d'une fonction positive
      est positive, on a la positivité de $u_n$.

      On peut donc écrire $\forall n \in\N,\ u_n \geqslant 0$.
    \item \begin{enumerate}
        \item Soit $n$ un entier positif. $u_n + u_{n+1} = \int_0^1
          \frac{e^{-nx}}{1 + e^{-x}}\diff x + \int_0^1
          \frac{e^{-(n+1)x}}{1 + e^{-x}}\diff x = \int_0^1 \frac{
          e^{-nx} + e^{-(n+1)x}}{1 + e^{-x}}\diff x = \int_0^1 \frac{
          e^{-nx} (1 + e^{-x})}{1 + e^{-x}}\diff x = \int_0^1 e^{-nx}
          \diff x$.

          Une primitive de $f\colon x\mapsto e^{-nx}$ est $F\colon
          x\mapsto \frac{-e^{-nx}}n$, on trouve que $u_n + u_{n+1} =
          F(1) - F(0) = \frac{-e^{-n}}n + \frac1n = \frac{1 - e^{-n}}n$.
        \item Pour tout entier $n$, $u_n \geqslant 0$ donc \emph{a
          fortiori} $u_{n+1}$. Donc $u_n \leqslant \frac{1 - e^{-n}}{n}$.
      \end{enumerate}
    \item $\lim_{n\to+\infty} \frac{1 - e^{-n}}n = 0$. De plus, pour
      tout entier naturel $n$ supérieur ou égal à 1, $0 \leqslant u_n
      \leqslant \frac{1 - e^{-n}}n$, donc avec le théorème d'encadrement
      on en déduit $\lim_{n\to+\infty} u_n = 0$.
  \end{enumerate}

\end{solution}


\newpage
\section*{Correction}
\printsolutions
\end{document}
