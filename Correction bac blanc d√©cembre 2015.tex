\documentclass[12pt,a4paper]{article}
\usepackage[utf8]{inputenc}
\usepackage[T1]{fontenc}
\usepackage[french]{babel}
\usepackage{ntheorem}
\usepackage{amsmath}
\usepackage{amsfonts}

\usepackage{kpfonts}
\usepackage{textcomp}
\newcommand{\euro}{\eurologo{}}

\usepackage[bookmarks=false,colorlinks,linkcolor=blue,pdfusetitle]{hyperref}
\usepackage{breakurl}

\pdfminorversion 7
\pdfobjcompresslevel 3

\frenchbsetup{ItemLabels=\textbullet,}

\usepackage{tabularx}
\usepackage{enumitem}

\usepackage{pgf}
\usepackage{tikz}
\usepackage{tkz-euclide}
\usetkzobj{all}
\usetikzlibrary{hobby}
\usepackage{tkz-tab}
\usepackage[top=1.4cm,bottom=1.4cm,left=2cm,right=2cm,includehead,
includefoot]{geometry}

\usepackage{lastpage}

\usepackage{fancybox}

\usepackage[autolanguage]{numprint}
\newcommand{\np}{\numprint}

\usepackage{ifthen}
\usepackage{fancyhdr}
\pagestyle{fancy}

\lhead{\ifthenelse{\value{page}=1}{Nom:\bsc{Jumel}\hfill Prénom:
    Vincent-Xavier \hfill ~}{}}
\rhead{}
\chead{}

\rfoot{\upshape \small page {\thepage} de \pageref{LastPage}}
\lfoot{}
\cfoot{}
\renewcommand{\headrulewidth}{0pt}
\renewcommand{\footrulewidth}{0pt}

\usepackage{pdfmarginpar}

\theoremstyle{break}
\newtheorem{definition}{Définition}
\theorembodyfont{\normalfont}
\theoremstyle{nobreak}
\newtheorem{exercice}{Exercice}

%\theoremstyle{plain}
\theoremstyle{nonumberplain}
\newtheorem{probleme}{Problème}

% Mise en forme des labels dans les énumérations
\renewcommand{\labelenumi}{\textbf{\theenumi.}}
\renewcommand{\labelenumii}{\textbf{\theenumii)}}
\renewcommand{\theenumi}{\arabic{enumi}}
\renewcommand{\theenumii}{\alph{enumii}}

\renewcommand{\labelitemi}{$\bullet$}

\setlength{\parsep}{0pt}
\setlength{\parskip}{5pt}
\setlength{\parindent}{0pt}
\setlength{\itemsep}{7pt}

\usepackage{multicol}
\setlength{\columnseprule}{0pt}

\everymath{\displaystyle\everymath{}}

\newcommand{\N}{\mathbf{N}}
\newcommand{\Z}{\mathbf{Z}}
\newcommand{\Q}{\mathbf{Q}}
\newcommand{\R}{\mathbf{R}}

\newcommand{\ligne}[1]{%
  \begin{tikzpicture}[]
    \draw[white] (0,#1+0.8) -- (15,#1+0.8) ;
    \foreach \i in {1,...,#1}
    { \draw[dotted] (0,\i) -- (\linewidth,\i) ; }
    \draw[white] (0,0.6) -- (15,0.6) ;
  \end{tikzpicture}%
}

\usepackage{xcolor}
\usepackage{framed}
\usepackage{algorithm}
\usepackage{algpseudocode}
\definecolor{fond}{RGB}{136,136,136}
\definecolor{sicolor}{RGB}{128,0,128}
\definecolor{tantquecolor}{RGB}{221,111,6}
\definecolor{pourcolor}{RGB}{187,136,0}
\definecolor{bloccolor}{RGB}{128,0,0}
\newenvironment{cadrecode}{%
  \def\FrameCommand{{\color{fond}\vrule width
  5pt}\fcolorbox{fond}{white}}%
  \MakeFramed {\hsize \linewidth \advance\hsize-\width
\FrameRestore}\begin{footnotesize}}%
{\end{footnotesize}\endMakeFramed}
\makeatletter
\def\therule{\makebox[\algorithmicindent][l]{\hspace*{.5em}\color{fond}
\vrule width 1pt height .75\baselineskip depth .25\baselineskip}}%
\newtoks\therules
\therules={}
\def\appendto#1#2{\expandafter#1\expandafter{\the#1#2}}
\def\gobblefirst#1{#1\expandafter\expandafter\expandafter{\expandafter\@gobble\the#1}}%
\def\Ligne{\State\unskip\the\therules}% 
\def\pushindent{\appendto\therules\therule}%
\def\popindent{\gobblefirst\therules}%
\def\printindent{\unskip\the\therules}%
\def\printandpush{\printindent\pushindent}%
\def\popandprint{\popindent\printindent}%
\def\Variables{\Ligne \textcolor{bloccolor}{\textbf{VARIABLES}}}
\def\Si#1{\Ligne \textcolor{sicolor}{\textbf{SI}} #1
\textcolor{sicolor}{\textbf{ALORS}}}%
\def\Sinon{\Ligne \textcolor{sicolor}{\textbf{SINON}}}%
\def\Pour#1#2#3{\Ligne \textcolor{pourcolor}{\textbf{POUR}} #1
  \textcolor{pourcolor}{\textbf{ALLANT\_DE}} #2
\textcolor{pourcolor}{\textbf{A}} #3}%
\def\Tantque#1{\Ligne \textcolor{tantquecolor}{\textbf{TANT\_QUE}} #1
\textcolor{tantquecolor}{\textbf{FAIRE}}}%
\algdef{SE}[WHILE]{DebutTantQue}{FinTantQue}
{\pushindent \printindent
\textcolor{tantquecolor}{\textbf{DEBUT\_TANT\_QUE}}}
{\printindent \popindent
\textcolor{tantquecolor}{\textbf{FIN\_TANT\_QUE}}}%
\algdef{SE}[FOR]{DebutPour}{FinPour}
{\pushindent \printindent
\textcolor{pourcolor}{\textbf{DEBUT\_POUR}}}
{\printindent \popindent
\textcolor{pourcolor}{\textbf{FIN\_POUR}}}%
\algdef{SE}[IF]{DebutSi}{FinSi}%
{\pushindent \printindent
\textcolor{sicolor}{\textbf{DEBUT\_SI}}}
{\printindent \popindent
\textcolor{sicolor}{\textbf{FIN\_SI}}}%
\algdef{SE}[IF]{DebutSinon}{FinSinon}
{\pushindent \printindent
\textcolor{sicolor}{\textbf{DEBUT\_SINON}}}
{\printindent \popindent
\textcolor{sicolor}{\textbf{FIN\_SINON}}}%
\algdef{SE}[PROCEDURE]{DebutAlgo}{FinAlgo}
{\printandpush
\textcolor{bloccolor}{\textbf{DEBUT\_ALGORITHME}}}%
{\popandprint
\textcolor{bloccolor}{\textbf{FIN\_ALGORITHME}}}%
\makeatother
\newenvironment{algobox}%
{%
  \begin{ttfamily}
    \begin{algorithmic}[1]
      \begin{cadrecode}
        \labelwidth 1.5em
        \leftmargin\labelwidth
        \addtolength{\leftmargin}{\labelsep}
      }
      {%
      \end{cadrecode}
    \end{algorithmic}
  \end{ttfamily}
}

\usepackage[gobble=auto]{pythontex}

\newcommand{\qed}{\hfill$\Box$}

\newcommand{\maximaout}[3]{%
\begin{framed}%
  {\ttfamily \textcolor{red}{(\%i#1)} #2\\%
      \textcolor{brown}{(\%o#1)} #3}%
\end{framed}%
}
\usepackage{wasysym}

\newcommand{\vf}{%
  \Square{}~ Vrai \-\- \Square{}~ Faux%
}

\renewcommand{\Vec}[1]{\overrightarrow{#1}}
\newcommand{\ProSca}[2]{#1 \cdot #2}
\newcommand{\abs}[1]{\left\lvert #1\right\rvert}

\usepackage{esvect}

\makeatletter
\renewcommand{\maketitle}%
{\framebox{%
    \begin{minipage}{0.98\linewidth}%
      \begin{center}%
        \Large \@title ~-- \@author \\%
        \@date%
      \end{center}%
  \end{minipage}}%
  \normalsize%
  %\vspace{1cm}%
}

\newcommand\vecteur[2][\\]{%
    \global\def\my@delim{#1}%
    \left(\negthinspace\begin{matrix}
        \my@vector #2,\relax\noexpand\@eolst%
    \end{matrix}\right)}

%Une macro pour les vecteurs
\def\my@vector #1,#2\@eolst{%
   \ifx\relax#2\relax
      #1
   \else
      #1\my@delim
      \my@vector #2\@eolst%
   \fi}
\makeatother

\title{Corrigé bac blanc \no{1}}
\author{\bsc{Jumel}}
\date{décembre 2015}

\begin{document}

\maketitle

\emph{Compétences :}
\begin{itemize}
  \item Étudier la limite d'une somme, d'un produit ou d'un quotient de
    deux suites ;
  \item Utilisation des théorèmes de comparaison ;
  \item Savoir mener un raisonnement par récurrence ;
  \item Construire un arbre pondéré de probabilités ;
  \item Théorème des valeurs intermédiaires ;
  \item Limites de fonctions (et interprétations graphiques) ;
  \item Dérivation et variation d'une fonction :
  \item Géométrie dans l'espace.
\end{itemize}

\begin{exercice}[Une étude de suites, dans un cas concret]~

  \underline{Cas \no{1}} : $u_0 \in \left[0;1\right], k\in
  \left]0;1\right[$
  \begin{enumerate}
    \item Soit $P_n$ la propritété «$0\leqslant u_n \leqslant k^n$».
      \begin{description}
        \item[Initialisation :] $P_0$ est vraie, en effet, $u_0 \in
          [0;1] \iff 0\leqslant u_0 \leqslant k^0 = 1$.
        \item[Hérédité :] Soit $i$ un entier fixé et supposons que $P_j$
          est vraie pour cet entier $j$.

          $0\leqslant u_j \leqslant k^j \implies 0\leqslant ku_j
          \leqslant k\times k^j$.

          De plus $0\leqslant u_j \implies -u_j \leqslant 0 \implies 1 -
          u_j \leqslant 1$.

          D'où $0\leqslant ku_j(1 - u_j) \leqslant k\times k^j (1 -
          u_j)$.

          On obtient donc $0\leqslant u_j \leqslant k^j \implies
          0\leqslant u_{j+1} \leqslant k^{j+1}$ \qed
        \item [Conclusion :] Par application du principe de récurrence,
          on a $\forall n\in\N,\ 0\leqslant u_n \leqslant k^n$.
      \end{description}
    \item Posons $v_n = k^n$. Pour tout $n$ entier naturel, la suite $u$
      est encadrée par deux suites de limites nulles ($\abs{k} < 1$). Le
      théorème d'encadrement nous permet de conclure que $u_n$ converge
      et que \[\boxed{\lim_{n\to\infty}u_n = 0.}\]
    \item Si le facteur $k$ est inférieur à 1, il y a exctinction de la
      population, quel que soit l'effectif de départ.
  \end{enumerate}
  \underline{Cas \no{2}}
  \begin{enumerate}
    \item
      \begin{enumerate}
        \item
          \footnotesize
          \begin{tabular}{|*{12}{c|}} \hline
            N & 0 & 1 & 2 & 3 & 4 & 5 & 6 & 7 & 8 & 9 & 10 \\ \hline
            U & 0,1 & 0,171 & 0,269 & 0,374 & 0,445 & 0,469 & 0,473 &
            0,474 & 0,474 & 0,474 & 0,474 \\ \hline
          \end{tabular}
          \normalsize
        \item La suite semble converger vers \np{0.474}.
      \end{enumerate}
    \item
      \begin{enumerate}
        \item $f$ est une fonction polynomiale de degré 2.

          $\forall x\in \R,\ f'(x) = -3,8x + 1,9$.

          \begin{tikzpicture}
            \tkzTabInit{ $x$ /1, $f'(x)$ /1, $f$ /2}
            {0,$\frac12$, 1}
            \tkzTabLine{ , + ,z , - }
            \tkzTabVar{ -/0 , +/\np{0.475} , -/0 }
          \end{tikzpicture}
        \item $f$ est continue sur $\left[0;\frac12\right]$. D'après le
          théorème des valeurs intermédiaires, l'image d'un intervalle
          est un intervalle et donc
          $f\left(\left[0;\frac12\right]\right) = \left[0;0,475\right]
          \subset \left[0;\frac12\right]$.
        \item Démontrons la proposition $P_(n)$ : « $0 \leqslant u_n
          \leqslant u_{n+1} \leqslant \frac12$» par récurrence.
          \begin{description}
            \item[Initialisation :] $P(0) : 0 \leqslant u_0 = 0,1
              \leqslant u_1 = 0,171 \leqslant \frac12 = 0,5$
            \item[Hérédité :] Soit $k$ un entier naturel fixé, et
              supposons $0 \leqslant u_k \leqslant u_{k+1} \leqslant
              \frac12$.

              Sur $\left[0;\frac12\right],\ f$ est croissante donc, on a
              $f(0) \leqslant f(u_k) \leqslant f(u_{k+1}) \leqslant
              f\left(\frac12\right)$, ce qui se lit encore $0 \leqslant
              u_{k+1} \leqslant u_{k+2} \leqslant 0,475 \leqslant
              \frac12$. \qed
              \item[Conclusion :] L'axiome de récurrence nous garantit
              que $\forall n\in\N, 0 \leqslant u_n \leqslant u_{n+1}
              \leqslant \frac12$
          \end{description}
        \item D'après ce qui précède, $(u_n)$ est croissante majorée,
          donc convergente.

          De plus, si elle converge, on peut noter sa limite $\ell$ qui
          est telle que $f(\ell) = \ell$, c'est-à-dire solution de
          l'équation $-1,9x^2+0,9x = 0$ dont les solutions sont 0 et
          $\boxed{\ell = \frac9{19} \approx \np{0,473684}}$.
        \item La population considéré convergera vers un effectif de
          \np{473684} individus.
      \end{enumerate}
  \end{enumerate}
\end{exercice}

\begin{exercice}[Une étude de fonction]~

  \noindent\textbf{\underline{Partie A}}

  \begin{enumerate}
    \item Déterminons la limite en $+\infty$ de $g$.

      $\lim_{x\to+\infty} 4x^3 - 3x -8 = \lim_{x\to+\infty} x(4x^2 - 3 -
      \frac8x)$. Or $\lim_{x\to+\infty} - 3 - \frac8x = 0$. On a donc
      $\lim_{x\to+\infty} g(x) = \lim_{x\to+\infty} x \times
      \lim_{x\to+\infty} 4x^2 = +\infty$.

      Le même raisonnement conduit à $\lim_{x\to-\infty} g(x) = -\infty$.
    \item $g$ est dérivable sur tout $\R$ et sa dérivée $g'$ s'écrit
      pour tout $x\in\R,\ g'(x) = 12x^2 - 3 = 3(2x -1 )(2x+1)$. Le signe
      de $g'(x)$ et les variations de $g$ peuvent être résumer dans le
      tableau suivant.

      \begin{center}
        \begin{tikzpicture}
          \tkzTabInit{ $x$ /1, $g'(x)$ /1, $g$ /2}
          {$-\infty$,$-\frac12$,$\frac12$, $+\infty$}
          \tkzTabLine{ , + ,z , - , z, + }
          \tkzTabVar{ -/$-\infty$ , +/-7, -/-9 , +/$+\infty$ }
        \end{tikzpicture}
      \end{center}

    \item $g$ est continue sur $\R$, strictement monotone sur
      $\left]-\infty;-\frac12\right]$, $\left[-\frac12;\frac12\right]$
      et $\left[\frac12;+\infty\right[$. Or le seul de ses intervalles
      dont l'image contient 0 est le dernier. En utilisant le théorème
      des valeurs intermédiaires, on en déduit l'existence et l'unicité
      de $\alpha$ dans l'intervalle $\left[\frac12;+\infty\right[$.
    \item On trouve avec le code Python suivant (algorithme de
      dichotomie.)
      \begin{pyblock}
        a = 0.5
        b = 2
        def g(x):
            return 4*x**3 - 3*x - 8

        while b-a > 10**(-2):
            c = (b+a)/2
            if g(a)*g(c) < 0:
                b = c
            else:
                a = c
        print("On trouve l'encadrement ${0:1.3f} \\leqslant \\alpha \
        \\leqslant {1:1.3f}$".format(a,b).replace('.',','),end='')
      \end{pyblock}
      \printpythontex\!.

      Remarque : la dernière ligne de cette algorithme permet surtout
      l'affichage de la ligne dessous avec \LaTeX{}.
    \item ~

      \begin{center}
        \begin{tikzpicture}
          \tkzTabInit{ $x$ /1, $g(x)$ /1}
          {$-\infty$,$\frac12$, $\alpha$ ,$+\infty$}
          \tkzTabLine{ , -, , - ,z ,  + }
        \end{tikzpicture}
      \end{center}

  \end{enumerate}

  \noindent\textbf{\underline{Partie B}}

  \begin{enumerate}
    \item
      \begin{enumerate}
        \item $\lim_{x\to \infty}f(x) = \lim_{x\to \infty}
          \frac{x^2\left(x+\frac1{x^2}\right)}{x^2\left(4 -
          \frac1{x^2}\right)} = +\infty$
        \item $\lim_{\substack{x\to \frac12\\x > \frac12}}f(x) =
          \lim_{\substack{x\to \frac12\\x > \frac12}}
          \frac{\frac98}{4x^2 -1} = +\infty$

          On a ici $x > \frac12 \implies 4x^2 > 1 \implies 4x^2 - 1 >
          0$.

          $\mathscr{C}$ possède une asymptote verticale d'équation $x =
          \frac12$.
      \end{enumerate}
    \item Soit $x\in \left]\frac12;+\infty\right[$.

        $f'(x) = \frac{(3x^2)\times(4x^2 -1) - 8x\times (x^3 +1) }{(4x^2 -
        1)^2}.$

        Or $3x^2(4x^2 -1) - 8x\times (x^3 +1) = 12x^4 - 3x^2 - 8x^4 -
        8x$

        Donc le numérateur vaut $xg(x)$.
    \item On en déduit les variations de $f$ :

      \begin{tikzpicture}
        \tkzTabInit{ $x$ /1, $f(x)$ /3}
        {$\frac12$, $\alpha$ ,$+\infty$}
        \tkzTabVar{ D+/$+\infty$ , -/$f(\alpha)$ , +/$+\infty$ }
      \end{tikzpicture}

    \item Par définition $4\alpha^3 - 3\alpha - 8 = \alpha$. Ainsi,
      $\alpha^2 = 1 + \frac2{\alpha}$ et $\alpha^3 = \alpha + 2$

      $4\alpha^2 - 1 = 3 + \frac8{\alpha}$

      Ainsi, $f(\alpha) = \frac{\alpha + 3}{3 + \frac8{\alpha}}
      = \frac{\alpha^2 + 3\alpha}{3\alpha + 8}$.

      D'autre part, $f(\alpha) - \frac38\alpha = \frac{8\alpha^3 -
      12\alpha^3 + 3\alpha + 8}{8(4\alpha^2 - 1)} = \frac{-g(\alpha)}{
      8(4\alpha^2 - 1)} = 0$.

      On en déduit que $\boxed{0,543 \leqslant f(\alpha) \leqslant
      0,547.}$

  \end{enumerate}

  \noindent\textbf{\underline{Partie C}}

  \begin{enumerate}
    \item Cf. annexe
    \item $\mathscr{C}$ est au dessus de $\mathcal{D}$.

      Pour le démontrer, montrons que $f(x) - \frac14 x \geqslant 0$.

      Or pour $x> \frac12$, $f(x) - \frac14 x = \frac{x^3 - x^3 +
      \frac14 x}{4x^2 - 1} > 0$.
    \item Lorsque $x$ tend vers $+\infty$, la distance $MN$ est en
      valeur absolue la différence précédente, qui est de limite nulle.
      Donc $MN$ devient nulle.
  \end{enumerate}

\end{exercice}

\begin{exercice}[Suite et probabilités]~

  \begin{enumerate}
    \item
      \begin{enumerate}
        \item ~

          \begin{center}
            \begin{tikzpicture}[level distance=25mm,sibling distance=15mm]
              \node {} [grow=right]
              child[sibling distance=35mm] { node {$R$}
                child[sibling distance=15mm] {
                  node {$\overline{S_1}$}
                  edge from parent node[below] {$\frac13$}
                }
                child[sibling distance=15mm] {
                  node {$S_1$}
                  edge from parent node[above] {$\frac23$}
                }
                edge from parent node[below] { $\frac13$ }
              }
              child[sibling distance=35mm] { node {$V$}
                child[sibling distance=15mm] {
                  node {$\overline{S_1}$}
                  edge from parent node[below] { $\frac56$ }
                }
                child[sibling distance=15mm] {
                  node {$S_1$}
                  edge from parent node[above] { $\frac16$ }
                }
                edge from parent node[above] { $\frac23$ }
              } ;
            \end{tikzpicture}
          \end{center}
        \item $P(S_1) = P(V\cap S_1) + P(R\cap S_1) = \frac29 +
          \frac2{18} = \frac13$ car $R$ et $V$ forment une partition,
          c'est à dire $R\cup V = E$ et $R\cap V = \emptyset$.
      \end{enumerate}
    \item
      \begin{enumerate}
        \item Soit $n\in\N^*$. $P_V(S_n)$ est la répétition d'un tirage
          aléatoire avec remise : donc $P_V(S_n) =
          \left(\frac16\right)^n$.

          De même, $P_R(S_n) = \left(\frac23\right)^n$
        \item On utilise la formule des probabilités totales comme à la
          question 1.b) : $P(V\cap S_n) = P(V)\times P_V(S_n)$ et
          $P(R\cap S_n) = P(R)\times P_R(S_n)$, d'où $P(S_n) = \frac23
          \left(\frac16\right)^n + \frac13 \left(\frac23\right)^n$
        \item $p_n$ est une probabilité conditionnelle. On a donc $p_n =
          \frac{P(R\cap S_n)}{P(S_n)} = \frac{\frac13
          \left(\frac23\right)^n}{\frac23 \left(\frac16\right)^n +
          \frac13 \left(\frac23\right)^n} = \frac{1}{1 + \frac23
          \left(\frac16\right)^n \times \frac31 \left(\frac32
          \right)^n} = \frac1{1 + 2\times\left( \frac16 \times \frac32
          \right)^n}$

          On a donc $\boxed{p_n = \frac1{1 + 2\times\left( \frac14
          \right)^n}.}$
        \item Calculons $\lim_{n\to+\infty} p_n$

          $\lim_{n\to+\infty} \left( \frac14 \right)^n = 0$ donc \[
          \boxed{\lim_{n\to+\infty} p_n =1} \]
      \end{enumerate}
    \item
      \begin{enumerate}
        \item On vient de montre à la question précédente que
          $p_n\mathop{\longrightarrow}_{n\to+\infty}1$. Donc par
          définition de la limite, il existe un rang $n_0$ tel que tous
          les intervalles de la forme $[\ell - \varepsilon ; \ell +
          \varepsilon]$ contiennent tous les termes de la suite d'un
          rang supérieur ou égal à $n_0$.

          Ici, $\ell = 1$ et $\varepsilon = 10^{-3}$.
        \item Cf. annexe.
      \end{enumerate}
  \end{enumerate}
\end{exercice}

\begin{exercice}[Géométrie dans l'espace]~

  \begin{enumerate}
    \item
      \begin{enumerate}
        \item Deux versions :
          \begin{itemize}
            \item $I$ milieu de $[OA]$ et $J$ milieu de $[OB]$, donc
              d'après le théorème des milieux, $(IJ)$ et $(AB)$ sont
              parallèles. \qed
            \item Dans le repère $(O,\vv{OI},\vv{OJ})$, on a $I :
              \vecteur{1;0}$, $A: \vecteur{2;0}$, $J: \vecteur{0;1}$ et
              $B : \vecteur{0;2}$. On en déduit que $\vv{IJ} =
              \vecteur{-1,1}$ et $\vv{AB} = \vecteur{-2,2}$ et donc que
              $\vv{IJ} = \frac12\vv{AB}$. Ils sont colinéaires et les
              droites $(IJ)$ et $(AB)$ sont parallèles. \qed
          \end{itemize}
        \item Les droites $(IK)$ et $(AC)$ sont coplanaires car
          $I\in(OA)$ et $K\in(OC)$.

          Raisonnons par l'absurde\footnote{On peut aussi calculer les
          coordonnées des vecteurs et conclure.} (ou contraposée du
          théorème de Thalès). Supposons que $(IK)$ et $(AC)$ sont
          parallèles. Alors dans ce cas, $\frac{OI}{OA} =
          \frac{OK}{OC}$, ce qui n'est pas le cas. Donc l'hypothèse
          «$(IK)$ et $(AC)$ sont parallèles» est absurde.

          Les droites $(IK)$ et $(AC)$ sont sécantes en $M$. \qed
        \item On peut écrire les relations suivantes :
          \begin{itemize}
            \item $(IK) \cap (AC) = M$ ;
            \item $(JK) \cap (AB) = N$ ;
            \item Donc $(IJK) \cap (ABC) = (MN)$.
          \end{itemize}
      \end{enumerate}
    \item
      \begin{enumerate}
        \item Démontrons que $\vv{OG} = \frac13\left( \vv{OI} + \vv{OJ}
          + \vv{OK} \right)$.

          Les égalités suivantes sont des équivalences.

          $\vv{OG} = \frac13\left( \vv{OI} + \vv{OJ} + \vv{OK} \right)$

          $3\vv{OG} = \vv{OI} + \vv{OJ} + \vv{OK}$

          $\vv{OG} + \vv{OG} + \vv{OG} = \vv{OI} + \vv{OJ} + \vv{OK}$

          $\vv{KO} + \vv{OG} = \vv{GO} + \vv{OI} + \vv{GO} + \vv{OJ}$

          \[\boxed{\vv{KG} = \vv{GI} + \vv{GJ}}\]

          Les vecteurs $\vv{KG}$, $\vv{GI}$ et $\vv{GJ}$ sont
          coplanaires donc les points $I,J,K$ et $J$ sont coplanaires.
          \qed
        \item $E$ milieu de $IJ$ entraîne que $\vv{IE} + \vv{JE} =
          \vv{0}$ et donc $\vv{KG} + \vv{0} = \vv{GI} + \vv{GJ} +
          \vv{IE} + \vv{JE} = 2\vv{GE}$.

          $G$ est le centre de gravité du triangle $IJK$. \qed
     \end{enumerate}
    \item
      \begin{enumerate}
        \item Prenons pour la droite, le point $O$ et comme vecteur
          directeur $3\vv{OG} = \vecteur{1,1,1}$

          Une représentation paramétrique de la droite $(OG)$ est
          \[ (OG) : \left\lbrace\begin{matrix} x = t \\ y = t \\ z = t
          \end{matrix} \right. , t\in\R \]
        \item Prenons $\vv{AC} = \vecteur{-2,0,3}$ et $\vv{BC} =
          \vecteur{0,-2,3}$ comme vecteurs directeurs du plan $(ABC)$ et
          le point $C : \vecteur{0,0,3}$.

          Une représentation paramétrique du plan $(ABC)$ est
          \[ (ABC) : \left\lbrace\begin{matrix} x = & \!\!-2t \\ y = &
          \!\! -2s \\ z = & \!\!3 + 3t + 3s \end{matrix} \right. ,
          (t,s)\in\R^2 \]
        \item Démontrons que le point $H : \vecteur{\frac34 ; \frac34 ;
          \frac34}$ appartient à la droite $(OG)$ et au plan $(ABC)$,
          c'est à dire qu'il existe un paramètre $t$ pour $(OG)$ et deux
          paramètres $(t,s)$ pour $(ABC)$ tel que les coordonnées du
          point vérifient la représentation paramétrique choisie.

          Pour la droite $(OG)$, $t=\frac34$ convient.

          Pour le plan $(ABC)$, les deux premières équations donnent
          $t = s = -\frac38$. La dernière équation donne donc $3 -
          3\frac38 - 3\frac38 = 3 - \frac94 = \frac34$.

          Le point $H : \vecteur{\frac34 ; \frac34 ; \frac34}$
          appartient bien au plan et à la droite. Comme celle-ci n'est
          ni coplanaire, ni parallèle, c'est bien l'unique point
          d'intersection. \qed
        \item Considérons le plan $(COF)$. Celui contient la droite
          $(OG)$, donc il contient le point $H$.

          $(CF)$ est l'intersection des plans $(COF)$ et $(ABC)$, donc
          le point $H$, comme point de $(ABC)$ appartient à la droite
          $(CF)$.\qed
      \end{enumerate}
  \end{enumerate}
\end{exercice}

\pagebreak
\section*{Annexe}

\textbf{\underline{Exercice \no{2}}}

\begin{center}
  \begin{tikzpicture}[scale=2.7,>=latex]
    \draw [dashed] (0,0) grid[step=0.5] (6.15,3.2) ;
    \draw [->,thick] (-0.25,0) -- (6.15,0) ;
    \draw [->,thick] (0,-0.25) -- (0,3.2) ;
    \draw [very thick] plot [smooth,samples=500,domain=0.589:6.19]
    (\x,{(\x*\x*\x + 1)/(4*\x*\x - 1)}) ;
    \draw [red,very thick] plot [domain=0:6.2] (\x,0.25*\x) ;
    \foreach \x in {1,...,6} {
      \draw (\x,0) -- (\x,-0.05) node [below] { $\x$ } ;
    }
    \foreach \y in {1,...,3} {
      \draw (0,\y) -- (-0.05,\y) node [left] { $\y$ } ;
    }
    \draw (0,0) node [below left] {0} ;
  \end{tikzpicture}
\end{center}

\textbf{\underline{Exercice \no{3}}}

\begin{tabular}{p{5em}p{0.1em}cp{20em}}
  Variables      &  & : & $E$ est un réel strictement compris entre 0 et 1 \\
                 &  &   & $N$ est un entier naturel non nul                           \\
  Entrée         &  & : & $E$ \textcolor{red}{$=10^{-3}$}                              \\
  Initialisation &  & : & Affecter à $N$ la valeur 1                                  \\
  Traitement     &  & : & Tant que \textcolor{red}{$1-p_N\geqslant E$}                \\
                 &  &   & \hspace{1em}\textcolor{red}{Affecter à $N$ la valeur $N+1$} \\
                 &  &   & Fin Tant que                                                \\
  Sortie         &  & : & Afficher $N$                                                \\
\end{tabular}

\end{document}
