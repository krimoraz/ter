\documentclass[a4paper,12pt,french]{article}

\usepackage[utf8]{inputenc}
\usepackage[T1]{fontenc}
\usepackage[upright]{kpfonts}
\usepackage[margin=1.4cm]{geometry}
\usepackage{amsmath,amsfonts,amssymb}

\usepackage[inline]{enumitem}
\setlist{noitemsep}
%\setlist[1]{\labelindent=\parindent} % < Usually a good idea
\setlist[itemize]{leftmargin=*}
\setlist[itemize,1]{label=$\triangleright$}
\setlist[enumerate]{labelsep=*, leftmargin=1.5pc}
\setlist[enumerate,1]{label=\arabic*., ref=\arabic*}
\setlist[enumerate,2]{label=\emph{\alph*}),
ref=\theenumi.\emph{\alph*}}
\setlist[enumerate,3]{label=\roman*), ref=\theenumii.\roman*}
\setlist[description]{font=\sffamily\bfseries}

\usepackage{multicol}
\setlength{\columnseprule}{0pt}

\usepackage[zerostyle=d]{newtxtt}
\usepackage[dvipsnames,table]{xcolor}
\usepackage{tipfr}

\usepackage{babel}
\usepackage[lastexercise]{exercise}

\renewcommand{\ExerciseName}{Exercice}
\renewcommand{\AnswerName}{Réponse de l'exercice}

\newcommand{\N}{\mathbf{N}}
\newcommand{\R}{\mathbf{R}}

\everymath{\displaystyle\everymath{}}

\title{Correction exercices suites rappels}
\date{\today}

\begin{document}

\maketitle

\begin{Exercise}[number=1,title={Vrai-faux}]
 \begin{enumerate}
    \item Pour tout $x$, $(x+1)^2 \geqslant x+1$.
    \item Si $x \leqslant \sqrt{x} \leqslant x^2$, alors $x=0$.
    \item Si $x > 1$, alors $\frac3{x^2} < \frac4x$.
    \item Si $x > 1$, alors $x^2 < x^3$.
    \item Si $x > 0$, $x-1 < (x-1)^2$
  \end{enumerate}
\end{Exercise}
\begin{Answer}
  \begin{enumerate}
    \item Faux. Deux arguments différents :
      \begin{itemize}
        \item $\forall x\in\R,\ (x+1)^2 \geqslant x+1 \iff x^2 + 2x +1
          \geqslant x+1 \iff x^2 + x \geqslant 0$. La fonction
          polynomiale $x\mapsto x^2 + x$ s'annule deux fois sur $\R$, ce
          qui invalide l'expression de départ.
        \item Pour $x=-\frac12$, le membre de gauche vaut $\frac14$ et
          celui de droite $\frac12$ ; or $\frac14 \leqslant \frac12$.
      \end{itemize}
    \item Faux, $1 \leqslant \sqrt1 \leqslant 1^2$.
    \item Vrai. Soit $x\in\R,\ x > 1 \implies x^2 > x \implies
      \frac{x^2}3 > \frac{x}{3} > \frac{x}{4} \implies \frac3{x^2} <
      \frac{4}{x}$
    \item Vrai. Soit $x\in\R,\ x > 1 \implies x^2 > x \implies x^3 >
      x^2$
    \item Faux. Deux arguments différents :
      \begin{itemize}
        \item Pour $x=\frac32$, le membre de gauche vaut $\frac12$ ; le
          membre de droite vaut $\frac14$.
        \item $\forall x\in\R,\ x-1 < (x-1)^2 \iff x-1 < x^2 -2x +1 \iff
          x^2 -3x +2 > 0$. Or cette fonction polynomiale s'annule en 1
          et en 2.
      \end{itemize}
  \end{enumerate}
\end{Answer}
\begin{Exercise}[title={QCM}]
  \emph{Pour chaque affirmation, une seule réponse est exacte.
  Identifiez-la en justifiant votre réponse.}

  Les fonctions $f$, $g$ et $h$ sont définies sur $\R$ par : \[ f(x) =
  2x+3 ,\ \ g(x) = 3 - x^2\ \ \text{et}\ \ h(x) = x^2 + 2x - 1 .\]
  \begin{enumerate}
    \item $f(x) \leqslant g(x)$ pour tout $x$ de l'intervalle :
      \begin{multicols}{3}
        \begin{enumerate}
          \item $\R$
          \item $[-2;0]$
          \item $[-2;2]$
        \end{enumerate}
      \end{multicols}
    \item Pour tout $x$ de l'intervalle $[-2;1]$ :
      \begin{multicols}{3}
        \begin{enumerate}
          \item $g(x) < f(x)$
          \item $h(x) \geqslant f(x)$
          \item $g(x) \geqslant h(x)$
        \end{enumerate}
      \end{multicols}
    \item \begin{enumerate}
        \item Pour tout $x$, $x > \frac32,\ h(x) < f(x)$.
        \item Pour tout $x$ de l'intervalle $]-2,0],\  \left[f(x) -
          g(x)\right]\times\left[f(x) - h(x)\right] \leqslant 0$.
        \item Il n'existe pas de nombre $x$ tel que : \[f(x) \leqslant
          g(x) \leqslant h(x).\]
      \end{enumerate}
  \end{enumerate}
\end{Exercise}
\begin{Answer}
  À la calculatrice, on peut obtenir ceci :
  \begin{center}
    \Ecran[graphic=true]{
      2*\x + 3/-6:6,
      3 - \x*\x/-6:6,
      \x*\x + 2*\x -1/-6:6
    }
  \end{center}
  \begin{enumerate}
    \item La différence $f-g$ est la fonction polynomiale $x\mapsto -x^2
      + 2x$. C'est donc la réponse \textbf{b)}.
    \item La différence $g -h$ est la fonction polynomiale $x\mapsto 4 -
      2x -2x^2$ dont les racines sont 1 et -2. C'est donc la réponse
      \textbf{c)}.
    \item $f(-2) = -1 ;\ g(-2) = -1 ;\ h(-2) = -1$. C'est donc la
      réponse \textbf{c)}.
  \end{enumerate}
\end{Answer}

\begin{Exercise}[title={Vrai ou faux}]
  \begin{enumerate}
    \item Pour tout $x$, $x-1 \leqslant x - \cos(x) \leqslant x +1$.
    \item Quelque soit $x$, $2\sin(x) + (\cos x)^2 \geqslant 0$.
    \item $f$ est une fonctin définie sur un intervalle $I$ telle que :

      pour tout $x$ de $I$: \[ -1 \leqslant f(x) \leqslant 3 .\] Alors,
      pour tout $x$ de $I$, le nombre $\frac{f(x) + \sin(x)}{2}$
      appartient à $[-1;2]$.
    \item Il existe des nombres $x$ de l'intervalle $]0;\pi]$ tels que
      $\sin(x) \leqslant -\frac1x$.
  \end{enumerate}
\end{Exercise}
\begin{Answer}
  \begin{enumerate}
    \item Vrai : $-1\leqslant\cos x\leqslant 1 \implies -1\leqslant
      -\cos x\leqslant 1 \implies x-1\leqslant x-\cos x\leqslant x+1$.
    \item Faux : $2\sin\left(\frac{-\pi}{2}\right) + \cos^2
      \left(\frac{-\pi}{2}\right) = 2\times (-1) + 0 \leqslant 0$.
    \item Vrai : $-1\leqslant\sin x\leqslant 1 \ \text{et}\ -1 \leqslant
      f(x) \leqslant 3 \implies -2\leqslant f(x) + \sin(x) \leqslant 4
      \implies -1 \leqslant \frac{f(x) + \sin(x)}{2} \leqslant 2
      \implies \frac{f(x) + \sin(x)}{2} \in [-1;2]$.
    \item Faux : $\forall x\in ]0;\pi],\ \sin x \geqslant 0 $ et $-\frac1x
      \leqslant 0$.
  \end{enumerate}
\end{Answer}
\end{document}
