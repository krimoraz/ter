\documentclass[a4paper,12pt,french]{article}

\usepackage[utf8]{inputenc}
\usepackage[T1]{fontenc}
\usepackage[upright]{kpfonts}
\usepackage[margin=1.4cm]{geometry}
\usepackage{amsmath,amsfonts,amssymb}

\usepackage[inline]{enumitem}
\setlist{noitemsep}
%\setlist[1]{\labelindent=\parindent} % < Usually a good idea
\setlist[itemize]{leftmargin=*}
\setlist[itemize,1]{label=$\triangleright$}
\setlist[enumerate]{labelsep=*, leftmargin=1.5pc}
\setlist[enumerate,1]{label=\arabic*., ref=\arabic*}
\setlist[enumerate,2]{label=\emph{\alph*}),
ref=\theenumi.\emph{\alph*}}
\setlist[enumerate,3]{label=\roman*), ref=\theenumii.\roman*}
\setlist[description]{font=\sffamily\bfseries}

\usepackage{multicol}
\setlength{\columnseprule}{0pt}

\usepackage[zerostyle=d]{newtxtt}
\usepackage[dvipsnames,table]{xcolor}
\usepackage{tipfr}

%\usepackage[pyfuture=all, autoprint=true, gobble=auto]{pythontex}

\usepackage{babel}
\usepackage[lastexercise,answerdelayed]{exercise}

\renewcommand{\ExerciseName}{Exercice}
\renewcommand{\AnswerName}{Réponse de l'exercice}

\newcommand{\N}{\mathbf{N}}
\newcommand{\R}{\mathbf{R}}

\everymath{\displaystyle\everymath{}}

\title{Exercices sur les limites}
\date{\today}

\begin{document}

\maketitle


\begin{Exercise}
  Calculer les limites suivantes :
  \begin{enumerate}
    \item $\lim_{x\to +\infty} x^2 - 4x +4$ (deux méthodes)
    \item $\lim_{x\to +\infty} x^2 - 3x + 2$
    \item $\lim_{x\to +\infty} x^3 - x^2 + 2x - 1$
    \item $\lim_{x\to +\infty} x^4 -2x^3 + 4x^2 -x + 6$
    \item $\lim_{x\to +\infty} -x^3 + 2x +1$
  \end{enumerate}
\end{Exercise}

\begin{Exercise}
  Calculer, si elles existent, les limites suivantes :
  \begin{enumerate}
    \item $\lim_{x\to +\infty} \frac{x - 1}{x + 1}$
    \item $\lim_{x\to +\infty} \frac{x^2  - x + 1}{x + 1}$
    \item $\lim_{x\to +\infty} \frac{x - 1}{x^2  -3x + 2}$
    \item $\lim_{x\to +\infty} \frac{3x^2 - 6x  - 5}{2x^2 +2x + 10}$
    \item $\lim_{x\to +\infty} \frac{x^3 - 2x^2 + 5x - 1}{x^2 + 3x - 1}$
    \item $\lim_{x\to +\infty} \frac{x^2 + 3x - 1}{4x^2 + 1}$
  \end{enumerate}
\end{Exercise}

On admet que $a^+$ désigne $x\geqslant a$ et $a^-$ désigne $x\leqslant
a$ pour $a$ un nombre réel.

\begin{Exercise}
  Calculer, si elles existtent, les limites suivantes :
  \begin{enumerate}
    \item $\lim_{x\to 5^-} \frac{1}{x - 5}$
    \item $\lim_{x\to 3^+} \frac{1}{3 - x}$
    \item $\lim_{x\to 7} \frac{x - 7}{x + 7}$
    \item $\lim_{x\to 3^+} \frac{x^2 + 3 x - 3}{x^2 - 9}$
    \item $\lim_{x\to 3^+} \frac{x^2 - 3 x - 3}{x^2 + 9}$
    \item $\lim_{x\to 3^+} \frac{x^2 - 3 x - 3}{x^2 - 9}$
  \end{enumerate}
\end{Exercise}

\pagebreak
\shipoutAnswer
\end{document}
