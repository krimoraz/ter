\documentclass[a4paper,12pt,french]{article}

\usepackage[utf8]{inputenc}
\usepackage[T1]{fontenc}
\usepackage[upright]{kpfonts}
\usepackage[margin=1.4cm]{geometry}
\usepackage{amsmath,amsfonts,amssymb}

\usepackage[inline]{enumitem}
\setlist{noitemsep}
%\setlist[1]{\labelindent=\parindent} % < Usually a good idea
\setlist[itemize]{leftmargin=*}
\setlist[itemize,1]{label=$\triangleright$}
\setlist[enumerate]{labelsep=*, leftmargin=1.5pc}
\setlist[enumerate,1]{label=\arabic*., ref=\arabic*}
\setlist[enumerate,2]{label=\emph{\alph*}),
ref=\theenumi.\emph{\alph*}}
\setlist[enumerate,3]{label=\roman*), ref=\theenumii.\roman*}
\setlist[description]{font=\sffamily\bfseries}

\usepackage{multicol}
\setlength{\columnseprule}{0pt}

\usepackage[zerostyle=d]{newtxtt}
\usepackage[dvipsnames,table]{xcolor}
\usepackage{tipfr}

\usepackage[pyfuture=all, autoprint=true, gobble=auto]{pythontex}

\usepackage{babel}
\usepackage[lastexercise,answerdelayed]{exercise}

\renewcommand{\ExerciseName}{Exercice}
\renewcommand{\AnswerName}{Réponse de l'exercice}

\newcommand{\N}{\mathbf{N}}
\newcommand{\R}{\mathbf{R}}

\everymath{\displaystyle\everymath{}}

\title{Exercices sur les limites}
\date{\today}

\begin{document}

\maketitle

\begin{sympyblock}
  var('x')
  funcslims = [
  ('x**2 - 2*x - 3','+oo'),
  ('-x**2 + 2*x + 3','-oo')
  ]

  print(r'\begin{Exercise}[number=1,title={Vrai-faux}]')
  for func,lim in funcslims:
      question = "Calculer la limite de ${}$ en ${}$".format(
      latex(eval(func)),latex(eval(lim)))
      print(r'\Question')
      print(question)

  print(r'\end{Exercise}')
  print(r'\begin{Answer}')
  for func,lim in funcslims:
      print(r'\Question')
      # Décomposition en somme : on initialise une liste vide dans
      #laquelle on ajoute les éléments de la somme
      funcn = []
      for k,v in enumerate(eval(func).as_coefficients_dict()):
          funcn.append(eval('{}*{}'.format(eval(func).as_coefficients_dict()[v],v)))
      print(r"L'expression se décompose comme la somme de", end=' ')
      # On affiche les différents termes de cette somme.
      for i in range(len(funcn) - 1):
          print(r"${}$,".format(latex(funcn[i])), end=' ')
          print(r"et ${}$.".format(latex(funcn[-1])))
      limits_mon =  map(lambda x: Limit(x,'x',lim).doit(), funcn)
      # On passe à la limite pour chacun des termes (dans la liste) Voir
      # les effets de 'map' et 'lambda: '
      # Si la somme des limites n'est pas indéterminée, alors on a le
      # résultat et on l'affiche
      if sum(limits_mon) != nan:
          print(r'La somme des limites donne ici le résultat :')
          limit = Limit(func,'x',lim)
          reponse = "${} = {}$".format(latex(limit),latex(limit.doit()))
          print(reponse)
      # Sinon on factorise
      else:
          print(r'La somme des limites ici conduit à une forme \
          indéterminée. On ne peut conclure ainsi.')
          print(r'Factorisons, si cela est possible.')
          # Si la factorisation «échoue», on «factorise par x != 0
          if factor(func) != func:
              print(r'Ici, on ne peut pas factoriser. Pour $x\neq 0$, on a :')
              x = Symbol('x',nonzero = true)
              funcx = expand(eval(func)/x)
              print(r'${} = x\times\left({}\right)$.'.format(
              latex(eval(func)),latex(funcx)))
              print(r'Le deuxième facteur de ce produit a pour limite')
              funcxn = []
              for k,v in enumerate(funcx.as_coefficients_dict()):
                  funcxn.append(eval('{}*{}'.format(
                  funcx.as_coefficients_dict()[v],v)))
              limits_monx =  map(lambda f: Limit(f,'x',lim).doit(), funcxn)
              print(r'${}$.'.format(latex(Limit(funcx,x,lim).doit())))
              limit = Limit(func,'x',lim)
              reponse = "${} = {}$".format(latex(limit),latex(limit.doit()))
              print("Le produit nous donne donc", reponse)

  print(r'\end{Answer}')
\end{sympyblock}

\pagebreak
\shipoutAnswer
\end{document}
