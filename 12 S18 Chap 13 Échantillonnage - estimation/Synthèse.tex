%vim: ft=tex
%!TEX encoding = UTF-8 Unicode
\documentclass[12pt,frenchb]{article}

\input{../../common.tex.inc}

\title{Nouvelle calédonie décembre 2015 : exercice de synthèse}
\date{mai \the\year}
\author{}

\begin{document}

Une usine produit de l'eau minérale en bouteilles. Lorsque le taux de calcium dans une
bouteille est inférieur à 6,5~mg par litre, on dit que l'eau de cette bouteille est très peu
calcaire.

\smallskip

\emph{Dans cet exercice les résultats approchés seront arrondis au millième.}

\smallskip

\textbf{Partie A}

\medskip

L'eau minérale provient de deux sources, notées \og source A \fg{} et \og source B \fg.

La probabilité que l'eau d'une bouteille prélevée au hasard dans la production d'une journée
de la source A soit très peu calcaire est $0,17$. La probabilité que l'eau d'une bouteille prélevée au hasard dans la production d'une journée de la source B soit très peu calcaire est $0,10$.\index{probabilités}

\smallskip

La source A fournit 70\,\% de la production quotidienne totale des bouteilles d'eau et la source
B le reste de cette production.

\smallskip

On prélève au hasard une bouteille d'eau dans la production totale de la journée. On considère
les évènements suivants :

$A$ : \og La bouteille d'eau provient de la source A \fg

$B$ : \og La bouteille d'eau provient de la source B \fg

$S$ : \og L'eau contenue dans la bouteille d'eau est très peu calcaire \fg.

\medskip

\begin{enumerate}
\item Déterminer la probabilité de l'évènement $A \cap S$.
\item Montrer que la probabilité de l'évènement $S$ vaut $0,149$.
\item Calculer la probabilité que l'eau contenue dans une bouteille provienne de la
source A sachant qu'elle est très peu calcaire.
\item Le lendemain d'une forte pluie, l'usine prélève un échantillon de \np{1000} bouteilles provenant de la source A. Parmi ces bouteilles, $211$ contiennent de l'eau très peu calcaire. 

Donner un intervalle permettant d'estimer au seuil de 95\,\% la proportion
de bouteilles contenant de l'eau très peu calcaire sur l'ensemble de la production
de la source A après cette intempérie.\index{intervalle de confiance}
\end{enumerate}

\bigskip

\textbf{Partie B}

\medskip

On note $X$ la variable aléatoire qui, à chaque bouteille prélevée au hasard dans la production d'une journée de la source A, associe le taux de calcium de l'eau qu'elle contient. On suppose que $X$ suit la loi normale de moyenne $8$ et d'écart-type $1,6$.\index{loi normale}

On note $Y$ la variable aléatoire qui, à chaque bouteille prélevée au hasard dans la production d'une journée de la source B, associe le taux de calcium qu'elle contient. On suppose que $Y$ suit la loi normale de moyenne $9$ et d'écart-type $\sigma$.

\medskip

\begin{enumerate}
\item Déterminer la probabilité pour que le taux de calcium mesuré dans une bouteille
prise au hasard dans la production d'une journée de la source A soit compris entre
$6,4$~mg et $9,6$~mg.
\item Calculer la probabilité $p(X \leqslant 6,5)$.
\item Déterminer $\sigma$ sachant que la probabilité qu'une bouteille prélevée au hasard dans la production d'une journée de la source B contienne de l'eau très peu calcaire est
$0,1$.
\end{enumerate}

\bigskip

\textbf{Partie C}

\medskip

Le service commercial a adopté pour les étiquettes des bouteilles la forme représentée ci-dessous dans un repère orthonormé du plan.

La forme de ces étiquettes est délimitée par l'axe des abscisses et la courbe $\mathcal{C}$ d'équation $y = a\cos x$ avec $x \in  \left[- \frac{\pi}{2}~;~\frac{\pi}{2}\right]$ et $a$ un réel strictement positif.

\smallskip

Un disque situé à l'intérieur est destiné à recevoir les informations données aux acheteurs. On
considère le disque de centre le point A de coordonnées $\left(0~;~\frac{a}{2}\right)$ et de rayon $\frac{a}{2}$. On admettra que ce disque se trouve entièrement en dessous de la courbe $\mathcal{C}$ pour des valeurs de $a$ inférieures à $1,4$.

\medskip

\begin{enumerate}
\item Justifier que l'aire du domaine compris entre l'axe des abscisses, les droites d'équation
$x = - \frac{\pi}{2}$ et $x =  \frac{\pi}{2}$, et la courbe $\mathcal{C}$ est égale à $2a$ unités d'aire.\index{aire et intégrale}
\item Pour des raisons esthétiques, on souhaite que l'aire du disque soit égale à l'aire de la surface grisée. Quelle valeur faut-il donner au réel $a$ pour respecter cette contrainte ?\hyperlink{Index}{*}
\end{enumerate}

\begin{center}
\psset{unit=3cm}
\begin{pspicture}(-1.7,-0.2)(1.7,1.7)
\psaxes[linewidth=1.25pt,Dx=2,Dy=2]{->}(0,0)(-1.7,-0.2)(1.7,1.7)
\psplot[plotpoints=3000,linewidth=1.25pt]{-1.57}{1.57}{x RadtoDeg cos 1.3 mul}
\pscustom[fillstyle=solid,fillcolor=lightgray]{
\psplot[plotpoints=3000,linewidth=1.25pt]{-1.57}{1.57}{x RadtoDeg cos 1.3 mul}
\psline(1.57,0)(-1.57,0)}
\pscircle[fillstyle=solid,fillcolor=white](0,0.65){0.65}
\psdots(0,0.65) \uput[ur](0,0.65){$A$}
\uput[ul](-1.4,0.3){$\mathcal{C}$}\uput[d](1.57,0){$\dfrac{\pi}{2}$}
\uput[d](-1.57,0){$- \dfrac{\pi}{2}$}\uput[dr](0,0){O}\uput[ul](0,1.3){$a$}
\end{pspicture}
\end{center}

