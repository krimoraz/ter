\documentclass[10pt,a4paper,french]{article}
\usepackage[utf8]{inputenc}
\usepackage[T1]{fontenc}
\usepackage{babel}
\usepackage[thmmarks]{ntheorem}
\usepackage{amsmath}
\usepackage{amsfonts}
\usepackage{amssymb}

\usepackage{array}

\usepackage{kpfonts}

\usepackage[bookmarks=false,colorlinks,linkcolor=blue,pdfusetitle]{hyperref}

\pdfminorversion 7
\pdfobjcompresslevel 3

\usepackage{tabularx}
\usepackage[autolanguage,np]{numprint}
\usepackage{enumitem}

\usepackage{tipfr}
\usepackage{pgf}
\usepackage{tikz}
\usepackage{tkz-base}
\usepackage{tkz-euclide}
\usetkzobj{all}
\usetikzlibrary{hobby}
\usepackage{tkz-tab}

\usepackage[top=1.7cm,bottom=2cm,left=2cm,right=2cm]{geometry}

\usepackage{lastpage}

\usepackage{esvect}
\usepackage{marginnote}

\usepackage{wrapfig}

\usepackage[defaultlines=5,all]{nowidow}


\makeatletter
\renewcommand{\@evenfoot}%
        {\hfil \upshape \small page {\thepage} de \pageref{LastPage}}
\renewcommand{\@oddfoot}{\@evenfoot}

\renewcommand{\maketitle}%
{\framebox{%
    \begin{minipage}{1.0\linewidth}%
      \begin{center}%
        \Large \@title ~-- \@author \\%
        \@date%
      \end{center}%
    \end{minipage}}%
  \normalsize%
  %\vspace{1cm}%
}

%Des macros pour les noms d'ensmbles
\newcommand{\R}{\mathbf{R}}
\newcommand{\Q}{\mathbf{Q}}
\newcommand{\Z}{\mathbf{Z}}
\newcommand{\C}{\mathbf{C}}
\newcommand{\N}{\mathbf{N}}

\newcommand{\Cf}{\mathscr{C}}

\newcommand{\norme}[1]{\left\lVert #1 \right\rVert}
\newcommand{\abs}[1]{\left\lvert #1 \right\rvert}
\newcommand{\diff}[1]{\mathrm{d}\,#1}

\newcommand{\p}{\mathcal{P}}

%Une macro récursive pour l'intérieru des vecteurs
%http://tex.stackexchange.com/questions/19693/arguments-of-custom-commands-as-comma-separated-list

\newcommand\vecteur[2][\\]{%
    \global\def\my@delim{#1}%
    \left(\negthinspace\begin{matrix}
        \my@vector #2,\relax\noexpand\@eolst%
    \end{matrix}\right)}

%Une macro pour les vecteurs
\def\my@vector #1,#2\@eolst{%
   \ifx\relax#2\relax
      #1
   \else
      #1\my@delim
      \my@vector #2\@eolst%
   \fi}

%Une macro récursive pour mettre formater l'intérieur des intervalles
\def\my@intervalle #1;#2\@eolst{%
  \ifx\relax#2\relax
    #1
  \else
    \my@intervalle #2\@eolst%
  \fi}

%Quatre macros pour les quatres types d'intervalles
\newcommand{\interff}[1]{%
  \left[\my@intervalle #1;\relax\noexpand\@eolst%
  \right]
}
\newcommand{\interfo}[1]{%
  \left[\my@intervalle #1;\relax\noexpand\@eolst%
  \right[}
\newcommand{\interof}[1]{%
  \left]\my@intervalle #1;\relax\noexpand\@eolst%
  \right]}
\newcommand{\interoo}[1]{%
  \left]\my@intervalle #1;\relax\noexpand\@eolst%
  \right[}

\makeatother


\usepackage{framed}

\theoremstyle{break}
\newtheorem{definition}{Définition}
\newtheorem{propriete}{Propriété}
\newtheorem{corollaire}{Corollaire}
\newtheorem{propdef}{Propriété - Définition}
\newtheorem{theoreme}{Théorème}
\newtheorem{lemme}{Lemme}
\theoremstyle{plain}
\theorembodyfont{\normalfont}
\newtheorem{exerciceT}{Exercice}
\theoremstyle{nonumberplain}
\newtheorem{remarque}{Remarque}
\newtheorem{notation}{Notation}
\newtheorem{probleme}{Problème}
\theoremsymbol{\ensuremath{\blacksquare}}
\newtheorem{preuve}{Preuve}
\theoremsymbol{}
\theoremstyle{nonumberbreak}
\newtheorem{exemple}{Exemple}

\newenvironment{exercice}{\begin{framed}\begin{exerciceT}}{\end{exerciceT}\end{framed}}

\setlength{\parsep}{0pt}
\setlength{\parskip}{5pt}
%\setlength{\parindent}{0pt}
\setlength{\itemsep}{7pt}

\setlist{noitemsep}
%\setlist[1]{\labelindent=\parindent} % < Usually a good idea
\setlist[itemize]{leftmargin=*}
\setlist[itemize,1]{label=$\triangleright$}
\setlist[enumerate]{labelsep=*, leftmargin=1.5pc}
\setlist[enumerate,1]{label=\arabic*., ref=\arabic*}
\setlist[enumerate,2]{label=\emph{\alph*}),
ref=\theenumi.\emph{\alph*}}
\setlist[enumerate,3]{label=\roman*), ref=\theenumii.\roman*}
\setlist[description]{font=\sffamily\bfseries}

\usepackage{multicol}
\setlength{\columnseprule}{0pt}

\everymath{\displaystyle\everymath{}}

\title{Éléments de statistique inférentielle}
\author{\bsc{Jumel}}
\date{mai 2018}

\begin{document}

\noindent\maketitle


\section*{Introduction}

Ce dernier cours de l'année est à envisager comme une initiation à la
statistiques inférentielle. Il repose principalement sur des
raisonnements de probabilité et permettra de donner une justification à
certains résultats utilisés depuis la classe de seconde.

Le but général de cette branche de la statistique est de permettre
d'inférer des résultats sur une population à partir de l'étude d'un
échantillon de cette population. Plusieurs questions se posent :
\begin{itemize}
  \item l'échantillon se comporte-t-il comme la population totale ;
  \item quel part prend le hasard dans la détermination d'un échantillon
    et donc des résultats qui en sont issus ;
  \item quelle confiance donner aux résultats obtenus par cette méthode.
\end{itemize}

\section{Intervalle de fluctuation}

On considère une caractéristique mesurable dans une population et on
considère la variable aléatoire $F$ qui a un échantillon associe la
proportion de la caractéristique mesurable.

$I$ est intervalle de fluctuation pour cette variable aléatoire au seuil
$\beta$ lorsque la probabilité pour une réalisation de $F$ d'appartenir
à $I$ est supérieur à $\beta$.

On a ainsi la définition suivante.

\begin{definition}
  $I$ est un intervalle de fluctuation de $F$ au seuil $\beta$ lorsque
  $\p(F \in I) \geqslant \beta$
\end{definition}

En pratique, il est rare d'utiliser cette définition assez théorique et
on utilise l'intervalle de fluctuation asymptotique telle que défini par
cette proposition.

\begin{propriete}~\\[-5mm]
  $I_n = \interff{ p - u_{\alpha}\frac{\sqrt{p(1-p)}}{\sqrt{n}} ; p +
  u_{\alpha}\frac{\sqrt{p(1-p)}}{\sqrt{n}} }$ est l'intervalle de
  fluctuation asymptotique de $\frac{X_n}{n}$ au seuil $1 - \alpha$ et
  \[ \lim_{n\to +\infty} \p\left(\frac{X_n}{n} \in I_n \right) \geqslant
  1 - \alpha.\]
  \begin{preuve}
    Soit $X_n$ une variable aléatoire qui suit une loi binomiale de
    paramètres $n$ et $p$ et $\alpha \in \interoo{0;1}$.

    On a d'une part, pour toute variable aléatoire $Z$ qui suit la loi
    normale centrée réduite $\mathscr{N}(0;1) \ \p(-u_{\alpha} \leqslant
    Z \leqslant u_{\alpha}) = 1 - \alpha$.

    D'autre part $E(X_n) = np $ et $V(X_n) = np(1 - p)$ donc d'après le
    théorème de Moivre-Laplace la proposition suivante est vraie à la
    limite pour la variable aléatoire $Z = \frac{X_n -
    np}{\sqrt{np(1-p)}}$.

    Or $-u_{\alpha} \leqslant \frac{X_n - np}{\sqrt{np(1-p)}} \leqslant
    u_{\alpha}  \iff p - u_{\alpha} \frac{\sqrt{p(1-p)}}{\sqrt{n}}
    \leqslant \frac{X_n}{n} \leqslant p + u_{\alpha}
    \frac{\sqrt{p(1-p)}}{\sqrt{n}}$.
  \end{preuve}
\end{propriete}

\begin{remarque}
  La notion d'intervalle de fluctuation asymptotique n'a réellement de
  sens que lorsque l'approximation d'une loi binomiale par une loi
  normale est possible (condition d'applications du théorème de
  Moivre-Laplace) c'est-à-dire $n \geqslant 30$ et $np$ et $n(1-p)
  \geqslant 5$.
\end{remarque}

Il est utile à ce stade là de mémoriser les valeurs $u_{0,05} \approx
1,96$ et $u_{0,01} \approx 2,58$. Ces valeurs là sont celles qui servent
régulièrement lors de la prise de décision.

\begin{remarque} $u_{\alpha}$ est parfois appelé «rayon de l'intervalle
  de fluctuation» au seuil d'acceptabilité $1 - \alpha$ (ou de risque
  $\alpha$)
\end{remarque}

La propriété suivante va permettre de justifier un résultat admis en
classe de seconde.

\begin{propriete}
  L'intervalle de fluctuation asymptotique au seuil de 95\% est inclus
  dans l'intervalle $I = \interff{p - \frac1{\sqrt{n}} ; p +
  \frac1{\sqrt{n}}}$.
  \begin{preuve}
    En étudiant les variations de la fonction $p\mapsto \sqrt{p(1-p)}$
    sur $\interff{0;1}$, on obtient que $1,96\sqrt{p(1-p)} \leqslant 1$.
  \end{preuve}
\end{propriete}

\begin{remarque}
  Cette inclusion n'est vraie que jusqu'à un certain seuil.
\end{remarque}

\section{Intervalle de confiance}

On veut souvent obtenir une information sur la fréquence (ou la
proportion) inconnue d'une population à partir de la mesure d'un
échantillon. C'est ce qui est courament fait lors des sondages.

Pour cela, on peut définir l'intervalle de confiance comme étant une
réalisation d'un intervalle choisi au hasard contenant la proportion $p$
inconnu. Cet intervalle est déterminé à partir de $F_n$ la fréquence
observée dans l'échantillon.

On pose $F_n = \frac{X_n}{n}$ qui a un échantillon de taille $n$ associe
la fréquence d'apparition $f$ de la caractéristique étudiée. $X_n$ suit,
quant à elle, une loi binomiale de paramètres $n$ et $p$.

Sous les conditions d'approximations usuelles ($n \leqslant 30, \dots$)
on a que si $n$ augmente $F_n$ appartient à $\interff{p -
\frac1{\sqrt{n}} ; p + \frac1{\sqrt{n}}}$ avec une probabilité supérieure
à $1 - \alpha$. On peut démontrer ensuite l'équivalence suivante.

\begin{propriete}~\\[-5mm]
  $\p\left(F_n \in \interff{p - \frac1{\sqrt{n}} ; p +
  \frac1{\sqrt{n}}}\right) \iff \p\left(p \in \interff{F_n -
  \frac1{\sqrt{n}} ; F_n + \frac1{\sqrt{n}}}\right)$
  \begin{preuve}
    Il suffit d'écrire ce que signifie l'appartenance aux intervalles et
    de raisonner sur l'inégalité de droite (ou de gauche.)
  \end{preuve}
\end{propriete}

On peut donc définir, pour une réalisation particulière l'intervalle de
confiance.

\begin{definition}
  L'intervalle de confiance au seuil de 95\% est l'intervalle
  $\interff{f_{\text{obs}} - \frac1{\sqrt{n}} ; f_{\text{obs}} +
  \frac1{\sqrt{n}}}$.
\end{definition}

Cet intervalle est d'amplitude $\frac2{\sqrt{n}}$.

\begin{remarque}
  On trouve parfois aussi l'intervalle d'amplitude
  $\frac{2\sqrt{f_{\text{obs}}(1-f_{\text{obs}})}}{\sqrt{n}}$.
\end{remarque}

\section{Prise de décision}

On suppose désormais que $p$ est connu et on cherche à vérifier dans une
population donnée si la proportion d'individus présentant la
caractéristique est bien $p$. On cherche pour cela à «éliminer le
hasard» dans le choix de notre échantillon ou du moins à quantifier la
fluctuation maximale admissible.

Pour cela, au seuil $\alpha$ fixé, on détermine un intervalle de
fluctuation asymptotique $I$ de la fréquence du caractère dans un
échantillon de taille $n$.

On observe ensuite dans un échantillon effectivement prelevé la
fréquence en question et on énonce puis applique la règle suivante :
\begin{itemize}
  \item si $f \notin I$, alors on rejette l'hypothèse au seuil $\alpha$
    ; le risque de rejet à tort est $\alpha$ ;
  \item si $f \in I$ alors on ne rejette pas l'hypothèse.
\end{itemize}

\end{document}
