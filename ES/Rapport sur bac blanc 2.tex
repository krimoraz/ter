\documentclass[a4paper,12Pt,french]{article}

\usepackage[utf8]{inputenc}
\usepackage[T1]{fontenc}

\usepackage{lmodern}

\usepackage[a4paper]{geometry}

\usepackage{babel}

\title{Rapport de correction du bac blanc 2 Terminale ES}
\author{Vincent-Xavier \bsc{Jumel}}
\date{20 avril 2016}

\begin{document}

\maketitle

\section*{Introduction}

L'épreuve proposée était de format classique pour une épreuve de type
bac à ce stade de l'année en terminale ES : 4 exercices portant sur des
points différents du programme, principalement celui de Terminale mais
avec des points du programme de Première.

Il est en revanche déplorable de trouver des erreurs graves, aussi bien
en terme d'interprétation que d'écriture. Je reviendrais sur ces points
dans le détail des exercices.

Les remarques que je formule ici sont à prendre dans leur ensemble :
aucune copie ne possède tous les défauts énumérés ici, mais aucune copie
ne comporte aucun défaut.

Enfin, la rédaction s'avère parfois inexistante ou peu appropriée.

\section{Exercice 1}

Il s'agissait d'un QCM conservateur, c'est à dire qu'il ne retirait pas
de point. On peut d'ailleurs montrer que si on réponds au hasard dans un
tel QCM, l'espérance est de 1.

L'objet de ce QCM était de tester les connaissances du cours en analyse.
Il est assez moyennement réussi, ce qui traduit un \textbf{manque
d'apprentissage du cours}.

Tracer la fonction sur la calculatrice permettait de répondre à la
question 4.

\section{Exercice 2}

Il s'agit de l'exercice différencié, mais dans les deux cas, l'exercice
conduisait à la recherche de la limite d'une suite géométrique.

Dans les deux cas, la première partie consistait en la description d'une
situation et l'explication du choix du modèle. Les explications sont
souvent \textbf{confuses, peu claires} et ne permettent pas
nécessairement de comprendre comment est construit le modèle
mathématique présenté.

Les candidats de spécialité sont invités à retrouver la différence entre
un \emph{graphe probabiliste} et un \emph{arbre pondéré de probabilité}.

Les candidats du tronc commun ont traité avec peu de succès les
questions portant sur la compréhension de l'algorithme. Celui-ci
permettait de trouver l'année qui conduirait la bibliothèque à ne plus
avoir de place.

La notion de suite géométrique est \textbf{peu maitrisée}. En effet, il
ne suffit pas de montre que $u_1 = qu_0$ ni d'écrire que $u_{n+1} =
qu_n$ pour que la suite soit géométrique, il faut le montrer
convenablement et pour tout $n$ entier naturel.

Il convient ensuite d'interroger la pertinence des résultats trouvés.
Pour le tronc commun, la limite de la suite $v_n$ correspond aux livres
dans la bibliothèque et donc ne peut être ni nulle ni infinie. Pour les
spécialistes, le total des deux partis, dans le modèle proposé, doit
toujours égaler 1.

\section{Exercice 3}

Un nombre substantiel d'élève n'a pas traité l'exercice du tout, ce qui
révèle un problème. Ceux qui ont abordé l'exercice ont rarement
justifiés les résultats qu'ils avancaient, ce qui est dommage. C'est
souvent le cas pour les variations de la fonction $C$.

Le graphique n'a pas toujours été tracé. À quelques exceptions près,
ceux qui ont fait le tracé l'ont fait correctement.

Le calcul de dérivée puis la mise au même dénominateur reste
\textbf{problématique} pour beaucoup d'élèves. C'est une compétence de
seconde.

Peu ont correctement cité le théorème des valeurs intermédiaires en
précisant toutes les hypothèses. À ce titre, vérifier, en explicitant,
que $B$ satisfait les hypothèes puis citer le théorème est correct.

Attention, la résolution algébrique directe n'est pas possible ici.

Puisque la solution est précisée $\alpha$, on peut l'utiliser pour le
tableau de signe de la question suivante.

Exprimer une quantité au dL près ne signifie pas qu'il faut l'exprimer
en dL. Il fallait également arrondir par excès pour être sûr que la
production serait suffisante.

La précision de l'arrondi était utile une fois de plus, lorsqu'il
fallait préciser à l'euro près et donc arrondir à un nombre entier.

\section{Exercice 4}

L'exercice a été globalement bien traité par les candidats, au moins
pour les 3 premières questions.

Pour la question 4, elle utilisait le résultat annoncé dans la question
3 qui était que l'expression de la \emph{formule des probabilités
totales}, expression jamais employée et une des formules dites de
\emph{Bayes} pour trouver le résultat.

Quant à la question 5, elle faisait appel à la loi binomiale et au
résultat de la question 3.

\end{document}
