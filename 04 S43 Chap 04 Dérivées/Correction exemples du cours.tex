\documentclass[a4paper,12pt,french]{article}

\usepackage[utf8]{inputenc}
\usepackage[T1]{fontenc}
\usepackage{lmodern}
\usepackage{cmbright}

\usepackage{hyperref}

\usepackage{amsmath}
\usepackage{ntheorem}

\theorembodyfont{\normalfont}
\theoremstyle{break}
\newtheorem{exercice}{Exercice}
\theoremstyle{nonumberplain}
\newtheorem{reponse}{Réponse}

\newcommand{\R}{\mathbf{R}}

\begin{document}

\begin{exercice}
  Sans développer dériver la fonction définie pour tout réel par
  $f(x)=(x^2 + 2x -3)^3$.

  \begin{reponse}
    On utilise $(u^n)' = nu'n^{n-1}$ avec $u(x) = x^2 + 2x - 3$. On a
    donc $u'(x) = 2x + 2$ et $f'(x) = 3\times(2x + 2)\times(x^2 + 2x
    -3)^2$.

    Il est ici inutile de développer, en particulier si on a besoin du
    signe de cette dérivée : $x^2 + 2x -3 = (x-1)(x+3)$ ce qui entraîne
    que la dérivée s'annule en changeant de signe en -1, 1 et 3.
  \end{reponse}
\end{exercice}

\begin{exercice}
  Dériver $x\mapsto \frac1{x^2 +1}$
  \begin{reponse}
    La fonction est de la forme $\frac1u$, avec $u = x\mapsto x^2 + 1$.

    $\forall x\in \R,\ x^2 + 1 > 0$ et on utilise ici $\left( \frac1u
    \right)' = \frac{-u}{u^2}$.

    On a donc $u' = x\mapsto 2x$ et donc la dérivée de la fonction
    considérée est la fonction $x \mapsto \frac{-2x}{(x^2 +1)^2}$.
  \end{reponse}
\end{exercice}

\begin{exercice}
  Dériver la fonction définie sur $\R$ par $f(x) = \sqrt{x^2 + 2x + 2}$
  \begin{reponse}
    Soit la fonction $u$ définie sur $\R$ par $u(x) = x^2 + 2x + 2$.

    Soit $x\in \R,\ u(x) = x^2 + 2x + 1 +1 = (x+1)^2 + 1 > 0$ (remarque
    : le calcul du discriminant montre ici que cette fonction
    polynômiale n'a pas de racines réelles.)

    $\forall x\in\R,\ u'(x) = 2x + 2$, donc $f'(x) = \frac{2x +
    2}{2\sqrt{x^2 + 2x + 2}} = \frac{x + 1}{\sqrt{x^2 + 2x + 2}}$.
  \end{reponse}
\end{exercice}
\end{document}
