\documentclass[12pt,a4paper,french]{article}
\usepackage[utf8]{inputenc}
\usepackage[T1]{fontenc}
\usepackage{babel}
\usepackage[thmmarks]{ntheorem}
\usepackage{amsmath}
\usepackage{amsfonts}
\usepackage{amssymb}

\usepackage{array}

\usepackage{lmodern}
\usepackage{kpfonts}
\usepackage{cmbright}

\usepackage[top=1.7cm,bottom=2cm,left=2cm,right=2cm]{geometry}

\usepackage{ifthen}
\usepackage{fancyhdr}
\pagestyle{fancy}

\count1=\year \count2=\year
\ifnum\month<8\advance\count1by-1\else\advance\count2by1\fi
\pagestyle{fancy}
\cfoot{\textsl{\footnotesize{Année \number\count1/\number\count2}}}
\lfoot{\textsl{\footnotesize{Lycée \textsc{LaSalle Saint-Denis}}}}
\rfoot{\footnotesize{Page \thepage/ \pageref{LastPage}}}
\rhead{}
%\lhead{\ifthenelse{\value{page}=1}{Nom:\dotfill\hfill Prénom: \dotfill
%\hfill ~}{}}
\renewcommand{\headrulewidth}{0pt}
\renewcommand{\footrulewidth}{0pt}

\pdfminorversion 7
\pdfobjcompresslevel 3

%\frenchbsetup{ItemLabels=\textbullet,}

\usepackage[bookmarks=false,colorlinks,linkcolor=blue]{hyperref}

\hypersetup{%
  pdftitle = {Exercices sur la dérivation},
  pdfauthor = {Vincent-Xavier Jumel},
  pdfkeywords = {TS, exercices, dérivation, TVI},
}

\usepackage{tabularx}
\usepackage[autolanguage,np]{numprint}
\usepackage[inline]{enumitem}

\usepackage{tipfr}
\usepackage{pgf}
\usepackage{tikz}
\usepackage{tkz-euclide}
\usetkzobj{all}
%\usetikzlibrary{hobby}
\usepackage{tkz-tab}

\usepackage{lastpage}
\usepackage{esvect}
\usepackage{marginnote}

\usepackage{wrapfig}

\usepackage[defaultlines=5,all]{nowidow}

\usepackage[tikz]{bclogo}


\makeatletter

\renewcommand{\maketitle}%
{\framebox{%
    \begin{minipage}{1.0\linewidth}%
      \begin{center}%
        \Large \@title ~-- \@author \\%
        \@date%
      \end{center}%
    \end{minipage}}%
  \normalsize%
  %\vspace{1cm}%
}

\pgfdeclarepatternformonly{mes_hachures}
{\pgfpoint{-0.1cm}{-0.1cm}}
{\pgfpoint{0.9cm}{0.5cm}}
{\pgfpoint{0.8cm}{0.4cm}}
{\pgfpathmoveto{\pgfpointorigin}
  \pgfpathlineto{\pgfpoint{0.8cm}{0.4cm}}
\pgfusepath{stroke}}

%Des macros pour les noms d'ensembles
\newcommand{\R}{\mathbf{R}}
\newcommand{\Q}{\mathbf{Q}}
\newcommand{\Z}{\mathbf{Z}}
\newcommand{\C}{\mathbf{C}}
\newcommand{\N}{\mathbf{N}}

\newcommand{\norme}[1]{\left\lVert #1 \right\rVert}
\newcommand{\abs}[1]{\left\lvert #1 \right\rvert}

%Une macro récursive pour l'intérieur des vecteurs
%http://tex.stackexchange.com/questions/19693/arguments-of-custom-commands-as-comma-separated-list

\newcommand\vecteur[2][\\]{%
    \global\def\my@delim{#1}%
    \left(\negthinspace\begin{matrix}
        \my@vector #2,\relax\noexpand\@eolst%
    \end{matrix}\right)}

%Une macro pour les vecteurs
\def\my@vector #1,#2\@eolst{%
   \ifx\relax#2\relax
      #1
   \else
      #1\my@delim
      \my@vector #2\@eolst%
   \fi}

%Une macro récursive pour mettre formater l'intérieur des intervalles
\def\my@intervalle #1;#2\@eolst{%
  \ifx\relax#2\relax
    #1
  \else
    \my@intervalle #2\@eolst%
  \fi}

%Quatre macros pour les quatre types d'intervalles
\newcommand{\interff}[1]{%
  \left[\my@intervalle #1;\relax\noexpand\@eolst%
  \right]
}
\newcommand{\interfo}[1]{%
  \left[\my@intervalle #1;\relax\noexpand\@eolst%
  \right[}
\newcommand{\interof}[1]{%
  \left]\my@intervalle #1;\relax\noexpand\@eolst%
  \right]}
\newcommand{\interoo}[1]{%
  \left]\my@intervalle #1;\relax\noexpand\@eolst%
  \right[}

\makeatother


\usepackage{mdframed}

\theoremstyle{break}
\newtheorem{definition}{Définition}
\newtheorem{propriete}{Propriété}
\newtheorem{corollaire}{Corollaire}
\newtheorem{propdef}{Propriété - Définition}
\newtheorem{theoreme}{Théorème}
\theoremstyle{plain}
\theorembodyfont{\normalfont}
\newtheorem{exerciceT}{Exercice}
\theoremstyle{nonumberplain}
\newtheorem{remarque}{Remarque}
\newtheorem{notation}{Notation}
\newtheorem{probleme}{Problème}
\theoremsymbol{\ensuremath{\blacksquare}}
\newtheorem{preuve}{Preuve}
\theoremsymbol{}
\theoremstyle{nonumberbreak}
\newtheorem{exemple}{Exemple}

\newenvironment{exercice}{\begin{framed}\begin{exerciceT}}{\end{exerciceT}\end{framed}}

\setlength{\parsep}{0pt}
\setlength{\parskip}{5pt}
\setlength{\parindent}{0pt}
\setlength{\itemsep}{7pt}

\setlist{noitemsep}
%\setlist[1]{\labelindent=\parindent} % < Usually a good idea
\setlist[itemize]{leftmargin=*}
\setlist[itemize,1]{label=$\triangleright$}
\setlist[enumerate]{labelsep=*, leftmargin=1.5pc}
\setlist[enumerate,1]{label=\arabic*., ref=\arabic*}
\setlist[enumerate,2]{label=\emph{\alph*}),
ref=\theenumi.\emph{\alph*}}
\setlist[enumerate,3]{label=\roman*), ref=\theenumii.\roman*}
\setlist[description]{font=\sffamily\bfseries}

\usepackage{multicol}
\setlength{\columnseprule}{0pt}

\usepackage[]{exsheets}
\SetupExSheets{
  headings = block-subtitle,
%  question/pre-hook = \mdframed,
%  question/post-hook = \endmdframed,
  counter-format = qu[1],
}
\usepackage{exsheets-listings}

\everymath{\displaystyle\everymath{}}

\title{Exercices sur la dérivation}
\author{\bsc{Ts 3}}
\date{novembre 2016}

\tikzstyle{inclue}=[draw, circle,fill, inner sep = 1pt]
\tikzstyle{exclue}=[draw, circle, fill = white,inner sep=1pt]

\begin{document}

\maketitle

\bigskip

\begin{question}[subtitle={Un peu de calcul}, class=1,
  topic={dérivabilité}]
  Donner l'expression de la dérivée des fonctions suivants en précisant
  l'ensemble de dérivabilité.

  \begin{enumerate*}[itemjoin=\hfill]
    \item $x\mapsto x^3 + 2x - 1$
    \item $x\mapsto 5x^2 + 3\sqrt{2x}$
    \item $x\mapsto x + \frac1x$
    \item $x\mapsto \frac{3}{2x}$
  \end{enumerate*}
\end{question}

\begin{question}[subtitle={Calculs de dérivées},
  class=3,topic={dérivabilité}]
  Pour les fonctions suivantes, préciser sur quels intervalles, elle
  sont dérivables et donner l'expression de la dérivée.

  \begin{enumerate*}[itemjoin=\hspace{5mm}\hfill]
    \item $x\mapsto (2x + 1)(3x -3)$
    \item $x\mapsto (x^2 - 6x + 9)^2$
    \item $x\mapsto \frac{x + 2}{x+1}$
    \item $x\mapsto \frac{9x^2 - 1}{1 + \sqrt{5}}$
    \item $x\mapsto \frac{3 + \sqrt{2}}{25 - 16x^2}$
    \item $x\mapsto x\sqrt{x}$
    \item $x\mapsto \sqrt{x^2 + 1}$
    \item $x\mapsto \sqrt{5x^2 - 5}$
  \end{enumerate*}
\end{question}

\begin{question}[subtitle={Version graphique},
  class=1,topic={dérivabilité}]
  Pour chacune des représentations graphiques ci-dessous, tracer la
  tangente lorsque c'est possible aux points $x_k$ indiqués sur l'axe
  des abscisses.

  \begin{multicols}{3}
    \begin{center}
      \begin{tikzpicture}[scale=0.5,>=latex]
        \begin{scope}
          \clip (-5,-5) rectangle (5,5) ;
          \draw [->] (-5,0) -- (5,0) ;
          \draw [->] (0,-5) -- (0,5) ;
          \draw [thin,dotted] (-5,-5) grid (5,5) ;
          \draw [thick] plot [smooth,domain=-2:0] (\x,{sqrt(1-(\x+1)^2)}) ;
          \draw [thick] plot [smooth,domain=0:4] (\x,{2*sqrt(1-(\x/2-1)^2)}) ;
          \draw [thick] plot [smooth,domain=4:5] (\x,{4*sqrt(\x-4)}) ;
          \draw (-2,0) node [inclue] {} ;
        \end{scope}

        \draw (0,-6) node {$x_0 = -1$ ; $x_1 = 1$ ; $x_2 = 4$ };
      \end{tikzpicture}
    \end{center}

    \begin{center}
      \begin{tikzpicture}[scale=0.5,>=latex]
        \begin{scope}
          \clip (-5,-5) rectangle (5,5) ;
          \draw [->] (-5,0) -- (5,0) ;
          \draw [->] (0,-5) -- (0,5) ;
          \draw [thin,dotted] (-5,-5) grid (5,5) ;
          \draw [thick] plot [smooth,domain=-5:0] (\x,{sqrt(-\x)}) ;
          \draw [thick] plot [smooth,domain=0:1] (\x,{-sqrt(\x)}) ;
          \draw [thick] plot [smooth,domain=1:3] (\x,-\x) ;
          \draw [thick] plot [smooth,domain=3:5] (\x,-6 + \x) ;
        \end{scope}

        \draw (0,-6) node {$x_0 = -1$ ; $x_1 = 1$ ; $x_2 = 4$ };
      \end{tikzpicture}
    \end{center}

    \begin{center}
      \begin{tikzpicture}[scale=0.5,>=latex]
        \begin{scope}
          \clip (-5,-5) rectangle (5,5) ;
          \draw [->] (-5,0) -- (5,0) ;
          \draw [->] (0,-5) -- (0,5) ;
          \draw [thin,dotted] (-5,-5) grid (5,5) ;
          \draw [thick] plot [smooth,domain=-2:0] (\x,{sqrt(1-(\x+1)^2)}) ;
          \draw [thick] plot [smooth,domain=0:4] (\x,{2*sqrt(1-(\x/2-1)^2)}) ;
          \draw [thick] plot [smooth,domain=4:5] (\x,{4*sqrt(\x-4)}) ;
          \draw (-2,0) node [inclue] {} ;
        \end{scope}

        \draw (0,-6) node {$x_0 = -1$ ; $x_1 = 1$ ; $x_2 = 4$ };
      \end{tikzpicture}
    \end{center}
  \end{multicols}
\end{question}


\end{document}

\pagebreak

\section*{Solutions des exercices}

\vspace{-7mm}

\printsolutions

\end{document}
