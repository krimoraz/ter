\documentclass[12pt,a4paper,french]{article}
\usepackage[utf8]{inputenc}
\usepackage[T1]{fontenc}
\usepackage{babel}
\usepackage{ntheorem}
\usepackage{amsmath}
\usepackage{amsfonts}
\usepackage{amssymb}

\usepackage{array}

\usepackage{kpfonts}

\usepackage[bookmarks=false,colorlinks,linkcolor=blue,pdfusetitle]{hyperref}

\pdfminorversion 7
\pdfobjcompresslevel 3

\usepackage{tabularx}
\usepackage[autolanguage,np]{numprint}
\usepackage{enumitem}

\usepackage{tipfr}
\usepackage{pgf}
\usepackage{tikz}
\usepackage{tkz-euclide}
\usetkzobj{all}
\usetikzlibrary{hobby}

\usepackage[top=1.7cm,bottom=2cm,left=2cm,right=2cm]{geometry}

\usepackage{lastpage}

\usepackage{esvect}
\usepackage{marginnote}

\usepackage{wrapfig}

\usepackage[defaultlines=5,all]{nowidow}


\makeatletter
\renewcommand{\@evenfoot}%
        {\hfil \upshape \small page {\thepage} de \pageref{LastPage}}
\renewcommand{\@oddfoot}{\@evenfoot}

\renewcommand{\maketitle}%
{\framebox{%
    \begin{minipage}{1.0\linewidth}%
      \begin{center}%
        \Large \@title ~-- \@author \\%
        \@date%
      \end{center}%
    \end{minipage}}%
  \normalsize%
  %\vspace{1cm}%
}

\pgfdeclarepatternformonly{mes_hachures}
{\pgfpoint{-0.1cm}{-0.1cm}}
{\pgfpoint{0.9cm}{0.5cm}}
{\pgfpoint{0.8cm}{0.4cm}}
{\pgfpathmoveto{\pgfpointorigin}
  \pgfpathlineto{\pgfpoint{0.8cm}{0.4cm}}
\pgfusepath{stroke}}

%Des macros pour les noms d'ensmbles
\newcommand{\R}{\mathbf{R}}
\newcommand{\Q}{\mathbf{Q}}
\newcommand{\Z}{\mathbf{Z}}
\newcommand{\C}{\mathbf{C}}
\newcommand{\N}{\mathbf{N}}

\newcommand{\norme}[1]{\left\lVert #1 \right\rVert}
\newcommand{\abs}[1]{\left\lvert #1 \right\rvert}

%Une macro récursive pour l'intérieru des vecteurs
%http://tex.stackexchange.com/questions/19693/arguments-of-custom-commands-as-comma-separated-list

\newcommand\vecteur[2][\\]{%
    \global\def\my@delim{#1}%
    \left(\negthinspace\begin{smallmatrix}
        \my@vector #2,\relax\noexpand\@eolst%
    \end{smallmatrix}\right)}

%Une macro pour les vecteurs
\def\my@vector #1,#2\@eolst{%
   \ifx\relax#2\relax
      #1
   \else
      #1\my@delim
      \my@vector #2\@eolst%
   \fi}

%Une macro récursive pour mettre formater l'intérieur des intervalles
\def\my@intervalle #1;#2\@eolst{%
  \ifx\relax#2\relax
    #1
  \else
    \my@intervalle #2\@eolst%
  \fi}

%Quatre macros pour les quatres types d'intervalles
\newcommand{\interff}[1]{%
  \left[\my@intervalle #1;\relax\noexpand\@eolst%
  \right]
}
\newcommand{\interfo}[1]{%
  \left[\my@intervalle #1;\relax\noexpand\@eolst%
  \right[}
\newcommand{\interof}[1]{%
  \left]\my@intervalle #1;\relax\noexpand\@eolst%
  \right]}
\newcommand{\interoo}[1]{%
  \left]\my@intervalle #1;\relax\noexpand\@eolst%
  \right[}

\newcommand*{\versioncomplet}{}
\newcommand{\complet}[2]{%
  \ifdefined\versioncomplet
    {\color{blue} #2}
  \else
   ~\\\hspace{0.98\linewidth}\vspace{#1}
  \fi
}


\makeatother


\theoremstyle{break}
\newtheorem{definition}{Définition}
\newtheorem{propriete}{Propriété}
\newtheorem{propdef}{Propriété - Définition}
\newtheorem{theoreme}{Théorème}
\theoremstyle{plain}
\theorembodyfont{\normalfont}
\newtheorem{exercice}{Exercice}
\theoremstyle{nonumberplain}
\newtheorem{remarque}{Remarque}
\newtheorem{probleme}{Problème}
\newtheorem{preuve}{Preuve}
\theoremstyle{nonumberbreak}
\newtheorem{exemple}{Exemple}

\setlength{\parsep}{0pt}
\setlength{\parskip}{5pt}
\setlength{\parindent}{0pt}
\setlength{\itemsep}{7pt}

\setlist{noitemsep}
%\setlist[1]{\labelindent=\parindent} % < Usually a good idea
\setlist[itemize]{leftmargin=*}
\setlist[itemize,1]{label=$\triangleright$}
\setlist[enumerate]{labelsep=*, leftmargin=1.5pc}
\setlist[enumerate,1]{label=\arabic*., ref=\arabic*}
\setlist[enumerate,2]{label=\emph{\alph*}),
ref=\theenumi.\emph{\alph*}}
\setlist[enumerate,3]{label=\roman*), ref=\theenumii.\roman*}
\setlist[description]{font=\sffamily\bfseries}

\usepackage{multicol}
\setlength{\columnseprule}{0pt}

\usepackage{exsheets}

\everymath{\displaystyle\everymath{}}

\title{Dérivation des fonctions : dégager une formule générale}
\author{Terminale S}
\date{novembre 2016}

\begin{document}

\maketitle

\section{Prise d'initiative}

Commençons par quelques calculs de dérivées.

\begin{question}
  Calculer de deux façons différentes les dérivées des fonctions
  suivantes :
  \begin{enumerate}
    \item $f:x\mapsto (x^2 + 3x -1)(x + 1)$
    \item $f:x\mapsto x^2\times x$
    \item $f:x\mapsto \sqrt{x}\sqrt{x}$
  \end{enumerate}
\end{question}

Qu'est ces deux derniers calculs permettent de démontrer ?

Proposer un calcul permettant de démontrer la dérivée de la fonction
inverse.

Rédiger une démonstration de la dérivée des fonctions puissances
($x\mapsto x^n$).

On va désormais chercher à démontrer la formule de dérivation d'un
produit que vous avez utilisé.

\begin{propriete}
  Soient $u$ et $v$ deux fonctions dérivables.

  La dérivée du produit $uv$ est \[u'v + uv'\] où $u'$ et $v'$ sont les
  dérivées respectives des fonctions $u$ et $v$.
\end{propriete}

Tracer les fonctions de l'exercice 1 de façon à obtenir ce graphique sur
la calculatrice :
\begin{center}
  \Ecran[width=12, height=8, graphic=true, xgrad=1, ygrad=1,
  origin={(6,4)}]%
  {%
    \x*\x+3*\x - 1/-5:5,%
    \x + 1/-5:5,%
    (\x*\x+3*\x - 1)*(\x+1)/-5:5%
  }
\end{center}


\pagebreak
\section{La dérivée du produit}

On considère deux fonctions $f$ et $g$ dérivables sur $\interoo{a,b}$.

\begin{enumerate}
  \item Citer un théorème du cours pour justifier que les fonctions sont
    continues sur $\interff{a,b}$.
  \item Soit $x_0$ un réel de l'intervalle $\interoo{a,b}$.
    \begin{enumerate}
      \item Que vaut $- f(x_0 + h)g(x_0) + f(x_0 + h)g(x_0)$ ?
      \item En utilisant la question précédente, écrire la différence
        $f(x_0+h)g(x_0+h) - f(x_0)g(x_0)$ sous la forme d'une somme de
        deux différences.
      \item Compléter : $\frac{f(x_0+h)g(x_0+h) - f(x_0)g(x_0)}h = \dots
        + \dots$.
    \end{enumerate}
  \item
    \begin{enumerate}
      \item Que vaut le membre de gauche de l'égalité précédente si on
        fait tendre $h$ vers 0 ?
      \item Peut-on passer à la limite dans le membre de droite ?

        Justifier la réponse.
      \item Écrire ce membre de droite sous la forme de la somme de deux
        nombres dérivés.
    \end{enumerate}
  \item Conclure : la dérivée $(fg)'$ du produit des fonctions
    dérivables est \dots.
\end{enumerate}

\section{Utilisation du produit pour en déduire d'autres égalités}

Soient $u$ une fonction définie sur $\interff{a,b}$ dérivable sur
$\interoo{a,b}$ et $x$ un nombre réel dans cet intervalle.

\begin{enumerate}
  \item
    \begin{enumerate}
      \item Que vaut $(u(x))^0$ ?
      \item En écrivant $(u(x))^2 = u(x)\times u(x)$, écrire la dérivée de
        $(u(x))^2$ en fonction de $u(x)$ et de $u'(x)$.
      \item Démontrer par récurrence que $(u(x))^n =
        nu'(x)(u(x))^{n-1}$.
    \end{enumerate}
  \item On suppose dans cette question que $u(x) > 0$ sur $\interff{a,b}$
    \begin{enumerate}
      \item Que vaut $\sqrt{u(x)}\sqrt{u(x)}$ ?
      \item En utilisant les des expressions précédentes, dériver à
        droite et à gauche l'égalité précédente.
      \item En déduire la dérivée de $(\sqrt{u(x)})'$.
    \end{enumerate}
  \item On suppose que dans cette question $u(x) \neq 0$ sur
    $\interff{a,b}$. On suppose de plus que si $u$ est dérivable,
    $v:x\mapsto \left(\frac1u(x)\right)$ est dérivable.

    \begin{enumerate}
      \item Que vaut le produit $u(x)v(x)$ pour $x$ fixé ?
      \item Écrire la dérivée du produit $uv$ de deux façons
        différentes.
      \item Exprimer $v'$ en fonction de $u$ et de $v$.
      \item En revenant à la définition de $v$, donner la dérivée de
        $\frac1u$.
    \end{enumerate}
\end{enumerate}

\section*{Bilan}

\begin{enumerate}
  \item Pour les trois résultats précédents, comparer les aux dérivées des
    fonctions $f:x\mapsto x^n$, $g:x\mapsto \sqrt{x}$ et $h:x\mapsto \frac1x$.
  \item Comment peut-on écrire les fonctions des trois questions
    précédentes et leurs dérivées en fonction de $f$, $g$, $h$ et $u$ ?
  \item Quelle est la forme qu'on peut écrire pour tradiuire une
    relation générale ?
\end{enumerate}

\end{document}
